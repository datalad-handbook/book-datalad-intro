\chapter*{Preface to the print edition}
\addcontentsline{toc}{chapter}{Preface}

The story behind this book began in June 2019 in Rome at the Organization for
Human Brain Mapping conference. We, Adina and Michael, sat in the front rows in
a symposium session ‘From "Open Science" to "Science": Shifting the status quo
in data sharing, software, and publishing’. The speaker highlighted DataLad as
a fantastic tool for open science and data sharing, but followed up with an
honest and crushing assessment of his learning experience with it. Out of the
mix of laughter, shame, and deeper thought that followed, we formed the idea
for better documentation -- we needed a handbook for DataLad that could make the
software accessible to everyone.

We started with a web page, and a rough idea. Adina knew the painful learning
experience all too well - when she had started to use DataLad, she didn’t know
about version control, had rarely touched a computer terminal, and had virtually no
experience in data management at all. Michael, a lead developer of DataLad, had
tried to document the tool beyond standard software documentation a few times
already, and other developers weren’t confident that yet another attempt would
be anything but wasted time.

Within a year, the \textit{DataLad Handbook} had grown into a comprehensive
documentation of almost all of DataLad’s basic functionality. Dozens of
contributors have engaged and contributed content, ideas, questions, and use
cases. Conjoint work on documentation and software helped to iron out kinks in
the software, and sparked ideas for improvements and new features. At the point
that we publish this book, there have been six releases of the
online-handbook and DataLad, dozens of workshops based on it, and countless web
visitors of \url{https://handbook.datalad.org}, where the online-handbook was, and will be
developed, free and open to everyone.

Even though you are already holding this book in your hands, a visit to the
online-handbook is highly recommended. The information available online extends
far beyond the scope of this book. Only about half the chapters have been used
for this basic introduction to research data management with DataLad. The
online-handbook also hosts a page with further information and updates to this
printed book at:

\hspace{1cm}\url{https://handbook.datalad.org/perma/book-intro-v1.html}.

We are proud and thankful that everything about this project is free and open
source. The tools that are the technical backbone of this book, Sphinx, LaTeX,
Python, and the fantastic open source illustrations by Katerina Limpitsouni
(undraw.co) have made this book more than just its contents.

Finally, and most
importantly, we want to thank and acknowledge everyone who has contributed to
this open resource: Laura Waite, Kyle Meyer, Marisa Heckner, Benjamin
Poldrack, Yaroslav Halchenko, Chris Markiewicz, Pattarawat Chormai, Lisa N.
Mochalski, Lisa Wiersch, Jean-Baptiste Poline, Nevena Kraljevic, Alex Waite,
Lya K. Paas, Niels Reuter, Peter Vavra, Tobias Kadelka, Peer Herholz, Alexandre
Hutton, Sarah Oliveira, Dorian Pustina, Hamzah Hamid Baagil, Tristan Glatard,
Giulia Ippoliti, Christian Mönch, Togaru Surya Teja, Dorien Huijser, Ariel
Rokem, Remi Gau, Judith Bomba, Konrad Hinsen, Jianxiao Wu, Małgorzata Wierzba,
Stefan Appelhoff, Michael Joseph, Tamara Cook, Stephan Heunis, Joerg Stadler,
Sin Kim, Oscar Esteban, Michał Szczepanik, Eduart Ort, Myrskyta, Thomas Guiot,
Julius Breuer, Ikko Ashimine, Arshitha Basavaraj, Anthony J Veltri, Isil Bilgin,
Julian Kosciessa, Isaac To, Austin Macdonald, Christopher S. Hall, John C. Ford,
Julien Colomb, and Danny Garside.
