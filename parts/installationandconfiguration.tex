\chapter{Installation and configuration}
\label{\detokenize{intro/installation:installation-and-configuration}}\label{\detokenize{intro/installation:install}}\label{\detokenize{intro/installation::doc}}
\noindent{\hspace*{\fill}\sphinxincludegraphics[height=.3\textheight]{{interior_design_bw}.pdf}\hspace*{\fill}}

\index{installation@\spxentry{installation}!DataLad@\spxentry{DataLad}}\ignorespaces 

\section{Install DataLad}
\label{\detokenize{intro/installation:install-datalad}}\label{\detokenize{intro/installation:index-0}}\begin{importantnote}[before title={\thetcbcounter\ }, check odd page=true]{Feedback on installation instructions}

\sphinxAtStartPar
The installation methods presented in this chapter are based on experience
and have been tested carefully. However, operating systems and other
software are continuously evolving, and these guides might have become
outdated. Be sure to check out the online\sphinxhyphen{}handbook for up\sphinxhyphen{}to\sphinxhyphen{}date information.


\end{importantnote}

\sphinxAtStartPar
In general, the DataLad installation requires Python 3 (see the
\textit{Find-out-more}~{\findoutmoreiconinline}\textit{\ref{fom-py2v3}} {\hyperref[\detokenize{intro/installation:fom-py2v3}]{\sphinxcrossref{\DUrole{std,std-ref}{on the difference between Python 2 and 3}}}} (\autopageref*{\detokenize{intro/installation:fom-py2v3}}) to learn
why this is required), {\hyperref[\detokenize{glossary:term-Git}]{\sphinxtermref{\DUrole{xref,std,std-term}{Git}}}}, and {\hyperref[\detokenize{glossary:term-git-annex}]{\sphinxtermref{\DUrole{xref,std,std-term}{git\sphinxhyphen{}annex}}}}, and for some
functionality \dlhbhref{71}{7\sphinxhyphen{}Zip}.  The instructions below detail how
to install the core DataLad tool and its dependencies on common operating
systems. They do not cover the various {\hyperref[\detokenize{glossary:term-DataLad-extension}]{\sphinxtermref{\DUrole{xref,std,std-term}{DataLad extension}}}}s that need to be installed separately, if desired.

\index{determine version@\spxentry{determine version}!with Python@\spxentry{with Python}}\index{with Python@\spxentry{with Python}!determine version@\spxentry{determine version}}\ignorespaces \begin{findoutmore}[label={fom-py2v3}, before title={\thetcbcounter\ }, float, floatplacement=tb, check odd page=true]{Python 2, Python 3, what’s the difference?}
\label{\detokenize{intro/installation:fom-py2v3}}

\sphinxAtStartPar
DataLad requires Python 3.8, or a more recent version, to be installed on your system.
The easiest way to verify that this is the case is to open a terminal and type \sphinxcode{\sphinxupquote{ python}} to start a Python session:

\begin{sphinxVerbatim}[commandchars=\\\{\}]
\PYG{g+gp}{\PYGZdl{} }python
\PYG{g+go}{Python 3.9.1+ (default, Jan 20 2021, 14:49:22)}
\PYG{g+go}{[GCC 10.2.1 20210110] on linux}
\PYG{g+go}{Type \PYGZdq{}help\PYGZdq{}, \PYGZdq{}copyright\PYGZdq{}, \PYGZdq{}credits\PYGZdq{} or \PYGZdq{}license\PYGZdq{} for more information.}
\PYG{g+go}{\PYGZgt{}\PYGZgt{}\PYGZgt{}}
\end{sphinxVerbatim}

\sphinxAtStartPar
If this fails, or reports a Python version with a leading \sphinxcode{\sphinxupquote{2}}, such as \sphinxcode{\sphinxupquote{Python 2.7.18}}, try starting \sphinxcode{\sphinxupquote{ python3}}, which some systems use to disambiguate between Python 2 and Python 3.
If this fails, too, you need to obtain a recent release of Python 3. On Windows, attempting to run commands that are not installed might cause a Windows Store window to pop up.
If this happens, Python may not yet be installed. Please check the {\hyperref[\detokenize{intro/installation:windows-10-and-11}]{\sphinxcrossref{Windows 10 and 11}}} (\autopageref*{\detokenize{intro/installation:windows-10-and-11}}) installation instructions, and \sphinxstyleemphasis{do not} install Python via the Windows Store.

\sphinxAtStartPar
Python 2 is an outdated, in technical terms “deprecated”, version of Python.
Although it still exist as the default Python version on many systems, it is no longer maintained since 2020, and thus, most software has dropped support for Python 2.
If you only run Python 2 on your system, most Python software, including DataLad, will be incompatible, and hence unusable,  resulting in errors during installation and execution.

\sphinxAtStartPar
But does that mean that you should uninstall Python 2?  \sphinxstylestrong{No}!  Keep it installed, especially if you are using Linux or MacOS.
Python 2 existed for 20 years and numerous software has been written for it.
It is quite likely that some basic operating system components or legacy software on your computer is depending on it, and uninstalling a preinstalled Python 2 from your system will likely render it unusable.
Install Python 3, and have both versions coexist peacefully.


\end{findoutmore}

\sphinxAtStartPar
The following sections provide targeted installation instructions for a set of
common scenarios, operating systems, or platforms.

\noindent{\hspace*{\fill}\sphinxincludegraphics[width=0.500\linewidth]{{install_bw}.pdf}\hspace*{\fill}}

\index{install DataLad@\spxentry{install DataLad}!on Windows@\spxentry{on Windows}}\index{on Windows@\spxentry{on Windows}!install DataLad@\spxentry{install DataLad}}\ignorespaces 

\subsection{Windows 10 and 11}
\label{\detokenize{intro/installation:windows-10-and-11}}\label{\detokenize{intro/installation:index-2}}
\sphinxAtStartPar
There are countless ways to install software on Windows. Here we describe \sphinxstyleemphasis{one}
possible approach that should work on any Windows computer, like one that you
may have just bought.
\begin{description}
\sphinxlineitem{Python:}
\index{install Python@\spxentry{install Python}!on Windows@\spxentry{on Windows}}\index{on Windows@\spxentry{on Windows}!install Python@\spxentry{install Python}}\index{installation@\spxentry{installation}!Python@\spxentry{Python}}\ignorespaces 
\sphinxAtStartPar
Windows itself does not ship with Python, it must be installed separately.
If you already did that, please check the \textit{Find-out-more}~{\findoutmoreiconinline}\textit{\ref{fom-py2v3}} {\hyperref[\detokenize{intro/installation:fom-py2v3}]{\sphinxcrossref{\DUrole{std,std-ref}{on Python
versions}}}} (\autopageref*{\detokenize{intro/installation:fom-py2v3}}), if it matches the requirements. Otherwise, head over
to the \dlhbhref{P8A}{download section of the Python website}, and download an installer. Unless you
have specific requirements, go with the 64bit installer of the latest
Python 3 release.
\begin{windowswit}[before title={\thetcbcounter\ }, check odd page=true]{Avoid installing Python from the Windows store}

\sphinxAtStartPar
We recommend to \sphinxstylestrong{not} install Python via the Windows store, even if it
opens after you typed \sphinxcode{\sphinxupquote{ python}}, as this version requires
additional configurations by hand (in particular of your \sphinxcode{\sphinxupquote{\$PATH}}
{\hyperref[\detokenize{glossary:term-environment-variable}]{\sphinxtermref{\DUrole{xref,std,std-term}{environment variable}}}}).


\end{windowswit}

\sphinxAtStartPar
When you run the installer, make sure to select the \sphinxstyleemphasis{Add Python to PATH} option,
as this is required for subsequent installation steps and interactive use later on.
Other than that, using the default installation settings is just fine.
\begin{windowswit}[before title={\thetcbcounter\ }, check odd page=true]{Verify Python installation}

\sphinxAtStartPar
It is not uncommon for multiple Python installations to co\sphinxhyphen{}exist on a Windows machine, because particular applications can ship their own.
Such alternative installations may even be or become the default.
This can cause confusing behavior, because each Python installation will have different package versions installed.

\sphinxAtStartPar
To verify if there are multiple installations, open the windows command line \sphinxcode{\sphinxupquote{cmd.exe}} and run \sphinxcode{\sphinxupquote{where python}}.
This will list all variants of \sphinxcode{\sphinxupquote{python.exe}}.
There will be one in \sphinxcode{\sphinxupquote{WindowsApps}}, which is only a link to the Windows app store.
Make sure the Python version that you installed is listed too.

\sphinxAtStartPar
If there are multiple Python installation, you can tell which one is default by running this command in \sphinxcode{\sphinxupquote{cmd.exe}}:

\begin{sphinxVerbatim}[commandchars=\\\{\}]
\PYG{o}{\PYGZgt{}} \PYG{n}{python} \PYG{o}{\PYGZhy{}}\PYG{n}{c} \PYG{l+s+s2}{\PYGZdq{}}\PYG{l+s+s2}{import sys; print(sys.executable)}\PYG{l+s+s2}{\PYGZdq{}}
\end{sphinxVerbatim}

\sphinxAtStartPar
This will print the path of the default \sphinxcode{\sphinxupquote{python.exe}}.
If the output is not matching the expected Python installation, likely the \sphinxcode{\sphinxupquote{\$PATH}} environment variable needs to be adjusted.
This can be done in the Windows system properties.
It is sufficient to move the entries created by the Python installer to the start of the declaration list.


\end{windowswit}

\sphinxlineitem{Git:}
\index{install Git@\spxentry{install Git}!on Windows@\spxentry{on Windows}}\index{on Windows@\spxentry{on Windows}!install Git@\spxentry{install Git}}\index{installation@\spxentry{installation}!Git@\spxentry{Git}}\ignorespaces 
\sphinxAtStartPar
Windows also does not come with Git. If you happen to have it installed already,
please check if you have configured it for command line use. You should be able
to open the Windows command prompt and run a command like \sphinxcode{\sphinxupquote{ git \sphinxhyphen{}\sphinxhyphen{}version}}.
It should return a version number and not an error.

\sphinxAtStartPar
To install Git, visit the \dlhbhref{G1O}{Git website} and
download an installer. If in doubt, go with the 64bit installer of the latest
version. The installer itself provides various customization options. We
recommend to leave the defaults as they are, in particular the target
directory, but configure the following settings (they are distributed over
multiple dialogs):
\begin{itemize}
\item {} 
\sphinxAtStartPar
Select \sphinxstyleemphasis{Git from the command line and also from 3rd\sphinxhyphen{}party software}

\item {} 
\sphinxAtStartPar
\sphinxstyleemphasis{Enable file system caching}

\item {} 
\sphinxAtStartPar
\sphinxstyleemphasis{Select Use external OpenSSH}

\item {} 
\sphinxAtStartPar
\sphinxstyleemphasis{Enable symbolic links}

\end{itemize}

\sphinxlineitem{Git\sphinxhyphen{}annex:}
\index{install git\sphinxhyphen{}annex@\spxentry{install git\sphinxhyphen{}annex}!on Windows@\spxentry{on Windows}}\index{on Windows@\spxentry{on Windows}!install git\sphinxhyphen{}annex@\spxentry{install git\sphinxhyphen{}annex}}\index{installation@\spxentry{installation}!git\sphinxhyphen{}annex@\spxentry{git\sphinxhyphen{}annex}}\ignorespaces 
\sphinxAtStartPar
There are two convenient ways to install git\sphinxhyphen{}annex. The first is \sphinxhref{https://git-annex.branchable.com/install/Windows}{downloading the installer from git\sphinxhyphen{}annex’ homepage}. The other is to deploy git\sphinxhyphen{}annex via the \dlhbhref{G2L}{DataLad installer}.
The latter option requires the installation of the \sphinxcode{\sphinxupquote{datalad\sphinxhyphen{}installer}} Python package.
Once Python is available, it can be done with the Python package manager
\sphinxcode{\sphinxupquote{ pip}}. Open a command prompt and run:

\begin{sphinxVerbatim}[commandchars=\\\{\}]
\PYG{p}{\PYGZgt{}} python \PYGZhy{}m pip install datalad\PYGZhy{}installer
\end{sphinxVerbatim}

\sphinxAtStartPar
Afterwards, open another command prompt in administrator mode and run:

\begin{sphinxVerbatim}[commandchars=\\\{\}]
\PYG{p}{\PYGZgt{}} datalad\PYGZhy{}installer git\PYGZhy{}annex \PYGZhy{}m datalad/git\PYGZhy{}annex:release
\end{sphinxVerbatim}

\sphinxAtStartPar
This will download a recent git\sphinxhyphen{}annex, and configure it for your Git installation.
The admin command prompt can be closed afterwards, all other steps do not need it.

\sphinxAtStartPar
For \dlhbhref{B1M}{performance improvements}, regardless of which installation method you chose, we recommend to also set the following git\sphinxhyphen{}annex configuration:

\begin{sphinxVerbatim}[commandchars=\\\{\}]
\PYG{p}{\PYGZgt{}} git config \PYGZhy{}\PYGZhy{}global filter.annex.process \PYG{l+s+s2}{\PYGZdq{}}\PYG{l+s+s2}{git\PYGZhy{}annex filter\PYGZhy{}process}\PYG{l+s+s2}{\PYGZdq{}}
\end{sphinxVerbatim}

\sphinxlineitem{DataLad:}
\sphinxAtStartPar
With Python, Git, and git\sphinxhyphen{}annex installed, DataLad can be installed, and later also
upgraded using \sphinxcode{\sphinxupquote{ pip}} by running:

\begin{sphinxVerbatim}[commandchars=\\\{\}]
\PYG{p}{\PYGZgt{}} python \PYGZhy{}m pip install datalad
\end{sphinxVerbatim}

\sphinxlineitem{7\sphinxhyphen{}Zip (optional, but highly recommended):}
\index{install 7\sphinxhyphen{}zip@\spxentry{install 7\sphinxhyphen{}zip}!on Windows@\spxentry{on Windows}}\index{on Windows@\spxentry{on Windows}!install 7\sphinxhyphen{}zip@\spxentry{install 7\sphinxhyphen{}zip}}\index{installation@\spxentry{installation}!7\sphinxhyphen{}Zip@\spxentry{7\sphinxhyphen{}Zip}}\ignorespaces 
\sphinxAtStartPar
Download it from the \dlhbhref{71}{7\sphinxhyphen{}zip website} (64bit
installer when in doubt), and install it into the default target directory.

\end{description}

\sphinxAtStartPar
There are many other ways to install DataLad on Windows, check for example the
\textit{Windows-wit}~{\windowswiticoninline}\textit{\ref{ww-wsl2}} {\hyperref[\detokenize{intro/installation:ww-wsl2}]{\sphinxcrossref{\DUrole{std,std-ref}{on the Windows Subsystem 2 for Linux}}}} (\autopageref*{\detokenize{intro/installation:ww-wsl2}}).
One attractive alternative approach is {\hyperref[\detokenize{intro/installation:conda}]{\sphinxcrossref{Conda}}} (\autopageref*{\detokenize{intro/installation:conda}}), a completely different approach is to install the {\hyperref[\detokenize{glossary:term-DataLad-Gooey}]{\sphinxtermref{\DUrole{xref,std,std-term}{DataLad Gooey}}}}, which is a standalone installation of DataLad’s graphical application (see \dlhbhref{D1F}{the DataLad Gooey documentation} for installation instructions).

\index{install DataLad@\spxentry{install DataLad}!on WSL2@\spxentry{on WSL2}}\index{on WSL2@\spxentry{on WSL2}!install DataLad@\spxentry{install DataLad}}\ignorespaces \begin{windowswit}[label={ww-wsl2}, before title={\thetcbcounter\ }, check odd page=true]{Install DataLad using the Windows Subsystem 2 for Linux}
\label{\detokenize{intro/installation:ww-wsl2}}

\sphinxAtStartPar
With the Windows Subsystem for Linux, you will be able to use a Unix system
despite being on Windows.  You need to have a recent build of Windows in
order to get WSL2 \textendash{} we do not recommend WSL1.

\sphinxAtStartPar
You can find out how to install the Windows Subsystem for Linux at
\dlhbhref{M1B}{docs.microsoft.com}.
Afterwards, proceed with your installation as described in the installation instructions
for Linux.


\end{windowswit}

\sphinxAtStartPar
Using DataLad on Windows has a few peculiarities. In general, DataLad can feel a bit
sluggish on non\sphinxhyphen{}WSL2 Windows systems. This is due to various file system issues
that also affect the version control system {\hyperref[\detokenize{glossary:term-Git}]{\sphinxtermref{\DUrole{xref,std,std-term}{Git}}}} itself, which DataLad
relies on. The core functionality of DataLad works, and you should be able to
follow most contents covered in this book.  You will notice, however, that some
Unix commands displayed in examples may not work, and that terminal output can
look different from what is displayed in the code examples of the book, and
that some dependencies for additional functionality are not available for
Windows. Dedicated notes,
“\sphinxcode{\sphinxupquote{Windows\sphinxhyphen{}wit}}s”, contain important information, alternative commands, or
warnings, and an overview of useful Windows commands and general information is included in {\hyperref[\detokenize{intro/howto:howto}]{\sphinxcrossref{\DUrole{std,std-ref}{The command line}}}} (\autopageref*{\detokenize{intro/howto:howto}}).

\index{install DataLad@\spxentry{install DataLad}!on Mac@\spxentry{on Mac}}\index{on Mac@\spxentry{on Mac}!install DataLad@\spxentry{install DataLad}}\ignorespaces 

\subsection{Mac (incl. M1)}
\label{\detokenize{intro/installation:mac-incl-m1}}\label{\detokenize{intro/installation:mac}}\label{\detokenize{intro/installation:index-8}}
\sphinxAtStartPar
Modern Macs come with a compatible Python 3 version installed by default. The
\textit{Find-out-more}~{\findoutmoreiconinline}\textit{\ref{fom-py2v3}} {\hyperref[\detokenize{intro/installation:fom-py2v3}]{\sphinxcrossref{\DUrole{std,std-ref}{on Python versions}}}} (\autopageref*{\detokenize{intro/installation:fom-py2v3}}) has instructions on how to
confirm that.

\sphinxAtStartPar
DataLad is available via OS X’s \dlhbhref{B3}{homebrew} package manager.
First, install the homebrew package manager, which requires \dlhbhref{A1A}{Xcode} to be installed from the
Mac App Store.

\sphinxAtStartPar
Next, install datalad and its dependencies:

\begin{sphinxVerbatim}[commandchars=\\\{\}]
\PYG{g+gp}{\PYGZdl{} }brew\PYG{+w}{ }install\PYG{+w}{ }datalad
\end{sphinxVerbatim}

\sphinxAtStartPar
Alternatively, you can exclusively use \sphinxcode{\sphinxupquote{ brew}} for DataLad’s non\sphinxhyphen{}Python
dependencies, and then check the \textit{Find-out-more}~{\findoutmoreiconinline}\textit{\ref{fom-macosx-pip}} {\hyperref[\detokenize{intro/installation:fom-macosx-pip}]{\sphinxcrossref{\DUrole{std,std-ref}{on how to install DataLad via
Python\textquotesingle{}s package manager}}}} (\autopageref*{\detokenize{intro/installation:fom-macosx-pip}}).
\begin{findoutmore}[label={fom-macosx-pip}, before title={\thetcbcounter\ }, float, floatplacement=tb, check odd page=true]{Install DataLad via pip on macOS}
\label{\detokenize{intro/installation:fom-macosx-pip}}

\sphinxAtStartPar
If Git/git\sphinxhyphen{}annex are installed already (via brew), DataLad can also be
installed via Python’s package manager \sphinxcode{\sphinxupquote{pip}}, which should be installed
by default on your system:

\begin{sphinxVerbatim}[commandchars=\\\{\}]
\PYG{g+gp}{\PYGZdl{} }python\PYG{+w}{ }\PYGZhy{}m\PYG{+w}{ }pip\PYG{+w}{ }install\PYG{+w}{ }datalad
\end{sphinxVerbatim}

\sphinxAtStartPar
Some macOS versions may use \sphinxcode{\sphinxupquote{python3}} instead of \sphinxcode{\sphinxupquote{python}} \textendash{} use {\hyperref[\detokenize{glossary:term-tab-completion}]{\sphinxtermref{\DUrole{xref,std,std-term}{tab
completion}}}} to find out which is installed.

\sphinxAtStartPar
Recent macOS versions may warn after installation that scripts were installed
into locations that were not on \sphinxcode{\sphinxupquote{PATH}}:

\begin{sphinxVerbatim}[commandchars=\\\{\}]
The script chardetect is installed in
\PYGZsq{}/Users/MYUSERNAME/Library/Python/3.11/bin\PYGZsq{} which is not on PATH.
Consider adding this directory to PATH or, if you prefer to
suppress this warning, use \PYGZhy{}\PYGZhy{}no\PYGZhy{}warn\PYGZhy{}script\PYGZhy{}location.
\end{sphinxVerbatim}

\sphinxAtStartPar
To fix this, add these paths to the \sphinxcode{\sphinxupquote{\$PATH}} environment variable.
You can do this for your own user account by adding something like the following
to the \sphinxstyleemphasis{profile} file of your shell (exchange the user name accordingly):

\begin{sphinxVerbatim}[commandchars=\\\{\}]
\PYG{g+gp}{\PYGZdl{} }\PYG{n+nb}{export}\PYG{+w}{ }\PYG{n+nv}{PATH}\PYG{o}{=}\PYG{n+nv}{\PYGZdl{}PATH}:/Users/MYUSERNAME/Library/Python/3.11/bin
\end{sphinxVerbatim}

\sphinxAtStartPar
If you use a {\hyperref[\detokenize{glossary:term-bash}]{\sphinxtermref{\DUrole{xref,std,std-term}{bash}}}} shell, this may be \sphinxcode{\sphinxupquote{\textasciitilde{}/.bashrc}} or
\sphinxcode{\sphinxupquote{\textasciitilde{}/.bash\_profile}}, if you are using a {\hyperref[\detokenize{glossary:term-zsh}]{\sphinxtermref{\DUrole{xref,std,std-term}{zsh}}}} shell, it may be
\sphinxcode{\sphinxupquote{\textasciitilde{}/.zshrc}} or \sphinxcode{\sphinxupquote{\textasciitilde{}/.zprofile}}. Find out which shell you are using by
typing \sphinxcode{\sphinxupquote{echo \$SHELL}} into your terminal.

\sphinxAtStartPar
Alternatively, you could configure it \sphinxstyleemphasis{system\sphinxhyphen{}wide}, i.e., for all users of
your computer by adding the path
\sphinxcode{\sphinxupquote{/Users/MYUSERNAME/Library/Python/3.11/bin}} to the file \sphinxcode{\sphinxupquote{/etc/paths}},
e.g., with the editor {\hyperref[\detokenize{glossary:term-nano}]{\sphinxtermref{\DUrole{xref,std,std-term}{nano}}}} (requires using \sphinxcode{\sphinxupquote{sudo}} and authenticating
with your password):

\begin{sphinxVerbatim}[commandchars=\\\{\}]
\PYG{g+gp}{\PYGZdl{} }sudo\PYG{+w}{ }nano\PYG{+w}{ }/etc/paths
\end{sphinxVerbatim}

\sphinxAtStartPar
The contents of this file could look like this afterwards (the last line was
added):

\begin{sphinxVerbatim}[commandchars=\\\{\}]
\PYG{g+go}{/usr/local/bin}
\PYG{g+go}{/usr/bin}
\PYG{g+go}{/bin}
\PYG{g+go}{/usr/sbin}
\PYG{g+go}{/sbin}
\PYG{g+go}{/Users/MYUSERNAME/Library/Python/3.11/bin}
\end{sphinxVerbatim}


\end{findoutmore}

\index{install DataLad@\spxentry{install DataLad}!on Debian/Ubuntu@\spxentry{on Debian/Ubuntu}}\index{on Debian/Ubuntu@\spxentry{on Debian/Ubuntu}!install DataLad@\spxentry{install DataLad}}\ignorespaces 

\subsection{Linux: (Neuro)Debian, Ubuntu, and similar systems}
\label{\detokenize{intro/installation:linux-neuro-debian-ubuntu-and-similar-systems}}\label{\detokenize{intro/installation:index-9}}
\sphinxAtStartPar
DataLad is part of the Debian and Ubuntu operating systems. However, the
particular DataLad version included in a release may be a bit older (check the
versions for \sphinxhref{https://packages.debian.org/datalad}{Debian} and \dlhbhref{U1A}{Ubuntu} to see which ones are available).

\sphinxAtStartPar
For some recent releases of Debian\sphinxhyphen{}based operating systems, \dlhbhref{D7}{NeuroDebian} provides more recent DataLad versions (check the
\dlhbhref{D7B}{availability table}).  In order to
install from NeuroDebian, follow \dlhbhref{D7A}{its installation documentation}, which only requires
copy\sphinxhyphen{}pasting three lines into a terminal.  Also, should you be confused by the
name: enabling this repository will not do any harm if your field is not
neuroscience.

\sphinxAtStartPar
Whichever repository you end up using, the following command installs DataLad
and all of its software dependencies (including {\hyperref[\detokenize{glossary:term-git-annex}]{\sphinxtermref{\DUrole{xref,std,std-term}{git\sphinxhyphen{}annex}}}} and \dlhbhref{S3}{p7zip}):

\begin{sphinxVerbatim}[commandchars=\\\{\}]
\PYG{g+gp}{\PYGZdl{} }sudo\PYG{+w}{ }apt\PYGZhy{}get\PYG{+w}{ }install\PYG{+w}{ }datalad
\end{sphinxVerbatim}

\sphinxAtStartPar
The command above will also upgrade existing installations to the most recent
available version.

\index{install DataLad@\spxentry{install DataLad}!on Redhat/Fedora@\spxentry{on Redhat/Fedora}}\index{on Redhat/Fedora@\spxentry{on Redhat/Fedora}!install DataLad@\spxentry{install DataLad}}\ignorespaces 

\subsection{Linux: CentOS, Redhat, Fedora, or similar systems}
\label{\detokenize{intro/installation:linux-centos-redhat-fedora-or-similar-systems}}\label{\detokenize{intro/installation:index-10}}
\sphinxAtStartPar
For CentOS, Redhat, Fedora, or similar distributions, there is an \dlhbhref{B1K}{RPM package for git\sphinxhyphen{}annex}.  A
suitable version of Python and {\hyperref[\detokenize{glossary:term-Git}]{\sphinxtermref{\DUrole{xref,std,std-term}{Git}}}} should come with the operating
system, although some servers may run fairly old releases.

\sphinxAtStartPar
DataLad itself can be installed via \sphinxcode{\sphinxupquote{pip}}:

\begin{sphinxVerbatim}[commandchars=\\\{\}]
\PYG{g+gp}{\PYGZdl{} }python\PYG{+w}{ }\PYGZhy{}m\PYG{+w}{ }pip\PYG{+w}{ }install\PYG{+w}{ }datalad
\end{sphinxVerbatim}

\sphinxAtStartPar
Alternatively, DataLad can be installed together with {\hyperref[\detokenize{glossary:term-Git}]{\sphinxtermref{\DUrole{xref,std,std-term}{Git}}}} and
{\hyperref[\detokenize{glossary:term-git-annex}]{\sphinxtermref{\DUrole{xref,std,std-term}{git\sphinxhyphen{}annex}}}} via {\hyperref[\detokenize{intro/installation:conda}]{\sphinxcrossref{Conda}}} (\autopageref*{\detokenize{intro/installation:conda}}).

\index{install DataLad@\spxentry{install DataLad}!on HPC@\spxentry{on HPC}}\index{on HPC@\spxentry{on HPC}!install DataLad@\spxentry{install DataLad}}\ignorespaces 

\subsection{Linux\sphinxhyphen{}machines with no root access (e.g. HPC systems)}
\label{\detokenize{intro/installation:linux-machines-with-no-root-access-e-g-hpc-systems}}\label{\detokenize{intro/installation:norootinstall}}\label{\detokenize{intro/installation:index-11}}
\sphinxAtStartPar
The most convenient user\sphinxhyphen{}based installation can be achieved via {\hyperref[\detokenize{intro/installation:conda}]{\sphinxcrossref{Conda}}} (\autopageref*{\detokenize{intro/installation:conda}}).

\index{install DataLad@\spxentry{install DataLad}!with Conda@\spxentry{with Conda}}\index{with Conda@\spxentry{with Conda}!install DataLad@\spxentry{install DataLad}}\ignorespaces 

\subsection{Conda}
\label{\detokenize{intro/installation:conda}}\label{\detokenize{intro/installation:index-12}}\label{\detokenize{intro/installation:id1}}
\sphinxAtStartPar
Conda is a software distribution available for all major operating systems, and
its \dlhbhref{C3A}{Miniconda} installer
offers a convenient way to bootstrap a DataLad installation. Importantly, it
does not require admin/root access to a system.

\sphinxAtStartPar
\dlhbhref{C3B}{Detailed, platform\sphinxhyphen{}specific installation instructions} are available
in the Conda documentation. In short: download and run the installer, or, from
the command line, run

\begin{sphinxVerbatim}[commandchars=\\\{\}]
\PYG{g+gp}{\PYGZdl{} }wget\PYG{+w}{ }https://repo.anaconda.com/miniconda/Miniconda3\PYGZhy{}latest\PYGZhy{}\PYGZlt{}YOUR\PYGZhy{}OS\PYGZgt{}\PYGZhy{}x86\PYGZus{}64.sh
\PYG{g+gp}{\PYGZdl{} }bash\PYG{+w}{ }Miniconda3\PYGZhy{}latest\PYGZhy{}\PYGZlt{}YOUR\PYGZhy{}OS\PYGZgt{}\PYGZhy{}x86\PYGZus{}64.sh
\end{sphinxVerbatim}

\sphinxAtStartPar
In the above call, replace \sphinxcode{\sphinxupquote{\textless{}YOUR\sphinxhyphen{}OS\textgreater{}}} with an identifier for your operating
system, such as “Linux” or “MacOSX”.  During the installation, you will need to
accept a license agreement (press Enter to scroll down, and type “yes” and
Enter to accept), confirm the installation into the default directory, and you
should respond “yes” to the prompt \sphinxcode{\sphinxupquote{“Do you wish the installer to initialize
Miniconda3 by running conda init? {[}yes|no{]}”}}.  Afterwards, you can remove the
installation script by running \sphinxcode{\sphinxupquote{rm ./Miniconda3\sphinxhyphen{}latest\sphinxhyphen{}*\sphinxhyphen{}x86\_64.sh}}.

\sphinxAtStartPar
The installer automatically configures the shell to make conda\sphinxhyphen{}installed tools
accessible, so no further configuration is necessary.  Once Conda is installed,
the DataLad package can be installed from the \sphinxcode{\sphinxupquote{conda\sphinxhyphen{}forge}} channel:

\begin{sphinxVerbatim}[commandchars=\\\{\}]
\PYG{g+gp}{\PYGZdl{} }conda\PYG{+w}{ }install\PYG{+w}{ }\PYGZhy{}c\PYG{+w}{ }conda\PYGZhy{}forge\PYG{+w}{ }datalad
\end{sphinxVerbatim}

\sphinxAtStartPar
In general, all of DataLad’s software dependencies are automatically installed, too.
This makes a conda\sphinxhyphen{}based deployment very convenient. A from\sphinxhyphen{}scratch DataLad installation
on a HPC system, as a normal user, is done in three lines:

\begin{sphinxVerbatim}[commandchars=\\\{\}]
\PYG{g+gp}{\PYGZdl{} }wget\PYG{+w}{ }https://repo.anaconda.com/miniconda/Miniconda3\PYGZhy{}latest\PYGZhy{}Linux\PYGZhy{}x86\PYGZus{}64.sh
\PYG{g+gp}{\PYGZdl{} }bash\PYG{+w}{ }Miniconda3\PYGZhy{}latest\PYGZhy{}Linux\PYGZhy{}x86\PYGZus{}64.sh
\PYG{g+gp}{\PYGZdl{} }\PYG{c+c1}{\PYGZsh{} acknowledge license, keep everything at default}
\PYG{g+gp}{\PYGZdl{} }conda\PYG{+w}{ }install\PYG{+w}{ }\PYGZhy{}c\PYG{+w}{ }conda\PYGZhy{}forge\PYG{+w}{ }datalad
\end{sphinxVerbatim}

\sphinxAtStartPar
In case a dependency is not available from Conda (e.g., there is no git\sphinxhyphen{}annex
package for Windows in Conda), please refer to the platform\sphinxhyphen{}specific
instructions above.

\sphinxAtStartPar
To update an existing installation with conda, use:

\begin{sphinxVerbatim}[commandchars=\\\{\}]
\PYG{g+gp}{\PYGZdl{} }conda\PYG{+w}{ }update\PYG{+w}{ }\PYGZhy{}c\PYG{+w}{ }conda\PYGZhy{}forge\PYG{+w}{ }datalad
\end{sphinxVerbatim}

\sphinxAtStartPar
The \dlhbhref{G2L}{DataLad installer} also supports setting up a Conda environment, in case
a suitable Python version is already available.

\index{install DataLad@\spxentry{install DataLad}!with pip@\spxentry{with pip}}\index{with pip@\spxentry{with pip}!install DataLad@\spxentry{install DataLad}}\ignorespaces 

\subsection{Using Python’s package manager \sphinxstyleliteralintitle{\sphinxupquote{pip}}}
\label{\detokenize{intro/installation:using-python-s-package-manager-pip}}\label{\detokenize{intro/installation:pipinstall}}\label{\detokenize{intro/installation:index-13}}
\sphinxAtStartPar
As mentioned above, DataLad can be installed via Python’s package manager \dlhbhref{P5}{pip}.  \sphinxcode{\sphinxupquote{pip}} comes with any Python distribution
from \dlhbhref{P8}{python.org}, and is available as a system\sphinxhyphen{}package
in nearly all GNU/Linux distributions.

\sphinxAtStartPar
If you have Python and \sphinxcode{\sphinxupquote{pip}} set up, to automatically install DataLad and
most of its software dependencies, type

\begin{sphinxVerbatim}[commandchars=\\\{\}]
\PYG{g+gp}{\PYGZdl{} }python\PYG{+w}{ }\PYGZhy{}m\PYG{+w}{ }pip\PYG{+w}{ }install\PYG{+w}{ }datalad
\end{sphinxVerbatim}

\sphinxAtStartPar
If this results in a \sphinxcode{\sphinxupquote{permission denied}} error, you can install DataLad into
a user’s home directory:

\begin{sphinxVerbatim}[commandchars=\\\{\}]
\PYG{g+gp}{\PYGZdl{} }python\PYG{+w}{ }\PYGZhy{}m\PYG{+w}{ }pip\PYG{+w}{ }install\PYG{+w}{ }\PYGZhy{}\PYGZhy{}user\PYG{+w}{ }datalad
\end{sphinxVerbatim}

\sphinxAtStartPar
On some systems, you may need to call \sphinxcode{\sphinxupquote{python3}} instead of \sphinxcode{\sphinxupquote{python}}:

\begin{sphinxVerbatim}[commandchars=\\\{\}]
\PYG{g+gp}{\PYGZdl{} }python3\PYG{+w}{ }\PYGZhy{}m\PYG{+w}{ }pip\PYG{+w}{ }install\PYG{+w}{ }datalad
\PYG{g+gp}{\PYGZdl{} }\PYG{c+c1}{\PYGZsh{} or, in case of a \PYGZdq{}permission denied error\PYGZdq{}:}
\PYG{g+gp}{\PYGZdl{} }python3\PYG{+w}{ }\PYGZhy{}m\PYG{+w}{ }pip\PYG{+w}{ }install\PYG{+w}{ }\PYGZhy{}\PYGZhy{}user\PYG{+w}{ }datalad
\end{sphinxVerbatim}

\sphinxAtStartPar
An existing installation can be upgraded with \sphinxcode{\sphinxupquote{python \sphinxhyphen{}m pip install \sphinxhyphen{}U datalad}}.

\sphinxAtStartPar
\sphinxcode{\sphinxupquote{pip}} is not able to install non\sphinxhyphen{}Python software, such as 7\sphinxhyphen{}zip or
{\hyperref[\detokenize{glossary:term-git-annex}]{\sphinxtermref{\DUrole{xref,std,std-term}{git\sphinxhyphen{}annex}}}}.  But you can install the \dlhbhref{G2L}{DataLad installer} via a \sphinxcode{\sphinxupquote{python \sphinxhyphen{}m pip install datalad\sphinxhyphen{}installer}}. This is a command\sphinxhyphen{}line tool that aids installation
of DataLad and its key software dependencies on a range of platforms.

\index{configure user identity@\spxentry{configure user identity}!with Git@\spxentry{with Git}|spxpagem}\ignorespaces 

\section{Initial configuration}
\label{\detokenize{intro/installation:initial-configuration}}\label{\detokenize{intro/installation:installconfig}}\label{\detokenize{intro/installation:index-14}}
\sphinxAtStartPar
Initial configurations only concern the setup of a {\hyperref[\detokenize{glossary:term-Git}]{\sphinxtermref{\DUrole{xref,std,std-term}{Git}}}} identity. If you
are a Git\sphinxhyphen{}user, you should hence be good to go.

\begin{figure}[tbp]
\centering

\noindent\sphinxincludegraphics[height=0.2\textheight]{{gitidentity_bw}.pdf}
\end{figure}

\sphinxAtStartPar
If you have not used the version control system Git before, you will need to
tell Git some information about you. This needs to be done only once.
In the following example, exchange \sphinxcode{\sphinxupquote{Bob McBobFace}} with your own name, and
\sphinxcode{\sphinxupquote{bob@example.com}} with your own email address.

\begin{sphinxVerbatim}[commandchars=\\\{\}]
\PYG{g+gp}{\PYGZdl{} }\PYG{c+c1}{\PYGZsh{} enter your home directory using the \PYGZti{} shortcut}
\PYG{g+gp}{\PYGZdl{} }\PYG{n+nb}{cd}\PYG{+w}{ }\PYGZti{}
\PYG{g+gp}{\PYGZdl{} }git\PYG{+w}{ }config\PYG{+w}{ }\PYGZhy{}\PYGZhy{}global\PYG{+w}{ }\PYGZhy{}\PYGZhy{}add\PYG{+w}{ }user.name\PYG{+w}{ }\PYG{l+s+s2}{\PYGZdq{}Bob McBobFace\PYGZdq{}}
\PYG{g+gp}{\PYGZdl{} }git\PYG{+w}{ }config\PYG{+w}{ }\PYGZhy{}\PYGZhy{}global\PYG{+w}{ }\PYGZhy{}\PYGZhy{}add\PYG{+w}{ }user.email\PYG{+w}{ }bob@example.com
\end{sphinxVerbatim}

\sphinxAtStartPar
This information is used to track changes in the DataLad projects you will
be working on. Based on this information, changes you make are associated
with your name and email address, and you should use a real email address
and name \textendash{} it does not establish a lot of trust nor is it helpful after a few
years if your history, especially in a collaborative project, shows
that changes were made by \sphinxcode{\sphinxupquote{Anonymous}} with the email
\sphinxcode{\sphinxupquote{youdontgetmy@email.fu}}.
And do not worry, you won’t get any emails from Git or DataLad.

\sphinxstepscope

\index{terminal@\spxentry{terminal}|spxpagem}\index{shell@\spxentry{shell}|spxpagem}\index{command Line@\spxentry{command Line}|spxpagem}\index{what is@\spxentry{what is}!command line@\spxentry{command line}}\index{command line@\spxentry{command line}!what is@\spxentry{what is}}\index{what is@\spxentry{what is}!terminal@\spxentry{terminal}}\index{terminal@\spxentry{terminal}!what is@\spxentry{what is}}\index{what is@\spxentry{what is}!shell@\spxentry{shell}}\index{shell@\spxentry{shell}!what is@\spxentry{what is}}\ignorespaces 

