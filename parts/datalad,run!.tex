\chapter{DataLad, run!}
\label{\detokenize{basics/basics-run:datalad-run}}\label{\detokenize{basics/basics-run:chapter-run}}\label{\detokenize{basics/basics-run:id1}}\label{\detokenize{basics/basics-run::doc}}
\noindent{\hspace*{\fill}\sphinxincludegraphics[width=0.500\linewidth]{{keeptrack}.pdf}\hspace*{\fill}}

\sphinxstepscope


\section{Keeping track}
\label{\detokenize{basics/101-108-run:keeping-track}}\label{\detokenize{basics/101-108-run:run}}\label{\detokenize{basics/101-108-run::doc}}
\sphinxAtStartPar
In previous examples, with the exception of \sphinxcode{\sphinxupquote{datalad download\sphinxhyphen{}url}}, all
changes that happened to the dataset or the files it contains were
saved to the dataset’s history by hand. We added larger and smaller
files and saved them, and we also modified smaller file contents and
saved these modifications.

\sphinxAtStartPar
Often, however, files get changed by shell commands
or by scripts.
Consider a data scientist.
She has data files with numeric data,
and code scripts in Python, R, Matlab or any other programming language
that will use the data to compute results or figures. Such output is
stored in new files, or modifies existing files.

\sphinxAtStartPar
But only a few weeks after these scripts were executed she finds it hard
to remember which script was modified for which reason or created which
output. How did this result came to be? Which script would she need
to run again on which data to produce this particular figure?

\sphinxAtStartPar
In this section we will experience how DataLad can help
to record the changes in a dataset after executing a script
from the shell. Just as \sphinxcode{\sphinxupquote{datalad download\sphinxhyphen{}url}} was able to associate
a file with its origin and store this information, we want to be
able to associate a particular file with the commands, scripts, and inputs
it was produced from, and thus capture and store full {\hyperref[\detokenize{glossary:term-provenance}]{\sphinxtermref{\DUrole{xref,std,std-term}{provenance}}}}.

\sphinxAtStartPar
Let’s say, for example, that you enjoyed the longnow podcasts a lot,
and you start a podcast\sphinxhyphen{}night with friends to wind down from all of
the exciting DataLad lectures. They propose to make a
list of speakers and titles to cross out what they’ve already listened
to, and ask you to prepare such a list.

\sphinxAtStartPar
“Mhh… probably there is a DataLad way to do this… wasn’t there also
a note about metadata extraction at some point?” But as we are not that
far into the lectures, you decide to write a short shell script
to generate a text file that lists speaker and title
name instead.

\sphinxAtStartPar
To do this, we are following a best practice that will reappear in the
later section on {\hyperref[\detokenize{basics/101-127-yoda:yoda}]{\sphinxcrossref{\DUrole{std,std-ref}{YODA principles}}}} (\autopageref*{\detokenize{basics/101-127-yoda:yoda}}): Collecting all
additional scripts that work with content of a subdataset \sphinxstyleemphasis{outside}
of this subdataset, in a dedicated \sphinxcode{\sphinxupquote{code/}} directory,
and collating the output of the execution of these scripts
\sphinxstyleemphasis{outside} of the subdataset as well \textendash{} and
therefore not modifying the subdataset.

\sphinxAtStartPar
The motivation behind this will become clear in later sections,
but for now we’ll start with best\sphinxhyphen{}practice building.
Therefore, create a subdirectory \sphinxcode{\sphinxupquote{code/}} in the \sphinxcode{\sphinxupquote{DataLad\sphinxhyphen{}101}}
superdataset:

\begin{sphinxVerbatim}[commandchars=\\\{\}]
\PYG{g+gp}{\PYGZdl{} }mkdir\PYG{+w}{ }code
\PYG{g+gp}{\PYGZdl{} }tree\PYG{+w}{ }\PYGZhy{}d
\PYG{g+go}{.}
\PYG{g+go}{├── books}
\PYG{g+go}{├── code}
\PYG{g+go}{└── recordings}
\PYG{g+go}{    └── longnow}
\PYG{g+go}{        ├── Long\PYGZus{}Now\PYGZus{}\PYGZus{}Conversations\PYGZus{}at\PYGZus{}The\PYGZus{}Interval}
\PYG{g+go}{        └── Long\PYGZus{}Now\PYGZus{}\PYGZus{}Seminars\PYGZus{}About\PYGZus{}Long\PYGZus{}term\PYGZus{}Thinking}

\PYG{g+go}{6 directories}
\end{sphinxVerbatim}

\sphinxAtStartPar
Inside of \sphinxcode{\sphinxupquote{DataLad\sphinxhyphen{}101/code}}, create a simple shell script \sphinxcode{\sphinxupquote{list\_titles.sh}}.
This script will carry out a simple task:
It will loop through the file names of the \sphinxcode{\sphinxupquote{.mp3}} files and
write out speaker names and talk titles in a very basic fashion.
The \sphinxcode{\sphinxupquote{cat}} command will write the script content into \sphinxcode{\sphinxupquote{code/list\_titles.sh}}.
Make sure to see the \textit{Windows-wit}~{\windowswiticoninline}\textit{\ref{ww-heredoc-alt}} {\hyperref[\detokenize{basics/101-108-run:ww-heredoc-alt}]{\sphinxcrossref{\DUrole{std,std-ref}{on an alternative}}}} (\autopageref*{\detokenize{basics/101-108-run:ww-heredoc-alt}}), and check
the \textit{Windows-wit}~{\windowswiticoninline}\textit{\ref{ww-filename-extensions}} {\hyperref[\detokenize{basics/101-108-run:ww-filename-extensions}]{\sphinxcrossref{\DUrole{std,std-ref}{on automatic file name extensions}}}} (\autopageref*{\detokenize{basics/101-108-run:ww-filename-extensions}})
to avoid confusion when creating files manually.
\begin{windowswit}[label={ww-heredoc-alt}, before title={\thetcbcounter\ }, float, floatplacement=tb, check odd page=true]{Here’s a script for Windows users}
\label{\detokenize{basics/101-108-run:ww-heredoc-alt}}

\sphinxAtStartPar
Please use an editor of your choice to create a file \sphinxcode{\sphinxupquote{list\_titles.sh}} inside of the \sphinxcode{\sphinxupquote{code}} directory.
These should be the contents:

\begin{sphinxVerbatim}[commandchars=\\\{\}]
\PYG{k}{for}\PYG{+w}{ }i\PYG{+w}{ }\PYG{k}{in}\PYG{+w}{ }recordings/longnow/Long\PYGZus{}Now\PYGZus{}\PYGZus{}Seminars*/*.mp3\PYG{p}{;}\PYG{+w}{ }\PYG{k}{do}
\PYG{+w}{   }\PYG{c+c1}{\PYGZsh{} get the filename}
\PYG{+w}{   }\PYG{n+nv}{base}\PYG{o}{=}\PYG{k}{\PYGZdl{}(}basename\PYG{+w}{ }\PYG{l+s+s2}{\PYGZdq{}}\PYG{n+nv}{\PYGZdl{}i}\PYG{l+s+s2}{\PYGZdq{}}\PYG{k}{)}\PYG{p}{;}
\PYG{+w}{   }\PYG{c+c1}{\PYGZsh{} strip the extension}
\PYG{+w}{   }\PYG{n+nv}{base}\PYG{o}{=}\PYG{l+s+si}{\PYGZdl{}\PYGZob{}}\PYG{n+nv}{base}\PYG{p}{\PYGZpc{}.mp3}\PYG{l+s+si}{\PYGZcb{}}\PYG{p}{;}
\PYG{+w}{   }\PYG{c+c1}{\PYGZsh{} date as yyyy\PYGZhy{}mm\PYGZhy{}dd}
\PYG{+w}{   }\PYG{n+nb}{printf}\PYG{+w}{ }\PYG{l+s+s2}{\PYGZdq{}}\PYG{l+s+si}{\PYGZdl{}\PYGZob{}}\PYG{n+nv}{base}\PYG{p}{\PYGZpc{}\PYGZpc{}\PYGZus{}\PYGZus{}*}\PYG{l+s+si}{\PYGZcb{}}\PYG{l+s+s2}{\PYGZbs{}t}\PYG{l+s+s2}{\PYGZdq{}}\PYG{+w}{ }\PYG{p}{|}\PYG{+w}{ }tr\PYG{+w}{ }\PYG{l+s+s1}{\PYGZsq{}\PYGZus{}\PYGZsq{}}\PYG{+w}{ }\PYG{l+s+s1}{\PYGZsq{}\PYGZhy{}\PYGZsq{}}\PYG{p}{;}
\PYG{+w}{   }\PYG{c+c1}{\PYGZsh{} name and title without underscores}
\PYG{+w}{   }\PYG{n+nb}{printf}\PYG{+w}{ }\PYG{l+s+s2}{\PYGZdq{}}\PYG{l+s+si}{\PYGZdl{}\PYGZob{}}\PYG{n+nv}{base}\PYG{p}{\PYGZsh{}*\PYGZus{}\PYGZus{}}\PYG{l+s+si}{\PYGZcb{}}\PYG{l+s+s2}{\PYGZbs{}n}\PYG{l+s+s2}{\PYGZdq{}}\PYG{+w}{ }\PYG{p}{|}\PYG{+w}{ }tr\PYG{+w}{ }\PYG{l+s+s1}{\PYGZsq{}\PYGZus{}\PYGZsq{}}\PYG{+w}{ }\PYG{l+s+s1}{\PYGZsq{} \PYGZsq{}}\PYG{p}{;}
\PYG{k}{done}
\end{sphinxVerbatim}

\sphinxAtStartPar
Note that this is not identical to the script in the text \textendash{} it lacks a few \sphinxcode{\sphinxupquote{\textbackslash{}}} characters, which is a meaningful difference.


\end{windowswit}

\index{hidden file name extensions@\spxentry{hidden file name extensions}!on Windows@\spxentry{on Windows}}\index{on Windows@\spxentry{on Windows}!hidden file name extensions@\spxentry{hidden file name extensions}}\ignorespaces \begin{windowswit}[label={ww-filename-extensions}, before title={\thetcbcounter\ }, float, floatplacement=tb, check odd page=true]{Be mindful of hidden extensions when creating files!}
\label{\detokenize{basics/101-108-run:ww-filename-extensions}}

\sphinxAtStartPar
By default, Windows does not show common file extensions when you view directory contents with a file explorer.
Instead, it only displays the base of the file name and indicates the file type with the display icon.
You can see if this is the case for you, too, by opening the \sphinxcode{\sphinxupquote{books\textbackslash{}}} directory in a file explorer, and checking if the file extension (\sphinxcode{\sphinxupquote{.pdf}}) is a part of the file name displayed underneath its PDF icon.

\sphinxAtStartPar
Hidden file extensions can be a confusing source of errors, because some Windows editors (for example, Notepad) automatically add a \sphinxcode{\sphinxupquote{.txt}} extension to your files \textendash{} when you save the script above under the name \sphinxcode{\sphinxupquote{list\_titles.sh}}, your editor may add an extension (\sphinxcode{\sphinxupquote{list\_titles.sh.txt}}), and the file explorer displays your file as \sphinxcode{\sphinxupquote{list\_titles.sh}} (hiding the \sphinxcode{\sphinxupquote{.txt}} extension).

\sphinxAtStartPar
To prevent confusion, configure the file explorer to always show you the file extension.
For this, open the Explorer, click on the “View” tab, and tick the box “File name extensions”.

\sphinxAtStartPar
Beyond this, double check the correct naming of your file, ideally in the terminal.


\end{windowswit}

\begin{sphinxVerbatim}[commandchars=\\\{\}]
\PYG{g+gp}{\PYGZdl{} }cat\PYG{+w}{ }\PYGZlt{}\PYGZlt{}\PYG{+w}{ }EOT\PYG{+w}{ }\PYGZgt{}\PYG{+w}{ }code/list\PYGZus{}titles.sh
\PYG{g+go}{for i in recordings/longnow/Long\PYGZus{}Now\PYGZus{}\PYGZus{}Seminars*/*.mp3; do}
\PYG{g+gp}{   \PYGZsh{} }get\PYG{+w}{ }the\PYG{+w}{ }filename
\PYG{g+go}{   base=\PYGZbs{}\PYGZdl{}(basename \PYGZdq{}\PYGZbs{}\PYGZdl{}i\PYGZdq{});}
\PYG{g+gp}{   \PYGZsh{} }strip\PYG{+w}{ }the\PYG{+w}{ }extension
\PYG{g+go}{   base=\PYGZbs{}\PYGZdl{}\PYGZob{}base\PYGZpc{}.mp3\PYGZcb{};}
\PYG{g+gp}{   \PYGZsh{} }date\PYG{+w}{ }as\PYG{+w}{ }yyyy\PYGZhy{}mm\PYGZhy{}dd
\PYG{g+go}{   printf \PYGZdq{}\PYGZbs{}\PYGZdl{}\PYGZob{}base\PYGZpc{}\PYGZpc{}\PYGZus{}\PYGZus{}*\PYGZcb{}\PYGZbs{}t\PYGZdq{} | tr \PYGZsq{}\PYGZus{}\PYGZsq{} \PYGZsq{}\PYGZhy{}\PYGZsq{};}
\PYG{g+gp}{   \PYGZsh{} }name\PYG{+w}{ }and\PYG{+w}{ }title\PYG{+w}{ }without\PYG{+w}{ }underscores
\PYG{g+go}{   printf \PYGZdq{}\PYGZbs{}\PYGZdl{}\PYGZob{}base\PYGZsh{}*\PYGZus{}\PYGZus{}\PYGZcb{}\PYGZbs{}n\PYGZdq{} | tr \PYGZsq{}\PYGZus{}\PYGZsq{} \PYGZsq{} \PYGZsq{};}
\PYG{g+go}{done}
\PYG{g+go}{EOT}
\end{sphinxVerbatim}

\sphinxAtStartPar
Save this script to the dataset.

\begin{sphinxVerbatim}[commandchars=\\\{\}]
\PYG{g+gp}{\PYGZdl{} }datalad\PYG{+w}{ }status
\PYG{g+go}{untracked: code (directory)}
\end{sphinxVerbatim}

\begin{sphinxVerbatim}[commandchars=\\\{\}]
\PYG{g+gp}{\PYGZdl{} }datalad\PYG{+w}{ }save\PYG{+w}{ }\PYGZhy{}m\PYG{+w}{ }\PYG{l+s+s2}{\PYGZdq{}Add short script to write a list of podcast speakers and titles\PYGZdq{}}
\PYG{g+go}{add(ok): code/list\PYGZus{}titles.sh (file)}
\PYG{g+go}{save(ok): . (dataset)}
\end{sphinxVerbatim}

\sphinxAtStartPar
Once we run this script, it will simply print dates, names and titles to
your terminal. We can save its outputs to a new file
\sphinxcode{\sphinxupquote{recordings/podcasts.tsv}} in the superdataset by redirecting these
outputs with \sphinxcode{\sphinxupquote{bash code/list\_titles.sh \textgreater{} recordings/podcasts.tsv}}.

\sphinxAtStartPar
Obviously, we could create this file, and subsequently save it to the superdataset.
However, just as in the example about the data scientist,
in a bit of time, we will forget how this file came into existence, or
that the script \sphinxcode{\sphinxupquote{code/list\_titles.sh}} is associated with this file, and
can be used to update it later on.

\index{run@\spxentry{run}!DataLad command@\spxentry{DataLad command}}\index{DataLad command@\spxentry{DataLad command}!run@\spxentry{run}}\index{run command with provenance capture@\spxentry{run command with provenance capture}!with DataLad@\spxentry{with DataLad}}\index{with DataLad@\spxentry{with DataLad}!run command with provenance capture@\spxentry{run command with provenance capture}}\index{run command with provenance capture@\spxentry{run command with provenance capture}!with DataLad run@\spxentry{with DataLad run}}\index{with DataLad run@\spxentry{with DataLad run}!run command with provenance capture@\spxentry{run command with provenance capture}}\ignorespaces 
\sphinxAtStartPar
The \sphinxcode{\sphinxupquote{datalad run}} command
can help with this. Put simply, it records a command’s impact on a dataset. Put
more technically, it will record a shell command, and \sphinxcode{\sphinxupquote{datalad save}} all changes
this command triggered in the dataset \textendash{} be that new files or changes to existing
files.

\sphinxAtStartPar
Let’s try the simplest way to use this command: \sphinxcode{\sphinxupquote{datalad run}},
followed by a commit message (\sphinxcode{\sphinxupquote{\sphinxhyphen{}m "a concise summary"}}), and the
command that executes the script from the shell: \sphinxcode{\sphinxupquote{bash code/list\_titles.sh \textgreater{} recordings/podcasts.tsv}}.
It is helpful to enclose the command in quotation marks.

\sphinxAtStartPar
Note that we execute the command from the root of the superdataset.
It is recommended to use \sphinxcode{\sphinxupquote{datalad run}} in the root of the dataset
you want to record the changes in, so make sure to run this
command from the root of \sphinxcode{\sphinxupquote{DataLad\sphinxhyphen{}101}}.

\begin{sphinxVerbatim}[commandchars=\\\{\}]
\PYG{g+gp}{\PYGZdl{} }datalad\PYG{+w}{ }run\PYG{+w}{ }\PYGZhy{}m\PYG{+w}{ }\PYG{l+s+s2}{\PYGZdq{}create a list of podcast titles\PYGZdq{}}\PYG{+w}{ }\PYG{l+s+se}{\PYGZbs{}}
\PYG{+w}{  }\PYG{l+s+s2}{\PYGZdq{}bash code/list\PYGZus{}titles.sh \PYGZgt{} recordings/podcasts.tsv\PYGZdq{}}
\PYG{g+go}{[INFO] == Command start (output follows) =====}
\PYG{g+go}{[INFO] == Command exit (modification check follows) =====}
\PYG{g+go}{run(ok): /home/me/dl\PYGZhy{}101/DataLad\PYGZhy{}101 (dataset) [bash code/list\PYGZus{}titles.sh \PYGZgt{} recordings/po...]}
\PYG{g+go}{add(ok): recordings/podcasts.tsv (file)}
\PYG{g+go}{save(ok): . (dataset)}
\end{sphinxVerbatim}

\sphinxAtStartPar
Let’s take a look into the history:

\fvset{hllines={, 6, 11, 25,}}%
\begin{sphinxVerbatim}[commandchars=\\\{\}]
\PYG{g+gp}{\PYGZdl{} }git\PYG{+w}{ }log\PYG{+w}{ }\PYGZhy{}p\PYG{+w}{ }\PYGZhy{}n\PYG{+w}{ }\PYG{l+m}{1}\PYG{+w}{   }\PYG{c+c1}{\PYGZsh{} On Windows, you may just want to type \PYGZdq{}git log\PYGZdq{}.}
\PYG{g+go}{commit e37c9fc9✂SHA1}
\PYG{g+go}{Author: Elena Piscopia \PYGZlt{}elena@example.net\PYGZgt{}}
\PYG{g+go}{Date:   Tue Jun 18 16:13:00 2019 +0000}

\PYG{g+go}{    [DATALAD RUNCMD] create a list of podcast titles}

\PYG{g+go}{    === Do not change lines below ===}
\PYG{g+go}{    \PYGZob{}}
\PYG{g+go}{     \PYGZdq{}chain\PYGZdq{}: [],}
\PYG{g+go}{     \PYGZdq{}cmd\PYGZdq{}: \PYGZdq{}bash code/list\PYGZus{}titles.sh \PYGZgt{} recordings/podcasts.tsv\PYGZdq{},}
\PYG{g+go}{     \PYGZdq{}dsid\PYGZdq{}: \PYGZdq{}e3e70682\PYGZhy{}c209\PYGZhy{}4cac\PYGZhy{}629f\PYGZhy{}6fbed82c07cd\PYGZdq{},}
\PYG{g+go}{     \PYGZdq{}exit\PYGZdq{}: 0,}
\PYG{g+go}{     \PYGZdq{}extra\PYGZus{}inputs\PYGZdq{}: [],}
\PYG{g+go}{     \PYGZdq{}inputs\PYGZdq{}: [],}
\PYG{g+go}{     \PYGZdq{}outputs\PYGZdq{}: [],}
\PYG{g+go}{     \PYGZdq{}pwd\PYGZdq{}: \PYGZdq{}.\PYGZdq{}}
\PYG{g+go}{    \PYGZcb{}}
\PYG{g+go}{    \PYGZca{}\PYGZca{}\PYGZca{} Do not change lines above \PYGZca{}\PYGZca{}\PYGZca{}}

\PYG{g+go}{diff \PYGZhy{}\PYGZhy{}git a/recordings/podcasts.tsv b/recordings/podcasts.tsv}
\PYG{g+go}{new file mode 100644}
\PYG{g+go}{index 0000000..f691b53}
\PYG{g+go}{\PYGZhy{}\PYGZhy{}\PYGZhy{} /dev/null}
\PYG{g+go}{+++ b/recordings/podcasts.tsv}
\PYG{g+go}{@@ \PYGZhy{}0,0 +1,206 @@}
\PYG{g+go}{+2003\PYGZhy{}11\PYGZhy{}15	Brian Eno  The Long Now}
\PYG{g+go}{+2003\PYGZhy{}12\PYGZhy{}13	Peter Schwartz  The Art Of The Really Long View}
\PYG{g+go}{+2004\PYGZhy{}01\PYGZhy{}10	George Dyson  There s Plenty of Room at the Top  Long term Thinking About Large scale Computing}
\PYG{g+go}{+2004\PYGZhy{}02\PYGZhy{}14	James Dewar  Long term Policy Analysis}
\end{sphinxVerbatim}
\sphinxresetverbatimhllines

\sphinxAtStartPar
The commit message we have supplied with \sphinxcode{\sphinxupquote{\sphinxhyphen{}m}} directly after \sphinxcode{\sphinxupquote{datalad run}} appears
in our history as a short summary.
Additionally, the output of the command, \sphinxcode{\sphinxupquote{recordings/podcasts.tsv}},
was saved right away.

\sphinxAtStartPar
But there is more in this log entry, a section in between the markers

\sphinxAtStartPar
\sphinxcode{\sphinxupquote{=== Do not change lines below ===}} and

\sphinxAtStartPar
\sphinxcode{\sphinxupquote{\textasciicircum{}\textasciicircum{}\textasciicircum{} Do not change lines above \textasciicircum{}\textasciicircum{}\textasciicircum{}}}.

\sphinxAtStartPar
This is the so\sphinxhyphen{}called \sphinxcode{\sphinxupquote{run record}} \textendash{} a recording of all of the
information in the \sphinxcode{\sphinxupquote{datalad run}} command, generated by DataLad.
In this case, it is a very simple summary. One informative
part is highlighted:
\sphinxcode{\sphinxupquote{"cmd": "bash code/list\_titles.sh"}} is the command that was run
in the terminal.
This information therefore maps the command, and with it the script,
to the output file, in one commit. Nice, isn’t it?

\sphinxAtStartPar
Arguably, the {\hyperref[\detokenize{glossary:term-run-record}]{\sphinxtermref{\DUrole{xref,std,std-term}{run record}}}} is not the most human\sphinxhyphen{}readable way to display information.
This representation however is less for the human user (the human user should
rely on their informative commit message), but for DataLad, in particular for the
\sphinxcode{\sphinxupquote{datalad rerun}} command, which you will see in action shortly. This
\sphinxcode{\sphinxupquote{run record}} is machine\sphinxhyphen{}readable provenance that associates an output with
the command that produced it.

\sphinxAtStartPar
You have probably already guessed that every \sphinxcode{\sphinxupquote{datalad run}} command
ends with a \sphinxcode{\sphinxupquote{datalad save}}. A logical consequence from this fact is that any
\sphinxcode{\sphinxupquote{datalad run}} that does not result in any changes in a dataset (no modification
of existing content; no additional files) will not produce any record in the
dataset’s history (just as a \sphinxcode{\sphinxupquote{datalad save}} with no modifications present
will not create a history entry). Try to run the exact same
command as before, and check whether anything in your log changes:

\begin{sphinxVerbatim}[commandchars=\\\{\}]
\PYG{g+gp}{\PYGZdl{} }datalad\PYG{+w}{ }run\PYG{+w}{ }\PYGZhy{}m\PYG{+w}{ }\PYG{l+s+s2}{\PYGZdq{}Try again to create a list of podcast titles\PYGZdq{}}\PYG{+w}{ }\PYG{l+s+se}{\PYGZbs{}}
\PYG{+w}{  }\PYG{l+s+s2}{\PYGZdq{}bash code/list\PYGZus{}titles.sh \PYGZgt{} recordings/podcasts.tsv\PYGZdq{}}
\PYG{g+go}{[INFO] == Command start (output follows) =====}
\PYG{g+go}{[INFO] == Command exit (modification check follows) =====}
\PYG{g+go}{run(ok): /home/me/dl\PYGZhy{}101/DataLad\PYGZhy{}101 (dataset) [bash code/list\PYGZus{}titles.sh \PYGZgt{} recordings/po...]}
\end{sphinxVerbatim}

\fvset{hllines={, 2,}}%
\begin{sphinxVerbatim}[commandchars=\\\{\}]
\PYG{g+gp}{\PYGZdl{} }git\PYG{+w}{ }log\PYG{+w}{ }\PYGZhy{}\PYGZhy{}oneline
\PYG{g+go}{e37c9fc [DATALAD RUNCMD] create a list of podcast titles}
\PYG{g+go}{e799b6b Add short script to write a list of podcast speakers and titles}
\PYG{g+go}{87609a3 Add note on datalad clone}
\PYG{g+go}{3c016f7 [DATALAD] Added subdataset}
\end{sphinxVerbatim}
\sphinxresetverbatimhllines

\sphinxAtStartPar
The most recent commit is still the \sphinxcode{\sphinxupquote{datalad run}} command from before,
and there was no second \sphinxcode{\sphinxupquote{datalad run}} commit created.

\sphinxAtStartPar
The \sphinxcode{\sphinxupquote{datalad run}} can therefore help you to keep track of what you are doing
in a dataset and capture provenance of your files: When, by whom, and how exactly
was a particular file created or modified?
The next sections will demonstrate how to make use of this information,
and also how to extend the command with additional arguments that will prove to
be helpful over the course of this chapter.

\sphinxstepscope

\index{rerun@\spxentry{rerun}!DataLad command@\spxentry{DataLad command}}\index{DataLad command@\spxentry{DataLad command}!rerun@\spxentry{rerun}}\ignorespaces 

\section{DataLad, rerun!}
\label{\detokenize{basics/101-109-rerun:datalad-rerun}}\label{\detokenize{basics/101-109-rerun:run2}}\label{\detokenize{basics/101-109-rerun:index-0}}\label{\detokenize{basics/101-109-rerun::doc}}
\sphinxAtStartPar
So far, you created a \sphinxcode{\sphinxupquote{.tsv}} file of all
speakers and talk titles in the \sphinxcode{\sphinxupquote{longnow/}} podcasts subdataset.
Let’s actually take a look into this file now:

\begin{sphinxVerbatim}[commandchars=\\\{\}]
\PYG{g+gp}{\PYGZdl{} }less\PYG{+w}{ }recordings/podcasts.tsv
\PYG{g+go}{2003\PYGZhy{}11\PYGZhy{}15	Brian Eno  The Long Now}
\PYG{g+go}{2003\PYGZhy{}12\PYGZhy{}13	Peter Schwartz  The Art Of The Really Long View}
\PYG{g+go}{2004\PYGZhy{}02\PYGZhy{}14	James Dewar  Long term Policy Analysis}
\PYG{g+go}{2004\PYGZhy{}03\PYGZhy{}13	Rusty Schweickart  The Asteroid Threat Over the Next 100 000 Years}
\PYG{g+go}{2004\PYGZhy{}04\PYGZhy{}10	Daniel Janzen  Third World Conservation  It s ALL Gardening}
\PYG{g+go}{\PYGZhy{}✂\PYGZhy{}\PYGZhy{}✂\PYGZhy{}}
\end{sphinxVerbatim}

\sphinxAtStartPar
Not too bad, and certainly good enough for the podcast night people.
What’s been cool about creating this file is that it was created with
a script within a \sphinxcode{\sphinxupquote{datalad run}} command. Thanks to \sphinxcode{\sphinxupquote{datalad run}},
the output file \sphinxcode{\sphinxupquote{podcasts.tsv}} is associated with the script it
generated.

\sphinxAtStartPar
Upon reviewing the list you realized that you made a mistake, though: you only
listed the talks in the SALT series (the
\sphinxcode{\sphinxupquote{Long\_Now\_\_Seminars\_About\_Long\_term\_Thinking/}} directory), but not
in the \sphinxcode{\sphinxupquote{Long\_Now\_\_Conversations\_at\_The\_Interval/}} directory.
Let’s fix this in the script. Replace the contents in \sphinxcode{\sphinxupquote{code/list\_titles.sh}}
with the following, fixed script. Again, make sure to see the
\textit{Windows-wit}~{\windowswiticoninline}\textit{\ref{ww-heredoc-alt2}} {\hyperref[\detokenize{basics/101-109-rerun:ww-heredoc-alt2}]{\sphinxcrossref{\DUrole{std,std-ref}{on an alternative}}}} (\autopageref*{\detokenize{basics/101-109-rerun:ww-heredoc-alt2}}).
\begin{windowswit}[label={ww-heredoc-alt2}, before title={\thetcbcounter\ }, float, floatplacement=tb, check odd page=true]{Here’s a script adjustment for Windows users}
\label{\detokenize{basics/101-109-rerun:ww-heredoc-alt2}}

\sphinxAtStartPar
Please use an editor of your choice to replace the contents of \sphinxcode{\sphinxupquote{list\_titles.sh}} inside of the \sphinxcode{\sphinxupquote{code}} directory with the following:

\begin{sphinxVerbatim}[commandchars=\\\{\}]
\PYG{k}{for}\PYG{+w}{ }i\PYG{+w}{ }\PYG{k}{in}\PYG{+w}{ }recordings/longnow/Long\PYGZus{}Now*/*.mp3\PYG{p}{;}\PYG{+w}{ }\PYG{k}{do}
\PYG{+w}{   }\PYG{c+c1}{\PYGZsh{} get the filename}
\PYG{+w}{   }\PYG{n+nv}{base}\PYG{o}{=}\PYG{k}{\PYGZdl{}(}basename\PYG{+w}{ }\PYG{l+s+s2}{\PYGZdq{}}\PYG{n+nv}{\PYGZdl{}i}\PYG{l+s+s2}{\PYGZdq{}}\PYG{k}{)}\PYG{p}{;}
\PYG{+w}{   }\PYG{c+c1}{\PYGZsh{} strip the extension}
\PYG{+w}{   }\PYG{n+nv}{base}\PYG{o}{=}\PYG{l+s+si}{\PYGZdl{}\PYGZob{}}\PYG{n+nv}{base}\PYG{p}{\PYGZpc{}.mp3}\PYG{l+s+si}{\PYGZcb{}}\PYG{p}{;}
\PYG{+w}{   }\PYG{c+c1}{\PYGZsh{} date as yyyy\PYGZhy{}mm\PYGZhy{}dd}
\PYG{+w}{   }\PYG{n+nb}{printf}\PYG{+w}{ }\PYG{l+s+s2}{\PYGZdq{}}\PYG{l+s+si}{\PYGZdl{}\PYGZob{}}\PYG{n+nv}{base}\PYG{p}{\PYGZpc{}\PYGZpc{}\PYGZus{}\PYGZus{}*}\PYG{l+s+si}{\PYGZcb{}}\PYG{l+s+s2}{\PYGZbs{}t}\PYG{l+s+s2}{\PYGZdq{}}\PYG{+w}{ }\PYG{p}{|}\PYG{+w}{ }tr\PYG{+w}{ }\PYG{l+s+s1}{\PYGZsq{}\PYGZus{}\PYGZsq{}}\PYG{+w}{ }\PYG{l+s+s1}{\PYGZsq{}\PYGZhy{}\PYGZsq{}}\PYG{p}{;}
\PYG{+w}{   }\PYG{c+c1}{\PYGZsh{} name and title without underscores}
\PYG{+w}{   }\PYG{n+nb}{printf}\PYG{+w}{ }\PYG{l+s+s2}{\PYGZdq{}}\PYG{l+s+si}{\PYGZdl{}\PYGZob{}}\PYG{n+nv}{base}\PYG{p}{\PYGZsh{}*\PYGZus{}\PYGZus{}}\PYG{l+s+si}{\PYGZcb{}}\PYG{l+s+s2}{\PYGZbs{}n}\PYG{l+s+s2}{\PYGZdq{}}\PYG{+w}{ }\PYG{p}{|}\PYG{+w}{ }tr\PYG{+w}{ }\PYG{l+s+s1}{\PYGZsq{}\PYGZus{}\PYGZsq{}}\PYG{+w}{ }\PYG{l+s+s1}{\PYGZsq{} \PYGZsq{}}\PYG{p}{;}
\PYG{k}{done}
\end{sphinxVerbatim}


\end{windowswit}

\fvset{hllines={, 2,}}%
\begin{sphinxVerbatim}[commandchars=\\\{\}]
\PYG{g+gp}{\PYGZdl{} }cat\PYG{+w}{ }\PYGZlt{}\PYGZlt{}\PYG{+w}{ }EOT\PYG{+w}{ }\PYGZgt{}\PYG{p}{|}\PYG{+w}{ }code/list\PYGZus{}titles.sh
\PYG{g+go}{for i in recordings/longnow/Long\PYGZus{}Now*/*.mp3; do}
\PYG{g+gp}{   \PYGZsh{} }get\PYG{+w}{ }the\PYG{+w}{ }filename
\PYG{g+go}{   base=\PYGZbs{}\PYGZdl{}(basename \PYGZdq{}\PYGZbs{}\PYGZdl{}i\PYGZdq{});}
\PYG{g+gp}{   \PYGZsh{} }strip\PYG{+w}{ }the\PYG{+w}{ }extension
\PYG{g+go}{   base=\PYGZbs{}\PYGZdl{}\PYGZob{}base\PYGZpc{}.mp3\PYGZcb{};}
\PYG{g+go}{   printf \PYGZdq{}\PYGZbs{}\PYGZdl{}\PYGZob{}base\PYGZpc{}\PYGZpc{}\PYGZus{}\PYGZus{}*\PYGZcb{}\PYGZbs{}t\PYGZdq{} | tr \PYGZsq{}\PYGZus{}\PYGZsq{} \PYGZsq{}\PYGZhy{}\PYGZsq{};}
\PYG{g+gp}{   \PYGZsh{} }name\PYG{+w}{ }and\PYG{+w}{ }title\PYG{+w}{ }without\PYG{+w}{ }underscores
\PYG{g+go}{   printf \PYGZdq{}\PYGZbs{}\PYGZdl{}\PYGZob{}base\PYGZsh{}*\PYGZus{}\PYGZus{}\PYGZcb{}\PYGZbs{}n\PYGZdq{} | tr \PYGZsq{}\PYGZus{}\PYGZsq{} \PYGZsq{} \PYGZsq{};}

\PYG{g+go}{done}
\PYG{g+go}{EOT}
\end{sphinxVerbatim}
\sphinxresetverbatimhllines

\sphinxAtStartPar
Because the script is now modified, save the modifications to the dataset.
We can use the shorthand “BF” to denote “Bug fix” in the commit message.

\begin{sphinxVerbatim}[commandchars=\\\{\}]
\PYG{g+gp}{\PYGZdl{} }datalad\PYG{+w}{ }status
\PYG{g+go}{ modified: code/list\PYGZus{}titles.sh (file)}
\end{sphinxVerbatim}

\begin{sphinxVerbatim}[commandchars=\\\{\}]
\PYG{g+gp}{\PYGZdl{} }datalad\PYG{+w}{ }save\PYG{+w}{ }\PYGZhy{}m\PYG{+w}{ }\PYG{l+s+s2}{\PYGZdq{}BF: list both directories content\PYGZdq{}}\PYG{+w}{ }\PYG{l+s+se}{\PYGZbs{}}
\PYG{+w}{  }code/list\PYGZus{}titles.sh
\PYG{g+go}{add(ok): code/list\PYGZus{}titles.sh (file)}
\PYG{g+go}{save(ok): . (dataset)}
\end{sphinxVerbatim}

\sphinxAtStartPar
What we \sphinxstyleemphasis{could} do is run the same \sphinxcode{\sphinxupquote{datalad run}} command as before to recreate
the file, but now with all of the contents:

\begin{sphinxVerbatim}[commandchars=\\\{\}]
\PYG{g+gp}{\PYGZdl{} }\PYG{c+c1}{\PYGZsh{} do not execute this!}
\PYG{g+gp}{\PYGZdl{} }datalad\PYG{+w}{ }run\PYG{+w}{ }\PYGZhy{}m\PYG{+w}{ }\PYG{l+s+s2}{\PYGZdq{}create a list of podcast titles\PYGZdq{}}\PYG{+w}{ }\PYG{l+s+se}{\PYGZbs{}}
\PYG{+w}{  }\PYG{l+s+s2}{\PYGZdq{}bash code/list\PYGZus{}titles.sh \PYGZgt{} recordings/podcasts.tsv\PYGZdq{}}
\end{sphinxVerbatim}

\sphinxAtStartPar
However, think about any situation where the command would be longer than this,
or that is many months past the first execution. It would not be easy to remember
the command, nor would it be very convenient to copy it from the \sphinxcode{\sphinxupquote{run record}}.

\sphinxAtStartPar
Luckily, a fellow student remembered the DataLad way of re\sphinxhyphen{}executing
a \sphinxcode{\sphinxupquote{run}} command, and he’s eager to show it to you.

\sphinxAtStartPar
“In order to re\sphinxhyphen{}execute a \sphinxcode{\sphinxupquote{datalad run}} command,
find the commit and use its {\hyperref[\detokenize{glossary:term-shasum}]{\sphinxtermref{\DUrole{xref,std,std-term}{shasum}}}} (or a {\hyperref[\detokenize{glossary:term-tag}]{\sphinxtermref{\DUrole{xref,std,std-term}{tag}}}}, or anything else that Git
understands) as an argument for the
\sphinxcode{\sphinxupquote{datalad rerun}} command! That’s it!”,
he says happily.

\sphinxAtStartPar
So you go ahead and find the commit {\hyperref[\detokenize{glossary:term-shasum}]{\sphinxtermref{\DUrole{xref,std,std-term}{shasum}}}} in your history:

\fvset{hllines={, 8,}}%
\begin{sphinxVerbatim}[commandchars=\\\{\}]
\PYG{g+gp}{\PYGZdl{} }git\PYG{+w}{ }log\PYG{+w}{ }\PYGZhy{}n\PYG{+w}{ }\PYG{l+m}{2}
\PYG{g+go}{commit f7ea9f3d✂SHA1}
\PYG{g+go}{Author: Elena Piscopia \PYGZlt{}elena@example.net\PYGZgt{}}
\PYG{g+go}{Date:   Tue Jun 18 16:13:00 2019 +0000}

\PYG{g+go}{    BF: list both directories content}

\PYG{g+go}{commit e37c9fc9✂SHA1}
\PYG{g+go}{Author: Elena Piscopia \PYGZlt{}elena@example.net\PYGZgt{}}
\PYG{g+go}{Date:   Tue Jun 18 16:13:00 2019 +0000}

\PYG{g+go}{    [DATALAD RUNCMD] create a list of podcast titles}
\end{sphinxVerbatim}
\sphinxresetverbatimhllines

\sphinxAtStartPar
Take that shasum and paste it after \sphinxcode{\sphinxupquote{datalad rerun}}
(the first 6\sphinxhyphen{}8 characters of the shasum would be sufficient,
here we are using all of them).

\begin{sphinxVerbatim}[commandchars=\\\{\}]
\PYG{g+gp}{\PYGZdl{} }datalad\PYG{+w}{ }rerun\PYG{+w}{ }e37c9fc9✂SHA1
\PYG{g+go}{[INFO] run commit e37c9fc; (create a list of ...)}
\PYG{g+go}{[INFO] == Command start (output follows) =====}
\PYG{g+go}{[INFO] == Command exit (modification check follows) =====}
\PYG{g+go}{run(ok): /home/me/dl\PYGZhy{}101/DataLad\PYGZhy{}101 (dataset) [bash code/list\PYGZus{}titles.sh \PYGZgt{} recordings/po...]}
\PYG{g+go}{add(ok): recordings/podcasts.tsv (file)}
\PYG{g+go}{save(ok): . (dataset)}
\PYG{g+go}{action summary:}
\PYG{g+go}{  add (ok: 1)}
\PYG{g+go}{  run (ok: 1)}
\PYG{g+go}{  save (notneeded: 1, ok: 1)}
\PYG{g+go}{  unlock (notneeded: 1)}
\end{sphinxVerbatim}

\sphinxAtStartPar
Now DataLad has made use of the \sphinxcode{\sphinxupquote{run record}}, and
re\sphinxhyphen{}executed the original command based on the information in it.
Because we updated the script, the output \sphinxcode{\sphinxupquote{podcasts.tsv}}
has changed and now contains the podcast
titles of both subdirectories.
You’ve probably already guessed it, but the easiest way
to check whether a \sphinxcode{\sphinxupquote{datalad rerun}}
has changed the desired output file is
to check whether the rerun command appears in the datasets history:
If a \sphinxcode{\sphinxupquote{datalad rerun}} does not add or change any content in the dataset,
it will also not be recorded in the history.

\begin{sphinxVerbatim}[commandchars=\\\{\}]
\PYG{g+gp}{\PYGZdl{} }git\PYG{+w}{ }log\PYG{+w}{ }\PYGZhy{}n\PYG{+w}{ }\PYG{l+m}{1}
\PYG{g+go}{commit 08120c38✂SHA1}
\PYG{g+go}{Author: Elena Piscopia \PYGZlt{}elena@example.net\PYGZgt{}}
\PYG{g+go}{Date:   Tue Jun 18 16:13:00 2019 +0000}

\PYG{g+go}{    [DATALAD RUNCMD] create a list of podcast titles}

\PYG{g+go}{    === Do not change lines below ===}
\PYG{g+go}{    \PYGZob{}}
\PYG{g+go}{     \PYGZdq{}chain\PYGZdq{}: [}
\PYG{g+go}{      \PYGZdq{}e37c9fc9✂SHA1\PYGZdq{}}
\PYG{g+go}{     ],}
\PYG{g+go}{     \PYGZdq{}cmd\PYGZdq{}: \PYGZdq{}bash code/list\PYGZus{}titles.sh \PYGZgt{} recordings/podcasts.tsv\PYGZdq{},}
\PYG{g+go}{     \PYGZdq{}dsid\PYGZdq{}: \PYGZdq{}e3e70682\PYGZhy{}c209\PYGZhy{}4cac\PYGZhy{}629f\PYGZhy{}6fbed82c07cd\PYGZdq{},}
\PYG{g+go}{     \PYGZdq{}exit\PYGZdq{}: 0,}
\PYG{g+go}{     \PYGZdq{}extra\PYGZus{}inputs\PYGZdq{}: [],}
\PYG{g+go}{     \PYGZdq{}inputs\PYGZdq{}: [],}
\PYG{g+go}{     \PYGZdq{}outputs\PYGZdq{}: [],}
\PYG{g+go}{     \PYGZdq{}pwd\PYGZdq{}: \PYGZdq{}.\PYGZdq{}}
\PYG{g+go}{    \PYGZcb{}}
\PYG{g+go}{    \PYGZca{}\PYGZca{}\PYGZca{} Do not change lines above \PYGZca{}\PYGZca{}\PYGZca{}}
\end{sphinxVerbatim}

\sphinxAtStartPar
In the dataset’s history,
we can see that a new \sphinxcode{\sphinxupquote{datalad run}} was recorded. This action is
committed by DataLad under the original commit message of the \sphinxcode{\sphinxupquote{run}}
command, and looks just like the previous \sphinxcode{\sphinxupquote{datalad run}} commit.

\index{diff@\spxentry{diff}!DataLad command@\spxentry{DataLad command}}\index{DataLad command@\spxentry{DataLad command}!diff@\spxentry{diff}}\ignorespaces 
\sphinxAtStartPar
Two cool tools that go beyond the \sphinxcode{\sphinxupquote{git log}}
are the \sphinxcode{\sphinxupquote{datalad diff}} and \sphinxcode{\sphinxupquote{git diff}} commands.
Both commands can report differences between two states of
a dataset. Thus, you can get an overview of what changed between two commits.
Both commands have a similar, but not identical structure: \sphinxcode{\sphinxupquote{datalad diff}}
compares one state (a commit specified with \sphinxcode{\sphinxupquote{\sphinxhyphen{}f}}/\sphinxcode{\sphinxupquote{\sphinxhyphen{}\sphinxhyphen{}from}},
by default the latest change)
and another state from the dataset’s history (a commit specified with
\sphinxcode{\sphinxupquote{\sphinxhyphen{}t}}/\sphinxcode{\sphinxupquote{\sphinxhyphen{}\sphinxhyphen{}to}}). Let’s do a \sphinxcode{\sphinxupquote{datalad diff}} between the current state
of the dataset and the previous commit (called “\sphinxcode{\sphinxupquote{HEAD\textasciitilde{}1}}” in Git terminology%
\begin{footnote}\sphinxAtStartFootnote
The section {\hyperref[\detokenize{basics/101-137-history:history}]{\sphinxcrossref{\DUrole{std,std-ref}{Git things done}}}} (\autopageref*{\detokenize{basics/101-137-history:history}}) will elaborate more on common {\hyperref[\detokenize{glossary:term-Git}]{\sphinxtermref{\DUrole{xref,std,std-term}{Git}}}} commands
and terminology.
%
\end{footnote}).
Check the \textit{Windows-wit}~{\windowswiticoninline}\textit{\ref{ww-diff-adjusted}} {\hyperref[\detokenize{basics/101-109-rerun:ww-diff-adjusted}]{\sphinxcrossref{\DUrole{std,std-ref}{on a command variant}}}} (\autopageref*{\detokenize{basics/101-109-rerun:ww-diff-adjusted}}).

\index{show dataset modification@\spxentry{show dataset modification}!on Windows with DataLad@\spxentry{on Windows with DataLad}}\index{on Windows with DataLad@\spxentry{on Windows with DataLad}!show dataset modification@\spxentry{show dataset modification}}\index{diff@\spxentry{diff}!DataLad command@\spxentry{DataLad command}}\index{DataLad command@\spxentry{DataLad command}!diff@\spxentry{diff}}\index{corresponding branch@\spxentry{corresponding branch}!in adjusted mode@\spxentry{in adjusted mode}}\index{in adjusted mode@\spxentry{in adjusted mode}!corresponding branch@\spxentry{corresponding branch}}\ignorespaces \begin{windowswit}[label={ww-diff-adjusted}, before title={\thetcbcounter\ }, float, floatplacement=tb, check odd page=true]{please use ‘datalad diff –from main –to HEAD~1’}
\label{\detokenize{basics/101-109-rerun:ww-diff-adjusted}}

\sphinxAtStartPar
While this example works on Unix file systems, it will not provide the same output on Windows.
This is due to different file handling on Windows.
When executing this command, you will see \sphinxstyleemphasis{all} files being modified between the most recent and the second\sphinxhyphen{}most recent commit.
On a technical level, this is correct given the underlying file handling on Windows, and chapter {\hyperref[\detokenize{basics/basics-annex:chapter-gitannex}]{\sphinxcrossref{\DUrole{std,std-ref}{Under the hood: git\sphinxhyphen{}annex}}}} (\autopageref*{\detokenize{basics/basics-annex:chapter-gitannex}}) will shed light on why that is.

\sphinxAtStartPar
For now, to get the same output as shown in the code snippet below, use the following command where \sphinxcode{\sphinxupquote{main}} (or \sphinxcode{\sphinxupquote{master}}) is the name of your default branch:

\begin{sphinxVerbatim}[commandchars=\\\{\}]
\PYG{g+gp}{\PYGZdl{} }datalad\PYG{+w}{ }diff\PYG{+w}{ }\PYGZhy{}\PYGZhy{}from\PYG{+w}{ }main\PYG{+w}{ }\PYGZhy{}\PYGZhy{}to\PYG{+w}{ }HEAD\PYGZti{}1
\end{sphinxVerbatim}

\sphinxAtStartPar
The \sphinxcode{\sphinxupquote{\sphinxhyphen{}\sphinxhyphen{}from}} argument specifies a different starting point for the comparison \sphinxhyphen{} the \sphinxcode{\sphinxupquote{main}} or {\hyperref[\detokenize{glossary:term-master}]{\sphinxtermref{\DUrole{xref,std,std-term}{master}}}} {\hyperref[\detokenize{glossary:term-branch}]{\sphinxtermref{\DUrole{xref,std,std-term}{branch}}}}, which would be the starting point on most Unix\sphinxhyphen{}based systems.


\end{windowswit}

\index{diff@\spxentry{diff}!Git command@\spxentry{Git command}}\index{Git command@\spxentry{Git command}!diff@\spxentry{diff}}\index{show dataset modification@\spxentry{show dataset modification}!with DataLad@\spxentry{with DataLad}}\index{with DataLad@\spxentry{with DataLad}!show dataset modification@\spxentry{show dataset modification}}\ignorespaces 
\def\sphinxLiteralBlockLabel{\label{\detokenize{basics/101-109-rerun:index-3}}}
\begin{sphinxVerbatim}[commandchars=\\\{\}]
\PYG{g+gp}{\PYGZdl{} }datalad\PYG{+w}{ }diff\PYG{+w}{ }\PYGZhy{}\PYGZhy{}to\PYG{+w}{ }HEAD\PYGZti{}1
\PYG{g+go}{ modified: recordings/podcasts.tsv (file)}
\end{sphinxVerbatim}

\index{diff@\spxentry{diff}!Git command@\spxentry{Git command}}\index{Git command@\spxentry{Git command}!diff@\spxentry{diff}}\index{show dataset modification@\spxentry{show dataset modification}!with Git@\spxentry{with Git}}\index{with Git@\spxentry{with Git}!show dataset modification@\spxentry{show dataset modification}}\ignorespaces 
\sphinxAtStartPar
This indeed shows the output file as “modified”. However, we do not know
what exactly changed. This is a task for \sphinxcode{\sphinxupquote{git diff}} (get out of the
diff view by pressing \sphinxcode{\sphinxupquote{q}}):

\begin{sphinxVerbatim}[commandchars=\\\{\}]
\PYG{g+gp}{\PYGZdl{} }git\PYG{+w}{ }diff\PYG{+w}{ }HEAD\PYGZti{}1
\PYG{g+go}{diff \PYGZhy{}\PYGZhy{}git a/recordings/podcasts.tsv b/recordings/podcasts.tsv}
\PYG{g+go}{index f691b53..d77891d 100644}
\PYG{g+go}{\PYGZhy{}\PYGZhy{}\PYGZhy{} a/recordings/podcasts.tsv}
\PYG{g+go}{+++ b/recordings/podcasts.tsv}
\PYG{g+go}{@@ \PYGZhy{}1,3 +1,31 @@}
\PYG{g+go}{+2017\PYGZhy{}06\PYGZhy{}09	How Digital Memory Is Shaping Our Future  Abby Smith Rumsey}
\PYG{g+go}{+2017\PYGZhy{}06\PYGZhy{}09	Pace Layers Thinking  Stewart Brand  Paul Saffo}
\PYG{g+go}{+2017\PYGZhy{}06\PYGZhy{}09	Proof  The Science of Booze  Adam Rogers}
\PYG{g+go}{+2017\PYGZhy{}06\PYGZhy{}09	Seveneves at The Interval  Neal Stephenson}
\PYG{g+go}{+2017\PYGZhy{}06\PYGZhy{}09	Talking with Robots about Architecture  Jeffrey McGrew}
\PYG{g+go}{+2017\PYGZhy{}06\PYGZhy{}09	The Red Planet for Real  Andy Weir}
\PYG{g+go}{+2017\PYGZhy{}07\PYGZhy{}03	Transforming Perception  One Sense at a Time  Kara Platoni}
\PYG{g+go}{+2017\PYGZhy{}08\PYGZhy{}01	How Climate Will Evolve Government and Society  Kim Stanley Robinson}
\PYG{g+go}{+2017\PYGZhy{}09\PYGZhy{}01	Envisioning Deep Time  Jonathon Keats}
\PYG{g+go}{+2017\PYGZhy{}10\PYGZhy{}01	Thinking Long term About the Evolving Global Challenge  The Refugee Reality}
\PYG{g+go}{+2017\PYGZhy{}11\PYGZhy{}01	The Web In An Eye Blink  Jason Scott}
\PYG{g+go}{+2017\PYGZhy{}12\PYGZhy{}01	Ideology in our Genes  The Biological Basis for Political Traits  Rose McDermott}
\PYG{g+go}{+2017\PYGZhy{}12\PYGZhy{}07	Can Democracy Survive the Internet   Nathaniel Persily}
\PYG{g+go}{+2018\PYGZhy{}01\PYGZhy{}02	The New Deal You Don t Know  Louis Hyman}
\end{sphinxVerbatim}

\sphinxAtStartPar
This output actually shows the precise changes between the contents created
with the first version of the script and the second script with the bug fix.
All of the files that are added after the second directory
was queried as well are shown in the \sphinxcode{\sphinxupquote{diff}}, preceded by a \sphinxcode{\sphinxupquote{+}}.

\sphinxAtStartPar
Quickly create a note about these two helpful commands in \sphinxcode{\sphinxupquote{notes.txt}}:

\begin{sphinxVerbatim}[commandchars=\\\{\}]
\PYG{g+gp}{\PYGZdl{} }cat\PYG{+w}{ }\PYGZlt{}\PYGZlt{}\PYG{+w}{ }EOT\PYG{+w}{ }\PYGZgt{}\PYGZgt{}\PYG{+w}{ }notes.txt
\PYG{g+go}{There are two useful functions to display changes between two}
\PYG{g+go}{states of a dataset: \PYGZdq{}datalad diff \PYGZhy{}f/\PYGZhy{}\PYGZhy{}from COMMIT \PYGZhy{}t/\PYGZhy{}\PYGZhy{}to COMMIT\PYGZdq{}}
\PYG{g+go}{and \PYGZdq{}git diff COMMIT COMMIT\PYGZdq{}, where COMMIT is a shasum of a commit}
\PYG{g+go}{in the history.}

\PYG{g+go}{EOT}
\end{sphinxVerbatim}

\sphinxAtStartPar
Finally, save this note.

\begin{sphinxVerbatim}[commandchars=\\\{\}]
\PYG{g+gp}{\PYGZdl{} }datalad\PYG{+w}{ }save\PYG{+w}{ }\PYGZhy{}m\PYG{+w}{ }\PYG{l+s+s2}{\PYGZdq{}add note datalad and git diff\PYGZdq{}}
\PYG{g+go}{add(ok): notes.txt (file)}
\PYG{g+go}{save(ok): . (dataset)}
\end{sphinxVerbatim}

\sphinxAtStartPar
Note that \sphinxcode{\sphinxupquote{datalad rerun}} can re\sphinxhyphen{}execute the run records of both a \sphinxcode{\sphinxupquote{datalad run}}
or a \sphinxcode{\sphinxupquote{datalad rerun}} command,
but not with any other type of DataLad command in your history
such as a \sphinxcode{\sphinxupquote{datalad save}} on results or outputs after you executed a script.
Therefore, make it a
habit to record the execution of scripts by plugging it into \sphinxcode{\sphinxupquote{datalad run}}.

\sphinxAtStartPar
This very basic example of a \sphinxcode{\sphinxupquote{datalad run}} is as simple as it can get, but it
is already
convenient from a memory\sphinxhyphen{}load perspective: Now you do not need to
remember the commands or scripts involved in creating an output. DataLad kept track
of what you did, and you can instruct it to “\sphinxcode{\sphinxupquote{rerun}}” it.
Also, incidentally, we have generated {\hyperref[\detokenize{glossary:term-provenance}]{\sphinxtermref{\DUrole{xref,std,std-term}{provenance}}}} information. It is
now recorded in the history of the dataset how the output \sphinxcode{\sphinxupquote{podcasts.tsv}} came
into existence. And we can interact with and use this provenance information with
other tools than from the machine\sphinxhyphen{}readable \sphinxcode{\sphinxupquote{run record}}.
For example, to find out who (or what) created or modified a file,
give the file path to \sphinxcode{\sphinxupquote{git log}} (prefixed by \sphinxcode{\sphinxupquote{\sphinxhyphen{}\sphinxhyphen{}}}).
Check the \textit{Windows-wit}~{\windowswiticoninline}\textit{\ref{ww-log-adjusted}} {\hyperref[\detokenize{basics/101-109-rerun:ww-log-adjusted}]{\sphinxcrossref{\DUrole{std,std-ref}{on a command variant}}}} (\autopageref*{\detokenize{basics/101-109-rerun:ww-log-adjusted}}).

\index{show history for particular paths@\spxentry{show history for particular paths}!on Windows with Git@\spxentry{on Windows with Git}}\index{on Windows with Git@\spxentry{on Windows with Git}!show history for particular paths@\spxentry{show history for particular paths}}\index{log@\spxentry{log}!Git command@\spxentry{Git command}}\index{Git command@\spxentry{Git command}!log@\spxentry{log}}\index{corresponding branch@\spxentry{corresponding branch}!in adjusted mode@\spxentry{in adjusted mode}}\index{in adjusted mode@\spxentry{in adjusted mode}!corresponding branch@\spxentry{corresponding branch}}\ignorespaces \begin{windowswit}[label={ww-log-adjusted}, before title={\thetcbcounter\ }, float, floatplacement=tb, check odd page=true]{use ‘git log main – recordings/podcasts.tsv’}
\label{\detokenize{basics/101-109-rerun:ww-log-adjusted}}

\sphinxAtStartPar
A previous Windows Wit already advised to append \sphinxcode{\sphinxupquote{main}} or \sphinxcode{\sphinxupquote{master}}, the common “default {\hyperref[\detokenize{glossary:term-branch}]{\sphinxtermref{\DUrole{xref,std,std-term}{branch}}}}”, to any command that starts with \sphinxcode{\sphinxupquote{git log}}.
Here, the last part of the command specifies a file (\sphinxcode{\sphinxupquote{\sphinxhyphen{}\sphinxhyphen{} recordings/podcasts.tsv}}).
Please append \sphinxcode{\sphinxupquote{main}} or \sphinxcode{\sphinxupquote{master}} to \sphinxcode{\sphinxupquote{git log}}, prior to the file specification.


\end{windowswit}

\index{show history for particular paths@\spxentry{show history for particular paths}!with Git@\spxentry{with Git}}\index{with Git@\spxentry{with Git}!show history for particular paths@\spxentry{show history for particular paths}}\ignorespaces 
\def\sphinxLiteralBlockLabel{\label{\detokenize{basics/101-109-rerun:index-6}}}
\begin{sphinxVerbatim}[commandchars=\\\{\}]
\PYG{g+gp}{\PYGZdl{} }git\PYG{+w}{ }log\PYG{+w}{ }\PYGZhy{}\PYGZhy{}\PYG{+w}{ }recordings/podcasts.tsv
\PYG{g+go}{commit 08120c38✂SHA1}
\PYG{g+go}{Author: Elena Piscopia \PYGZlt{}elena@example.net\PYGZgt{}}
\PYG{g+go}{Date:   Tue Jun 18 16:13:00 2019 +0000}

\PYG{g+go}{    [DATALAD RUNCMD] create a list of podcast titles}

\PYG{g+go}{    === Do not change lines below ===}
\PYG{g+go}{    \PYGZob{}}
\PYG{g+go}{     \PYGZdq{}chain\PYGZdq{}: [}
\PYG{g+go}{      \PYGZdq{}e37c9fc9✂SHA1\PYGZdq{}}
\PYG{g+go}{     ],}
\PYG{g+go}{     \PYGZdq{}cmd\PYGZdq{}: \PYGZdq{}bash code/list\PYGZus{}titles.sh \PYGZgt{} recordings/podcasts.tsv\PYGZdq{},}
\PYG{g+go}{     \PYGZdq{}dsid\PYGZdq{}: \PYGZdq{}e3e70682\PYGZhy{}c209\PYGZhy{}4cac\PYGZhy{}629f\PYGZhy{}6fbed82c07cd\PYGZdq{},}
\PYG{g+go}{     \PYGZdq{}exit\PYGZdq{}: 0,}
\PYG{g+go}{     \PYGZdq{}extra\PYGZus{}inputs\PYGZdq{}: [],}
\PYG{g+go}{     \PYGZdq{}inputs\PYGZdq{}: [],}
\PYG{g+go}{     \PYGZdq{}outputs\PYGZdq{}: [],}
\PYG{g+go}{     \PYGZdq{}pwd\PYGZdq{}: \PYGZdq{}.\PYGZdq{}}
\PYG{g+go}{    \PYGZcb{}}
\PYG{g+go}{    \PYGZca{}\PYGZca{}\PYGZca{} Do not change lines above \PYGZca{}\PYGZca{}\PYGZca{}}

\PYG{g+go}{commit e37c9fc9✂SHA1}
\PYG{g+go}{Author: Elena Piscopia \PYGZlt{}elena@example.net\PYGZgt{}}
\PYG{g+go}{Date:   Tue Jun 18 16:13:00 2019 +0000}

\PYG{g+go}{    [DATALAD RUNCMD] create a list of podcast titles}

\PYG{g+go}{    === Do not change lines below ===}
\PYG{g+go}{    \PYGZob{}}
\PYG{g+go}{     \PYGZdq{}chain\PYGZdq{}: [],}
\PYG{g+go}{     \PYGZdq{}cmd\PYGZdq{}: \PYGZdq{}bash code/list\PYGZus{}titles.sh \PYGZgt{} recordings/podcasts.tsv\PYGZdq{},}
\PYG{g+go}{     \PYGZdq{}dsid\PYGZdq{}: \PYGZdq{}e3e70682\PYGZhy{}c209\PYGZhy{}4cac\PYGZhy{}629f\PYGZhy{}6fbed82c07cd\PYGZdq{},}
\PYG{g+go}{     \PYGZdq{}exit\PYGZdq{}: 0,}
\PYG{g+go}{     \PYGZdq{}extra\PYGZus{}inputs\PYGZdq{}: [],}
\PYG{g+go}{     \PYGZdq{}inputs\PYGZdq{}: [],}
\PYG{g+go}{     \PYGZdq{}outputs\PYGZdq{}: [],}
\PYG{g+go}{     \PYGZdq{}pwd\PYGZdq{}: \PYGZdq{}.\PYGZdq{}}
\PYG{g+go}{    \PYGZcb{}}
\PYG{g+go}{    \PYGZca{}\PYGZca{}\PYGZca{} Do not change lines above \PYGZca{}\PYGZca{}\PYGZca{}}
\end{sphinxVerbatim}

\sphinxAtStartPar
Neat, isn’t it?

\sphinxAtStartPar
Still, this \sphinxcode{\sphinxupquote{datalad run}} was very simple.
The next section will demonstrate how \sphinxcode{\sphinxupquote{datalad run}} becomes handy in
more complex standard use cases: situations with \sphinxstyleemphasis{locked} contents.

\sphinxAtStartPar
But prior to that, make a note about \sphinxcode{\sphinxupquote{datalad run}} and \sphinxcode{\sphinxupquote{datalad rerun}} in your
\sphinxcode{\sphinxupquote{notes.txt}} file.

\begin{sphinxVerbatim}[commandchars=\\\{\}]
\PYG{g+gp}{\PYGZdl{} }cat\PYG{+w}{ }\PYGZlt{}\PYGZlt{}\PYG{+w}{ }EOT\PYG{+w}{ }\PYGZgt{}\PYGZgt{}\PYG{+w}{ }notes.txt
\PYG{g+go}{The datalad run command can record the impact a script or command has}
\PYG{g+go}{on a Dataset. In its simplest form, datalad run only takes a commit}
\PYG{g+go}{message and the command that should be executed.}

\PYG{g+go}{Any datalad run command can be re\PYGZhy{}executed by using its commit shasum}
\PYG{g+go}{as an argument in datalad rerun CHECKSUM. DataLad will take}
\PYG{g+go}{information from the run record of the original commit, and re\PYGZhy{}execute}
\PYG{g+go}{it. If no changes happen with a rerun, the command will not be written}
\PYG{g+go}{to history. Note: you can also rerun a datalad rerun command!}

\PYG{g+go}{EOT}
\end{sphinxVerbatim}

\sphinxAtStartPar
Finally, save this note.

\begin{sphinxVerbatim}[commandchars=\\\{\}]
\PYG{g+gp}{\PYGZdl{} }datalad\PYG{+w}{ }save\PYG{+w}{ }\PYGZhy{}m\PYG{+w}{ }\PYG{l+s+s2}{\PYGZdq{}add note on basic datalad run and datalad rerun\PYGZdq{}}
\PYG{g+go}{add(ok): notes.txt (file)}
\PYG{g+go}{save(ok): . (dataset)}
\end{sphinxVerbatim}

\sphinxstepscope


\section{Input and output}
\label{\detokenize{basics/101-110-run2:input-and-output}}\label{\detokenize{basics/101-110-run2:run3}}\label{\detokenize{basics/101-110-run2::doc}}
\sphinxAtStartPar
In the previous two sections, you created a simple \sphinxcode{\sphinxupquote{.tsv}} file of all
speakers and talk titles in the \sphinxcode{\sphinxupquote{longnow/}} podcasts subdataset, and you have
re\sphinxhyphen{}executed a \sphinxcode{\sphinxupquote{datalad run}} command after a bug\sphinxhyphen{}fix in your script.

\sphinxAtStartPar
But these previous \sphinxcode{\sphinxupquote{datalad run}} and \sphinxcode{\sphinxupquote{datalad rerun}} command were very simple.
Maybe you noticed some values in the \sphinxcode{\sphinxupquote{run record}} were empty:
\sphinxcode{\sphinxupquote{inputs}} and \sphinxcode{\sphinxupquote{outputs}} for example did not have an entry. Let’s experience
a few situations in which
these two arguments can become necessary.

\sphinxAtStartPar
In our DataLad\sphinxhyphen{}101 course we were given a group assignment. Everyone should
give a small presentation about an open DataLad dataset they found. Conveniently,
you decided to settle for the longnow podcasts right away.
After all, you know the dataset quite well already,
and after listening to almost a third of the podcasts
and enjoying them a lot,
you also want to recommend them to the others.

\sphinxAtStartPar
Almost all of the slides are ready, but what’s still missing is the logo of the
longnow podcasts. Good thing that this is part of the subdataset,
so you can simply retrieve it from there.

\sphinxAtStartPar
The logos (one for the SALT series, one for the Interval series \textendash{} the two
directories in the subdataset)
were originally extracted from the podcasts metadata information by DataLad.
The online\sphinxhyphen{}handbook has a chapter on the metadata aggregation capabilities of DataLad,
but for now, let’s just use the logos instead of finding out where they
come from \textendash{} this will come later.
As part of the metadata of the dataset, the logos are
in the hidden paths
\sphinxcode{\sphinxupquote{.datalad/feed\_metadata/logo\_salt.jpg}} and
\sphinxcode{\sphinxupquote{.datalad/feed\_metadata/logo\_interval.jpg}}:

\begin{sphinxVerbatim}[commandchars=\\\{\}]
\PYG{g+gp}{\PYGZdl{} }ls\PYG{+w}{ }recordings/longnow/.datalad/feed\PYGZus{}metadata/*jpg
\PYG{g+go}{recordings/longnow/.datalad/feed\PYGZus{}metadata/logo\PYGZus{}interval.jpg}
\PYG{g+go}{recordings/longnow/.datalad/feed\PYGZus{}metadata/logo\PYGZus{}salt.jpg}
\end{sphinxVerbatim}

\sphinxAtStartPar
For the slides you decide to prepare images of size 400x400 px, but
the logos’ original size is much larger (both are 3000x3000 pixel). Therefore
let’s try to resize the images \textendash{} currently, they are far too large to fit on a slide.

\sphinxAtStartPar
To resize an image from the command line we can use the Unix
command \sphinxcode{\sphinxupquote{convert \sphinxhyphen{}resize}} from the \dlhbhref{I1A}{ImageMagick tool}.
See the \textit{Windows-wit}~{\windowswiticoninline}\textit{\ref{ww-install-imagemagick}} {\hyperref[\detokenize{basics/101-110-run2:ww-install-imagemagick}]{\sphinxcrossref{\DUrole{std,std-ref}{on installing it}}}} (\autopageref*{\detokenize{basics/101-110-run2:ww-install-imagemagick}}), if your system does not provide it.
The command takes a new size in pixels as an argument, a path to the file that should be
resized, and a filename and path under which a new,
resized image will be saved.
To resize one image to 400x400 px, the command would thus be
\sphinxcode{\sphinxupquote{convert \sphinxhyphen{}resize 400x400 path/to/file.jpg path/to/newfilename.jpg}}.

\index{install ImageMagick@\spxentry{install ImageMagick}!on Windows@\spxentry{on Windows}}\index{on Windows@\spxentry{on Windows}!install ImageMagick@\spxentry{install ImageMagick}}\index{installation@\spxentry{installation}!ImageMagick@\spxentry{ImageMagick}}\ignorespaces \begin{windowswit}[label={ww-install-imagemagick}, before title={\thetcbcounter\ }, float, floatplacement=tb, check odd page=true]{Tool installation}
\label{\detokenize{basics/101-110-run2:ww-install-imagemagick}}

\sphinxAtStartPar
\dlhbhref{I1A}{ImageMagick} is not installed on Windows systems by default.
To use it, you need to install it, using the provided \dlhbhref{I1B}{Windows Binary Release on the Download page}.
During installation, it is important to install the tool into a place where it is easily accessible to your terminal, for example the \sphinxcode{\sphinxupquote{Program Files}} folder.
Do also make sure to tick the box “install legacy commands” in the installation wizard.


\end{windowswit}

\sphinxAtStartPar
Remembering the last lecture on \sphinxcode{\sphinxupquote{datalad run}}, you decide to plug this into
\sphinxcode{\sphinxupquote{datalad run}}. Even though this is not a script, it is a command, and you can wrap
commands like this conveniently with \sphinxcode{\sphinxupquote{datalad run}}.
Because they will be quite long, we line break the commands in the upcoming examples
for better readability \textendash{} in your terminal, you can always write the commands into
a single line.

\index{run command with provenance capture@\spxentry{run command with provenance capture}!with DataLad run@\spxentry{with DataLad run}}\index{with DataLad run@\spxentry{with DataLad run}!run command with provenance capture@\spxentry{run command with provenance capture}}\ignorespaces 
\def\sphinxLiteralBlockLabel{\label{\detokenize{basics/101-110-run2:index-1}}}
\fvset{hllines={, 4,}}%
\begin{sphinxVerbatim}[commandchars=\\\{\}]
\PYG{g+gp}{\PYGZdl{} }datalad\PYG{+w}{ }run\PYG{+w}{ }\PYGZhy{}m\PYG{+w}{ }\PYG{l+s+s2}{\PYGZdq{}Resize logo for slides\PYGZdq{}}\PYG{+w}{ }\PYG{l+s+se}{\PYGZbs{}}
\PYG{l+s+s2}{\PYGZdq{}convert \PYGZhy{}resize 400x400 recordings/longnow/.datalad/feed\PYGZus{}metadata/logo\PYGZus{}salt.jpg recordings/salt\PYGZus{}logo\PYGZus{}small.jpg\PYGZdq{}}
\PYG{g+go}{[INFO] == Command start (output follows) =====}
\PYG{g+go}{convert convert: Unable to open file (recordings/longnow/.datalad/feed\PYGZus{}metadata/logo\PYGZus{}salt.jpg) [No such file or directory].}
\PYG{g+go}{[INFO] == Command exit (modification check follows) =====}
\PYG{g+go}{[INFO] The command had a non\PYGZhy{}zero exit code. If this is expected, you can save the changes with \PYGZsq{}datalad save \PYGZhy{}d . \PYGZhy{}r \PYGZhy{}F .git/COMMIT\PYGZus{}EDITMSG\PYGZsq{}}
\PYG{g+go}{run(error): /home/me/dl\PYGZhy{}101/DataLad\PYGZhy{}101 (dataset) [convert \PYGZhy{}resize 400x400 recordings/longn...]}
\end{sphinxVerbatim}
\sphinxresetverbatimhllines

\sphinxAtStartPar
\sphinxstyleemphasis{Oh, crap!} Why didn’t this work?

\sphinxAtStartPar
Let’s take a look at the error message DataLad provides. In general, these error messages
might seem wordy, and maybe a bit intimidating as well, but usually they provide helpful
information to find out what is wrong. Whenever you encounter an error message,
make sure to read it, even if it feels like a mushroom cloud exploded in your terminal.

\sphinxAtStartPar
A \sphinxcode{\sphinxupquote{datalad run}} error message has several parts. The first starts after

\sphinxAtStartPar
\sphinxcode{\sphinxupquote{{[}INFO   {]} == Command start (output follows) =====}}.

\sphinxAtStartPar
This is displaying errors that the
terminal command threw: The \sphinxcode{\sphinxupquote{convert}} tool complains that it cannot open
the file, because there is “No such file or directory”.

\sphinxAtStartPar
The second part starts after

\sphinxAtStartPar
\sphinxcode{\sphinxupquote{{[}INFO   {]} == Command exit (modification check follows) =====}}.

\sphinxAtStartPar
DataLad adds information about a “non\sphinxhyphen{}zero exit code”. A non\sphinxhyphen{}zero exit code indicates
that something went wrong%
\begin{footnote}\sphinxAtStartFootnote
In shell programming, commands exit with a specific code that indicates
whether they failed, and if so, how. Successful commands have the exit code zero. All failures
have exit codes greater than zero.
%
\end{footnote}. In principle, you could go ahead and google what this
specific exit status indicates. However, the solution might have already occurred to you when
reading the first error report: The file is not present.

\sphinxAtStartPar
How can that be?

\sphinxAtStartPar
“Right!”, you exclaim with a facepalm.
Just as the \sphinxcode{\sphinxupquote{.mp3}} files, the \sphinxcode{\sphinxupquote{.jpg}} file content is not present
locally after a \sphinxcode{\sphinxupquote{datalad clone}}, and we did not \sphinxcode{\sphinxupquote{datalad get}} it yet!

\index{declare command input@\spxentry{declare command input}!with DataLad run@\spxentry{with DataLad run}}\index{with DataLad run@\spxentry{with DataLad run}!declare command input@\spxentry{declare command input}}\ignorespaces 
\sphinxAtStartPar
This is where the \sphinxcode{\sphinxupquote{\sphinxhyphen{}i}}/\sphinxcode{\sphinxupquote{\sphinxhyphen{}\sphinxhyphen{}input}} option for a \sphinxcode{\sphinxupquote{datalad run}} becomes useful.
The content of everything that is specified as an \sphinxcode{\sphinxupquote{input}} will be retrieved
prior to running the command.

\fvset{hllines={, 8,}}%
\begin{sphinxVerbatim}[commandchars=\\\{\}]
\PYG{g+gp}{\PYGZdl{} }datalad\PYG{+w}{ }run\PYG{+w}{ }\PYGZhy{}m\PYG{+w}{ }\PYG{l+s+s2}{\PYGZdq{}Resize logo for slides\PYGZdq{}}\PYG{+w}{ }\PYG{l+s+se}{\PYGZbs{}}
\PYGZhy{}\PYGZhy{}input\PYG{+w}{ }\PYG{l+s+s2}{\PYGZdq{}recordings/longnow/.datalad/feed\PYGZus{}metadata/logo\PYGZus{}salt.jpg\PYGZdq{}}\PYG{+w}{ }\PYG{l+s+se}{\PYGZbs{}}
\PYG{l+s+s2}{\PYGZdq{}convert \PYGZhy{}resize 400x400 recordings/longnow/.datalad/feed\PYGZus{}metadata/logo\PYGZus{}salt.jpg recordings/salt\PYGZus{}logo\PYGZus{}small.jpg\PYGZdq{}}
\PYG{g+gp}{\PYGZdl{} }\PYG{c+c1}{\PYGZsh{} or shorter:}
\PYG{g+gp}{\PYGZdl{} }datalad\PYG{+w}{ }run\PYG{+w}{ }\PYGZhy{}m\PYG{+w}{ }\PYG{l+s+s2}{\PYGZdq{}Resize logo for slides\PYGZdq{}}\PYG{+w}{ }\PYG{l+s+se}{\PYGZbs{}}
\PYGZhy{}i\PYG{+w}{ }\PYG{l+s+s2}{\PYGZdq{}recordings/longnow/.datalad/feed\PYGZus{}metadata/logo\PYGZus{}salt.jpg\PYGZdq{}}\PYG{+w}{ }\PYG{l+s+se}{\PYGZbs{}}
\PYG{l+s+s2}{\PYGZdq{}convert \PYGZhy{}resize 400x400 recordings/longnow/.datalad/feed\PYGZus{}metadata/logo\PYGZus{}salt.jpg recordings/salt\PYGZus{}logo\PYGZus{}small.jpg\PYGZdq{}}
\PYG{g+go}{get(ok): recordings/longnow/.datalad/feed\PYGZus{}metadata/logo\PYGZus{}salt.jpg (file) [from web...]}
\PYG{g+go}{[INFO] == Command start (output follows) =====}
\PYG{g+go}{[INFO] == Command exit (modification check follows) =====}
\PYG{g+go}{run(ok): /home/me/dl\PYGZhy{}101/DataLad\PYGZhy{}101 (dataset) [convert \PYGZhy{}resize 400x400 recordings/longn...]}
\PYG{g+go}{add(ok): recordings/salt\PYGZus{}logo\PYGZus{}small.jpg (file)}
\PYG{g+go}{save(ok): . (dataset)}
\end{sphinxVerbatim}
\sphinxresetverbatimhllines

\sphinxAtStartPar
Cool! You can see in this output that prior to the data command execution, DataLad did a \sphinxcode{\sphinxupquote{datalad get}}.
This is useful for several reasons. For one, it saved us the work of manually
getting content. But moreover, this is useful for anyone with whom we might share the
dataset: With an installed dataset one can very simply rerun \sphinxcode{\sphinxupquote{datalad run}} commands
if they have the input argument appropriately specified. It is therefore good practice to
specify the inputs appropriately. Remember from section {\hyperref[\detokenize{basics/101-105-install:installds}]{\sphinxcrossref{\DUrole{std,std-ref}{Install datasets}}}} (\autopageref*{\detokenize{basics/101-105-install:installds}})
that \sphinxcode{\sphinxupquote{datalad get}} will only retrieve content if
it is not yet present, all input already downloaded will not be downloaded again \textendash{} so
specifying inputs even though they are already present will not do any harm.
The \textit{Find-out-more}~{\findoutmoreiconinline}\textit{\ref{fom-run-multiple-inputs}} {\hyperref[\detokenize{basics/101-110-run2:fom-run-multiple-inputs}]{\sphinxcrossref{\DUrole{std,std-ref}{on using more than one input}}}} (\autopageref*{\detokenize{basics/101-110-run2:fom-run-multiple-inputs}}) provides
some more details.

\index{path globbing@\spxentry{path globbing}!with DataLad run@\spxentry{with DataLad run}}\index{with DataLad run@\spxentry{with DataLad run}!path globbing@\spxentry{path globbing}}\ignorespaces \begin{findoutmore}[label={fom-run-multiple-inputs}, before title={\thetcbcounter\ }, float, floatplacement=tb, check odd page=true]{What if there are several inputs?}
\label{\detokenize{basics/101-110-run2:fom-run-multiple-inputs}}

\sphinxAtStartPar
Often, a command needs several inputs. In principle, every input (which could be files, directories, or subdatasets) gets its own \sphinxcode{\sphinxupquote{\sphinxhyphen{}i}}/\sphinxcode{\sphinxupquote{\sphinxhyphen{}\sphinxhyphen{}input}}
flag. However, you can make use of {\hyperref[\detokenize{glossary:term-globbing}]{\sphinxtermref{\DUrole{xref,std,std-term}{globbing}}}}. For example,

\begin{sphinxVerbatim}[commandchars=\\\{\}]
\PYG{g+gp}{\PYGZdl{} }datalad\PYG{+w}{ }run\PYG{+w}{ }\PYGZhy{}\PYGZhy{}input\PYG{+w}{ }\PYG{l+s+s2}{\PYGZdq{}*.jpg\PYGZdq{}}\PYG{+w}{ }\PYG{l+s+s2}{\PYGZdq{}COMMAND\PYGZdq{}}
\end{sphinxVerbatim}

\sphinxAtStartPar
will retrieve all \sphinxcode{\sphinxupquote{.jpg}} files prior to command execution.


\end{findoutmore}


\subsection{If outputs already exist…}
\label{\detokenize{basics/101-110-run2:if-outputs-already-exist}}
\index{files are unlocked by default@\spxentry{files are unlocked by default}!on Windows@\spxentry{on Windows}}\index{on Windows@\spxentry{on Windows}!files are unlocked by default@\spxentry{files are unlocked by default}}\index{unlocked files@\spxentry{unlocked files}!in adjusted mode@\spxentry{in adjusted mode}}\index{in adjusted mode@\spxentry{in adjusted mode}!unlocked files@\spxentry{unlocked files}}\ignorespaces \begin{windowswit}[label={fom-rerun-unlocked}, before title={\thetcbcounter\ }, float, floatplacement=tb, check odd page=true]{Good news! Here is something that is easier on Windows}
\label{\detokenize{basics/101-110-run2:fom-rerun-unlocked}}

\sphinxAtStartPar
The section below describes something that is very confusing for people that have just started with DataLad: Some files in a dataset can’t be modified, and if one tries, it results in a “permission denied” error.
Why is that?
The remainder of this section and the upcoming chapter {\hyperref[\detokenize{basics/basics-annex:chapter-gitannex}]{\sphinxcrossref{\DUrole{std,std-ref}{Under the hood: git\sphinxhyphen{}annex}}}} (\autopageref*{\detokenize{basics/basics-annex:chapter-gitannex}}) contain a procedural explanation.
However: This doesn’t happen on Windows.
The “unlocking” that is necessary on almost all other systems to modify a file is already done on Windows.
Thus, all files in your dataset will be readily modifiable, sparing you the need to adjust to the unexpected behavior that is described below.
While it is easier, it isn’t a “more useful” behavior, though.
A different Windows Wit in the next chapter will highlight how it rather is a suboptimal workaround.

\sphinxAtStartPar
Please don’t skip the next section \textendash{} it is useful to know how datasets behave on other systems.
Just be mindful that you will not encounter the errors that the book displays next.
And while this all sounds quite cryptic and vague, an upcoming Windows Wit will provide more information.


\end{windowswit}

\sphinxAtStartPar
Looking at the resulting image, you wonder whether 400x400 might be a tiny bit to small.
Maybe we should try to resize it to 450x450, and see whether that looks better?

\sphinxAtStartPar
Note that we cannot use a \sphinxcode{\sphinxupquote{datalad rerun}} for this: if we want to change the dimension option
in the command, we have to define a new \sphinxcode{\sphinxupquote{datalad run}} command.
If you would benefit from an extra bit of motivation, check the \textit{Windows-wit}~{\windowswiticoninline}\textit{\ref{fom-rerun-unlocked}} {\hyperref[\detokenize{basics/101-110-run2:fom-rerun-unlocked}]{\sphinxcrossref{\DUrole{std,std-ref}{on why what we will do now is, for once, simpler on Windows}}}} (\autopageref*{\detokenize{basics/101-110-run2:fom-rerun-unlocked}}).

\sphinxAtStartPar
To establish best\sphinxhyphen{}practices, let’s specify the input even though it is already present:

\fvset{hllines={, 9,}}%
\begin{sphinxVerbatim}[commandchars=\\\{\}]
\PYG{g+gp}{\PYGZdl{} }datalad\PYG{+w}{ }run\PYG{+w}{ }\PYGZhy{}m\PYG{+w}{ }\PYG{l+s+s2}{\PYGZdq{}Resize logo for slides\PYGZdq{}}\PYG{+w}{ }\PYG{l+s+se}{\PYGZbs{}}
\PYGZhy{}\PYGZhy{}input\PYG{+w}{ }\PYG{l+s+s2}{\PYGZdq{}recordings/longnow/.datalad/feed\PYGZus{}metadata/logo\PYGZus{}salt.jpg\PYGZdq{}}\PYG{+w}{ }\PYG{l+s+se}{\PYGZbs{}}
\PYG{l+s+s2}{\PYGZdq{}convert \PYGZhy{}resize 450x450 recordings/longnow/.datalad/feed\PYGZus{}metadata/logo\PYGZus{}salt.jpg recordings/salt\PYGZus{}logo\PYGZus{}small.jpg\PYGZdq{}}
\PYG{g+gp}{\PYGZdl{} }\PYG{c+c1}{\PYGZsh{} or shorter:}
\PYG{g+gp}{\PYGZdl{} }datalad\PYG{+w}{ }run\PYG{+w}{ }\PYGZhy{}m\PYG{+w}{ }\PYG{l+s+s2}{\PYGZdq{}Resize logo for slides\PYGZdq{}}\PYG{+w}{ }\PYG{l+s+se}{\PYGZbs{}}
\PYGZhy{}i\PYG{+w}{ }\PYG{l+s+s2}{\PYGZdq{}recordings/longnow/.datalad/feed\PYGZus{}metadata/logo\PYGZus{}salt.jpg\PYGZdq{}}\PYG{+w}{ }\PYG{l+s+se}{\PYGZbs{}}
\PYG{l+s+s2}{\PYGZdq{}convert \PYGZhy{}resize 450x450 recordings/longnow/.datalad/feed\PYGZus{}metadata/logo\PYGZus{}salt.jpg recordings/salt\PYGZus{}logo\PYGZus{}small.jpg\PYGZdq{}}
\PYG{g+go}{[INFO] == Command start (output follows) =====}
\PYG{g+go}{convert convert: Unable to open file (recordings/salt\PYGZus{}logo\PYGZus{}small.jpg) [Permission denied].}
\PYG{g+go}{[INFO] == Command exit (modification check follows) =====}
\PYG{g+go}{[INFO] The command had a non\PYGZhy{}zero exit code. If this is expected, you can save the changes with \PYGZsq{}datalad save \PYGZhy{}d . \PYGZhy{}r \PYGZhy{}F .git/COMMIT\PYGZus{}EDITMSG\PYGZsq{}}
\PYG{g+go}{run(error): /home/me/dl\PYGZhy{}101/DataLad\PYGZhy{}101 (dataset) [convert \PYGZhy{}resize 450x450 recordings/longn...]}
\end{sphinxVerbatim}
\sphinxresetverbatimhllines

\sphinxAtStartPar
\sphinxstylestrong{Oh wtf}… \sphinxstyleemphasis{What is it now?}

\sphinxAtStartPar
A quick glimpse into the error message shows a different error than before:
The tool complains that it is “unable to open” the image, because the “Permission {[}is{]} denied”.

\sphinxAtStartPar
We have not seen anything like this before, and we need to turn to our lecturer for help.
Confused about what we might have
done wrong, we raise our hand to ask the instructor.
Knowingly, she smiles, and tells you about how DataLad protects content given
to it:

\sphinxAtStartPar
“Content in your DataLad dataset is protected by {\hyperref[\detokenize{glossary:term-git-annex}]{\sphinxtermref{\DUrole{xref,std,std-term}{git\sphinxhyphen{}annex}}}} from
accidental changes” our instructor begins.

\sphinxAtStartPar
“Wait!” we interrupt. “First off, that wasn’t accidental. And second, I was told this
course does not have \sphinxcode{\sphinxupquote{git\sphinxhyphen{}annex\sphinxhyphen{}101}} as a prerequisite?”

\sphinxAtStartPar
“Yes, hear me out” she says. “I promise you two different solutions at
the end of this explanation, and the concept behind this is quite relevant”.

\sphinxAtStartPar
DataLad usually gives content to {\hyperref[\detokenize{glossary:term-git-annex}]{\sphinxtermref{\DUrole{xref,std,std-term}{git\sphinxhyphen{}annex}}}} to store and track.
git\sphinxhyphen{}annex, let’s just say, takes this task \sphinxstyleemphasis{really} seriously. One of its
features that you have just experienced is that it \sphinxstyleemphasis{locks} content.

\sphinxAtStartPar
If files are \sphinxstyleemphasis{locked down}, their content cannot be modified. In principle,
that’s not a bad thing: It could be your late grandma’s secret cherry\sphinxhyphen{}pie
recipe, and you do not want to \sphinxstyleemphasis{accidentally} change that.
Therefore, a file needs to be consciously \sphinxstyleemphasis{unlocked} to apply modifications.

\sphinxAtStartPar
In the attempt to resize the image to 450x450 you tried to overwrite
\sphinxcode{\sphinxupquote{recordings/salt\_logo\_small.jpg}}, a file that was given to DataLad
and thus protected by git\sphinxhyphen{}annex.

\index{unlock@\spxentry{unlock}!DataLad command@\spxentry{DataLad command}}\index{DataLad command@\spxentry{DataLad command}!unlock@\spxentry{unlock}}\index{unlock file@\spxentry{unlock file}!with DataLad@\spxentry{with DataLad}}\index{with DataLad@\spxentry{with DataLad}!unlock file@\spxentry{unlock file}}\ignorespaces 
\sphinxAtStartPar
There is a DataLad command that takes care of unlocking file content,
and thus making locked files modifiable again: \sphinxcode{\sphinxupquote{datalad unlock}}.
Let us check out what it does, and also compare that to the
\textit{Windows-wit}~{\windowswiticoninline}\textit{\ref{ww-unlocking-unlocked}} {\hyperref[\detokenize{basics/101-110-run2:ww-unlocking-unlocked}]{\sphinxcrossref{\DUrole{std,std-ref}{on an entirely different explanation}}}} (\autopageref*{\detokenize{basics/101-110-run2:ww-unlocking-unlocked}}).

\index{files are unlocked by default@\spxentry{files are unlocked by default}!on Windows@\spxentry{on Windows}}\index{on Windows@\spxentry{on Windows}!files are unlocked by default@\spxentry{files are unlocked by default}}\index{adjusted branch@\spxentry{adjusted branch}!unlocked files@\spxentry{unlocked files}}\ignorespaces \begin{windowswit}[label={ww-unlocking-unlocked}, before title={\thetcbcounter\ }, float, floatplacement=tb, check odd page=true]{What happens if I run ‘datalad unlock’ on Windows?}
\label{\detokenize{basics/101-110-run2:ww-unlocking-unlocked}}

\sphinxAtStartPar
Nothing. All of the files in your dataset are always unlocked, and actually \sphinxstyleemphasis{cannot} be locked at all.
Consequently, there will be nothing to show for \sphinxcode{\sphinxupquote{datalad status}} afterwards (as shown a few paragraphs below).
This is due to a file system limitation, and will be explained in more detail in chapter {\hyperref[\detokenize{basics/basics-annex:chapter-gitannex}]{\sphinxcrossref{\DUrole{std,std-ref}{Under the hood: git\sphinxhyphen{}annex}}}} (\autopageref*{\detokenize{basics/basics-annex:chapter-gitannex}}).


\end{windowswit}

\begin{sphinxVerbatim}[commandchars=\\\{\}]
\PYG{g+gp}{\PYGZdl{} }datalad\PYG{+w}{ }unlock\PYG{+w}{ }recordings/salt\PYGZus{}logo\PYGZus{}small.jpg
\PYG{g+go}{unlock(ok): recordings/salt\PYGZus{}logo\PYGZus{}small.jpg (file)}
\end{sphinxVerbatim}

\sphinxAtStartPar
Well, \sphinxcode{\sphinxupquote{unlock(ok)}} does not sound too bad for a start. As always, we
feel the urge to run a \sphinxcode{\sphinxupquote{datalad status}} on this:

\begin{sphinxVerbatim}[commandchars=\\\{\}]
\PYG{g+gp}{\PYGZdl{} }datalad\PYG{+w}{ }status
\PYG{g+go}{ modified: recordings/salt\PYGZus{}logo\PYGZus{}small.jpg (file)}
\end{sphinxVerbatim}

\sphinxAtStartPar
“Ah, do not mind that for now”, our instructor says, and with a wink she
continues: “We’ll talk about symlinks and object trees a while later”.
You are not really sure whether that’s a good thing, but you have a task to focus
on. Hastily, you run the command right from the terminal:

\begin{sphinxVerbatim}[commandchars=\\\{\}]
\PYG{g+gp}{\PYGZdl{} }convert\PYG{+w}{ }\PYGZhy{}resize\PYG{+w}{ }450x450\PYG{+w}{ }recordings/longnow/.datalad/feed\PYGZus{}metadata/logo\PYGZus{}salt.jpg\PYG{+w}{ }recordings/salt\PYGZus{}logo\PYGZus{}small.jpg
\end{sphinxVerbatim}

\sphinxAtStartPar
Hey, no permission denied error! You note that the instructor still stands
right next to you. “Sooo… now what do I do to \sphinxstyleemphasis{lock} the file again?” you ask.

\sphinxAtStartPar
“Well… what you just did there was quite suboptimal. Didn’t you want to
use \sphinxcode{\sphinxupquote{datalad run}}? But, anyway, in order to lock the file again, you would need to
run a \sphinxcode{\sphinxupquote{datalad save}}.”

\begin{sphinxVerbatim}[commandchars=\\\{\}]
\PYG{g+gp}{\PYGZdl{} }datalad\PYG{+w}{ }save\PYG{+w}{ }\PYGZhy{}m\PYG{+w}{ }\PYG{l+s+s2}{\PYGZdq{}resized picture by hand\PYGZdq{}}
\PYG{g+go}{add(ok): recordings/salt\PYGZus{}logo\PYGZus{}small.jpg (file)}
\PYG{g+go}{save(ok): . (dataset)}
\end{sphinxVerbatim}

\sphinxAtStartPar
“So”, you wonder aloud, “whenever I want to modify I need to
\sphinxcode{\sphinxupquote{datalad unlock}} it, do the modifications, and then \sphinxcode{\sphinxupquote{datalad save}} it?”

\sphinxAtStartPar
“Well, this is certainly one way of doing it, and a completely valid workflow
if you would do that outside of a \sphinxcode{\sphinxupquote{datalad run}} command.
But within \sphinxcode{\sphinxupquote{datalad run}} there is actually a much easier way of doing this.
Let’s use the \sphinxcode{\sphinxupquote{\sphinxhyphen{}\sphinxhyphen{}output}} argument.”

\sphinxAtStartPar
\sphinxcode{\sphinxupquote{datalad run}} \sphinxstyleemphasis{retrieves} everything that is specified as \sphinxcode{\sphinxupquote{\sphinxhyphen{}\sphinxhyphen{}input}} prior to
command execution, and it \sphinxstyleemphasis{unlocks} everything specified as \sphinxcode{\sphinxupquote{\sphinxhyphen{}\sphinxhyphen{}output}} prior to
command execution. Therefore, whenever the output of a \sphinxcode{\sphinxupquote{datalad run}} command already
exists and is tracked, it should be specified as an argument in
the \sphinxcode{\sphinxupquote{\sphinxhyphen{}o}}/\sphinxcode{\sphinxupquote{\sphinxhyphen{}\sphinxhyphen{}output}} option. See \textit{Find-out-more}~{\findoutmoreiconinline}\textit{\ref{fom-output-glob}} {\hyperref[\detokenize{basics/101-110-run2:fom-output-glob}]{\sphinxcrossref{\DUrole{std,std-ref}{on dealing with many
outputs}}}} (\autopageref*{\detokenize{basics/101-110-run2:fom-output-glob}}).

\index{path globbing@\spxentry{path globbing}!with DataLad run@\spxentry{with DataLad run}}\index{with DataLad run@\spxentry{with DataLad run}!path globbing@\spxentry{path globbing}}\ignorespaces \begin{findoutmore}[label={fom-output-glob}, before title={\thetcbcounter\ }, float, floatplacement=tb, check odd page=true]{But what if I have a lot of outputs?}
\label{\detokenize{basics/101-110-run2:fom-output-glob}}

\sphinxAtStartPar
The use case here is simplistic \textendash{} a single file gets modified.
But there are commands and tools that create full directories with
many files as an output.
The easiest way to specify this type of output
is by supplying the directory name, or the directory name and a {\hyperref[\detokenize{glossary:term-globbing}]{\sphinxtermref{\DUrole{xref,std,std-term}{globbing}}}} character, such as
\sphinxcode{\sphinxupquote{\sphinxhyphen{}o directory/*.dat}}.
This would unlock all files with a \sphinxcode{\sphinxupquote{.dat}} extension inside of \sphinxcode{\sphinxupquote{directory}}.
To glob for files in multiple levels of directories, use \sphinxcode{\sphinxupquote{**}} (a so\sphinxhyphen{}called \dlhbhref{L2A}{globstar}) for a recursive glob through any number directories.
And, just as for \sphinxcode{\sphinxupquote{\sphinxhyphen{}i}}/\sphinxcode{\sphinxupquote{\sphinxhyphen{}\sphinxhyphen{}input}}, you could use multiple \sphinxcode{\sphinxupquote{\sphinxhyphen{}\sphinxhyphen{}output}} specifications.


\end{findoutmore}

\index{declare command output@\spxentry{declare command output}!with DataLad run@\spxentry{with DataLad run}}\index{with DataLad run@\spxentry{with DataLad run}!declare command output@\spxentry{declare command output}}\ignorespaces 
\sphinxAtStartPar
In order to execute \sphinxcode{\sphinxupquote{datalad run}} with both the \sphinxcode{\sphinxupquote{\sphinxhyphen{}i}}/\sphinxcode{\sphinxupquote{\sphinxhyphen{}\sphinxhyphen{}input}} and \sphinxcode{\sphinxupquote{\sphinxhyphen{}o}}/\sphinxcode{\sphinxupquote{\sphinxhyphen{}\sphinxhyphen{}output}}
flag and see their magic, let’s crop the second logo, \sphinxcode{\sphinxupquote{logo\_interval.jpg}}.
The \textit{Windows-wit}~{\windowswiticoninline}\textit{\ref{ww-run-outputs}} {\hyperref[\detokenize{basics/101-110-run2:ww-run-outputs}]{\sphinxcrossref{\DUrole{std,std-ref}{on output declaration}}}} (\autopageref*{\detokenize{basics/101-110-run2:ww-run-outputs}}) clarifies the impact
of not doing so, when it is technically not needed.

\index{files are unlocked by default@\spxentry{files are unlocked by default}!on Windows@\spxentry{on Windows}}\index{on Windows@\spxentry{on Windows}!files are unlocked by default@\spxentry{files are unlocked by default}}\index{run@\spxentry{run}!DataLad command@\spxentry{DataLad command}}\index{DataLad command@\spxentry{DataLad command}!run@\spxentry{run}}\index{unlocked files@\spxentry{unlocked files}!in adjusted mode@\spxentry{in adjusted mode}}\index{in adjusted mode@\spxentry{in adjusted mode}!unlocked files@\spxentry{unlocked files}}\ignorespaces \begin{windowswit}[label={ww-run-outputs}, before title={\thetcbcounter\ }, float, floatplacement=tb, check odd page=true]{Wait, would I need to specify outputs, too?}
\label{\detokenize{basics/101-110-run2:ww-run-outputs}}

\sphinxAtStartPar
Given that nothing in your dataset is locked, is there a \sphinxstyleemphasis{need} for you to bother with creating \sphinxcode{\sphinxupquote{\sphinxhyphen{}\sphinxhyphen{}output}} flags?
Not for you personally, if you only stay on your Windows machine.
However, you will be doing others that you share your dataset with a favor if they are not using Windows \textendash{} should you or others want to rerun a run record, \sphinxcode{\sphinxupquote{\sphinxhyphen{}\sphinxhyphen{}output}} flags will make it work on all operating systems.


\end{windowswit}

\fvset{hllines={, 11,}}%
\begin{sphinxVerbatim}[commandchars=\\\{\}]
\PYG{g+gp}{\PYGZdl{} }datalad\PYG{+w}{ }run\PYG{+w}{ }\PYGZhy{}m\PYG{+w}{ }\PYG{l+s+s2}{\PYGZdq{}Resize logo for slides\PYGZdq{}}\PYG{+w}{ }\PYG{l+s+se}{\PYGZbs{}}
\PYGZhy{}\PYGZhy{}input\PYG{+w}{ }\PYG{l+s+s2}{\PYGZdq{}recordings/longnow/.datalad/feed\PYGZus{}metadata/logo\PYGZus{}interval.jpg\PYGZdq{}}\PYG{+w}{ }\PYG{l+s+se}{\PYGZbs{}}
\PYGZhy{}\PYGZhy{}output\PYG{+w}{ }\PYG{l+s+s2}{\PYGZdq{}recordings/interval\PYGZus{}logo\PYGZus{}small.jpg\PYGZdq{}}\PYG{+w}{ }\PYG{l+s+se}{\PYGZbs{}}
\PYG{l+s+s2}{\PYGZdq{}convert \PYGZhy{}resize 450x450 recordings/longnow/.datalad/feed\PYGZus{}metadata/logo\PYGZus{}interval.jpg recordings/interval\PYGZus{}logo\PYGZus{}small.jpg\PYGZdq{}}

\PYG{g+gp}{\PYGZdl{} }\PYG{c+c1}{\PYGZsh{} or shorter:}
\PYG{g+gp}{\PYGZdl{} }datalad\PYG{+w}{ }run\PYG{+w}{ }\PYGZhy{}m\PYG{+w}{ }\PYG{l+s+s2}{\PYGZdq{}Resize logo for slides\PYGZdq{}}\PYG{+w}{ }\PYG{l+s+se}{\PYGZbs{}}
\PYGZhy{}i\PYG{+w}{ }\PYG{l+s+s2}{\PYGZdq{}recordings/longnow/.datalad/feed\PYGZus{}metadata/logo\PYGZus{}interval.jpg\PYGZdq{}}\PYG{+w}{ }\PYG{l+s+se}{\PYGZbs{}}
\PYGZhy{}o\PYG{+w}{ }\PYG{l+s+s2}{\PYGZdq{}recordings/interval\PYGZus{}logo\PYGZus{}small.jpg\PYGZdq{}}\PYG{+w}{ }\PYG{l+s+se}{\PYGZbs{}}
\PYG{l+s+s2}{\PYGZdq{}convert \PYGZhy{}resize 450x450 recordings/longnow/.datalad/feed\PYGZus{}metadata/logo\PYGZus{}interval.jpg recordings/interval\PYGZus{}logo\PYGZus{}small.jpg\PYGZdq{}}
\PYG{g+go}{get(ok): recordings/longnow/.datalad/feed\PYGZus{}metadata/logo\PYGZus{}interval.jpg (file) [from web...]}
\PYG{g+go}{[INFO] == Command start (output follows) =====}
\PYG{g+go}{[INFO] == Command exit (modification check follows) =====}
\PYG{g+go}{run(ok): /home/me/dl\PYGZhy{}101/DataLad\PYGZhy{}101 (dataset) [convert \PYGZhy{}resize 450x450 recordings/longn...]}
\PYG{g+go}{add(ok): recordings/interval\PYGZus{}logo\PYGZus{}small.jpg (file)}
\PYG{g+go}{save(ok): . (dataset)}
\end{sphinxVerbatim}
\sphinxresetverbatimhllines

\sphinxAtStartPar
This time, with both \sphinxcode{\sphinxupquote{\sphinxhyphen{}\sphinxhyphen{}input}} and \sphinxcode{\sphinxupquote{\sphinxhyphen{}\sphinxhyphen{}output}}
options specified, DataLad informs about the \sphinxcode{\sphinxupquote{datalad get}}
operations it performs prior to the command
execution, and \sphinxcode{\sphinxupquote{datalad run}} executes the command successfully.
It does \sphinxstyleemphasis{not} inform about any \sphinxcode{\sphinxupquote{datalad unlock}} operation,
because the output \sphinxcode{\sphinxupquote{recordings/interval\_logo\_small.jpg}} does not
exist before the command is run. Should you rerun this command however,
the summary will include a statement about content unlocking. You will
see an example of this in the next section.

\sphinxAtStartPar
Note now how many individual commands a \sphinxcode{\sphinxupquote{datalad run}} saves us:
\sphinxcode{\sphinxupquote{datalad get}}, \sphinxcode{\sphinxupquote{datalad unlock}}, and \sphinxcode{\sphinxupquote{datalad save}}!
But even better: Beyond saving time \sphinxstyleemphasis{now}, running commands reproducibly and
recorded with \sphinxcode{\sphinxupquote{datalad run}} saves us plenty of time in the future as soon
as we want to rerun a command, or find out how a file came into existence.

\sphinxAtStartPar
With this last code snippet, you have experienced a full \sphinxcode{\sphinxupquote{datalad run}} command: commit message,
input and output definitions (the order in which you give those two options is irrelevant),
and the command to be executed. Whenever a command takes input or produces output you should specify
this with the appropriate option.

\sphinxAtStartPar
Make a note of this behavior in your \sphinxcode{\sphinxupquote{notes.txt}} file.

\begin{sphinxVerbatim}[commandchars=\\\{\}]
\PYG{g+gp}{\PYGZdl{} }cat\PYG{+w}{ }\PYGZlt{}\PYGZlt{}\PYG{+w}{ }EOT\PYG{+w}{ }\PYGZgt{}\PYGZgt{}\PYG{+w}{ }notes.txt
\PYG{g+go}{You should specify all files that a command takes as input with an}
\PYG{g+go}{\PYGZhy{}i/\PYGZhy{}\PYGZhy{}input flag. These files will be retrieved prior to the command}
\PYG{g+go}{execution. Any content that is modified or produced by the command}
\PYG{g+go}{should be specified with an \PYGZhy{}o/\PYGZhy{}\PYGZhy{}output flag. Upon a run or rerun of}
\PYG{g+go}{the command, the contents of these files will get unlocked so that}
\PYG{g+go}{they can be modified.}

\PYG{g+go}{EOT}
\end{sphinxVerbatim}


\subsection{Save yourself the preparation time}
\label{\detokenize{basics/101-110-run2:save-yourself-the-preparation-time}}
\sphinxAtStartPar
Its generally good practice to specify \sphinxcode{\sphinxupquote{\sphinxhyphen{}\sphinxhyphen{}input}} and \sphinxcode{\sphinxupquote{\sphinxhyphen{}\sphinxhyphen{}output}} even if your input files are already retrieved and your output files unlocked \textendash{} it makes sure that a recomputation can succeed, even if inputs are not yet retrieved, or if output needs to be unlocked.
However, the internal preparation steps of checking that inputs exist or that outputs are unlocked can take a bit of time, especially if it involves checking a large number of files.

\sphinxAtStartPar
If you want to avoid the expense of unnecessary preparation steps you can make use of the \sphinxcode{\sphinxupquote{\sphinxhyphen{}\sphinxhyphen{}assume\sphinxhyphen{}ready}} argument of \sphinxcode{\sphinxupquote{datalad run}}.
Depending on whether your inputs are already retrieved, your outputs already unlocked (or not needed to be unlocked), or both, specify \sphinxcode{\sphinxupquote{\sphinxhyphen{}\sphinxhyphen{}assume\sphinxhyphen{}ready}} with the argument \sphinxcode{\sphinxupquote{inputs}}, \sphinxcode{\sphinxupquote{outputs}} or \sphinxcode{\sphinxupquote{both}} and save yourself a few seconds, without sacrificing the ability to rerun your command under conditions in which the preparation would be necessary.


\subsection{Placeholders}
\label{\detokenize{basics/101-110-run2:placeholders}}
\sphinxAtStartPar
Just after writing the note, you had to relax your fingers a bit. “Man, this was
so much typing. Not only did I need to specify the inputs and outputs, I also had
to repeat all of these lengthy paths in the command line call…” you think.

\sphinxAtStartPar
There is a neat little trick to spare you half of this typing effort, though: \sphinxstyleemphasis{Placeholders}
for inputs and outputs. This is how it works:

\sphinxAtStartPar
Instead of running

\begin{sphinxVerbatim}[commandchars=\\\{\}]
\PYG{g+gp}{\PYGZdl{} }datalad\PYG{+w}{ }run\PYG{+w}{ }\PYGZhy{}m\PYG{+w}{ }\PYG{l+s+s2}{\PYGZdq{}Resize logo for slides\PYGZdq{}}\PYG{+w}{ }\PYG{l+s+se}{\PYGZbs{}}
\PYGZhy{}\PYGZhy{}input\PYG{+w}{ }\PYG{l+s+s2}{\PYGZdq{}recordings/longnow/.datalad/feed\PYGZus{}metadata/logo\PYGZus{}interval.jpg\PYGZdq{}}\PYG{+w}{ }\PYG{l+s+se}{\PYGZbs{}}
\PYGZhy{}\PYGZhy{}output\PYG{+w}{ }\PYG{l+s+s2}{\PYGZdq{}recordings/interval\PYGZus{}logo\PYGZus{}small.jpg\PYGZdq{}}\PYG{+w}{ }\PYG{l+s+se}{\PYGZbs{}}
\PYG{l+s+s2}{\PYGZdq{}convert \PYGZhy{}resize 450x450 recordings/longnow/.datalad/feed\PYGZus{}metadata/logo\PYGZus{}interval.jpg recordings/interval\PYGZus{}logo\PYGZus{}small.jpg\PYGZdq{}}
\end{sphinxVerbatim}

\sphinxAtStartPar
you could shorten this to

\fvset{hllines={, 4,}}%
\begin{sphinxVerbatim}[commandchars=\\\{\}]
\PYG{g+gp}{\PYGZdl{} }datalad\PYG{+w}{ }run\PYG{+w}{ }\PYGZhy{}m\PYG{+w}{ }\PYG{l+s+s2}{\PYGZdq{}Resize logo for slides\PYGZdq{}}\PYG{+w}{ }\PYG{l+s+se}{\PYGZbs{}}
\PYGZhy{}\PYGZhy{}input\PYG{+w}{ }\PYG{l+s+s2}{\PYGZdq{}recordings/longnow/.datalad/feed\PYGZus{}metadata/logo\PYGZus{}interval.jpg\PYGZdq{}}\PYG{+w}{ }\PYG{l+s+se}{\PYGZbs{}}
\PYGZhy{}\PYGZhy{}output\PYG{+w}{ }\PYG{l+s+s2}{\PYGZdq{}recordings/interval\PYGZus{}logo\PYGZus{}small.jpg\PYGZdq{}}\PYG{+w}{ }\PYG{l+s+se}{\PYGZbs{}}
\PYG{l+s+s2}{\PYGZdq{}convert \PYGZhy{}resize 450x450 \PYGZob{}inputs\PYGZcb{} \PYGZob{}outputs\PYGZcb{}\PYGZdq{}}
\end{sphinxVerbatim}
\sphinxresetverbatimhllines

\sphinxAtStartPar
The placeholder \sphinxcode{\sphinxupquote{\{inputs\}}} will expand to the path given as \sphinxcode{\sphinxupquote{\sphinxhyphen{}\sphinxhyphen{}input}}, and
the placeholder \sphinxcode{\sphinxupquote{\{outputs\}}} will expand to the path given as \sphinxcode{\sphinxupquote{\sphinxhyphen{}\sphinxhyphen{}output}}.
This means instead of writing the full paths in the command, you can simply reuse
the \sphinxcode{\sphinxupquote{\sphinxhyphen{}\sphinxhyphen{}input}} and \sphinxcode{\sphinxupquote{\sphinxhyphen{}\sphinxhyphen{}output}} specification done before.
The \textit{Find-out-more}~{\findoutmoreiconinline}\textit{\ref{fom-multiple-inoutputs}} {\hyperref[\detokenize{basics/101-110-run2:fom-multiple-inoutputs}]{\sphinxcrossref{\DUrole{std,std-ref}{on dealing with multiple inputs and outputs}}}} (\autopageref*{\detokenize{basics/101-110-run2:fom-multiple-inoutputs}}) provides more details on the general usage of this feature.

\index{multiple command inputs@\spxentry{multiple command inputs}!with DataLad run@\spxentry{with DataLad run}}\index{with DataLad run@\spxentry{with DataLad run}!multiple command inputs@\spxentry{multiple command inputs}}\ignorespaces \begin{findoutmore}[label={fom-multiple-inoutputs}, before title={\thetcbcounter\ }, float, floatplacement=tb, check odd page=true]{What if I have multiple inputs or outputs?}
\label{\detokenize{basics/101-110-run2:fom-multiple-inoutputs}}

\sphinxAtStartPar
If multiple values are specified, e.g., as in

\begin{sphinxVerbatim}[commandchars=\\\{\}]
\PYG{g+gp}{\PYGZdl{} }datalad\PYG{+w}{ }run\PYG{+w}{ }\PYGZhy{}m\PYG{+w}{ }\PYG{l+s+s2}{\PYGZdq{}move a few files around\PYGZdq{}}\PYG{+w}{ }\PYG{l+s+se}{\PYGZbs{}}
\PYGZhy{}\PYGZhy{}input\PYG{+w}{ }\PYG{l+s+s2}{\PYGZdq{}file1\PYGZdq{}}\PYG{+w}{ }\PYGZhy{}\PYGZhy{}input\PYG{+w}{ }\PYG{l+s+s2}{\PYGZdq{}file2\PYGZdq{}}\PYG{+w}{ }\PYGZhy{}\PYGZhy{}input\PYG{+w}{ }\PYG{l+s+s2}{\PYGZdq{}file3\PYGZdq{}}\PYG{+w}{ }\PYG{l+s+se}{\PYGZbs{}}
\PYGZhy{}\PYGZhy{}output\PYG{+w}{ }\PYG{l+s+s2}{\PYGZdq{}directory\PYGZus{}a/\PYGZdq{}}\PYG{+w}{ }\PYG{l+s+se}{\PYGZbs{}}
\PYG{l+s+s2}{\PYGZdq{}mv \PYGZob{}inputs\PYGZcb{} \PYGZob{}outputs\PYGZcb{}\PYGZdq{}}
\end{sphinxVerbatim}

\sphinxAtStartPar
the values will be joined by a space like this:

\begin{sphinxVerbatim}[commandchars=\\\{\}]
\PYG{g+gp}{\PYGZdl{} }datalad\PYG{+w}{ }run\PYG{+w}{ }\PYGZhy{}m\PYG{+w}{ }\PYG{l+s+s2}{\PYGZdq{}move a few files around\PYGZdq{}}\PYG{+w}{ }\PYG{l+s+se}{\PYGZbs{}}
\PYGZhy{}\PYGZhy{}input\PYG{+w}{ }\PYG{l+s+s2}{\PYGZdq{}file1\PYGZdq{}}\PYG{+w}{ }\PYGZhy{}\PYGZhy{}input\PYG{+w}{ }\PYG{l+s+s2}{\PYGZdq{}file2\PYGZdq{}}\PYG{+w}{ }\PYGZhy{}\PYGZhy{}input\PYG{+w}{ }\PYG{l+s+s2}{\PYGZdq{}file3\PYGZdq{}}\PYG{+w}{ }\PYG{l+s+se}{\PYGZbs{}}
\PYGZhy{}\PYGZhy{}output\PYG{+w}{ }\PYG{l+s+s2}{\PYGZdq{}directory\PYGZus{}a/\PYGZdq{}}\PYG{+w}{ }\PYG{l+s+se}{\PYGZbs{}}
\PYG{l+s+s2}{\PYGZdq{}mv file1 file2 file3 directory\PYGZus{}a/\PYGZdq{}}
\end{sphinxVerbatim}

\sphinxAtStartPar
The order of the values will match that order from the command line.

\sphinxAtStartPar
If you use globs for input specification, as in

\begin{sphinxVerbatim}[commandchars=\\\{\}]
\PYG{g+gp}{\PYGZdl{} }datalad\PYG{+w}{ }run\PYG{+w}{ }\PYGZhy{}m\PYG{+w}{ }\PYG{l+s+s2}{\PYGZdq{}move a few files around\PYGZdq{}}\PYG{+w}{ }\PYG{l+s+se}{\PYGZbs{}}
\PYGZhy{}\PYGZhy{}input\PYG{+w}{ }\PYG{l+s+s2}{\PYGZdq{}file*\PYGZdq{}}\PYG{+w}{ }\PYG{l+s+se}{\PYGZbs{}}
\PYGZhy{}\PYGZhy{}output\PYG{+w}{ }\PYG{l+s+s2}{\PYGZdq{}directory\PYGZus{}a/\PYGZdq{}}\PYG{+w}{ }\PYG{l+s+se}{\PYGZbs{}}
\PYG{l+s+s2}{\PYGZdq{}mv \PYGZob{}inputs\PYGZcb{} \PYGZob{}outputs\PYGZcb{}\PYGZdq{}}
\end{sphinxVerbatim}

\sphinxAtStartPar
the globs will expanded in alphabetical order (like bash):

\begin{sphinxVerbatim}[commandchars=\\\{\}]
\PYG{g+gp}{\PYGZdl{} }datalad\PYG{+w}{ }run\PYG{+w}{ }\PYGZhy{}m\PYG{+w}{ }\PYG{l+s+s2}{\PYGZdq{}move a few files around\PYGZdq{}}\PYG{+w}{ }\PYG{l+s+se}{\PYGZbs{}}
\PYGZhy{}\PYGZhy{}input\PYG{+w}{ }\PYG{l+s+s2}{\PYGZdq{}file1\PYGZdq{}}\PYG{+w}{ }\PYGZhy{}\PYGZhy{}input\PYG{+w}{ }\PYG{l+s+s2}{\PYGZdq{}file2\PYGZdq{}}\PYG{+w}{ }\PYGZhy{}\PYGZhy{}input\PYG{+w}{ }\PYG{l+s+s2}{\PYGZdq{}file3\PYGZdq{}}\PYG{+w}{ }\PYG{l+s+se}{\PYGZbs{}}
\PYGZhy{}\PYGZhy{}output\PYG{+w}{ }\PYG{l+s+s2}{\PYGZdq{}directory\PYGZus{}a/\PYGZdq{}}\PYG{+w}{ }\PYG{l+s+se}{\PYGZbs{}}
\PYG{l+s+s2}{\PYGZdq{}mv file1 file2 file3 directory\PYGZus{}a/\PYGZdq{}}
\end{sphinxVerbatim}

\sphinxAtStartPar
If the command only needs a subset of the inputs or outputs, individual values
can be accessed with an integer index, e.g., \sphinxcode{\sphinxupquote{\{inputs{[}0{]}\}}} for the very first
input.

\sphinxAtStartPar
Check the \textit{Find-out-more}~{\findoutmoreiconinline}\textit{\ref{fom-run-curly}} {\hyperref[\detokenize{basics/101-110-run2:fom-run-curly}]{\sphinxcrossref{\DUrole{std,std-ref}{on how to deal with commands that include
\{ or \} for other purposes}}}} (\autopageref*{\detokenize{basics/101-110-run2:fom-run-curly}}).


\end{findoutmore}

\index{run command with curly brackets@\spxentry{run command with curly brackets}!with DataLad run@\spxentry{with DataLad run}}\index{with DataLad run@\spxentry{with DataLad run}!run command with curly brackets@\spxentry{run command with curly brackets}}\ignorespaces \begin{findoutmore}[label={fom-run-curly}, before title={\thetcbcounter\ }, float, floatplacement=tb, check odd page=true]{Wait, what if I need a curly bracket in my ‘datalad run’ call?}
\label{\detokenize{basics/101-110-run2:fom-run-curly}}

\sphinxAtStartPar
If your command call involves a \sphinxcode{\sphinxupquote{\{}} or \sphinxcode{\sphinxupquote{\}}} character, you will need to escape
this brace character by doubling it, i.e., \sphinxcode{\sphinxupquote{\{\{}} or \sphinxcode{\sphinxupquote{\}\}}}.


\end{findoutmore}

\index{dry\sphinxhyphen{}run@\spxentry{dry\sphinxhyphen{}run}!with DataLad run@\spxentry{with DataLad run}}\index{with DataLad run@\spxentry{with DataLad run}!dry\sphinxhyphen{}run@\spxentry{dry\sphinxhyphen{}run}}\ignorespaces 

\subsection{Dry\sphinxhyphen{}running your run call}
\label{\detokenize{basics/101-110-run2:dry-running-your-run-call}}\label{\detokenize{basics/101-110-run2:dryrun}}\label{\detokenize{basics/101-110-run2:index-12}}
\sphinxAtStartPar
\sphinxcode{\sphinxupquote{datalad run}} commands can become confusing and long, especially when you make heavy use of placeholders or wrap a complex bash commands.
To better anticipate what you will be running, or help debug a failed command, you can make use of the \sphinxcode{\sphinxupquote{\sphinxhyphen{}\sphinxhyphen{}dry\sphinxhyphen{}run}}  flag of \sphinxcode{\sphinxupquote{datalad run}}.
This option needs a mode specification (\sphinxcode{\sphinxupquote{\sphinxhyphen{}\sphinxhyphen{}dry\sphinxhyphen{}run=basic}} or \sphinxcode{\sphinxupquote{dry\sphinxhyphen{}run=command}}), followed by the \sphinxcode{\sphinxupquote{run}} command you want to execute, and it will decipher the commands elements:
The mode \sphinxcode{\sphinxupquote{command}} will display the command that is about to be ran.
The mode \sphinxcode{\sphinxupquote{basic}} will report a few important details about the execution:
Apart from displaying the command that will be ran, you will learn \sphinxstyleemphasis{where} the command runs, what its \sphinxstyleemphasis{inputs} are (helpful if your \sphinxcode{\sphinxupquote{\sphinxhyphen{}\sphinxhyphen{}input}} specification includes a {\hyperref[\detokenize{glossary:term-globbing}]{\sphinxtermref{\DUrole{xref,std,std-term}{globbing}}}} term), and what its \sphinxstyleemphasis{outputs} are.

\sphinxstepscope


\section{Clean desk}
\label{\detokenize{basics/101-112-run4:clean-desk}}\label{\detokenize{basics/101-112-run4:run5}}\label{\detokenize{basics/101-112-run4::doc}}
\sphinxAtStartPar
Just now you realize that you need to fit both logos onto the same slide.
“Ah, damn, I might then really need to have them 400 by 400 pixel to fit”,
you think. “Good that I know how to not run into the permission denied errors anymore!”

\sphinxAtStartPar
Therefore, we need to do the \sphinxcode{\sphinxupquote{datalad run}} command yet again \sphinxhyphen{} we wanted to have
the image in 400x400 px size. “Now this definitely will be the last time I’m running this”,
you think.

\fvset{hllines={, 5,}}%
\begin{sphinxVerbatim}[commandchars=\\\{\}]
\PYG{g+gp}{\PYGZdl{} }datalad\PYG{+w}{ }run\PYG{+w}{ }\PYGZhy{}m\PYG{+w}{ }\PYG{l+s+s2}{\PYGZdq{}Resize logo for slides\PYGZdq{}}\PYG{+w}{ }\PYG{l+s+se}{\PYGZbs{}}
\PYGZhy{}\PYGZhy{}input\PYG{+w}{ }\PYG{l+s+s2}{\PYGZdq{}recordings/longnow/.datalad/feed\PYGZus{}metadata/logo\PYGZus{}interval.jpg\PYGZdq{}}\PYG{+w}{ }\PYG{l+s+se}{\PYGZbs{}}
\PYGZhy{}\PYGZhy{}output\PYG{+w}{ }\PYG{l+s+s2}{\PYGZdq{}recordings/interval\PYGZus{}logo\PYGZus{}small.jpg\PYGZdq{}}\PYG{+w}{ }\PYG{l+s+se}{\PYGZbs{}}
\PYG{l+s+s2}{\PYGZdq{}convert \PYGZhy{}resize 400x400 recordings/longnow/.datalad/feed\PYGZus{}metadata/logo\PYGZus{}interval.jpg recordings/interval\PYGZus{}logo\PYGZus{}small.jpg\PYGZdq{}}
\PYG{g+go}{run(impossible): /home/me/dl\PYGZhy{}101/DataLad\PYGZhy{}101 (dataset) [clean dataset required to detect changes from command; use `datalad status` to inspect unsaved changes]}
\end{sphinxVerbatim}
\sphinxresetverbatimhllines


\subsection{Oh for f**** sake… run is “impossible”?}
\label{\detokenize{basics/101-112-run4:oh-for-f-sake-run-is-impossible}}
\sphinxAtStartPar
Weird. After the initial annoyance about yet another error message faded,
and you read on,
DataLad informs that a “clean dataset” is required.
Run a \sphinxcode{\sphinxupquote{datalad status}} to see what is meant by this:

\begin{sphinxVerbatim}[commandchars=\\\{\}]
\PYG{g+gp}{\PYGZdl{} }datalad\PYG{+w}{ }status
\PYG{g+go}{ modified: notes.txt (file)}
\end{sphinxVerbatim}

\sphinxAtStartPar
Ah right. We forgot to save the notes we added, and thus there are
unsaved modifications present in \sphinxcode{\sphinxupquote{DataLad\sphinxhyphen{}101}}.
But why is this a problem?

\sphinxAtStartPar
By default, at the end of a \sphinxcode{\sphinxupquote{datalad run}} is a \sphinxcode{\sphinxupquote{datalad save}}.
Remember the section {\hyperref[\detokenize{basics/101-102-populate:populate}]{\sphinxcrossref{\DUrole{std,std-ref}{Populate a dataset}}}} (\autopageref*{\detokenize{basics/101-102-populate:populate}}): A general \sphinxcode{\sphinxupquote{datalad save}} without
a path specification will save \sphinxstyleemphasis{all} of the modified or untracked
contents to the dataset.

\sphinxAtStartPar
Therefore, in order to not mix any changes in the dataset that are unrelated
to the command plugged into \sphinxcode{\sphinxupquote{datalad run}}, by default it will only run
on a clean dataset with no changes or untracked files present.

\sphinxAtStartPar
There are two ways to get around this error message:
The more obvious \textendash{} and recommended \textendash{} one is to save the modifications,
and run the command in a clean dataset.
We will try this way with the \sphinxcode{\sphinxupquote{logo\_interval.jpg}}.
It would look like this:
First, save the changes,

\begin{sphinxVerbatim}[commandchars=\\\{\}]
\PYG{g+gp}{\PYGZdl{} }datalad\PYG{+w}{ }save\PYG{+w}{ }\PYGZhy{}m\PYG{+w}{ }\PYG{l+s+s2}{\PYGZdq{}add additional notes on run options\PYGZdq{}}
\PYG{g+go}{add(ok): notes.txt (file)}
\PYG{g+go}{save(ok): . (dataset)}
\end{sphinxVerbatim}

\sphinxAtStartPar
and then try again:

\begin{sphinxVerbatim}[commandchars=\\\{\}]
\PYG{g+gp}{\PYGZdl{} }datalad\PYG{+w}{ }run\PYG{+w}{ }\PYGZhy{}m\PYG{+w}{ }\PYG{l+s+s2}{\PYGZdq{}Resize logo for slides\PYGZdq{}}\PYG{+w}{ }\PYG{l+s+se}{\PYGZbs{}}
\PYGZhy{}\PYGZhy{}input\PYG{+w}{ }\PYG{l+s+s2}{\PYGZdq{}recordings/longnow/.datalad/feed\PYGZus{}metadata/logo\PYGZus{}interval.jpg\PYGZdq{}}\PYG{+w}{ }\PYG{l+s+se}{\PYGZbs{}}
\PYGZhy{}\PYGZhy{}output\PYG{+w}{ }\PYG{l+s+s2}{\PYGZdq{}recordings/interval\PYGZus{}logo\PYGZus{}small.jpg\PYGZdq{}}\PYG{+w}{ }\PYG{l+s+se}{\PYGZbs{}}
\PYG{l+s+s2}{\PYGZdq{}convert \PYGZhy{}resize 400x400 recordings/longnow/.datalad/feed\PYGZus{}metadata/logo\PYGZus{}interval.jpg recordings/interval\PYGZus{}logo\PYGZus{}small.jpg\PYGZdq{}}
\PYG{g+go}{unlock(ok): recordings/interval\PYGZus{}logo\PYGZus{}small.jpg (file)}
\PYG{g+go}{[INFO] == Command start (output follows) =====}
\PYG{g+go}{[INFO] == Command exit (modification check follows) =====}
\PYG{g+go}{run(ok): /home/me/dl\PYGZhy{}101/DataLad\PYGZhy{}101 (dataset) [convert \PYGZhy{}resize 400x400 recordings/longn...]}
\PYG{g+go}{add(ok): recordings/interval\PYGZus{}logo\PYGZus{}small.jpg (file)}
\PYG{g+go}{save(ok): . (dataset)}
\end{sphinxVerbatim}

\sphinxAtStartPar
Note how in this execution of \sphinxcode{\sphinxupquote{datalad run}}, output unlocking was actually
necessary and DataLad provides a summary of this action in its output.

\sphinxAtStartPar
Add a quick addition to your notes about this way of cleaning up prior
to a \sphinxcode{\sphinxupquote{datalad run}}:

\begin{sphinxVerbatim}[commandchars=\\\{\}]
\PYG{g+gp}{\PYGZdl{} }cat\PYG{+w}{ }\PYGZlt{}\PYGZlt{}\PYG{+w}{ }EOT\PYG{+w}{ }\PYGZgt{}\PYGZgt{}\PYG{+w}{ }notes.txt
\PYG{g+go}{Important! If the dataset is not \PYGZdq{}clean\PYGZdq{} (a datalad status output is}
\PYG{g+go}{empty), datalad run will not work \PYGZhy{} you will have to save}
\PYG{g+go}{modifications present in your dataset.}
\PYG{g+go}{EOT}
\end{sphinxVerbatim}

\index{run command on dirty dataset@\spxentry{run command on dirty dataset}!with DataLad run@\spxentry{with DataLad run}}\index{with DataLad run@\spxentry{with DataLad run}!run command on dirty dataset@\spxentry{run command on dirty dataset}}\ignorespaces 
\sphinxAtStartPar
A way of executing a \sphinxcode{\sphinxupquote{datalad run}} \sphinxstyleemphasis{despite} an “unclean” dataset,
though, is to add the \sphinxcode{\sphinxupquote{\sphinxhyphen{}\sphinxhyphen{}explicit}} flag to \sphinxcode{\sphinxupquote{datalad run}}.
We will try this flag with the remaining \sphinxcode{\sphinxupquote{logo\_salt.jpg}}. Note that
we have an “unclean dataset” again because of the
additional note in \sphinxcode{\sphinxupquote{notes.txt}}.

\begin{sphinxVerbatim}[commandchars=\\\{\}]
\PYG{g+gp}{\PYGZdl{} }datalad\PYG{+w}{ }run\PYG{+w}{ }\PYGZhy{}m\PYG{+w}{ }\PYG{l+s+s2}{\PYGZdq{}Resize logo for slides\PYGZdq{}}\PYG{+w}{ }\PYG{l+s+se}{\PYGZbs{}}
\PYGZhy{}\PYGZhy{}input\PYG{+w}{ }\PYG{l+s+s2}{\PYGZdq{}recordings/longnow/.datalad/feed\PYGZus{}metadata/logo\PYGZus{}salt.jpg\PYGZdq{}}\PYG{+w}{ }\PYG{l+s+se}{\PYGZbs{}}
\PYGZhy{}\PYGZhy{}output\PYG{+w}{ }\PYG{l+s+s2}{\PYGZdq{}recordings/salt\PYGZus{}logo\PYGZus{}small.jpg\PYGZdq{}}\PYG{+w}{ }\PYG{l+s+se}{\PYGZbs{}}
\PYGZhy{}\PYGZhy{}explicit\PYG{+w}{ }\PYG{l+s+se}{\PYGZbs{}}
\PYG{l+s+s2}{\PYGZdq{}convert \PYGZhy{}resize 400x400 recordings/longnow/.datalad/feed\PYGZus{}metadata/logo\PYGZus{}salt.jpg recordings/salt\PYGZus{}logo\PYGZus{}small.jpg\PYGZdq{}}
\PYG{g+go}{unlock(ok): recordings/salt\PYGZus{}logo\PYGZus{}small.jpg (file)}
\PYG{g+go}{[INFO] == Command start (output follows) =====}
\PYG{g+go}{[INFO] == Command exit (modification check follows) =====}
\PYG{g+go}{run(ok): /home/me/dl\PYGZhy{}101/DataLad\PYGZhy{}101 (dataset) [convert \PYGZhy{}resize 400x400 recordings/longn...]}
\PYG{g+go}{add(ok): recordings/salt\PYGZus{}logo\PYGZus{}small.jpg (file)}
\PYG{g+go}{save(ok): . (dataset)}
\end{sphinxVerbatim}

\sphinxAtStartPar
With this flag, DataLad considers the specification of inputs and outputs to be “explicit”.
It does not warn if the repository is dirty, but importantly, it
\sphinxstylestrong{only} saves modifications to the \sphinxstyleemphasis{listed outputs} (which is a problem in the
vast amount of cases where one does not exactly know which outputs are produced).

\index{explicit input/output declaration@\spxentry{explicit input/output declaration}!with DataLad run@\spxentry{with DataLad run}}\index{with DataLad run@\spxentry{with DataLad run}!explicit input/output declaration@\spxentry{explicit input/output declaration}}\ignorespaces \begin{importantnote}[label={index-1}, before title={\thetcbcounter\ }, check odd page=true]{Put explicit first!}
\label{\detokenize{basics/101-112-run4:index-1}}

\sphinxAtStartPar
The \sphinxcode{\sphinxupquote{\sphinxhyphen{}\sphinxhyphen{}explicit}} flag has to be given anywhere \sphinxstyleemphasis{prior} to the command that
should be run \textendash{} the command needs to be the last element of a
\sphinxcode{\sphinxupquote{datalad run}} call.


\end{importantnote}

\sphinxAtStartPar
A \sphinxcode{\sphinxupquote{datalad status}} will show that your previously modified \sphinxcode{\sphinxupquote{notes.txt}}
is still modified:

\begin{sphinxVerbatim}[commandchars=\\\{\}]
\PYG{g+gp}{\PYGZdl{} }datalad\PYG{+w}{ }status
\PYG{g+go}{ modified: notes.txt (file)}
\end{sphinxVerbatim}

\sphinxAtStartPar
Add an additional note on the \sphinxcode{\sphinxupquote{\sphinxhyphen{}\sphinxhyphen{}explicit}} flag, and finally save your changes to \sphinxcode{\sphinxupquote{notes.txt}}.

\begin{sphinxVerbatim}[commandchars=\\\{\}]
\PYG{g+gp}{\PYGZdl{} }cat\PYG{+w}{ }\PYGZlt{}\PYGZlt{}\PYG{+w}{ }EOT\PYG{+w}{ }\PYGZgt{}\PYGZgt{}\PYG{+w}{ }notes.txt
\PYG{g+go}{A suboptimal alternative is the \PYGZhy{}\PYGZhy{}explicit flag, used to record only}
\PYG{g+go}{those changes done to the files listed with \PYGZhy{}\PYGZhy{}output flags.}

\PYG{g+go}{EOT}
\end{sphinxVerbatim}

\begin{sphinxVerbatim}[commandchars=\\\{\}]
\PYG{g+gp}{\PYGZdl{} }datalad\PYG{+w}{ }save\PYG{+w}{ }\PYGZhy{}m\PYG{+w}{ }\PYG{l+s+s2}{\PYGZdq{}add note on clean datasets\PYGZdq{}}
\PYG{g+go}{add(ok): notes.txt (file)}
\PYG{g+go}{save(ok): . (dataset)}
\end{sphinxVerbatim}

\sphinxAtStartPar
To conclude this section on \sphinxcode{\sphinxupquote{datalad run}}, take a look at the last \sphinxcode{\sphinxupquote{datalad run}}
commit to see a {\hyperref[\detokenize{glossary:term-run-record}]{\sphinxtermref{\DUrole{xref,std,std-term}{run record}}}} with more content:

\fvset{hllines={, 10, 14, 15, 16, 17, 18, 19,}}%
\begin{sphinxVerbatim}[commandchars=\\\{\}]
\PYG{g+gp}{\PYGZdl{} }git\PYG{+w}{ }log\PYG{+w}{ }\PYGZhy{}p\PYG{+w}{ }\PYGZhy{}n\PYG{+w}{ }\PYG{l+m}{2}
\PYG{g+go}{Author: Elena Piscopia \PYGZlt{}elena@example.net\PYGZgt{}}
\PYG{g+go}{Date:   Tue Jun 18 16:13:00 2019 +0000}

\PYG{g+go}{    [DATALAD RUNCMD] Resize logo for slides}

\PYG{g+go}{    === Do not change lines below ===}
\PYG{g+go}{    \PYGZob{}}
\PYG{g+go}{     \PYGZdq{}chain\PYGZdq{}: [],}
\PYG{g+go}{     \PYGZdq{}cmd\PYGZdq{}: \PYGZdq{}convert \PYGZhy{}resize 400x400 recordings/longnow/.datalad/feed\PYGZus{}metadata/logo\PYGZus{}salt.jpg recordings/salt\PYGZus{}logo\PYGZus{}small.jpg\PYGZdq{},}
\PYG{g+go}{     \PYGZdq{}dsid\PYGZdq{}: \PYGZdq{}e3e70682\PYGZhy{}c209\PYGZhy{}4cac\PYGZhy{}629f\PYGZhy{}6fbed82c07cd\PYGZdq{},}
\PYG{g+go}{     \PYGZdq{}exit\PYGZdq{}: 0,}
\PYG{g+go}{     \PYGZdq{}extra\PYGZus{}inputs\PYGZdq{}: [],}
\PYG{g+go}{     \PYGZdq{}inputs\PYGZdq{}: [}
\PYG{g+go}{      \PYGZdq{}recordings/longnow/.datalad/feed\PYGZus{}metadata/logo\PYGZus{}salt.jpg\PYGZdq{}}
\PYG{g+go}{     ],}
\PYG{g+go}{     \PYGZdq{}outputs\PYGZdq{}: [}
\PYG{g+go}{      \PYGZdq{}recordings/salt\PYGZus{}logo\PYGZus{}small.jpg\PYGZdq{}}
\PYG{g+go}{     ],}
\PYG{g+go}{     \PYGZdq{}pwd\PYGZdq{}: \PYGZdq{}.\PYGZdq{}}
\PYG{g+go}{    \PYGZcb{}}
\PYG{g+go}{    \PYGZca{}\PYGZca{}\PYGZca{} Do not change lines above \PYGZca{}\PYGZca{}\PYGZca{}}

\PYG{g+go}{diff \PYGZhy{}\PYGZhy{}git a/recordings/salt\PYGZus{}logo\PYGZus{}small.jpg b/recordings/salt\PYGZus{}logo\PYGZus{}small.jpg}
\PYG{g+go}{index 0985399..d90c601 120000}
\PYG{g+go}{\PYGZhy{}\PYGZhy{}\PYGZhy{} a/recordings/salt\PYGZus{}logo\PYGZus{}small.jpg}
\PYG{g+go}{+++ b/recordings/salt\PYGZus{}logo\PYGZus{}small.jpg}
\PYG{g+go}{@@ \PYGZhy{}1 +1 @@}
\end{sphinxVerbatim}
\sphinxresetverbatimhllines

\sphinxstepscope


\section{Summary}
\label{\detokenize{basics/101-113-summary:summary}}\label{\detokenize{basics/101-113-summary:run6}}\label{\detokenize{basics/101-113-summary::doc}}
\sphinxAtStartPar
In the last four sections, we demonstrated how to create a proper \sphinxcode{\sphinxupquote{datalad run}}
command (\hyperref[\detokenize{basics/101-113-summary:fig-run-schema}]{Fig.\@ \ref{\detokenize{basics/101-113-summary:fig-run-schema}}}), and discovered the concept of \sphinxstyleemphasis{locked} content.
\begin{itemize}
\item {} 
\sphinxAtStartPar
\sphinxcode{\sphinxupquote{datalad run}} records and saves the changes a command makes in a dataset. That means
that modifications to existing content or new content are associated with a specific command
and saved to the dataset’s history. Essentially, \sphinxcode{\sphinxupquote{datalad run}} helps you to keep
track of what you do in your dataset by capturing all {\hyperref[\detokenize{glossary:term-provenance}]{\sphinxtermref{\DUrole{xref,std,std-term}{provenance}}}}.

\item {} 
\sphinxAtStartPar
A \sphinxcode{\sphinxupquote{datalad run}} command generates a \sphinxcode{\sphinxupquote{run record}} in the commit. This {\hyperref[\detokenize{glossary:term-run-record}]{\sphinxtermref{\DUrole{xref,std,std-term}{run record}}}} can be used
by DataLad to re\sphinxhyphen{}execute a command with \sphinxcode{\sphinxupquote{datalad rerun SHASUM}}, where SHASUM is the
commit hash of the \sphinxcode{\sphinxupquote{datalad run}} command that should be re\sphinxhyphen{}executed.

\item {} 
\sphinxAtStartPar
If a \sphinxcode{\sphinxupquote{datalad run}} or \sphinxcode{\sphinxupquote{datalad rerun}} does not modify any content, it will not write a
record to history.

\item {} 
\sphinxAtStartPar
With any \sphinxcode{\sphinxupquote{datalad run}}, specify a commit message, and whenever appropriate, specify its inputs
to the executed command (using the \sphinxcode{\sphinxupquote{\sphinxhyphen{}i}}/\sphinxcode{\sphinxupquote{\sphinxhyphen{}\sphinxhyphen{}input}} flag) and/or its output (using the \sphinxcode{\sphinxupquote{\sphinxhyphen{}o}}/
\sphinxcode{\sphinxupquote{\sphinxhyphen{}\sphinxhyphen{}output}} flag). The full command structure is:

\begin{sphinxVerbatim}[commandchars=\\\{\}]
\PYG{g+gp}{\PYGZdl{} }datalad\PYG{+w}{ }run\PYG{+w}{ }\PYG{l+s+se}{\PYGZbs{}}
\PYG{+w}{    }\PYGZhy{}m\PYG{+w}{ }\PYG{l+s+s2}{\PYGZdq{}commit message here\PYGZdq{}}\PYG{+w}{ }\PYG{l+s+se}{\PYGZbs{}}
\PYG{+w}{    }\PYGZhy{}\PYGZhy{}input\PYG{+w}{ }\PYG{l+s+s2}{\PYGZdq{}path/to/input/\PYGZdq{}}\PYG{+w}{ }\PYG{l+s+se}{\PYGZbs{}}
\PYG{+w}{    }\PYGZhy{}\PYGZhy{}output\PYG{+w}{ }\PYG{l+s+s2}{\PYGZdq{}path/to/output\PYGZdq{}}\PYG{+w}{ }\PYG{l+s+se}{\PYGZbs{}}
\PYG{+w}{    }\PYG{l+s+s2}{\PYGZdq{}command\PYGZdq{}}
\end{sphinxVerbatim}

\item {} 
\sphinxAtStartPar
Anything specified as \sphinxcode{\sphinxupquote{input}} will be retrieved if necessary with a \sphinxcode{\sphinxupquote{datalad get}} prior to command
execution. Anything specified as \sphinxcode{\sphinxupquote{output}} will be \sphinxcode{\sphinxupquote{unlocked}} prior to modifications.

\item {} 
\sphinxAtStartPar
It is good practice to specify \sphinxcode{\sphinxupquote{input}} and \sphinxcode{\sphinxupquote{output}} to ensure that a \sphinxcode{\sphinxupquote{datalad rerun}} works, and to capture the relevant elements of a computation in a machine\sphinxhyphen{}readable record.
If you want to spare yourself preparation time in case everything is already retrieved and unlocked, you can use \sphinxcode{\sphinxupquote{\sphinxhyphen{}\sphinxhyphen{}assume\sphinxhyphen{}ready \{input|output|both\}}} to skip a check on whether inputs are already present or outputs already unlocked.

\end{itemize}

\begin{figure}[tbp]
\centering
\capstart

\noindent\sphinxincludegraphics[width=0.800\linewidth]{{run}.pdf}
\caption{Overview of \sphinxcode{\sphinxupquote{datalad run}}.}\label{\detokenize{basics/101-113-summary:id1}}\label{\detokenize{basics/101-113-summary:fig-run-schema}}\end{figure}
\begin{itemize}
\item {} 
\sphinxAtStartPar
Getting and unlocking content is not only convenient for yourself, but enormously helpful
for anyone you share your dataset with, but this will be demonstrated in an upcoming section
in detail.

\item {} 
\sphinxAtStartPar
To execute a \sphinxcode{\sphinxupquote{datalad run}} or \sphinxcode{\sphinxupquote{datalad rerun}}, a \sphinxcode{\sphinxupquote{datalad status}}
either needs to report that the dataset has no uncommitted changes (the dataset state
should be “clean”), or the command needs to be extended with the \sphinxcode{\sphinxupquote{\sphinxhyphen{}\sphinxhyphen{}explicit}} option.

\end{itemize}


\subsection{Now what can I do with that?}
\label{\detokenize{basics/101-113-summary:now-what-can-i-do-with-that}}
\sphinxAtStartPar
You have procedurally experienced how to use \sphinxcode{\sphinxupquote{datalad run}} and \sphinxcode{\sphinxupquote{datalad rerun}}. Both
of these commands make it easier for you and others to associate changes in a dataset with
a script or command, and are helpful as the exact command for a given task is stored by
DataLad, and does not need to be remembered.

\sphinxAtStartPar
Furthermore, by experiencing many common error messages in the context of \sphinxcode{\sphinxupquote{datalad run}}
commands, you have gotten some clues on where to look for problems, should you encounter
those errors in your own work.

\sphinxAtStartPar
Lastly, we’ve started to unveil some principles of {\hyperref[\detokenize{glossary:term-git-annex}]{\sphinxtermref{\DUrole{xref,std,std-term}{git\sphinxhyphen{}annex}}}} that are relevant to
understanding how certain commands work and why certain commands may fail. We have seen that
git\sphinxhyphen{}annex locks large files’ content to prevent accidental modifications, and how the \sphinxcode{\sphinxupquote{\sphinxhyphen{}\sphinxhyphen{}output}}
flag in \sphinxcode{\sphinxupquote{datalad run}} can save us an intermediate \sphinxcode{\sphinxupquote{datalad unlock}} to unlock this content.
The next section will elaborate on this a bit more.


\subsection{Further reading}
\label{\detokenize{basics/101-113-summary:further-reading}}
\sphinxAtStartPar
The chapter on \sphinxcode{\sphinxupquote{datalad run}} provided an almost complete feature overview of the command.
If you want, you can extend this knowledge with computational environments and \sphinxcode{\sphinxupquote{datalad containers\sphinxhyphen{}run}} in chapter {\hyperref[\detokenize{basics/101-133-containersrun:containersrun}]{\sphinxcrossref{\DUrole{std,std-ref}{Computational reproducibility with software containers}}}} (\autopageref*{\detokenize{basics/101-133-containersrun:containersrun}}).
In addition, you can read up on other forms of computing usecases \sphinxhyphen{} for example, how to use \sphinxcode{\sphinxupquote{datalad run}} in interactive computing environments such as \dlhbhref{P2A}{Jupyter Notebooks}.

\sphinxstepscope


