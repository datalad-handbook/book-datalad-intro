\chapter{Tuning datasets to your needs}
\label{\detokenize{basics/basics-configuration:tuning-datasets-to-your-needs}}\label{\detokenize{basics/basics-configuration:chapter-config}}\label{\detokenize{basics/basics-configuration:id1}}\label{\detokenize{basics/basics-configuration::doc}}
\noindent{\hspace*{\fill}\sphinxincludegraphics[height=.33\textheight]{{settings_bw}.pdf}\hspace*{\fill}}

\sphinxstepscope

\index{configuration@\spxentry{configuration}!with Git@\spxentry{with Git}}\index{with Git@\spxentry{with Git}!configuration@\spxentry{configuration}}\index{configuration@\spxentry{configuration}!with DataLad@\spxentry{with DataLad}}\index{with DataLad@\spxentry{with DataLad}!configuration@\spxentry{configuration}}\ignorespaces 

\section{Local configuration}
\label{\detokenize{basics/101-122-config:local-configuration}}\label{\detokenize{basics/101-122-config:config}}\label{\detokenize{basics/101-122-config:index-0}}\label{\detokenize{basics/101-122-config::doc}}
\sphinxAtStartPar
Back in section {\hyperref[\detokenize{basics/101-114-txt2git:text2git}]{\sphinxcrossref{\DUrole{std,std-ref}{Data safety}}}} (\autopageref*{\detokenize{basics/101-114-txt2git:text2git}}), you already learned that there
are dataset configurations, and that these configurations can
be modified, for example, with the \sphinxcode{\sphinxupquote{\sphinxhyphen{}c text2git}} option.
This option applies a configuration template to store text
files in {\hyperref[\detokenize{glossary:term-Git}]{\sphinxtermref{\DUrole{xref,std,std-term}{Git}}}} instead of {\hyperref[\detokenize{glossary:term-git-annex}]{\sphinxtermref{\DUrole{xref,std,std-term}{git\sphinxhyphen{}annex}}}}, and thereby
modifies the DataLad dataset’s default configuration to store
every file in git\sphinxhyphen{}annex.

\sphinxAtStartPar
The lecture today focuses entirely on the topic of configurations,
and aims to equip everyone with the basics to configure
their general and dataset specific setup to their needs.
This is not only a handy way to tune a dataset to one’s
wishes, but also helpful to understand potential differences in
command execution and file handling between two users,
computers, or datasets.

\sphinxAtStartPar
“First of all, when we talk about configurations, we have
to differentiate between different scopes of configuration,
and different tools the configuration belongs or applies to”,
our lecturer starts. “In DataLad datasets, different tools can
have a configuration: {\hyperref[\detokenize{glossary:term-Git}]{\sphinxtermref{\DUrole{xref,std,std-term}{Git}}}}, {\hyperref[\detokenize{glossary:term-git-annex}]{\sphinxtermref{\DUrole{xref,std,std-term}{git\sphinxhyphen{}annex}}}}, and
DataLad itself. Because these tools are all
combined by DataLad to help you manage your data,
it is important to understand how the configuration of one
software is used by or influences a second tool, or the overall
dataset performance.”

\sphinxAtStartPar
“Oh crap, one of these theoretical lectures again” mourns a
student from the row behind you. Personally, you’d also
be much more excited
about any hands\sphinxhyphen{}on lecture filled with commands. But the
recent lecture about {\hyperref[\detokenize{glossary:term-git-annex}]{\sphinxtermref{\DUrole{xref,std,std-term}{git\sphinxhyphen{}annex}}}} and the {\hyperref[\detokenize{glossary:term-object-tree}]{\sphinxtermref{\DUrole{xref,std,std-term}{object\sphinxhyphen{}tree}}}}
was surprisingly captivating, so you are actually looking forward to today.
“Shht! I want to hear this!”, you shush him with a wink.

\sphinxAtStartPar
“We will start by looking into the very first configuration
you did, already before the course started: The \sphinxstyleemphasis{global}
Git configuration.” the lecturer says.

\index{config@\spxentry{config}!Git command@\spxentry{Git command}}\index{Git command@\spxentry{Git command}!config@\spxentry{config}}\ignorespaces 
\sphinxAtStartPar
At one point in time, you likely followed instructions such as
in {\hyperref[\detokenize{intro/installation:install}]{\sphinxcrossref{\DUrole{std,std-ref}{Installation and configuration}}}} (\autopageref*{\detokenize{intro/installation:install}}) and configured your
\sphinxstyleemphasis{Git identity} with the commands:

\begin{sphinxVerbatim}[commandchars=\\\{\}]
\PYG{g+gp}{\PYGZdl{} }git\PYG{+w}{ }config\PYG{+w}{ }\PYGZhy{}\PYGZhy{}global\PYG{+w}{ }\PYGZhy{}\PYGZhy{}add\PYG{+w}{ }user.name\PYG{+w}{ }\PYG{l+s+s2}{\PYGZdq{}Elena Piscopia\PYGZdq{}}
\PYG{g+gp}{\PYGZdl{} }git\PYG{+w}{ }config\PYG{+w}{ }\PYGZhy{}\PYGZhy{}global\PYG{+w}{ }\PYGZhy{}\PYGZhy{}add\PYG{+w}{ }user.email\PYG{+w}{ }elena@example.net
\end{sphinxVerbatim}

\sphinxAtStartPar
“What the above commands do is very simple: They search for
a specific configuration file, and set the variables specified
in the command \textendash{} in this case user name and user email address
\textendash{} to the values provided with the command.” she explains.

\sphinxAtStartPar
“This general procedure, specifying a value for a configuration
variable in a configuration file, is how you can configure the
different tools to your needs. The configuration, therefore,
is really easy. Even if you are only used to ticking boxes
in the \sphinxcode{\sphinxupquote{settings}} tab of a software tool so far, it’s intuitive
to understand how a configuration file in principle works and also
how to use it. The only piece of information you will need
are the necessary files, or the command that writes to them, and
the available options for configuration, that’s it. And what’s
really cool is that all tools we’ll be looking at \textendash{} Git, git\sphinxhyphen{}annex,
and DataLad \textendash{} can be configured using the \sphinxcode{\sphinxupquote{git config}}
command%
\begin{footnote}\sphinxAtStartFootnote
As an alternative to a \sphinxcode{\sphinxupquote{git config}} command, you could also run configuration
templates or procedures that apply predefined configurations or in some cases even
add the information to the configuration file by hand and save it using an editor of your choice. See {\hyperref[\detokenize{basics/101-124-procedures:procedures}]{\sphinxcrossref{\DUrole{std,std-ref}{Configurations to go}}}} (\autopageref*{\detokenize{basics/101-124-procedures:procedures}}) for more info.
%
\end{footnote}. Therefore, once you understand the syntax of this
command, you already know half of what’s relevant. The other half
is understanding what you are doing. Now then, let’s learn \sphinxstyleemphasis{how}
to configure settings, but also \sphinxstyleemphasis{understand} what we are doing
with these configurations.”

\sphinxAtStartPar
“This seems easy enough”, you think. Let’s see what types of
configurations there are.


\subsection{Git config files}
\label{\detokenize{basics/101-122-config:git-config-files}}
\sphinxAtStartPar
The user name and email configuration
is a \sphinxstyleemphasis{user\sphinxhyphen{}specific} configuration (called \sphinxstyleemphasis{global}
configuration by Git), and therefore applies to your user account.
Wherever on your computer \sphinxstyleemphasis{you} run a Git, git\sphinxhyphen{}annex, or DataLad
command, this global configuration will
associate the name and email address you supplied in
the \sphinxcode{\sphinxupquote{git config}} commands above with this action.
For example, whenever you
\sphinxcode{\sphinxupquote{datalad save}}, the information in this file is used for the
history entry about commit author and email.

\sphinxAtStartPar
Apart from \sphinxstyleemphasis{global} Git configurations, there are also \sphinxstyleemphasis{system\sphinxhyphen{}wide}%
\begin{footnote}\sphinxAtStartFootnote
The third scope of a Git configuration are the system wide configurations.
These are stored (if they exist) in \sphinxcode{\sphinxupquote{/etc/gitconfig}} and contain settings that would
apply to every user on the computer you are using. These configurations
are not relevant for DataLad\sphinxhyphen{}101, and we will thus skip them. You can
read more about Git’s configurations and different files
\dlhbhref{G1K}{here}.
%
\end{footnote}
and \sphinxstyleemphasis{repository} configurations. Each of these configurations
resides in its own file. The global configuration is stored in a file called
\sphinxcode{\sphinxupquote{.gitconfig}} in your home directory. Among
your name and email address, this file can store general
per\sphinxhyphen{}user configurations, such as a default editor%
\begin{footnote}\sphinxAtStartFootnote
If your default editor is {\hyperref[\detokenize{glossary:term-vim}]{\sphinxtermref{\DUrole{xref,std,std-term}{vim}}}} and you do not like this, now can be the time
to change it! Chose either of two options:
\begin{enumerate}
\sphinxsetlistlabels{\arabic}{enumi}{enumii}{}{)}%
\item {} 
\sphinxAtStartPar
Open up the file with an editor for your choice (e.g., \dlhbhref{H2A}{nano}), and either paste the following configuration or edit it if it already exists:

\end{enumerate}

\sphinxSetupCodeBlockInFootnote
\begin{sphinxVerbatim}[commandchars=\\\{\}]
\PYG{k}{[core]}
\PYG{+w}{    }\PYG{n+na}{editor}\PYG{+w}{ }\PYG{o}{=}\PYG{+w}{ }\PYG{l+s}{nano}
\end{sphinxVerbatim}
\begin{enumerate}
\sphinxsetlistlabels{\arabic}{enumi}{enumii}{}{)}%
\setcounter{enumi}{1}
\item {} 
\sphinxAtStartPar
Run the following command, but exchange \sphinxcode{\sphinxupquote{nano}} with an editor of your choice:

\end{enumerate}

\sphinxSetupCodeBlockInFootnote
\begin{sphinxVerbatim}[commandchars=\\\{\}]
\PYG{n+na}{\PYGZdl{} git config \PYGZhy{}\PYGZhy{}global \PYGZhy{}\PYGZhy{}add core.editor \PYGZdq{}nano\PYGZdq{}}
\end{sphinxVerbatim}
%
\end{footnote}, or highlighting
options.

\sphinxAtStartPar
The \sphinxstyleemphasis{repository\sphinxhyphen{}specific} configurations apply to each individual
repository. Their scope is more limited than the \sphinxstyleemphasis{global}
configuration (namely to a single repository), but it can overrule global
configurations: The more specific the scope of a configuration file is, the more
important it is, and the variables in the more specific configuration
will take precedence over variables in less specific configuration files.
One could, for example, have {\hyperref[\detokenize{glossary:term-vim}]{\sphinxtermref{\DUrole{xref,std,std-term}{vim}}}} configured to be the default editor
on a global scope, but could overrule this by setting the editor to \sphinxcode{\sphinxupquote{nano}}
in a given repository. For this reason, the repository\sphinxhyphen{}specific configuration
does not reside in a file in your home directory, but in \sphinxcode{\sphinxupquote{.git/config}}
within every Git repository (and thus DataLad dataset).

\sphinxAtStartPar
Thus, there are three different scopes of Git configuration, and each is defined
in a \sphinxcode{\sphinxupquote{config}} file in a different location. The configurations will determine
how Git behaves. In principle, all of these files can configure
the same variables differently, but more specific scopes take precedence over broader
scopes. Conveniently, not only can DataLad and git\sphinxhyphen{}annex be configured with
the same command as Git, but in many cases they will also use exactly the same
files as Git for their own configurations.

\index{configuration file@\spxentry{configuration file}!.git/config@\spxentry{.git/config}|spxpagem}\ignorespaces 
\sphinxAtStartPar
Let’s find out how the repository\sphinxhyphen{}specific configuration file in the \sphinxcode{\sphinxupquote{DataLad\sphinxhyphen{}101}}
superdataset looks like:

\begin{sphinxVerbatim}[commandchars=\\\{\}]
\PYG{g+gp}{\PYGZdl{} }cat\PYG{+w}{ }.git/config
\PYG{g+go}{[core]}
\PYG{g+go}{	repositoryformatversion = 0}
\PYG{g+go}{	filemode = true}
\PYG{g+go}{	bare = false}
\PYG{g+go}{	logallrefupdates = true}
\PYG{g+go}{[annex]}
\PYG{g+go}{	uuid = 46b169aa\PYGZhy{}bb91\PYGZhy{}42d6\PYGZhy{}be06\PYGZhy{}355d957fb4f7}
\PYG{g+go}{	version = 10}
\PYG{g+go}{[filter \PYGZdq{}annex\PYGZdq{}]}
\PYG{g+go}{	smudge = git\PYGZhy{}annex smudge \PYGZhy{}\PYGZhy{} \PYGZpc{}f}
\PYG{g+go}{	clean = git\PYGZhy{}annex smudge \PYGZhy{}\PYGZhy{}clean \PYGZhy{}\PYGZhy{} \PYGZpc{}f}
\PYG{g+go}{	process = git\PYGZhy{}annex filter\PYGZhy{}process}
\PYG{g+go}{[submodule \PYGZdq{}recordings/longnow\PYGZdq{}]}
\PYG{g+go}{	active = true}
\PYG{g+go}{	url = https://github.com/datalad\PYGZhy{}datasets/longnow\PYGZhy{}podcasts.git}
\PYG{g+go}{[remote \PYGZdq{}roommate\PYGZdq{}]}
\PYG{g+go}{	url = ../mock\PYGZus{}user/DataLad\PYGZhy{}101}
\PYG{g+go}{	fetch = +refs/heads/*:refs/remotes/roommate/*}
\PYG{g+go}{	annex\PYGZhy{}uuid = ✂UUID✂}
\PYG{g+go}{	annex\PYGZhy{}ignore = false}
\end{sphinxVerbatim}

\sphinxAtStartPar
This file consists of so called “sections” with the section names
in square brackets (e.g., \sphinxcode{\sphinxupquote{core}}). Occasionally, a section can have
subsections: This is indicated by subsection names in
quotation marks after the section name. For example, \sphinxcode{\sphinxupquote{roommate}} is a subsection
of the section \sphinxcode{\sphinxupquote{remote}}.
Within each section, \sphinxcode{\sphinxupquote{variable = value}} pairs specify configurations
for the given (sub)section.

\index{configure editor@\spxentry{configure editor}!with Git@\spxentry{with Git}}\index{with Git@\spxentry{with Git}!configure editor@\spxentry{configure editor}}\ignorespaces 
\sphinxAtStartPar
The first section is called \sphinxcode{\sphinxupquote{core}} \textendash{} as the name suggests,
this configures core Git functionality. There are
\dlhbhref{G1L}{many more}
configurations than the ones in this config file, but
they are related to Git, and less related or important to the configuration of
a DataLad dataset. We will use this section to showcase the anatomy of the
\sphinxcode{\sphinxupquote{git config}} command. If, for example, you would want to specifically
configure {\hyperref[\detokenize{glossary:term-nano}]{\sphinxtermref{\DUrole{xref,std,std-term}{nano}}}} to be the default editor in this dataset, you
can do it like this:

\begin{sphinxVerbatim}[commandchars=\\\{\}]
\PYG{g+gp}{\PYGZdl{} }git\PYG{+w}{ }config\PYG{+w}{ }\PYGZhy{}\PYGZhy{}local\PYG{+w}{ }\PYGZhy{}\PYGZhy{}add\PYG{+w}{ }core.editor\PYG{+w}{ }\PYG{l+s+s2}{\PYGZdq{}nano\PYGZdq{}}
\end{sphinxVerbatim}

\sphinxAtStartPar
The command consists of the base command \sphinxcode{\sphinxupquote{git config}},
a specification of the scope of the configuration with the \sphinxcode{\sphinxupquote{\sphinxhyphen{}\sphinxhyphen{}local}}
flag, a \sphinxcode{\sphinxupquote{name}} specification consisting of section and key with the
notation \sphinxcode{\sphinxupquote{section.variable}} (here: \sphinxcode{\sphinxupquote{core.editor}}), and finally the value
specification \sphinxcode{\sphinxupquote{"nano"}}.

\sphinxAtStartPar
Let’s see what has changed:

\fvset{hllines={, 7,}}%
\begin{sphinxVerbatim}[commandchars=\\\{\}]
\PYG{g+gp}{\PYGZdl{} }cat\PYG{+w}{ }.git/config
\PYG{g+go}{[core]}
\PYG{g+go}{	repositoryformatversion = 0}
\PYG{g+go}{	filemode = true}
\PYG{g+go}{	bare = false}
\PYG{g+go}{	logallrefupdates = true}
\PYG{g+go}{	editor = nano}
\PYG{g+go}{[annex]}
\PYG{g+go}{	uuid = 46b169aa\PYGZhy{}bb91\PYGZhy{}42d6\PYGZhy{}be06\PYGZhy{}355d957fb4f7}
\PYG{g+go}{	version = 10}
\PYG{g+go}{[filter \PYGZdq{}annex\PYGZdq{}]}
\PYG{g+go}{	smudge = git\PYGZhy{}annex smudge \PYGZhy{}\PYGZhy{} \PYGZpc{}f}
\PYG{g+go}{	clean = git\PYGZhy{}annex smudge \PYGZhy{}\PYGZhy{}clean \PYGZhy{}\PYGZhy{} \PYGZpc{}f}
\PYG{g+go}{	process = git\PYGZhy{}annex filter\PYGZhy{}process}
\PYG{g+go}{[submodule \PYGZdq{}recordings/longnow\PYGZdq{}]}
\PYG{g+go}{	active = true}
\PYG{g+go}{	url = https://github.com/datalad\PYGZhy{}datasets/longnow\PYGZhy{}podcasts.git}
\PYG{g+go}{[remote \PYGZdq{}roommate\PYGZdq{}]}
\PYG{g+go}{	url = ../mock\PYGZus{}user/DataLad\PYGZhy{}101}
\PYG{g+go}{	fetch = +refs/heads/*:refs/remotes/roommate/*}
\PYG{g+go}{	annex\PYGZhy{}uuid = ✂UUID✂}
\PYG{g+go}{	annex\PYGZhy{}ignore = false}
\end{sphinxVerbatim}
\sphinxresetverbatimhllines

\sphinxAtStartPar
With this additional line in your repository’s Git configuration, \sphinxcode{\sphinxupquote{nano}} will
be used as a default editor regardless of the configuration in your global
or system\sphinxhyphen{}wide configuration. Note that the flag \sphinxcode{\sphinxupquote{\sphinxhyphen{}\sphinxhyphen{}local}} applies the
configuration to your repository’s \sphinxcode{\sphinxupquote{.git/config}} file, whereas \sphinxcode{\sphinxupquote{\sphinxhyphen{}\sphinxhyphen{}global}}
would apply it as a user specific configuration, and \sphinxcode{\sphinxupquote{\sphinxhyphen{}\sphinxhyphen{}system}} as a
system\sphinxhyphen{}wide configuration.
If you would want to change this existing line in your \sphinxcode{\sphinxupquote{.git/config}}
file, you would replace \sphinxcode{\sphinxupquote{\sphinxhyphen{}\sphinxhyphen{}add}} with \sphinxcode{\sphinxupquote{\sphinxhyphen{}\sphinxhyphen{}replace\sphinxhyphen{}all}} such as in:

\begin{sphinxVerbatim}[commandchars=\\\{\}]
\PYG{g+gp}{\PYGZdl{} }git\PYG{+w}{ }config\PYG{+w}{ }\PYGZhy{}\PYGZhy{}local\PYG{+w}{ }\PYGZhy{}\PYGZhy{}replace\PYGZhy{}all\PYG{+w}{ }core.editor\PYG{+w}{ }\PYG{l+s+s2}{\PYGZdq{}vim\PYGZdq{}}
\end{sphinxVerbatim}

\sphinxAtStartPar
to configure {\hyperref[\detokenize{glossary:term-vim}]{\sphinxtermref{\DUrole{xref,std,std-term}{vim}}}} to be your default editor.
Note that while being a good toy example, it is not a common thing to
configure repository\sphinxhyphen{}specific editors.

\sphinxAtStartPar
This example demonstrated the structure of a \sphinxcode{\sphinxupquote{git config}}
command. By specifying the \sphinxcode{\sphinxupquote{name}} option with \sphinxcode{\sphinxupquote{section.variable}}
(or \sphinxcode{\sphinxupquote{section.subsection.variable}} if there is a subsection), and
a value, one can configure Git, git\sphinxhyphen{}annex, and DataLad.
\sphinxstyleemphasis{Most} of these configurations will be written to a \sphinxcode{\sphinxupquote{config}} file
of Git, depending on the scope (local, global, system\sphinxhyphen{}wide)
specified in the command. Should you happen to make a mistake,
it is typically simple to rectify things again. Check the
\textit{Find-out-more}~{\findoutmoreiconinline}\textit{\ref{fom-git-config}} {\hyperref[\detokenize{basics/101-122-config:fom-git-config}]{\sphinxcrossref{\DUrole{std,std-ref}{on \textquotesingle{}git config\textquotesingle{}}}}} (\autopageref*{\detokenize{basics/101-122-config:fom-git-config}}).

\index{unset configuration@\spxentry{unset configuration}!with Git@\spxentry{with Git}}\index{with Git@\spxentry{with Git}!unset configuration@\spxentry{unset configuration}}\ignorespaces \begin{findoutmore}[label={fom-git-config}, before title={\thetcbcounter\ }, float, floatplacement=tb, check odd page=true]{If things go wrong during Git config}
\label{\detokenize{basics/101-122-config:fom-git-config}}

\sphinxAtStartPar
If something goes wrong during the \sphinxcode{\sphinxupquote{git config}} command,
for example, you end up having two keys of the same name because you
added a key instead of replacing an existing one, you can use the
\sphinxcode{\sphinxupquote{\sphinxhyphen{}\sphinxhyphen{}unset}} option to remove the line. Alternatively, you can also open
the config file in an editor and remove or change sections by hand.


\end{findoutmore}

\sphinxAtStartPar
The only information you need, therefore, is the name of a section and
variable to configure, and the value you want to specify. But in many cases
it is also useful to find out which configurations are already set in
which way and where. For this, the \sphinxcode{\sphinxupquote{git config \sphinxhyphen{}\sphinxhyphen{}list \sphinxhyphen{}\sphinxhyphen{}show\sphinxhyphen{}origin}}
is useful. It will display all configurations and their location:

\begin{sphinxVerbatim}[commandchars=\\\{\}]
\PYG{g+gp}{\PYGZdl{} }git\PYG{+w}{ }config\PYG{+w}{ }\PYGZhy{}\PYGZhy{}list\PYG{+w}{ }\PYGZhy{}\PYGZhy{}show\PYGZhy{}origin
\PYG{g+go}{file:/home/bob/.gitconfig   user.name=Bob McBobface}
\PYG{g+go}{file:/home/bob/.gitconfig   user.email=bob@mcbobface.com}
\PYG{g+go}{file:.git/config    annex.uuid=1f83595e\PYGZhy{}bcba\PYGZhy{}4226\PYGZhy{}aa2c\PYGZhy{}6f0153eb3c54}
\PYG{g+go}{file:.git/config    annex.backends=MD5E}
\PYG{g+go}{file:.git/config    submodule.recordings/longnow.url=https://github.com/✂}
\PYG{g+go}{file:.git/config    submodule.recordings/longnow.active=true}
\PYG{g+go}{file:.git/config    remote.roommate.url=../mock\PYGZus{}user/onemoredir/DataLad\PYGZhy{}101}
\PYG{g+go}{file:.git/config    remote.roommate.annex\PYGZhy{}uuid=a5ae24de\PYGZhy{}1533\PYGZhy{}4b09\PYGZhy{}98b9\PYGZhy{}cd9ba6bf303c}
\PYG{g+go}{file:.git/config    submodule.longnow.url=https://github.com/✂}
\PYG{g+go}{file:.git/config    submodule.longnow.active=true}
\PYG{g+go}{...}
\end{sphinxVerbatim}

\sphinxAtStartPar
This example shows some configurations in the global \sphinxcode{\sphinxupquote{.gitconfig}}
file, and the configurations within \sphinxcode{\sphinxupquote{DataLad\sphinxhyphen{}101/.git/config}}.
The command is very handy to display all configurations at once to identify
configuration problems, find the right configuration file to make a change to,
or simply remind oneself of the existing configurations, and it is a useful
helper to keep in the back of your head.

\sphinxAtStartPar
At this point you may feel like many of these configurations or the configuration file
inside of \sphinxcode{\sphinxupquote{DataLad\sphinxhyphen{}101}} do not appear to be
intuitively understandable enough to confidently apply changes to them,
or identify necessary changes. And indeed, most of the sections and variables
or values in there are irrelevant for understanding the book, your dataset,
or DataLad, and can just be left as they are. The previous section merely served
to de\sphinxhyphen{}mystify the \sphinxcode{\sphinxupquote{git config}} command and the configuration files.
Nevertheless, it can be helpful to get an overview on the meaning of the
remaining sections in that file, and the \textit{Find-out-more}~{\findoutmoreiconinline}\textit{\ref{fom-gitconfig}} {\hyperref[\detokenize{basics/101-122-config:fom-gitconfig}]{\sphinxcrossref{\DUrole{std,std-ref}{on the Git config file}}}} (\autopageref*{\detokenize{basics/101-122-config:fom-gitconfig}}) offers a glimpse at that.

\noindent{\hspace*{\fill}\sphinxincludegraphics[height=.33\textheight]{{personal_settings_bw}.pdf}\hspace*{\fill}}

\index{dataset configuration@\spxentry{dataset configuration}}\ignorespaces \begin{findoutmore}[label={fom-gitconfig}, before title={\thetcbcounter\ }, float, floatplacement=tb, check odd page=true]{Dissecting a Git config file further}
\label{\detokenize{basics/101-122-config:fom-gitconfig}}

\sphinxAtStartPar
Let’s walk through the Git config file of \sphinxcode{\sphinxupquote{DataLad\sphinxhyphen{}101}}:
As mentioned above, git\sphinxhyphen{}annex will use the
{\hyperref[\detokenize{glossary:term-Git-config-file}]{\sphinxtermref{\DUrole{xref,std,std-term}{Git config file}}}} for some of its configurations, such as the second section.
It lists the repository version and {\hyperref[\detokenize{glossary:term-annex-UUID}]{\sphinxtermref{\DUrole{xref,std,std-term}{annex UUID}}}}%
\begin{footnote}\sphinxAtStartFootnote
A UUID is a universally unique identifier \textendash{} a 128\sphinxhyphen{}bit number
that unambiguously identifies information.
%
\end{footnote} (\sphinxcode{\sphinxupquote{git annex whereis}} displays information about where the
annexed content is with these UUIDs).

\sphinxAtStartPar
You may recognize the fourth part of the configuration, the subsection
\sphinxcode{\sphinxupquote{"recordings/longnow"}} in the section \sphinxcode{\sphinxupquote{submodule}}.
Clearly, this is a reference to the \sphinxcode{\sphinxupquote{longnow}} podcasts
we cloned as a subdataset. The name \sphinxstyleemphasis{submodule} is Git
terminology, and describes a Git repository inside of
another Git repository \textendash{} just like
the super\sphinxhyphen{} and subdataset principles you discovered in the
section {\hyperref[\detokenize{basics/101-106-nesting:nesting}]{\sphinxcrossref{\DUrole{std,std-ref}{Dataset nesting}}}} (\autopageref*{\detokenize{basics/101-106-nesting:nesting}}). When you clone a DataLad dataset
as a subdataset, it gets \sphinxstyleemphasis{registered} in this file.
For each subdataset, an individual submodule entry
will store the information about the subdataset’s
\sphinxcode{\sphinxupquote{\sphinxhyphen{}\sphinxhyphen{}source}} or \sphinxstyleemphasis{origin} (the “url”).
Thus, every subdataset in your dataset
will be listed in this file.
If you want, go back to section {\hyperref[\detokenize{basics/101-105-install:installds}]{\sphinxcrossref{\DUrole{std,std-ref}{Install datasets}}}} (\autopageref*{\detokenize{basics/101-105-install:installds}}) to see that the
“url” is the same URL we cloned the longnow dataset from, and
go back to section {\hyperref[\detokenize{basics/101-116-sharelocal:sharelocal1}]{\sphinxcrossref{\DUrole{std,std-ref}{Looking without touching}}}} (\autopageref*{\detokenize{basics/101-116-sharelocal:sharelocal1}}) to remind yourself of
how cloning a dataset with subdatasets looked and felt like.

\sphinxAtStartPar
Another interesting part is the last section, “remote”.
Here we can find the {\hyperref[\detokenize{glossary:term-sibling}]{\sphinxtermref{\DUrole{xref,std,std-term}{sibling}}}} “roommate” we defined
in {\hyperref[\detokenize{basics/101-121-siblings:sibling}]{\sphinxcrossref{\DUrole{std,std-ref}{Networking}}}} (\autopageref*{\detokenize{basics/101-121-siblings:sibling}}). The term {\hyperref[\detokenize{glossary:term-remote}]{\sphinxtermref{\DUrole{xref,std,std-term}{remote}}}} is Git\sphinxhyphen{}terminology and is
used to describe other repositories or DataLad datasets that the
repository knows about.
This file, therefore, is where DataLad \sphinxstyleemphasis{registered} the sibling
with \sphinxcode{\sphinxupquote{datalad siblings add}}, and thanks to it you can
collaborate with your room mate.
The value to the \sphinxcode{\sphinxupquote{url}} variable is a \sphinxstyleemphasis{path}. If at any point
either your superdataset or the remote moves on your file system,
the association between the two datasets breaks \textendash{} this can be fixed by adjusting this
path, and a demonstration of this is in section {\hyperref[\detokenize{basics/101-136-filesystem:file-system}]{\sphinxcrossref{\DUrole{std,std-ref}{Dataset ops}}}} (\autopageref*{\detokenize{basics/101-136-filesystem:file-system}}).
\sphinxtitleref{fetch} contains a specification which parts of the repository are
updated \textendash{} in this case everything (all of the branches).
Lastly, the \sphinxcode{\sphinxupquote{annex\sphinxhyphen{}ignore = false}} configuration allows git\sphinxhyphen{}annex
to query the remote when it tries to retrieve data from annexed content.


\end{findoutmore}

\index{configuration@\spxentry{configuration}!DataLad command@\spxentry{DataLad command}}\index{DataLad command@\spxentry{DataLad command}!configuration@\spxentry{configuration}}\index{set configuration@\spxentry{set configuration}!with DataLad@\spxentry{with DataLad}}\index{with DataLad@\spxentry{with DataLad}!set configuration@\spxentry{set configuration}}\ignorespaces 

\subsection{The \sphinxstyleliteralintitle{\sphinxupquote{datalad configuration}} command}
\label{\detokenize{basics/101-122-config:the-datalad-configuration-command}}\label{\detokenize{basics/101-122-config:index-6}}
\sphinxAtStartPar
Although this section put a focus on the \sphinxcode{\sphinxupquote{git config}} command, it is important to mention that there also is a \sphinxcode{\sphinxupquote{datalad configuration}} command.
It is not identical to \sphinxcode{\sphinxupquote{git config}}, but while it lacks some feature of \sphinxcode{\sphinxupquote{git config}}, such as the ability to set system\sphinxhyphen{}wide configuration, it has additional features.
Beyond the \sphinxcode{\sphinxupquote{local}} and \sphinxcode{\sphinxupquote{global}} scopes, it also supports {\hyperref[\detokenize{glossary:term-branch}]{\sphinxtermref{\DUrole{xref,std,std-term}{branch}}}} specific configurations in the \sphinxcode{\sphinxupquote{.datalad/config}} file (further discussed in the next section), setting configurations recursively through dataset hierarchies, and multi\sphinxhyphen{}configuration queries (such as \sphinxcode{\sphinxupquote{datalad configuration get user.name user.email}}).
By default, \sphinxcode{\sphinxupquote{datalad configuration}} will \sphinxcode{\sphinxupquote{dump}} (list) the effective configuration including relevant \sphinxcode{\sphinxupquote{DATALAD\_*}} {\hyperref[\detokenize{glossary:term-environment-variable}]{\sphinxtermref{\DUrole{xref,std,std-term}{environment variable}}}}s, and also annotate the purpose of many common configuration items.
The subcommands \sphinxcode{\sphinxupquote{datalad configuration get}} or \sphinxcode{\sphinxupquote{datalad configuration set}} perform queries or set configurations.
You can find out more information on this command in the command documentation.


\subsection{\sphinxstyleliteralintitle{\sphinxupquote{.git/config}} versus other (configuration) files}
\label{\detokenize{basics/101-122-config:git-config-versus-other-configuration-files}}
\sphinxAtStartPar
One crucial aspect distinguishes the \sphinxcode{\sphinxupquote{.git/config}} file from many other files
in your dataset: Even though it is part of your dataset, it won’t be shared together
with the dataset. The reason for this is that this file is not version
controlled, as it lies within the \sphinxcode{\sphinxupquote{.git}} directory.
Repository\sphinxhyphen{}specific configurations within your \sphinxcode{\sphinxupquote{.git/config}}
file are thus not written to history. Any local configuration in \sphinxcode{\sphinxupquote{.git/config}}
applies to the dataset, but it does not \sphinxstyleemphasis{stick} to the dataset.
One can have the misconception that because the configurations were made \sphinxstyleemphasis{in}
the dataset, these configurations will also be shared together with the dataset.
\sphinxcode{\sphinxupquote{.git/config}}, however, behaves just as your global or system\sphinxhyphen{}wide configurations.
These configurations are in effect on a system, or for a user, or for a dataset,
but are not shared.
A \sphinxcode{\sphinxupquote{datalad clone}} command of someone’s dataset will not get you their
editor configuration, should they have included one in their config file.
Instead, upon a \sphinxcode{\sphinxupquote{datalad clone}}, a new config file will be created.

\sphinxAtStartPar
This means, however, that configurations that should “stick” to a dataset%
\begin{footnote}\sphinxAtStartFootnote
Please note that not all configurations can be written to files other than \sphinxcode{\sphinxupquote{.git/config}}.
Some of the files introduced in the next section will not be queried by Git, and in principle, it is a good thing that one cannot share arbitrary configurations together with a dataset, as this could be a potential security threat.
In those cases where you need dataset clones to inherit certain non\sphinxhyphen{}sticky configurations, it is advised to write a custom procedure and distribute it together with the dataset.
The next two sections contain concrete usecases and tutorials.
%
\end{footnote}
need to be defined in different files \textendash{} files that are version controlled.
The next section will talk about them.

\sphinxstepscope

\index{dataset configuration@\spxentry{dataset configuration}!with DataLad@\spxentry{with DataLad}}\index{with DataLad@\spxentry{with DataLad}!dataset configuration@\spxentry{dataset configuration}}\ignorespaces 

\section{Shipping and overriding configuration}
\label{\detokenize{basics/101-123-config2:shipping-and-overriding-configuration}}\label{\detokenize{basics/101-123-config2:config2}}\label{\detokenize{basics/101-123-config2:index-0}}\label{\detokenize{basics/101-123-config2::doc}}
\sphinxAtStartPar
As the last section already suggest, within a Git repository,
\sphinxcode{\sphinxupquote{.git/config}} is not the only configuration file.
There are also \sphinxcode{\sphinxupquote{.gitmodules}} and \sphinxcode{\sphinxupquote{.gitattributes}}, and in DataLad datasets
there also is a \sphinxcode{\sphinxupquote{.datalad/config}} file.

\sphinxAtStartPar
All of these files store configurations, but have an important difference:
They are version controlled, and upon sharing a dataset these configurations
will be shared as well. An example for a shared configuration
is the one that the \sphinxcode{\sphinxupquote{text2git}} configuration template applied:
In the shared copy of your dataset from {\hyperref[\detokenize{basics/101-121-siblings:sibling}]{\sphinxcrossref{\DUrole{std,std-ref}{Networking}}}} (\autopageref*{\detokenize{basics/101-121-siblings:sibling}}), text files are also saved with Git,
and not git\sphinxhyphen{}annex. The configuration responsible
for this behavior is in a \sphinxcode{\sphinxupquote{.gitattributes}} file, and we’ll start this
section by looking into it.

\index{configuration file@\spxentry{configuration file}!.gitattributes@\spxentry{.gitattributes}|spxpagem}\ignorespaces 

\subsection{\sphinxstyleliteralintitle{\sphinxupquote{.gitattributes}}}
\label{\detokenize{basics/101-123-config2:gitattributes}}\label{\detokenize{basics/101-123-config2:index-1}}
\sphinxAtStartPar
This file lies right in the root of your superdataset:

\begin{sphinxVerbatim}[commandchars=\\\{\}]
\PYG{g+gp}{\PYGZdl{} }cat\PYG{+w}{ }.gitattributes
\PYG{g+go}{* annex.backend=MD5E}
\PYG{g+go}{**/.git* annex.largefiles=nothing}
\PYG{g+go}{* annex.largefiles=((mimeencoding=binary)and(largerthan=0))}
\end{sphinxVerbatim}

\sphinxAtStartPar
This looks neither spectacular nor pretty. Also, it does not follow the \sphinxcode{\sphinxupquote{section\sphinxhyphen{}option\sphinxhyphen{}value}}
organization of the \sphinxcode{\sphinxupquote{.git/config}} file anymore. Instead, there are three lines,
and all of these seem to have something to do with the configuration of git\sphinxhyphen{}annex.
There even is one key word that you recognize: MD5E.
If you have read the {\hyperref[\detokenize{basics/101-115-symlinks:objecttree}]{\sphinxcrossref{\DUrole{std,std-ref}{Find\sphinxhyphen{}out\sphinxhyphen{}more on object trees}}}} (\autopageref*{\detokenize{basics/101-115-symlinks:objecttree}})
you will recognize it as a reference to the type of
key used by git\sphinxhyphen{}annex to identify and store file content in the object\sphinxhyphen{}tree.
The first row, \sphinxcode{\sphinxupquote{* annex.backend=MD5E}}, therefore translates to “The \sphinxcode{\sphinxupquote{MD5E}} git\sphinxhyphen{}annex backend should be used for any file”.
But what is the rest? We’ll start with the last row:

\begin{sphinxVerbatim}[commandchars=\\\{\}]
*\PYG{+w}{ }annex.largefiles\PYG{o}{=}\PYG{o}{(}\PYG{o}{(}\PYG{n+nv}{mimeencoding}\PYG{o}{=}binary\PYG{o}{)}and\PYG{o}{(}\PYG{n+nv}{largerthan}\PYG{o}{=}\PYG{l+m}{0}\PYG{o}{)}\PYG{o}{)}
\end{sphinxVerbatim}

\sphinxAtStartPar
Uhhh, cryptic. The lecturer explains: “git\sphinxhyphen{}annex will \sphinxstyleemphasis{annex}, that is, \sphinxstyleemphasis{store in the object\sphinxhyphen{}tree},
anything it considers to be a “large file”. By default, anything
in your dataset would be a “large file”, that means anything would be annexed.
However, in section {\hyperref[\detokenize{basics/101-115-symlinks:symlink}]{\sphinxcrossref{\DUrole{std,std-ref}{Data integrity}}}} (\autopageref*{\detokenize{basics/101-115-symlinks:symlink}}) I already mentioned that exceptions to this
behavior can be defined based on
\begin{enumerate}
\sphinxsetlistlabels{\arabic}{enumi}{enumii}{}{.}%
\item {} 
\sphinxAtStartPar
file size

\item {} 
\sphinxAtStartPar
and/or path/pattern, and thus for example file extensions,
or names, or file types (e.g., text files, as with the
\sphinxcode{\sphinxupquote{text2git}} configuration template).

\end{enumerate}

\sphinxAtStartPar
“In \sphinxcode{\sphinxupquote{.gitattributes}}, you can define what a large file and what is not
by simply telling git\sphinxhyphen{}annex by writing such rules.”

\sphinxAtStartPar
What you can see in this \sphinxcode{\sphinxupquote{.gitattributes}} file is a rule based on \sphinxstylestrong{file types}:
With \sphinxcode{\sphinxupquote{(mimeencoding=binary)}}%
\begin{footnote}\sphinxAtStartFootnote
When opening any file on a UNIX system, the file does not need to have a file
extension (such as \sphinxcode{\sphinxupquote{.txt}}, \sphinxcode{\sphinxupquote{.pdf}}, \sphinxcode{\sphinxupquote{.jpg}}) for the operating system to know
how to open or use this file (in contrast to Windows, which does not know how to
open a file without an extension). To do this, Unix systems rely on a file’s
MIME type \textendash{} an information about a file’s content. A \sphinxcode{\sphinxupquote{.txt}} file, for example,
has MIME type \sphinxcode{\sphinxupquote{text/plain}} as does a bash script (\sphinxcode{\sphinxupquote{.sh}}), a Python
script has MIME type \sphinxcode{\sphinxupquote{text/x\sphinxhyphen{}python}}, a \sphinxcode{\sphinxupquote{.jpg}} file is \sphinxcode{\sphinxupquote{image/jpg}}, and
a \sphinxcode{\sphinxupquote{.pdf}} file has MIME type \sphinxcode{\sphinxupquote{application/pdf}}. You can find out the MIME type
of a file by running:

\sphinxSetupCodeBlockInFootnote
\begin{sphinxVerbatim}[commandchars=\\\{\}]
\PYG{g+gp}{\PYGZdl{} }file\PYG{+w}{ }\PYGZhy{}\PYGZhy{}mime\PYGZhy{}type\PYG{+w}{ }path/to/file
\end{sphinxVerbatim}
%
\end{footnote}, the \sphinxcode{\sphinxupquote{text2git}} configuration template
configured git\sphinxhyphen{}annex to regard all files of type “binary” as a large file.
Thanks to this little line, your text files are not annexed, but stored
directly in Git.

\sphinxAtStartPar
The patterns \sphinxcode{\sphinxupquote{*}} and \sphinxcode{\sphinxupquote{**}} are so\sphinxhyphen{}called “wildcards” you might recognize from used in {\hyperref[\detokenize{glossary:term-globbing}]{\sphinxtermref{\DUrole{xref,std,std-term}{globbing}}}}.
In Git configuration files, an asterisk “*” matches anything except a slash.
The third row therefore
translates to “Do not annex anything that is a text file” for git\sphinxhyphen{}annex.
Two leading “\sphinxcode{\sphinxupquote{**}}” followed by a slash matches
\sphinxstyleemphasis{recursively} in all directories.
Therefore, the second row instructs git\sphinxhyphen{}annex to regard nothing starting with \sphinxcode{\sphinxupquote{.git}} as a “large file”, including contents inside of \sphinxcode{\sphinxupquote{.git}} directories.
This way, the \sphinxcode{\sphinxupquote{.git}} repositories are protected from being annexed.
If you had a single file (\sphinxcode{\sphinxupquote{myfile.pdf}}) you would not want annexed, specifying a rule such as:

\begin{sphinxVerbatim}[commandchars=\\\{\}]
myfile.pdf\PYG{+w}{ }annex.largefiles\PYG{o}{=}nothing
\end{sphinxVerbatim}

\sphinxAtStartPar
will keep it stored in Git. To see an example of this, navigate into the longnow subdataset,
and view this dataset’s \sphinxcode{\sphinxupquote{.gitattributes}} file:

\begin{sphinxVerbatim}[commandchars=\\\{\}]
\PYG{g+gp}{\PYGZdl{} }cat\PYG{+w}{ }recordings/longnow/.gitattributes
\PYG{g+go}{* annex.backend=MD5E}
\PYG{g+go}{**/.git* annex.largefiles=nothing}
\PYG{g+go}{README.md annex.largefiles=nothing}
\end{sphinxVerbatim}

\sphinxAtStartPar
The relevant part is \sphinxcode{\sphinxupquote{README.md annex.largefiles=nothing}}.
This instructs git\sphinxhyphen{}annex to specifically not annex \sphinxcode{\sphinxupquote{README.md}}.

\sphinxAtStartPar
Lastly, if you wanted to configure a rule based on \sphinxstylestrong{size}, you could add a row such as:

\begin{sphinxVerbatim}[commandchars=\\\{\}]
**\PYG{+w}{ }annex.largefiles\PYG{o}{(}\PYG{n+nv}{largerthan}\PYG{o}{=}20kb\PYG{o}{)}
\end{sphinxVerbatim}

\sphinxAtStartPar
to store only files exceeding 20KB in size in git\sphinxhyphen{}annex%
\begin{footnote}\sphinxAtStartFootnote
Specifying annex.largefiles in your .gitattributes file will make the configuration
“portable” \textendash{} shared copies of your dataset will retain these configurations.
You could however also set largefiles rules in your \sphinxcode{\sphinxupquote{.git/config}} file. Rules
specified in there take precedence over rules in \sphinxcode{\sphinxupquote{.gitattributes}}. You can set
them using the \sphinxcode{\sphinxupquote{git config}} command:

\sphinxSetupCodeBlockInFootnote
\begin{sphinxVerbatim}[commandchars=\\\{\}]
\PYG{g+gp}{\PYGZdl{} }git\PYG{+w}{ }config\PYG{+w}{ }annex.largefiles\PYG{+w}{ }\PYG{l+s+s1}{\PYGZsq{}largerthan=100kb and not (include=*.c or include=*.h)\PYGZsq{}}
\end{sphinxVerbatim}

\sphinxAtStartPar
The above command annexes files larger than 100KB, and will never annex files with a
\sphinxcode{\sphinxupquote{.c}} or \sphinxcode{\sphinxupquote{.h}} extension.
%
\end{footnote}.

\sphinxAtStartPar
As you may have noticed, unlike \sphinxcode{\sphinxupquote{.git/config}} files,
there can be multiple \sphinxcode{\sphinxupquote{.gitattributes}} files within a dataset. So far, you have seen one
in the root of the superdataset, and in the root of the \sphinxcode{\sphinxupquote{longnow}} subdataset.
In principle, you can add one to every directory\sphinxhyphen{}level of your dataset.
For example, there is another \sphinxcode{\sphinxupquote{.gitattributes}} file within the
\sphinxcode{\sphinxupquote{.datalad}} directory:

\begin{sphinxVerbatim}[commandchars=\\\{\}]
\PYG{g+gp}{\PYGZdl{} }cat\PYG{+w}{ }.datalad/.gitattributes
\PYG{g+go}{config annex.largefiles=nothing}
\end{sphinxVerbatim}

\sphinxAtStartPar
As with Git configuration files, more specific or lower\sphinxhyphen{}level configurations take precedence
over more general or higher\sphinxhyphen{}level configurations. Specifications in a subdirectory can
therefore overrule specifications made in the \sphinxcode{\sphinxupquote{.gitattributes}} file of the parent
directory.

\sphinxAtStartPar
In summary, the \sphinxcode{\sphinxupquote{.gitattributes}} files will give you the possibility to configure
what should be annexed and what should not be annexed up to individual file level.
This can be very handy, and allows you to tune your dataset to your custom needs.
For example, files you will often edit by hand could be stored in Git if they are
not too large to ease modifying them%
\begin{footnote}\sphinxAtStartFootnote
Should you ever need to, this file is also where one would change the git\sphinxhyphen{}annex
backend in order to store new files with a new backend. Switching the backend of
\sphinxstyleemphasis{all} files (new as well as existing ones) requires the \sphinxcode{\sphinxupquote{git annex migrate}}
command
(see \dlhbhref{B1E}{the documentation} for
more information on this command).
%
\end{footnote}.
Once you know the basics of this type of configuration syntax, writing
your own rules is easy. For more tips on how configure git\sphinxhyphen{}annex’s content
management in \sphinxcode{\sphinxupquote{.gitattributes}}, take a look at \dlhbhref{B1O}{the git\sphinxhyphen{}annex documentation}.
Later however you will see preconfigured DataLad \sphinxstyleemphasis{procedures} such as \sphinxcode{\sphinxupquote{text2git}} that
can apply useful configurations for you, just as \sphinxcode{\sphinxupquote{text2git}} added the last line
in the root \sphinxcode{\sphinxupquote{.gitattributes}} file.

\index{configuration file@\spxentry{configuration file}!.gitmodules@\spxentry{.gitmodules}|spxpagem}\ignorespaces 

\subsection{\sphinxstyleliteralintitle{\sphinxupquote{.gitmodules}}}
\label{\detokenize{basics/101-123-config2:gitmodules}}\label{\detokenize{basics/101-123-config2:index-2}}
\sphinxAtStartPar
On last configuration file that Git creates is the \sphinxcode{\sphinxupquote{.gitmodules}} file.
There is one right in the root of your dataset:

\begin{sphinxVerbatim}[commandchars=\\\{\}]
\PYG{g+gp}{\PYGZdl{} }cat\PYG{+w}{ }.gitmodules
\PYG{g+go}{[submodule \PYGZdq{}recordings/longnow\PYGZdq{}]}
\PYG{g+go}{	path = recordings/longnow}
\PYG{g+go}{	url = https://github.com/datalad\PYGZhy{}datasets/longnow\PYGZhy{}podcasts.git}
\PYG{g+go}{	datalad\PYGZhy{}id = b3ca2718\PYGZhy{}8901\PYGZhy{}11e8\PYGZhy{}99aa\PYGZhy{}a0369f7c647e}
\PYG{g+go}{	datalad\PYGZhy{}url = https://github.com/datalad\PYGZhy{}datasets/longnow\PYGZhy{}podcasts.git}
\end{sphinxVerbatim}

\sphinxAtStartPar
Based on these contents, you might have already guessed what this file
stores. The \sphinxcode{\sphinxupquote{.gitmodules}} file is a configuration file that stores the mapping between
your own dataset and any subdatasets you have installed in it.
There will be an entry for each submodule (subdataset) in your dataset.
The name \sphinxstyleemphasis{submodule} is Git terminology, and describes a Git repository inside of
another Git repository, i.e., the super\sphinxhyphen{} and subdataset principles.
Upon sharing your dataset, the information about subdatasets and where to retrieve
them from is stored and shared with this file.
In addition to modifying it with the \sphinxcode{\sphinxupquote{git config}} command or by hand, the \sphinxcode{\sphinxupquote{datalad subdatasets}} command also has a \sphinxcode{\sphinxupquote{\sphinxhyphen{}\sphinxhyphen{}set\sphinxhyphen{}property NAME VALUE}} option that you can use to set subdataset properties.

\sphinxAtStartPar
Section {\hyperref[\detokenize{basics/101-116-sharelocal:sharelocal1}]{\sphinxcrossref{\DUrole{std,std-ref}{Looking without touching}}}} (\autopageref*{\detokenize{basics/101-116-sharelocal:sharelocal1}}) already mentioned one additional configuration option in a footnote: The \sphinxcode{\sphinxupquote{datalad\sphinxhyphen{}recursiveinstall}} key.
This key is defined on a per subdataset basis, and if set to “\sphinxcode{\sphinxupquote{skip}}”, the given subdataset will not be recursively installed unless it is explicitly specified as a path to \sphinxcode{\sphinxupquote{datalad get {[}\sphinxhyphen{}n/\sphinxhyphen{}\sphinxhyphen{}no\sphinxhyphen{}data{]} \sphinxhyphen{}r}}.
If you are a maintainer of a superdataset with monstrous amounts of subdatasets, you can set this option and share it together with the dataset to prevent an accidental, large recursive installation in particularly deeply nested subdatasets.
Below is a minimally functional example on how to apply the configuration and how it works:

\sphinxAtStartPar
Let’s create a dataset hierarchy to work with (note that we concatenate multiple commands into a single line using bash’s “and” \sphinxcode{\sphinxupquote{\&\&}} operator):

% beauty pagebreak
\newpage

\begin{sphinxVerbatim}[commandchars=\\\{\}]
\PYG{g+gp}{\PYGZdl{} }\PYG{c+c1}{\PYGZsh{} create a superdataset with two subdatasets}
\PYG{g+gp}{\PYGZdl{} }datalad\PYG{+w}{ }create\PYG{+w}{ }superds\PYG{+w}{ }\PYG{o}{\PYGZam{}\PYGZam{}}\PYG{+w}{ }datalad\PYG{+w}{ }\PYGZhy{}C\PYG{+w}{ }superds\PYG{+w}{ }create\PYG{+w}{ }\PYGZhy{}d\PYG{+w}{ }.\PYG{+w}{ }subds1\PYG{+w}{ }\PYG{o}{\PYGZam{}\PYGZam{}}\PYG{+w}{ }datalad\PYG{+w}{ }\PYGZhy{}C\PYG{+w}{ }superds\PYG{+w}{ }create\PYG{+w}{ }\PYGZhy{}d\PYG{+w}{ }.\PYG{+w}{ }subds2
\PYG{g+go}{create(ok): /tmp/superds (dataset)}
\PYG{g+go}{add(ok): subds1 (file)}
\PYG{g+go}{add(ok): .gitmodules (file)}
\PYG{g+go}{save(ok): . (dataset)}
\PYG{g+go}{create(ok): subds1 (dataset)}
\PYG{g+go}{add(ok): subds2 (file)}
\PYG{g+go}{add(ok): .gitmodules (file)}
\PYG{g+go}{save(ok): . (dataset)}
\PYG{g+go}{create(ok): subds2 (dataset)}
\end{sphinxVerbatim}

\sphinxAtStartPar
Next, we create subdatasets in the subdatasets:

\begin{sphinxVerbatim}[commandchars=\\\{\}]
\PYG{g+gp}{\PYGZdl{} }\PYG{c+c1}{\PYGZsh{} create two subdatasets in subds1}
\PYG{g+gp}{\PYGZdl{} }datalad\PYG{+w}{ }\PYGZhy{}C\PYG{+w}{ }superds/subds1\PYG{+w}{ }create\PYG{+w}{ }\PYGZhy{}d\PYG{+w}{ }.\PYG{+w}{ }subsubds1\PYG{+w}{ }\PYG{o}{\PYGZam{}\PYGZam{}}\PYG{+w}{ }datalad\PYG{+w}{ }\PYGZhy{}C\PYG{+w}{ }superds/subds1\PYG{+w}{ }create\PYG{+w}{ }\PYGZhy{}d\PYG{+w}{ }.\PYG{+w}{ }subsubds2
\PYG{g+go}{add(ok): subsubds1 (file)}
\PYG{g+go}{add(ok): .gitmodules (file)}
\PYG{g+go}{save(ok): . (dataset)}
\PYG{g+go}{create(ok): subsubds1 (dataset)}
\PYG{g+go}{add(ok): subsubds2 (file)}
\PYG{g+go}{add(ok): .gitmodules (file)}
\PYG{g+go}{save(ok): . (dataset)}
\PYG{g+go}{create(ok): subsubds2 (dataset)}

\PYG{g+gp}{\PYGZdl{} }\PYG{c+c1}{\PYGZsh{} create two subdatasets in subds2}
\PYG{g+gp}{\PYGZdl{} }datalad\PYG{+w}{ }\PYGZhy{}C\PYG{+w}{ }superds/subds2\PYG{+w}{ }create\PYG{+w}{ }\PYGZhy{}d\PYG{+w}{ }.\PYG{+w}{ }subsubds1\PYG{+w}{ }\PYG{o}{\PYGZam{}\PYGZam{}}\PYG{+w}{ }datalad\PYG{+w}{ }\PYGZhy{}C\PYG{+w}{ }superds/subds2\PYG{+w}{ }create\PYG{+w}{ }\PYGZhy{}d\PYG{+w}{ }.\PYG{+w}{ }subsubds2
\PYG{g+go}{add(ok): subsubds1 (file)}
\PYG{g+go}{add(ok): .gitmodules (file)}
\PYG{g+go}{save(ok): . (dataset)}
\PYG{g+go}{create(ok): subsubds1 (dataset)}
\PYG{g+go}{add(ok): subsubds2 (file)}
\PYG{g+go}{add(ok): .gitmodules (file)}
\PYG{g+go}{save(ok): . (dataset)}
\PYG{g+go}{create(ok): subsubds2 (dataset)}
\end{sphinxVerbatim}

\sphinxAtStartPar
Here is the directory structure:

\begin{sphinxVerbatim}[commandchars=\\\{\}]
\PYG{g+gp}{\PYGZdl{} }\PYG{n+nb}{cd}\PYG{+w}{ }../\PYG{+w}{ }\PYG{o}{\PYGZam{}\PYGZam{}}\PYG{+w}{ }tree
\PYG{g+go}{.}
\PYG{g+go}{├── subds1}
\PYG{g+go}{│   ├── subsubds1}
\PYG{g+go}{│   └── subsubds2}
\PYG{g+go}{└── subds2}
\PYG{g+go}{    ├── subsubds1}
\PYG{g+go}{    └── subsubds2}

\PYG{g+gp}{\PYGZdl{} }\PYG{c+c1}{\PYGZsh{} save in the superdataset}
\PYG{g+go}{datalad save \PYGZhy{}m \PYGZdq{}add a few sub and subsub datasets\PYGZdq{}}
\PYG{g+go}{add(ok): subds1 (file)}
\PYG{g+go}{add(ok): subds2 (file)}
\PYG{g+go}{save(ok): . (dataset)}
\end{sphinxVerbatim}

\sphinxAtStartPar
Now, we can apply the \sphinxcode{\sphinxupquote{datalad\sphinxhyphen{}recursiveinstall}} configuration to skip recursive installations for \sphinxcode{\sphinxupquote{subds1}}

\begin{sphinxVerbatim}[commandchars=\\\{\}]
\PYG{g+gp}{\PYGZdl{} }git\PYG{+w}{ }config\PYG{+w}{ }\PYGZhy{}f\PYG{+w}{ }.gitmodules\PYG{+w}{ }\PYGZhy{}\PYGZhy{}add\PYG{+w}{ }submodule.subds1.datalad\PYGZhy{}recursiveinstall\PYG{+w}{ }skip

\PYG{g+gp}{\PYGZdl{} }\PYG{c+c1}{\PYGZsh{} save this configuration}
\PYG{g+gp}{\PYGZdl{} }datalad\PYG{+w}{ }save\PYG{+w}{ }\PYGZhy{}m\PYG{+w}{ }\PYG{l+s+s2}{\PYGZdq{}prevent recursion into subds1, unless explicitly given as path\PYGZdq{}}
\PYG{g+go}{add(ok): .gitmodules (file)}
\PYG{g+go}{save(ok): . (dataset)}
\end{sphinxVerbatim}

\sphinxAtStartPar
If the dataset is cloned, and someone runs a recursive \sphinxcode{\sphinxupquote{datalad get}}, the subdatasets of \sphinxcode{\sphinxupquote{subds1}} will not be installed, the subdatasets of \sphinxcode{\sphinxupquote{subds2}}, however, will be.

\begin{sphinxVerbatim}[commandchars=\\\{\}]
\PYG{g+gp}{\PYGZdl{} }\PYG{c+c1}{\PYGZsh{} clone the dataset somewhere else}
\PYG{g+gp}{\PYGZdl{} }\PYG{n+nb}{cd}\PYG{+w}{ }../\PYG{+w}{ }\PYG{o}{\PYGZam{}\PYGZam{}}\PYG{+w}{ }datalad\PYG{+w}{ }clone\PYG{+w}{ }superds\PYG{+w}{ }clone\PYGZus{}of\PYGZus{}superds
\PYG{g+go}{[INFO   ] Cloning superds into \PYGZsq{}/tmp/clone\PYGZus{}of\PYGZus{}superds\PYGZsq{}}
\PYG{g+go}{install(ok): /tmp/clone\PYGZus{}of\PYGZus{}superds (dataset)}

\PYG{g+gp}{\PYGZdl{} }\PYG{c+c1}{\PYGZsh{} recursively get all contents (without data)}
\PYG{g+gp}{\PYGZdl{} }\PYG{n+nb}{cd}\PYG{+w}{ }clone\PYGZus{}of\PYGZus{}superds\PYG{+w}{ }\PYG{o}{\PYGZam{}\PYGZam{}}\PYG{+w}{ }datalad\PYG{+w}{ }get\PYG{+w}{ }\PYGZhy{}n\PYG{+w}{ }\PYGZhy{}r\PYG{+w}{ }.
\PYG{g+go}{get(ok): /tmp/clone\PYGZus{}of\PYGZus{}superds/subds2 (dataset)}
\PYG{g+go}{get(ok): /tmp/clone\PYGZus{}of\PYGZus{}superds/subds2/subsubds1 (dataset)}
\PYG{g+go}{get(ok): /tmp/clone\PYGZus{}of\PYGZus{}superds/subds2/subsubds2 (dataset)}

\PYG{g+gp}{\PYGZdl{} }\PYG{c+c1}{\PYGZsh{} only subsubds of subds2 are installed, not of subds1:}
\PYG{g+gp}{\PYGZdl{} }tree
\PYG{g+go}{.}
\PYG{g+go}{├── subds1}
\PYG{g+go}{└── subds2}
\PYG{g+go}{    ├── subsubds1}
\PYG{g+go}{    └── subsubds2}

\PYG{g+go}{4 directories, 0 files}
\end{sphinxVerbatim}

\sphinxAtStartPar
Nevertheless, if \sphinxcode{\sphinxupquote{subds1}} is provided with an explicit path, its subdataset \sphinxcode{\sphinxupquote{subsubds}} will be cloned, essentially overriding the configuration:

% beauty pagebreak
\newpage

\begin{sphinxVerbatim}[commandchars=\\\{\}]
\PYG{g+gp}{\PYGZdl{}  }datalad\PYG{+w}{ }get\PYG{+w}{ }\PYGZhy{}n\PYG{+w}{ }\PYGZhy{}r\PYG{+w}{ }subds1\PYG{+w}{ }\PYG{o}{\PYGZam{}\PYGZam{}}\PYG{+w}{ }tree
\PYG{g+go}{install(ok): /tmp/clone\PYGZus{}of\PYGZus{}superds/subds1 (dataset) [Installed subdataset in order to get /tmp/clone\PYGZus{}of\PYGZus{}superds/subds1]}
\PYG{g+go}{.}
\PYG{g+go}{├── subds1}
\PYG{g+go}{│   ├── subsubds1}
\PYG{g+go}{│   └── subsubds2}
\PYG{g+go}{└── subds2}
\PYG{g+go}{    ├── subsubds1}
\PYG{g+go}{    └── subsubds2}

\PYG{g+go}{6 directories, 0 files}
\end{sphinxVerbatim}

\index{configuration file@\spxentry{configuration file}!.datalad/config@\spxentry{.datalad/config}|spxpagem}\ignorespaces 

\subsection{\sphinxstyleliteralintitle{\sphinxupquote{.datalad/config}}}
\label{\detokenize{basics/101-123-config2:datalad-config}}\label{\detokenize{basics/101-123-config2:index-3}}
\sphinxAtStartPar
DataLad adds a repository\sphinxhyphen{}specific configuration file as well.
It can be found in the \sphinxcode{\sphinxupquote{.datalad}} directory, and just like \sphinxcode{\sphinxupquote{.gitattributes}}
and \sphinxcode{\sphinxupquote{.gitmodules}} it is version controlled and is thus shared together with
the dataset. One can configure
\dlhbhref{D1A}{many options},
but currently, our \sphinxcode{\sphinxupquote{.datalad/config}} file only stores a {\hyperref[\detokenize{glossary:term-dataset-ID}]{\sphinxtermref{\DUrole{xref,std,std-term}{dataset ID}}}}.
This ID serves to identify a dataset as a unit, across its entire history and flavors.
In a geeky way, this is your dataset’s social security number: It will only exist
one time on this planet.

\begin{sphinxVerbatim}[commandchars=\\\{\}]
\PYG{g+gp}{\PYGZdl{} }cat\PYG{+w}{ }.datalad/config
\PYG{g+go}{[datalad \PYGZdq{}dataset\PYGZdq{}]}
\PYG{g+go}{	id = e3e70682\PYGZhy{}c209\PYGZhy{}4cac\PYGZhy{}629f\PYGZhy{}6fbed82c07cd}
\end{sphinxVerbatim}

\sphinxAtStartPar
Note, though, that local configurations within a Git configuration file
will take precedence over configurations that can be distributed with a dataset.
Otherwise, dataset updates with \sphinxcode{\sphinxupquote{datalad update}} (or, for Git\sphinxhyphen{}users,
\sphinxcode{\sphinxupquote{git pull}}) could suddenly and unintentionally alter local DataLad
behavior that was specifically configured.
Also, {\hyperref[\detokenize{glossary:term-Git}]{\sphinxtermref{\DUrole{xref,std,std-term}{Git}}}} and {\hyperref[\detokenize{glossary:term-git-annex}]{\sphinxtermref{\DUrole{xref,std,std-term}{git\sphinxhyphen{}annex}}}} will not query this file for configurations, so please store only sticky options that are specific to DataLad (i.e., under the \sphinxcode{\sphinxupquote{datalad.*}} namespace) in it.

\index{modify configuration@\spxentry{modify configuration}!with Git@\spxentry{with Git}}\index{with Git@\spxentry{with Git}!modify configuration@\spxentry{modify configuration}}\ignorespaces 

\subsection{Writing to configuration files other than \sphinxstyleliteralintitle{\sphinxupquote{.git/config}}}
\label{\detokenize{basics/101-123-config2:writing-to-configuration-files-other-than-git-config}}\label{\detokenize{basics/101-123-config2:index-4}}
\sphinxAtStartPar
“Didn’t you say that knowing the \sphinxcode{\sphinxupquote{git config}} command is already
half of what I need to know?” you ask. “Now there are three other configuration
files, and I do not know with which command I can write into these files.”

\sphinxAtStartPar
“Excellent question”, you hear in return, “but in reality, you \sphinxstylestrong{do} know:
it’s also the \sphinxcode{\sphinxupquote{git config}} command. The only part of it you need to
adjust is the \sphinxcode{\sphinxupquote{\sphinxhyphen{}f}}, \sphinxcode{\sphinxupquote{\sphinxhyphen{}\sphinxhyphen{}file}} parameter. By default, the command writes to
a Git config file. But it can write to a different file if you specify it
appropriately. For example,
\begin{quote}

\sphinxAtStartPar
\sphinxcode{\sphinxupquote{git config \sphinxhyphen{}\sphinxhyphen{}file=.gitmodules \sphinxhyphen{}\sphinxhyphen{}replace\sphinxhyphen{}all submodule."name".url "new URL"}}
\end{quote}

\sphinxAtStartPar
will update your submodule’s URL. Keep in mind though that you would need
to commit this change, as \sphinxcode{\sphinxupquote{.gitmodules}} is version controlled”.

\sphinxAtStartPar
Let’s try this:

\begin{sphinxVerbatim}[commandchars=\\\{\}]
\PYG{g+gp}{\PYGZdl{} }git\PYG{+w}{ }config\PYG{+w}{ }\PYGZhy{}\PYGZhy{}file\PYG{o}{=}.gitmodules\PYG{+w}{ }\PYGZhy{}\PYGZhy{}replace\PYGZhy{}all\PYG{+w}{ }submodule.\PYG{l+s+s2}{\PYGZdq{}recordings/longnow\PYGZdq{}}.url\PYG{+w}{ }\PYG{l+s+s2}{\PYGZdq{}git@github.com:datalad\PYGZhy{}datasets/longnow\PYGZhy{}podcasts.git\PYGZdq{}}
\end{sphinxVerbatim}

\sphinxAtStartPar
This command will replace the submodule’s https URL with an SSH URL.
The latter is often used if someone has an \sphinxstyleemphasis{SSH key pair} and added the
public key to their GitHub account (you can read more about this
\dlhbhref{G3C}{here}).
We will revert this change shortly, but use it to show the difference between
a \sphinxcode{\sphinxupquote{git config}} on a \sphinxcode{\sphinxupquote{.git/config}} file and on a version controlled file:

\begin{sphinxVerbatim}[commandchars=\\\{\}]
\PYG{g+gp}{\PYGZdl{} }datalad\PYG{+w}{ }status
\PYG{g+go}{ modified: .gitmodules (file)}
\end{sphinxVerbatim}

\begin{sphinxVerbatim}[commandchars=\\\{\}]
\PYG{g+gp}{\PYGZdl{} }git\PYG{+w}{ }diff
\PYG{g+go}{diff \PYGZhy{}\PYGZhy{}git a/.gitmodules b/.gitmodules}
\PYG{g+go}{index 9bc9ee9..11273e1 100644}
\PYG{g+go}{\PYGZhy{}\PYGZhy{}\PYGZhy{} a/.gitmodules}
\PYG{g+go}{+++ b/.gitmodules}
\PYG{g+go}{@@ \PYGZhy{}1,5 +1,5 @@}
\PYG{g+go}{ [submodule \PYGZdq{}recordings/longnow\PYGZdq{}]}
\PYG{g+go}{ 	path = recordings/longnow}
\PYG{g+go}{\PYGZhy{}	url = https://github.com/datalad\PYGZhy{}datasets/longnow\PYGZhy{}podcasts.git}
\PYG{g+go}{+	url = git@github.com:datalad\PYGZhy{}datasets/longnow\PYGZhy{}podcasts.git}
\PYG{g+go}{ 	datalad\PYGZhy{}id = b3ca2718\PYGZhy{}8901\PYGZhy{}11e8\PYGZhy{}99aa\PYGZhy{}a0369f7c647e}
\PYG{g+go}{ 	datalad\PYGZhy{}url = https://github.com/datalad\PYGZhy{}datasets/longnow\PYGZhy{}podcasts.git}
\end{sphinxVerbatim}

\sphinxAtStartPar
As these two commands show, the \sphinxcode{\sphinxupquote{.gitmodules}} file is modified. The https URL
has been deleted (note the \sphinxcode{\sphinxupquote{\sphinxhyphen{}}}), and a SSH URL has been added. To keep these
changes, we would need to \sphinxcode{\sphinxupquote{datalad save}} them. However, as we want to stay with
https URLs, we will just \sphinxstyleemphasis{checkout} this change \textendash{} using a Git tool to undo an
unstaged modification.

\begin{sphinxVerbatim}[commandchars=\\\{\}]
\PYG{g+gp}{\PYGZdl{} }git\PYG{+w}{ }checkout\PYG{+w}{ }.gitmodules
\PYG{g+gp}{\PYGZdl{} }datalad\PYG{+w}{ }status
\PYG{g+go}{Updated 1 path from the index}
\PYG{g+go}{nothing to save, working tree clean}
\end{sphinxVerbatim}

\sphinxAtStartPar
Note, though, that the \sphinxcode{\sphinxupquote{.gitattributes}} file cannot be modified with a \sphinxcode{\sphinxupquote{git config}}
command. This is due to its different format that does not comply to the
\sphinxcode{\sphinxupquote{section.variable.value}} structure of all other configuration files. This file, therefore,
has to be edited by hand, with an editor of your choice.

\index{environment variable@\spxentry{environment variable}|spxpagem}\ignorespaces 

\subsection{Environment variables}
\label{\detokenize{basics/101-123-config2:environment-variables}}\label{\detokenize{basics/101-123-config2:envvars}}\label{\detokenize{basics/101-123-config2:index-5}}
\sphinxAtStartPar
An {\hyperref[\detokenize{glossary:term-environment-variable}]{\sphinxtermref{\DUrole{xref,std,std-term}{environment variable}}}} is a variable set up in your shell
that affects the way the shell or certain software works \textendash{} for example,
the environment variables \sphinxcode{\sphinxupquote{HOME}}, \sphinxcode{\sphinxupquote{PWD}}, or \sphinxcode{\sphinxupquote{PATH}}.
Configuration options that determine the behavior of Git, git\sphinxhyphen{}annex, and
DataLad that could be defined in a configuration file can also be set (or overridden)
by the associated environment variables of these configuration options.
Many configuration items have associated environment variables.
If this environment variable is set, it takes precedence over options set in
configuration files, thus providing both an alternative way to define configurations
as well as an override mechanism. For example, the \sphinxcode{\sphinxupquote{user.name}}
configuration of Git can be overridden by its associated environment variable,
\sphinxcode{\sphinxupquote{GIT\_AUTHOR\_NAME}}. Likewise, one can define the environment variable instead
of setting the \sphinxcode{\sphinxupquote{user.name}} configuration in a configuration file.

\index{configuration item@\spxentry{configuration item}!datalad.log.level@\spxentry{datalad.log.level}}\ignorespaces 
\sphinxAtStartPar
Git, git\sphinxhyphen{}annex, and DataLad have more environment variables than anyone would want to
remember. \dlhbhref{G1C}{The ProGit book}
has a good overview on Git’s most useful available environment variables for a start.
All of DataLad’s configuration options can be translated to their
associated environment variables. Any environment variable with a name that starts with \sphinxcode{\sphinxupquote{DATALAD\_}}
will be available as the corresponding \sphinxcode{\sphinxupquote{datalad.}} configuration variable,
replacing any \sphinxcode{\sphinxupquote{\_\_}} (two underscores) with a hyphen, then any \sphinxcode{\sphinxupquote{\_}} (single underscore)
with a dot, and finally converting all letters to lower case. The \sphinxcode{\sphinxupquote{datalad.log.level}}
configuration option thus is the environment variable \sphinxcode{\sphinxupquote{DATALAD\_LOG\_LEVEL}}.
If you are new to the concept of environment variables, check out the
\textit{Find-out-more}~{\findoutmoreiconinline}\textit{\ref{fom-envvar}} {\hyperref[\detokenize{basics/101-123-config2:fom-envvar}]{\sphinxcrossref{\DUrole{std,std-ref}{on them}}}} (\autopageref*{\detokenize{basics/101-123-config2:fom-envvar}}).

\index{operating system concept@\spxentry{operating system concept}!environment variable@\spxentry{environment variable}}\ignorespaces \begin{findoutmore}[label={fom-envvar}, before title={\thetcbcounter\ }, float, floatplacement=tb, check odd page=true]{Some more general information on environment variables}
\label{\detokenize{basics/101-123-config2:fom-envvar}}

\sphinxAtStartPar
Names of environment variables are often all\sphinxhyphen{}uppercase, but they need not be.
While the \sphinxcode{\sphinxupquote{\$}} is not part of
the name of the environment variable, it is necessary to \sphinxstyleemphasis{refer} to the environment
variable: To reference the value of the environment variable \sphinxcode{\sphinxupquote{HOME}}, for example, you would
need to use \sphinxcode{\sphinxupquote{echo \$HOME}} and not \sphinxcode{\sphinxupquote{echo HOME}}. However, environment variables are
set without a leading \sphinxcode{\sphinxupquote{\$}}. There are several ways to set an environment variable
(note that there are no spaces before and after the equals sign!), leading to different
levels of availability of the variable:
\begin{itemize}
\item {} 
\sphinxAtStartPar
\sphinxcode{\sphinxupquote{THEANSWER=42 \textless{}command\textgreater{}}} makes the variable \sphinxcode{\sphinxupquote{THEANSWER}} available for the process in \sphinxcode{\sphinxupquote{\textless{}command\textgreater{}}}.
For example, \sphinxcode{\sphinxupquote{DATALAD\_LOG\_LEVEL=debug datalad get \textless{}file\textgreater{}}} will execute the \sphinxcode{\sphinxupquote{datalad get}}
command (and only this one) with the log level set to “debug”.

\item {} 
\sphinxAtStartPar
\sphinxcode{\sphinxupquote{export THEANSWER=42}} makes the variable \sphinxcode{\sphinxupquote{THEANSWER}} available for other processes in the
same session, but it will not be available to other shells.

\item {} 
\sphinxAtStartPar
\sphinxcode{\sphinxupquote{echo \textquotesingle{}export THEANSWER=42\textquotesingle{} \textgreater{}\textgreater{} \textasciitilde{}/.bashrc}} will write the variable definition in the
\sphinxcode{\sphinxupquote{.bashrc}} file and thus available to all future shells of the user (i.e., this will make
the variable permanent for the user)

\end{itemize}

\sphinxAtStartPar
To list all of the configured environment variables, type \sphinxcode{\sphinxupquote{env}} into your terminal.


\end{findoutmore}


\subsection{Summary}
\label{\detokenize{basics/101-123-config2:summary}}
\sphinxAtStartPar
This has been an intense lecture, you have to admit. One definite
take\sphinxhyphen{}away from it has been that you now know a second reason why the hidden
\sphinxcode{\sphinxupquote{.git}} and \sphinxcode{\sphinxupquote{.datalad}} directory contents and also the contents of \sphinxcode{\sphinxupquote{.gitmodules}} and
\sphinxcode{\sphinxupquote{.gitattributes}} should not be carelessly tampered with \textendash{} they contain all of
the repository’s configurations.

\sphinxAtStartPar
But you now also know how to modify these configurations with enough
care and background knowledge such that nothing should go wrong once you
want to work with and change them. You can use the \sphinxcode{\sphinxupquote{git config}} command
for Git configuration files on different scopes, and even the \sphinxcode{\sphinxupquote{.gitmodules}} or \sphinxcode{\sphinxupquote{datalad/config}}
files. Of course you do not yet know all of the available configuration options. However,
you already know some core Git configurations such as name, email, and editor. Even more
important, you know how to configure git\sphinxhyphen{}annex’s content management based on \sphinxcode{\sphinxupquote{largefile}}
rules, and you understand the  variables within \sphinxcode{\sphinxupquote{.gitmodules}} or the sections
in \sphinxcode{\sphinxupquote{.git/config}}. Slowly, you realize with pride,
you are more and more becoming a DataLad power\sphinxhyphen{}user.

\sphinxAtStartPar
Write a note about configurations in datasets into \sphinxcode{\sphinxupquote{notes.txt}}.

\begin{sphinxVerbatim}[commandchars=\\\{\}]
\PYG{g+gp}{\PYGZdl{} }cat\PYG{+w}{ }\PYGZlt{}\PYGZlt{}\PYG{+w}{ }EOT\PYG{+w}{ }\PYGZgt{}\PYGZgt{}\PYG{+w}{ }notes.txt
\PYG{g+go}{Configurations for datasets exist on different levels (systemwide,}
\PYG{g+go}{global, and local), and in different types of files (not version}
\PYG{g+go}{controlled (git)config files, or version controlled .datalad/config,}
\PYG{g+go}{.gitattributes, or gitmodules files), or environment variables.}
\PYG{g+go}{With the exception of .gitattributes, all configuration files share a}
\PYG{g+go}{common structure, and can be modified with the git config command, but}
\PYG{g+go}{also with an editor by hand.}

\PYG{g+go}{Depending on whether a configuration file is version controlled or}
\PYG{g+go}{not, the configurations will be shared together with the dataset.}
\PYG{g+go}{More specific configurations and not\PYGZhy{}shared configurations will always}
\PYG{g+go}{take precedence over more global or hared configurations, and}
\PYG{g+go}{environment variables take precedence over configurations in files.}

\PYG{g+go}{The git config \PYGZhy{}\PYGZhy{}list \PYGZhy{}\PYGZhy{}show\PYGZhy{}origin command is a useful tool to give}
\PYG{g+go}{an overview over existing configurations. Particularly important may}
\PYG{g+go}{be the .gitattributes file, in which one can set rules for git\PYGZhy{}annex}
\PYG{g+go}{about which files should be version\PYGZhy{}controlled with Git instead of}
\PYG{g+go}{being annexed.}

\PYG{g+go}{EOT}
\end{sphinxVerbatim}

\begin{sphinxVerbatim}[commandchars=\\\{\}]
\PYG{g+gp}{\PYGZdl{} }datalad\PYG{+w}{ }save\PYG{+w}{ }\PYGZhy{}m\PYG{+w}{ }\PYG{l+s+s2}{\PYGZdq{}add note on configurations and git config\PYGZdq{}}
\PYG{g+go}{add(ok): notes.txt (file)}
\PYG{g+go}{save(ok): . (dataset)}
\end{sphinxVerbatim}

\sphinxstepscope

\index{procedures@\spxentry{procedures}|spxpagem}\index{run\sphinxhyphen{}procedures@\spxentry{run\sphinxhyphen{}procedures}}\index{standardized configuration@\spxentry{standardized configuration}!with DataLad@\spxentry{with DataLad}}\index{with DataLad@\spxentry{with DataLad}!standardized configuration@\spxentry{standardized configuration}}\ignorespaces 

\section{Configurations to go}
\label{\detokenize{basics/101-124-procedures:configurations-to-go}}\label{\detokenize{basics/101-124-procedures:procedures}}\label{\detokenize{basics/101-124-procedures:index-0}}\label{\detokenize{basics/101-124-procedures::doc}}
\sphinxAtStartPar
The past two sections should have given you a comprehensive
overview on the different configuration options the tools
Git, git\sphinxhyphen{}annex, and DataLad provide. They not only
showed you a way to configure everything you may need to
configure, but also gave explanations about what the
configuration options actually mean.

\sphinxAtStartPar
But figuring out which configurations are useful and how
to apply them are also not the easiest tasks. Therefore,
some clever people decided to assist with
these tasks, and created pre\sphinxhyphen{}configured \sphinxstyleemphasis{procedures}
that process datasets in a particular way.
These procedures can be shipped within DataLad or its extensions,
lie on a system, or can be shared together with datasets.

\sphinxAtStartPar
One of such procedures is the \sphinxcode{\sphinxupquote{text2git}} configuration.
In order to learn about procedures in general, let’s demystify
what the \sphinxcode{\sphinxupquote{text2git}} procedure exactly is: It is
nothing more than a simple script that
\begin{itemize}
\item {} 
\sphinxAtStartPar
writes the relevant \sphinxcode{\sphinxupquote{annex\_largefiles}} configuration,  i.e., “Do not put anything that is a text file in the annex”) to the \sphinxcode{\sphinxupquote{.gitattributes}} file of a dataset, and

\item {} 
\sphinxAtStartPar
saves this modification with the commit message “Instruct annex to add text files to Git”.

\end{itemize}

\sphinxAtStartPar
This particular procedure lives in a script called
\sphinxcode{\sphinxupquote{cfg\_text2git}} in the sourcecode of DataLad. The amount of code
in this script is not large, and the relevant lines of code
are highlighted:

\fvset{hllines={, 12, 16, 17,}}%
\begin{sphinxVerbatim}[commandchars=\\\{\}]
 \PYG{k+kn}{import} \PYG{n+nn}{sys}
 \PYG{k+kn}{import} \PYG{n+nn}{os}\PYG{n+nn}{.}\PYG{n+nn}{path} \PYG{k}{as} \PYG{n+nn}{op}

 \PYG{k+kn}{from} \PYG{n+nn}{datalad}\PYG{n+nn}{.}\PYG{n+nn}{distribution}\PYG{n+nn}{.}\PYG{n+nn}{dataset} \PYG{k+kn}{import} \PYG{n}{require\PYGZus{}dataset}

 \PYG{n}{ds} \PYG{o}{=} \PYG{n}{require\PYGZus{}dataset}\PYG{p}{(}
     \PYG{n}{sys}\PYG{o}{.}\PYG{n}{argv}\PYG{p}{[}\PYG{l+m+mi}{1}\PYG{p}{]}\PYG{p}{,}
     \PYG{n}{check\PYGZus{}installed}\PYG{o}{=}\PYG{k+kc}{True}\PYG{p}{,}
     \PYG{n}{purpose}\PYG{o}{=}\PYG{l+s+s1}{\PYGZsq{}}\PYG{l+s+s1}{configuration}\PYG{l+s+s1}{\PYGZsq{}}\PYG{p}{)}

 \PYG{c+c1}{\PYGZsh{} the relevant configuration:}
 \PYG{n}{annex\PYGZus{}largefiles} \PYG{o}{=} \PYG{l+s+s1}{\PYGZsq{}}\PYG{l+s+s1}{((mimeencoding=binary)and(largerthan=0))}\PYG{l+s+s1}{\PYGZsq{}}
 \PYG{n}{attrs} \PYG{o}{=} \PYG{n}{ds}\PYG{o}{.}\PYG{n}{repo}\PYG{o}{.}\PYG{n}{get\PYGZus{}gitattributes}\PYG{p}{(}\PYG{l+s+s1}{\PYGZsq{}}\PYG{l+s+s1}{*}\PYG{l+s+s1}{\PYGZsq{}}\PYG{p}{)}
 \PYG{k}{if} \PYG{o+ow}{not} \PYG{n}{attrs}\PYG{o}{.}\PYG{n}{get}\PYG{p}{(}\PYG{l+s+s1}{\PYGZsq{}}\PYG{l+s+s1}{*}\PYG{l+s+s1}{\PYGZsq{}}\PYG{p}{,} \PYG{p}{\PYGZob{}}\PYG{p}{\PYGZcb{}}\PYG{p}{)}\PYG{o}{.}\PYG{n}{get}\PYG{p}{(}
         \PYG{l+s+s1}{\PYGZsq{}}\PYG{l+s+s1}{annex.largefiles}\PYG{l+s+s1}{\PYGZsq{}}\PYG{p}{,} \PYG{k+kc}{None}\PYG{p}{)} \PYG{o}{==} \PYG{n}{annex\PYGZus{}largefiles}\PYG{p}{:}
     \PYG{n}{ds}\PYG{o}{.}\PYG{n}{repo}\PYG{o}{.}\PYG{n}{set\PYGZus{}gitattributes}\PYG{p}{(}\PYG{p}{[}
         \PYG{p}{(}\PYG{l+s+s1}{\PYGZsq{}}\PYG{l+s+s1}{*}\PYG{l+s+s1}{\PYGZsq{}}\PYG{p}{,} \PYG{p}{\PYGZob{}}\PYG{l+s+s1}{\PYGZsq{}}\PYG{l+s+s1}{annex.largefiles}\PYG{l+s+s1}{\PYGZsq{}}\PYG{p}{:} \PYG{n}{annex\PYGZus{}largefiles}\PYG{p}{\PYGZcb{}}\PYG{p}{)}\PYG{p}{]}\PYG{p}{)}

 \PYG{n}{git\PYGZus{}attributes\PYGZus{}file} \PYG{o}{=} \PYG{n}{op}\PYG{o}{.}\PYG{n}{join}\PYG{p}{(}\PYG{n}{ds}\PYG{o}{.}\PYG{n}{path}\PYG{p}{,} \PYG{l+s+s1}{\PYGZsq{}}\PYG{l+s+s1}{.gitattributes}\PYG{l+s+s1}{\PYGZsq{}}\PYG{p}{)}
 \PYG{n}{ds}\PYG{o}{.}\PYG{n}{save}\PYG{p}{(}
     \PYG{n}{git\PYGZus{}attributes\PYGZus{}file}\PYG{p}{,}
     \PYG{n}{message}\PYG{o}{=}\PYG{l+s+s2}{\PYGZdq{}}\PYG{l+s+s2}{Instruct annex to add text files to Git}\PYG{l+s+s2}{\PYGZdq{}}\PYG{p}{,}
 \PYG{p}{)}
\end{sphinxVerbatim}
\sphinxresetverbatimhllines

\sphinxAtStartPar
Just like \sphinxcode{\sphinxupquote{cfg\_text2git}}, all DataLad procedures are
executables (such as a script, or compiled code).
In principle, they can be written in any language, and perform
any task inside of a dataset.
The \sphinxcode{\sphinxupquote{text2git}} configuration, for example, applies a configuration for how
git\sphinxhyphen{}annex treats different file types. Other procedures do not
only modify \sphinxcode{\sphinxupquote{.gitattributes}}, but can also populate a dataset
with particular content, or automate routine tasks such as
synchronizing dataset content with certain siblings.
What makes them a particularly versatile and flexible tool is
that anyone can write their own procedures.
If a workflow is a standard in a team and needs to be applied often, turning it into
a script can save time and effort.
To learn how to do this, read the tutorial on writing own procedures in an upcoming section.
By pointing DataLad to the location the procedures reside in they can be applied, and by
including them in a dataset they can even be shared.
And even if the script is simple, it is very handy to have preconfigured
procedures that can be run in a single command line call. In the
case of \sphinxcode{\sphinxupquote{text2git}}, all text files in a dataset will be stored
in Git \textendash{} this is a useful configuration that is applicable to a
wide range of datasets. It is a shortcut that
spares naive users the necessity to learn about the \sphinxcode{\sphinxupquote{.gitattributes}}
file when setting up a dataset.

\index{run\sphinxhyphen{}procedure@\spxentry{run\sphinxhyphen{}procedure}!DataLad command@\spxentry{DataLad command}}\index{DataLad command@\spxentry{DataLad command}!run\sphinxhyphen{}procedure@\spxentry{run\sphinxhyphen{}procedure}}\index{discover dataset procedures@\spxentry{discover dataset procedures}!with DataLad@\spxentry{with DataLad}}\index{with DataLad@\spxentry{with DataLad}!discover dataset procedures@\spxentry{discover dataset procedures}}\index{discover@\spxentry{discover}!dataset procedure@\spxentry{dataset procedure}}\index{dataset procedure@\spxentry{dataset procedure}!discover@\spxentry{discover}}\ignorespaces 
\sphinxAtStartPar
To find out available procedures, the command
\sphinxcode{\sphinxupquote{datalad run\sphinxhyphen{}procedure \sphinxhyphen{}\sphinxhyphen{}discover}} is helpful.
This command will make DataLad search the default location for
procedures in a dataset, the source code of DataLad or
installed DataLad extensions, and the default locations for
procedures on the system for available procedures:

\begin{sphinxVerbatim}[commandchars=\\\{\}]
\PYG{g+gp}{\PYGZdl{} }datalad\PYG{+w}{ }run\PYGZhy{}procedure\PYG{+w}{ }\PYGZhy{}\PYGZhy{}discover
\PYG{g+go}{cfg\PYGZus{}metadatatypes (VIRTUALENV/lib/python3.8/site\PYGZhy{}packages/datalad\PYGZus{}deprecated/resources/procedures/cfg\PYGZus{}metadatatypes.py) [python\PYGZus{}script]}
\PYG{g+go}{cfg\PYGZus{}noannex (VIRTUALENV/lib/python3.8/site\PYGZhy{}packages/datalad/resources/procedures/cfg\PYGZus{}noannex.py) [python\PYGZus{}script]}
\PYG{g+go}{cfg\PYGZus{}text2git (VIRTUALENV/lib/python3.8/site\PYGZhy{}packages/datalad/resources/procedures/cfg\PYGZus{}text2git.py) [python\PYGZus{}script]}
\PYG{g+go}{cfg\PYGZus{}yoda (VIRTUALENV/lib/python3.8/site\PYGZhy{}packages/datalad/resources/procedures/cfg\PYGZus{}yoda.py) [python\PYGZus{}script]}
\end{sphinxVerbatim}

\sphinxAtStartPar
The output shows that four procedures available in this particular dataset and the system it exists on:
\sphinxcode{\sphinxupquote{cfg\_metadatatypes}}, \sphinxcode{\sphinxupquote{cfg\_text2git}}, \sphinxcode{\sphinxupquote{cfg\_yoda}}, and \sphinxcode{\sphinxupquote{cfg\_noannex}}.
It also lists where they are stored \textendash{} in this case,
they are all part of the source code of DataLad%
\begin{footnote}\sphinxAtStartFootnote
In theory, because procedures can exist on different levels, and
because anyone can create (and thus name) their own procedures, there
can be name conflicts. The order of precedence in such cases is:
user\sphinxhyphen{}level, system\sphinxhyphen{}level, dataset, DataLad extension, DataLad, i.e.,
local procedures take precedence over those coming from “outside” via
datasets or DataLad extensions.
If procedures in a higher\sphinxhyphen{}level dataset and a subdataset have the same
name, the procedure closer to the dataset \sphinxcode{\sphinxupquote{run\sphinxhyphen{}procedure}} is
operating on takes precedence.
%
\end{footnote}.
\begin{itemize}
\item {} 
\sphinxAtStartPar
\sphinxcode{\sphinxupquote{cfg\_noannex}} configures a dataset to not have an annex at all.

\item {} 
\sphinxAtStartPar
\sphinxcode{\sphinxupquote{cfg\_yoda}} configures a dataset according to the yoda
principles \textendash{} the section {\hyperref[\detokenize{basics/101-127-yoda:yoda}]{\sphinxcrossref{\DUrole{std,std-ref}{YODA: Best practices for data analyses in a dataset}}}} (\autopageref*{\detokenize{basics/101-127-yoda:yoda}}) talks about this in detail.

\item {} 
\sphinxAtStartPar
\sphinxcode{\sphinxupquote{cfg\_text2git}} configures text files to be stored in Git.

\item {} 
\sphinxAtStartPar
\sphinxcode{\sphinxupquote{cfg\_metadatatypes}} lets users configure additional metadata
types.

\end{itemize}

\index{run dataset procedure@\spxentry{run dataset procedure}!with DataLad@\spxentry{with DataLad}}\index{with DataLad@\spxentry{with DataLad}!run dataset procedure@\spxentry{run dataset procedure}}\index{run@\spxentry{run}!dataset procedure@\spxentry{dataset procedure}}\index{dataset procedure@\spxentry{dataset procedure}!run@\spxentry{run}}\ignorespaces 

\subsection{Applying procedures}
\label{\detokenize{basics/101-124-procedures:applying-procedures}}\label{\detokenize{basics/101-124-procedures:index-2}}
\sphinxAtStartPar
\sphinxcode{\sphinxupquote{datalad run\sphinxhyphen{}procedure}} not only \sphinxstyleemphasis{discovers}
but also \sphinxstyleemphasis{executes} procedures. If given the name of
a procedure, this command will apply the procedure to
the current dataset, or the dataset that is specified
with the \sphinxcode{\sphinxupquote{\sphinxhyphen{}d/\sphinxhyphen{}\sphinxhyphen{}dataset}} flag:

\begin{sphinxVerbatim}[commandchars=\\\{\}]
datalad\PYG{+w}{ }run\PYGZhy{}procedure\PYG{+w}{ }\PYG{o}{[}\PYGZhy{}d\PYG{+w}{ }\PYGZlt{}PATH\PYGZgt{}\PYG{o}{]}\PYG{+w}{ }cfg\PYGZus{}text2git
\end{sphinxVerbatim}

\index{run dataset procedure on dataset creation@\spxentry{run dataset procedure on dataset creation}!with DataLad@\spxentry{with DataLad}}\index{with DataLad@\spxentry{with DataLad}!run dataset procedure on dataset creation@\spxentry{run dataset procedure on dataset creation}}\index{run on dataset creation@\spxentry{run on dataset creation}!dataset procedure@\spxentry{dataset procedure}}\index{dataset procedure@\spxentry{dataset procedure}!run on dataset creation@\spxentry{run on dataset creation}}\ignorespaces 
\sphinxAtStartPar
The typical workflow is to create a dataset and apply
a procedure afterwards.
However, some procedures shipped with DataLad or its extensions with a
\sphinxcode{\sphinxupquote{cfg\_}} prefix can also be applied right at the creation of a dataset
with the \sphinxcode{\sphinxupquote{\sphinxhyphen{}c/\sphinxhyphen{}\sphinxhyphen{}cfg\sphinxhyphen{}proc \textless{}name\textgreater{}}} option in a \sphinxcode{\sphinxupquote{datalad create}}
command. This is a peculiarity of these procedures because, by convention,
all of these procedures are written to not require arguments.
The command structure looks like this:

\begin{sphinxVerbatim}[commandchars=\\\{\}]
\PYG{g+gp}{\PYGZdl{} }datalad\PYG{+w}{ }create\PYG{+w}{ }\PYGZhy{}c\PYG{+w}{ }text2git\PYG{+w}{ }DataLad\PYGZhy{}101
\end{sphinxVerbatim}

\sphinxAtStartPar
Note that the \sphinxcode{\sphinxupquote{cfg\_}} prefix of the procedures is omitted in these
calls to keep it extra simple and short. The
available procedures in this example (\sphinxcode{\sphinxupquote{cfg\_yoda}}, \sphinxcode{\sphinxupquote{cfg\_text2git}})
could thus be applied within a \sphinxcode{\sphinxupquote{datalad create}} as
\begin{itemize}
\item {} 
\sphinxAtStartPar
\sphinxcode{\sphinxupquote{datalad create \sphinxhyphen{}c yoda \textless{}DSname\textgreater{}}}

\item {} 
\sphinxAtStartPar
\sphinxcode{\sphinxupquote{datalad create \sphinxhyphen{}c text2git \textless{}DSname\textgreater{}}}

\end{itemize}

\index{dataset procedure@\spxentry{dataset procedure}!apply more than one configuration@\spxentry{apply more than one configuration}}\ignorespaces 
\sphinxAtStartPar
You can also apply multiple configuration procedures at once. However,
the need to be compatible and complement each other. DataLad does not
guarantee that per\sphinxhyphen{}se. This is how it could look:

\begin{sphinxVerbatim}[commandchars=\\\{\}]
\PYG{g+gp}{\PYGZdl{} }datalad\PYG{+w}{ }create\PYG{+w}{ }\PYGZhy{}c\PYG{+w}{ }yoda\PYG{+w}{ }\PYGZhy{}c\PYG{+w}{ }text2git
\end{sphinxVerbatim}

\sphinxAtStartPar
And lastly, procedures can also be applied to subdatasets. See the
\textit{Find-out-more}~{\findoutmoreiconinline}\textit{\ref{fom-subds-procedures}} {\hyperref[\detokenize{basics/101-124-procedures:fom-subds-procedures}]{\sphinxcrossref{\DUrole{std,std-ref}{on this topic}}}} (\autopageref*{\detokenize{basics/101-124-procedures:fom-subds-procedures}}) for details.

\index{dataset procedure@\spxentry{dataset procedure}!apply to subdatasets@\spxentry{apply to subdatasets}}\ignorespaces \begin{findoutmore}[label={fom-subds-procedures}, before title={\thetcbcounter\ }, float, floatplacement=tb, check odd page=true]{Applying procedures in subdatasets}
\label{\detokenize{basics/101-124-procedures:fom-subds-procedures}}

\sphinxAtStartPar
Procedures can be applied in datasets on any level in the dataset hierarchy, i.e.,
also in subdatasets. Note, though, that a subdataset will show up as being
\sphinxcode{\sphinxupquote{modified}} in \sphinxcode{\sphinxupquote{datalad status}} \sphinxstyleemphasis{in the superdataset}
after applying a procedure.
This is expected, and it would also be the case with any other modification
(saved or not) in the subdataset, as the version of the subdataset that is tracked
in the superdataset simply changed. A \sphinxcode{\sphinxupquote{datalad save}} in the superdataset
will make sure that the version of the subdataset gets updated in the superdataset.
The section {\hyperref[\detokenize{basics/101-132-advancednesting:nesting2}]{\sphinxcrossref{\DUrole{std,std-ref}{Turtles all the way down}}}} (\autopageref*{\detokenize{basics/101-132-advancednesting:nesting2}}) will elaborate on this general principle later in this
book.


\end{findoutmore}

\sphinxAtStartPar
As a general note, it can be useful to apply procedures
early in the life of a dataset. Procedures such
as \sphinxcode{\sphinxupquote{cfg\_yoda}}, explained in detail in section {\hyperref[\detokenize{basics/101-127-yoda:yoda}]{\sphinxcrossref{\DUrole{std,std-ref}{YODA: Best practices for data analyses in a dataset}}}} (\autopageref*{\detokenize{basics/101-127-yoda:yoda}}),
create files, change \sphinxcode{\sphinxupquote{.gitattributes}}, or apply other configurations.
If many other (possibly complex) configurations are
already in place, or if files of the same name as the ones created by
a procedure are already in existence, this can lead to unexpected
problems or failures, especially for naive users. Applying \sphinxcode{\sphinxupquote{cfg\_text2git}}
to a default dataset in which one has saved many text files already
(as per default added to the annex) will not place the existing, saved
files into Git \textendash{} only those text files created \sphinxstyleemphasis{after} the configuration
was applied.

\index{configuration item@\spxentry{configuration item}!datalad.locations.system\sphinxhyphen{}procedures@\spxentry{datalad.locations.system\sphinxhyphen{}procedures}}\index{configuration item@\spxentry{configuration item}!datalad.locations.user\sphinxhyphen{}procedures@\spxentry{datalad.locations.user\sphinxhyphen{}procedures}}\index{configuration item@\spxentry{configuration item}!datalad.locations.dataset\sphinxhyphen{}procedures@\spxentry{datalad.locations.dataset\sphinxhyphen{}procedures}}\index{configuration item@\spxentry{configuration item}!datalad.procedures.\textless{}name\textgreater{}.call\sphinxhyphen{}format@\spxentry{datalad.procedures.\textless{}name\textgreater{}.call\sphinxhyphen{}format}}\index{configuration item@\spxentry{configuration item}!datalad.procedures.\textless{}name\textgreater{}.help@\spxentry{datalad.procedures.\textless{}name\textgreater{}.help}}\index{datasets procedures@\spxentry{datasets procedures}!write your own@\spxentry{write your own}}\ignorespaces 

\subsection{Write your own procedures}
\label{\detokenize{basics/101-124-procedures:write-your-own-procedures}}\label{\detokenize{basics/101-124-procedures:index-6}}
\sphinxAtStartPar
Procedures can come with DataLad or its extensions, but anyone can
write their own ones in addition, and deploy them on individual machines,
or ship them within DataLad datasets. This allows to
automate routine configurations or tasks in a dataset, or share configurations that would otherwise not “stick” to the dataset.
Here are some general rules for creating a custom procedure:
\begin{itemize}
\item {} 
\sphinxAtStartPar
A procedure can be any executable. Executables must have the
appropriate permissions and, in the case of a script,
must contain an appropriate {\hyperref[\detokenize{glossary:term-shebang}]{\sphinxtermref{\DUrole{xref,std,std-term}{shebang}}}}.
\begin{itemize}
\item {} 
\sphinxAtStartPar
If a procedure is not executable, but its filename ends with
\sphinxcode{\sphinxupquote{.sh}}, it is automatically executed via {\hyperref[\detokenize{glossary:term-bash}]{\sphinxtermref{\DUrole{xref,std,std-term}{bash}}}}.

\end{itemize}

\item {} 
\sphinxAtStartPar
Procedures can implement any argument handling, but must be capable
of taking at least one positional argument (the absolute path to the
dataset they shall operate on).

\item {} 
\sphinxAtStartPar
Custom procedures rely heavily on configurations in \sphinxcode{\sphinxupquote{.datalad/config}}
(or the associated environment variables). Within \sphinxcode{\sphinxupquote{.datalad/config}},
each procedure should get an individual entry that contains at least
a short “help” description on what the procedure does. Below is a minimal
\sphinxcode{\sphinxupquote{.datalad/config}} entry for a custom procedure:

\begin{sphinxVerbatim}[commandchars=\\\{\}]
\PYG{k}{[datalad \PYGZdq{}procedures.\PYGZlt{}NAME\PYGZgt{}\PYGZdq{}]}
\PYG{+w}{   }\PYG{n+na}{help}\PYG{+w}{ }\PYG{o}{=}\PYG{+w}{ }\PYG{l+s}{This is a string to describe what the procedure does}
\end{sphinxVerbatim}

\item {} 
\sphinxAtStartPar
By default, on GNU/Linux systems, DataLad will search for system\sphinxhyphen{}wide procedures
(i.e., procedures on the \sphinxstyleemphasis{system} level) in \sphinxcode{\sphinxupquote{/etc/xdg/datalad/procedures}},
for user procedures (i.e., procedures on the \sphinxstyleemphasis{global} level) in \sphinxcode{\sphinxupquote{\textasciitilde{}/.config/datalad/procedures}},
and for dataset procedures (i.e., the \sphinxstyleemphasis{local} level%
\begin{footnote}\sphinxAtStartFootnote
Note that we simplify the level of procedures that exist within a dataset
by calling them \sphinxstyleemphasis{local}. Even though they apply to a dataset just as \sphinxstyleemphasis{local}
Git configurations, unlike Git’s \sphinxstyleemphasis{local} configurations in \sphinxcode{\sphinxupquote{.git/config}},
the procedures and procedure configurations in \sphinxcode{\sphinxupquote{.datalad/config}} are committed
and can be shared together with a dataset. The procedure level \sphinxstyleemphasis{local} therefore
does not exactly corresponds to the \sphinxstyleemphasis{local} scope in the sense that Git uses it.
%
\end{footnote}) in \sphinxcode{\sphinxupquote{.datalad/procedures}}
relative to a dataset root.
Note that \sphinxcode{\sphinxupquote{.datalad/procedures}} does not exist by default, and the \sphinxcode{\sphinxupquote{procedures}}
directory needs to be created first.

\item {} 
\sphinxAtStartPar
Alternatively to the default locations, DataLad can be pointed to the location of a procedure with a configuration in \sphinxcode{\sphinxupquote{.datalad/config}} (or with the help of the associated {\hyperref[\detokenize{glossary:term-environment-variable}]{\sphinxtermref{\DUrole{xref,std,std-term}{environment variable}}}}s).
The appropriate configuration keys for \sphinxcode{\sphinxupquote{.datalad/config}} are either \sphinxcode{\sphinxupquote{datalad.locations.system\sphinxhyphen{}procedures}} (for changing the \sphinxstyleemphasis{system} default), \sphinxcode{\sphinxupquote{datalad.locations.user\sphinxhyphen{}procedures}} (for changing the \sphinxstyleemphasis{global} default), or \sphinxcode{\sphinxupquote{datalad.locations.dataset\sphinxhyphen{}procedures}} (for changing the \sphinxstyleemphasis{local} default).
An example \sphinxcode{\sphinxupquote{.datalad/config}} entry for the local scope is shown below.

\begin{sphinxVerbatim}[commandchars=\\\{\}]
\PYG{k}{[datalad \PYGZdq{}locations\PYGZdq{}]}
\PYG{+w}{        }\PYG{n+na}{dataset\PYGZhy{}procedures}\PYG{+w}{ }\PYG{o}{=}\PYG{+w}{ }\PYG{l+s}{relative/path/from/dataset\PYGZhy{}root}
\end{sphinxVerbatim}

\item {} 
\sphinxAtStartPar
By default, DataLad will call a procedure with a standard template defined by a format string:
\begin{quote}

\begin{sphinxVerbatim}[commandchars=\\\{\}]
\PYG{n}{interpreter} \PYG{p}{\PYGZob{}}\PYG{n}{script}\PYG{p}{\PYGZcb{}} \PYG{p}{\PYGZob{}}\PYG{n}{ds}\PYG{p}{\PYGZcb{}} \PYG{p}{\PYGZob{}}\PYG{n}{arguments}\PYG{p}{\PYGZcb{}}
\end{sphinxVerbatim}
\end{quote}

\end{itemize}

\sphinxAtStartPar
where arguments can be any additional command line arguments a script (procedure) takes or requires.
This default format string can be customized within \sphinxcode{\sphinxupquote{.datalad/config}} in \sphinxcode{\sphinxupquote{datalad.procedures.\textless{}NAME\textgreater{}.call\sphinxhyphen{}format}}.
An example \sphinxcode{\sphinxupquote{.datalad/config}} entry with a changed call format string is shown below.
\begin{quote}

\begin{sphinxVerbatim}[commandchars=\\\{\}]
\PYG{k}{[datalad \PYGZdq{}procedures.\PYGZlt{}NAME\PYGZgt{}\PYGZdq{}]}
\PYG{+w}{   }\PYG{n+na}{help}\PYG{+w}{ }\PYG{o}{=}\PYG{+w}{ }\PYG{l+s}{This is a string to describe what the procedure does}
\PYG{+w}{   }\PYG{n+na}{call\PYGZhy{}format}\PYG{+w}{ }\PYG{o}{=}\PYG{+w}{ }\PYG{l+s}{python \PYGZob{}script\PYGZcb{} \PYGZob{}ds\PYGZcb{} \PYGZob{}somearg1\PYGZcb{} \PYGZob{}somearg2\PYGZcb{}}
\end{sphinxVerbatim}
\end{quote}
\begin{itemize}
\item {} 
\sphinxAtStartPar
By convention, procedures should leave a dataset in a clean state.

\end{itemize}

\sphinxAtStartPar
Therefore, in order to create a custom procedure, an executable script
in the appropriate location is fine. Placing a script \sphinxcode{\sphinxupquote{myprocedure}}
into \sphinxcode{\sphinxupquote{.datalad/procedures}} will allow running
\sphinxcode{\sphinxupquote{datalad run\sphinxhyphen{}procedure myprocedure}} in your dataset, and because
it is part of the dataset it will also allow distributing the procedure.
Below is a toy\sphinxhyphen{}example for a custom procedure:

\begin{sphinxVerbatim}[commandchars=\\\{\}]
\PYG{g+gp}{\PYGZdl{} }datalad\PYG{+w}{ }create\PYG{+w}{ }somedataset\PYG{p}{;}\PYG{+w}{ }\PYG{n+nb}{cd}\PYG{+w}{ }somedataset
\PYG{g+go}{create(ok): /home/me/procs/somedataset (dataset)}
\end{sphinxVerbatim}

\begin{sphinxVerbatim}[commandchars=\\\{\}]
\PYG{g+gp}{\PYGZdl{} }mkdir\PYG{+w}{ }.datalad/procedures
\PYG{g+gp}{\PYGZdl{} }cat\PYG{+w}{ }\PYGZlt{}\PYGZlt{}\PYG{+w}{ }EOT\PYG{+w}{ }\PYGZgt{}\PYG{+w}{ }.datalad/procedures/example.py
\PYG{g+go}{\PYGZdq{}\PYGZdq{}\PYGZdq{}A simple procedure to create a file \PYGZsq{}example\PYGZsq{} and store}
\PYG{g+go}{it in Git, and a file \PYGZsq{}example2\PYGZsq{} and annex it. The contents}
\PYG{g+go}{of \PYGZsq{}example\PYGZsq{} must be defined with a positional argument.\PYGZdq{}\PYGZdq{}\PYGZdq{}}

\PYG{g+go}{import sys}
\PYG{g+go}{import os.path as op}
\PYG{g+go}{from datalad.distribution.dataset import require\PYGZus{}dataset}
\PYG{g+go}{from datalad.utils import create\PYGZus{}tree}

\PYG{g+go}{ds = require\PYGZus{}dataset(}
\PYG{g+go}{    sys.argv[1],}
\PYG{g+go}{    check\PYGZus{}installed=True,}
\PYG{g+go}{    purpose=\PYGZsq{}showcase an example procedure\PYGZsq{})}

\PYG{g+gp}{\PYGZsh{} }this\PYG{+w}{ }is\PYG{+w}{ }the\PYG{+w}{ }content\PYG{+w}{ }\PYG{k}{for}\PYG{+w}{ }file\PYG{+w}{ }\PYG{l+s+s2}{\PYGZdq{}example\PYGZdq{}}
\PYG{g+go}{content = \PYGZdq{}\PYGZdq{}\PYGZdq{}\PYGZbs{}}
\PYG{g+go}{This file was created by a custom procedure! Neat, huh?}
\PYG{g+go}{\PYGZdq{}\PYGZdq{}\PYGZdq{}}

\PYG{g+gp}{\PYGZsh{} }create\PYG{+w}{ }a\PYG{+w}{ }directory\PYG{+w}{ }structure\PYG{+w}{ }template.\PYG{+w}{ }Write
\PYG{g+go}{tmpl = \PYGZob{}}
\PYG{g+go}{    \PYGZsq{}somedir\PYGZsq{}: \PYGZob{}}
\PYG{g+go}{        \PYGZsq{}example\PYGZsq{}: content,}
\PYG{g+go}{    \PYGZcb{},}
\PYG{g+go}{    \PYGZsq{}example2\PYGZsq{}: sys.argv[2] if sys.argv[2] else \PYGZdq{}got no input\PYGZdq{}}
\PYG{g+go}{\PYGZcb{}}

\PYG{g+gp}{\PYGZsh{} }actually\PYG{+w}{ }create\PYG{+w}{ }the\PYG{+w}{ }structure\PYG{+w}{ }\PYG{k}{in}\PYG{+w}{ }the\PYG{+w}{ }dataset
\PYG{g+go}{create\PYGZus{}tree(ds.path, tmpl)}

\PYG{g+gp}{\PYGZsh{} }rule\PYG{+w}{ }to\PYG{+w}{ }store\PYG{+w}{ }\PYG{l+s+s1}{\PYGZsq{}example\PYGZsq{}}\PYG{+w}{ }Git
\PYG{g+go}{ds.repo.set\PYGZus{}gitattributes([(\PYGZsq{}example\PYGZsq{}, \PYGZob{}\PYGZsq{}annex.largefiles\PYGZsq{}: \PYGZsq{}nothing\PYGZsq{}\PYGZcb{})])}

\PYG{g+gp}{\PYGZsh{} }save\PYG{+w}{ }the\PYG{+w}{ }dataset\PYG{+w}{ }modifications
\PYG{g+go}{ds.save(message=\PYGZdq{}Apply custom procedure\PYGZdq{})}

\PYG{g+go}{EOT}
\end{sphinxVerbatim}

\begin{sphinxVerbatim}[commandchars=\\\{\}]
\PYG{g+gp}{\PYGZdl{} }datalad\PYG{+w}{ }save\PYG{+w}{ }\PYGZhy{}m\PYG{+w}{ }\PYG{l+s+s2}{\PYGZdq{}add custom procedure\PYGZdq{}}
\PYG{g+go}{add(ok): .datalad/procedures/example.py (file)}
\PYG{g+go}{save(ok): . (dataset)}
\end{sphinxVerbatim}

\sphinxAtStartPar
At this point, the dataset contains the custom procedure \sphinxcode{\sphinxupquote{example}}.
This is how it can be executed and what it does:

\begin{sphinxVerbatim}[commandchars=\\\{\}]
\PYG{g+gp}{\PYGZdl{} }datalad\PYG{+w}{ }run\PYGZhy{}procedure\PYG{+w}{ }example\PYG{+w}{ }\PYG{l+s+s2}{\PYGZdq{}this text will be in the file \PYGZsq{}example2\PYGZsq{}\PYGZdq{}}
\PYG{g+go}{[INFO] Running procedure example}
\PYG{g+go}{[INFO] == Command start (output follows) =====}
\PYG{g+go}{add(ok): example2 (file)}
\PYG{g+go}{add(ok): somedir/example (file)}
\PYG{g+go}{add(ok): .gitattributes (file)}
\PYG{g+go}{save(ok): . (dataset)}
\PYG{g+go}{[INFO] == Command exit (modification check follows) =====}
\PYG{g+go}{run(ok): /home/me/procs/somedataset (dataset) [VIRTUALENV/bin/python /home/m...]}
\end{sphinxVerbatim}

\begin{sphinxVerbatim}[commandchars=\\\{\}]
\PYG{g+gp}{\PYGZdl{} }\PYG{c+c1}{\PYGZsh{} the directory structure has been created}
\PYG{g+gp}{\PYGZdl{} }tree
\PYG{g+go}{.}
\PYG{g+go}{├── example2 \PYGZhy{}\PYGZgt{} .git/annex/objects/G6/zw/✂/MD5E\PYGZhy{}s40\PYGZhy{}\PYGZhy{}2ed1bce0✂MD5}
\PYG{g+go}{└── somedir}
\PYG{g+go}{    └── example}
\end{sphinxVerbatim}

\begin{sphinxVerbatim}[commandchars=\\\{\}]
\PYG{g+gp}{\PYGZdl{} }\PYG{c+c1}{\PYGZsh{} lets check out the contents in the files}
\PYG{g+gp}{\PYGZdl{} }cat\PYG{+w}{ }example2\PYG{+w}{  }\PYG{o}{\PYGZam{}\PYGZam{}}\PYG{+w}{ }\PYG{n+nb}{echo}\PYG{+w}{ }\PYG{l+s+s1}{\PYGZsq{}\PYGZsq{}}\PYG{+w}{ }\PYG{o}{\PYGZam{}\PYGZam{}}\PYG{+w}{ }cat\PYG{+w}{ }somedir/example
\PYG{g+go}{this text will be in the file \PYGZsq{}example2\PYGZsq{}}
\PYG{g+go}{This file was created by a custom procedure! Neat, huh?}
\end{sphinxVerbatim}

\begin{sphinxVerbatim}[commandchars=\\\{\}]
\PYG{g+gp}{\PYGZdl{} }git\PYG{+w}{ }config\PYG{+w}{ }\PYGZhy{}f\PYG{+w}{ }.datalad/config\PYG{+w}{ }datalad.procedures.example.help\PYG{+w}{ }\PYG{l+s+s2}{\PYGZdq{}A toy example\PYGZdq{}}
\PYG{g+gp}{\PYGZdl{} }datalad\PYG{+w}{ }save\PYG{+w}{ }\PYGZhy{}m\PYG{+w}{ }\PYG{l+s+s2}{\PYGZdq{}add help description\PYGZdq{}}
\PYG{g+go}{add(ok): .datalad/config (file)}
\PYG{g+go}{save(ok): . (dataset)}
\end{sphinxVerbatim}

\sphinxAtStartPar
To find out more about a given procedure, you can ask for help:

\begin{sphinxVerbatim}[commandchars=\\\{\}]
\PYG{g+gp}{\PYGZdl{} }datalad\PYG{+w}{ }run\PYGZhy{}procedure\PYG{+w}{ }\PYGZhy{}\PYGZhy{}help\PYGZhy{}proc\PYG{+w}{ }example
\PYG{g+go}{example (.datalad/procedures/example.py)}
\PYG{g+go}{A toy example}
\end{sphinxVerbatim}


\subsection{Conclusions}
\label{\detokenize{basics/101-124-procedures:conclusions}}
\sphinxAtStartPar
Summing up, DataLad’s \sphinxcode{\sphinxupquote{datalad run\sphinxhyphen{}procedure}} command is a handy tool
with useful existing procedures but much flexibility for your own
DIY procedure scripts. With the information of the last three sections
you should be able to write and understand necessary configurations,
but you can also rely on existing, preconfigured templates in the
form of procedures, and even write and distribute your own.

\sphinxAtStartPar
Therefore, envision procedures as
helper\sphinxhyphen{}tools that can minimize technical complexities
in a dataset \textendash{} users can concentrate on the actual task while
the dataset is set\sphinxhyphen{}up, structured, processed, or configured automatically
with the help of a procedure.
Especially in the case of trainees and new users, applying procedures
instead of doing relevant routines “by hand” can help to ease
working with the dataset. Other than by users, procedures can also be triggered to automatically
run after any command execution if a command results matches a specific
requirement.

\sphinxAtStartPar
Finally, make a note about running procedures inside of \sphinxcode{\sphinxupquote{notes.txt}}:

\begin{sphinxVerbatim}[commandchars=\\\{\}]
\PYG{g+gp}{\PYGZdl{} }cat\PYG{+w}{ }\PYGZlt{}\PYGZlt{}\PYG{+w}{ }EOT\PYG{+w}{ }\PYGZgt{}\PYGZgt{}\PYG{+w}{ }notes.txt
\PYG{g+go}{It can be useful to use pre\PYGZhy{}configured procedures that can apply}
\PYG{g+go}{configurations, create files or file hierarchies, or perform arbitrary}
\PYG{g+go}{tasks in datasets. They can be shipped with DataLad, its extensions,}
\PYG{g+go}{or datasets, and you can even write your own procedures and distribute}
\PYG{g+go}{them.}
\PYG{g+go}{The \PYGZdq{}datalad run\PYGZhy{}procedure\PYGZdq{} command is used to apply such a procedure}
\PYG{g+go}{to a dataset. Procedures shipped with DataLad or its extensions}
\PYG{g+go}{starting with a \PYGZdq{}cfg\PYGZdq{} prefix can also be applied at the creation of a}
\PYG{g+go}{dataset with \PYGZdq{}datalad create \PYGZhy{}c \PYGZlt{}PROC\PYGZhy{}NAME\PYGZgt{} \PYGZlt{}PATH\PYGZgt{}\PYGZdq{} (omitting the}
\PYG{g+go}{\PYGZdq{}cfg\PYGZdq{} prefix).}

\PYG{g+go}{EOT}
\end{sphinxVerbatim}

\begin{sphinxVerbatim}[commandchars=\\\{\}]
\PYG{g+gp}{\PYGZdl{} }datalad\PYG{+w}{ }save\PYG{+w}{ }\PYGZhy{}m\PYG{+w}{ }\PYG{l+s+s2}{\PYGZdq{}add note on DataLad\PYGZsq{}s procedures\PYGZdq{}}
\PYG{g+go}{add(ok): notes.txt (file)}
\PYG{g+go}{save(ok): . (dataset)}
\end{sphinxVerbatim}

\sphinxstepscope


\section{Summary}
\label{\detokenize{basics/101-125-summary:summary}}\label{\detokenize{basics/101-125-summary:summary-config}}\label{\detokenize{basics/101-125-summary::doc}}
\sphinxAtStartPar
This has been a substantial amount of information regarding various configuration
types, methods, and files. After this lecture, you have greatly broadened
your horizon about configurations of datasets:
\begin{itemize}
\item {} 
\sphinxAtStartPar
Configurations exist at different scopes and for different tools. Each
of such configuration scopes exists in an individual file, on a \sphinxstyleemphasis{system\sphinxhyphen{}wide},
\sphinxstyleemphasis{global} (user\sphinxhyphen{}specific) or \sphinxstyleemphasis{local} (repository specific) level. In addition
to Git’s \sphinxstyleemphasis{local} scope in \sphinxcode{\sphinxupquote{.git/config}}, DataLad introduces its own configurations within
\sphinxcode{\sphinxupquote{.datalad/config}} that apply to a specific dataset, but are committed and
therefore distributed. More specialized scopes take precedence over more global scopes.

\item {} 
\sphinxAtStartPar
Almost all configurations can be set with the \sphinxcode{\sphinxupquote{git config}} command.
Its structure looks like this:

\begin{sphinxVerbatim}[commandchars=\\\{\}]
git\PYG{+w}{ }config\PYG{+w}{ }\PYGZhy{}\PYGZhy{}local/\PYGZhy{}\PYGZhy{}global/\PYGZhy{}\PYGZhy{}system\PYG{+w}{ }\PYGZhy{}\PYGZhy{}add/remove\PYGZhy{}all/\PYGZhy{}\PYGZhy{}list\PYG{+w}{ }section.\PYG{o}{[}subsection.\PYG{o}{]}variable\PYG{+w}{ }\PYG{l+s+s2}{\PYGZdq{}value\PYGZdq{}}
\end{sphinxVerbatim}

\item {} 
\sphinxAtStartPar
The \sphinxcode{\sphinxupquote{.git/config}} configuration file is not version controlled, other
configuration files (\sphinxcode{\sphinxupquote{.gitmodules}}, \sphinxcode{\sphinxupquote{.gitattributes}}, \sphinxcode{\sphinxupquote{.datalad/config}})
however are, and can be shared together with the dataset. Non\sphinxhyphen{}shared configurations
will take precedence over shared configurations in a dataset clone.

\item {} 
\sphinxAtStartPar
Other tools than Git can be configured with the \sphinxcode{\sphinxupquote{git config}} command
as well. If configurations needs to be written to a file other than a
\sphinxcode{\sphinxupquote{.git(/)config}} file, supply a path to this file with the \sphinxcode{\sphinxupquote{\sphinxhyphen{}f/\sphinxhyphen{}\sphinxhyphen{}file}} flag
in a \sphinxcode{\sphinxupquote{git config}} command.

\item {} 
\sphinxAtStartPar
The \sphinxcode{\sphinxupquote{.gitattributes}} file is the only configuration file the \sphinxcode{\sphinxupquote{git config}}
command cannot write to, because it has a different layout. However, run\sphinxhyphen{}procedures or
the user can write simple rules into it that determine which files are annexed
and which are stored in Git.

\item {} 
\sphinxAtStartPar
DataLad’s \sphinxcode{\sphinxupquote{run\sphinxhyphen{}procedure}}s offer an easy and fast alternative to DIY
configurations, structuring, or processing of the dataset, and offer means to share or ship configurations together with a dataset.
They can be applied already at creation of a dataset with \sphinxcode{\sphinxupquote{datalad create \sphinxhyphen{}c \textless{}procedure\textgreater{}}},
or executed later with a \sphinxcode{\sphinxupquote{datalad run\sphinxhyphen{}procedure}} command.

\end{itemize}


\subsection{Now what can I do with it?}
\label{\detokenize{basics/101-125-summary:now-what-can-i-do-with-it}}
\sphinxAtStartPar
Configurations are not a closed book for you anymore. What will probably be
especially helpful is your new knowledge about \sphinxcode{\sphinxupquote{.gitattributes}} and
DataLad’s \sphinxcode{\sphinxupquote{run\sphinxhyphen{}procedure}} command that allow you to configure the behavior
of git\sphinxhyphen{}annex in your dataset.

\sphinxstepscope


