\chapter{DataLad datasets}
\label{\detokenize{basics/basics-datasets:datalad-datasets}}\label{\detokenize{basics/basics-datasets:chapter-datasets}}\label{\detokenize{basics/basics-datasets::doc}}
\noindent{\hspace*{\fill}\sphinxincludegraphics[width=0.500\linewidth]{{datasets}.pdf}\hspace*{\fill}}

\sphinxstepscope

\index{create@\spxentry{create}!DataLad command@\spxentry{DataLad command}}\index{DataLad command@\spxentry{DataLad command}!create@\spxentry{create}}\index{create dataset@\spxentry{create dataset}!with DataLad@\spxentry{with DataLad}}\index{with DataLad@\spxentry{with DataLad}!create dataset@\spxentry{create dataset}}\ignorespaces 

\section{Create a dataset}
\label{\detokenize{basics/101-101-create:create-a-dataset}}\label{\detokenize{basics/101-101-create:createds}}\label{\detokenize{basics/101-101-create:index-0}}\label{\detokenize{basics/101-101-create::doc}}
\sphinxAtStartPar
We are about to start the educational course \sphinxcode{\sphinxupquote{DataLad\sphinxhyphen{}101}}.
In order to follow along and organize course content, let us create
a directory on our computer to collate the materials, assignments, and
notes in.

\sphinxAtStartPar
Since this is \sphinxcode{\sphinxupquote{DataLad\sphinxhyphen{}101}}, let’s do it as a {\hyperref[\detokenize{glossary:term-DataLad-dataset}]{\sphinxtermref{\DUrole{xref,std,std-term}{DataLad dataset}}}}.
You might associate the term “dataset” with a large spreadsheet containing
variables and data.
But for DataLad, a dataset is the core data type:
As noted in {\hyperref[\detokenize{intro/philosophy:philo}]{\sphinxcrossref{\DUrole{std,std-ref}{A brief overview of DataLad}}}} (\autopageref*{\detokenize{intro/philosophy:philo}}), a dataset is a collection of \sphinxstyleemphasis{files}
in folders, and a file is the smallest unit any dataset can contain.
Although this is a very simple concept, datasets come with many
useful features.
Because experiencing is more insightful than just reading, we will explore the
concepts of DataLad datasets together by creating one.

\sphinxAtStartPar
Find a nice place on your computer’s file system to put a dataset for \sphinxcode{\sphinxupquote{DataLad\sphinxhyphen{}101}},
and create a fresh, empty dataset with the \sphinxcode{\sphinxupquote{datalad create}} command.

\sphinxAtStartPar
Note the command structure of \sphinxcode{\sphinxupquote{datalad create}} (optional bits are enclosed in \sphinxcode{\sphinxupquote{{[} {]}}}):

\begin{sphinxVerbatim}[commandchars=\\\{\}]
\PYG{n}{datalad} \PYG{n}{create} \PYG{p}{[}\PYG{o}{\PYGZhy{}}\PYG{o}{\PYGZhy{}}\PYG{n}{description} \PYG{l+s+s2}{\PYGZdq{}}\PYG{l+s+s2}{...}\PYG{l+s+s2}{\PYGZdq{}}\PYG{p}{]} \PYG{p}{[}\PYG{o}{\PYGZhy{}}\PYG{n}{c} \PYG{o}{\PYGZlt{}}\PYG{n}{config} \PYG{n}{options}\PYG{o}{\PYGZgt{}}\PYG{p}{]} \PYG{n}{PATH}
\end{sphinxVerbatim}

\index{set description for dataset location@\spxentry{set description for dataset location}!with DataLad@\spxentry{with DataLad}}\index{with DataLad@\spxentry{with DataLad}!set description for dataset location@\spxentry{set description for dataset location}}\ignorespaces \phantomsection\label{\detokenize{basics/101-101-create:createdescription}}\begin{findoutmore}[label={index-1}, before title={\thetcbcounter\ }, check odd page=true]{What is the description option of ‘datalad create’?}
\label{\detokenize{basics/101-101-create:index-1}}

\sphinxAtStartPar
The optional \sphinxcode{\sphinxupquote{\sphinxhyphen{}\sphinxhyphen{}description}} flag allows you to provide a short description of
the \sphinxstyleemphasis{location} of your dataset, for example with

\begin{sphinxVerbatim}[commandchars=\\\{\}]
\PYG{g+gp}{\PYGZdl{} }datalad\PYG{+w}{ }create\PYG{+w}{ }\PYGZhy{}\PYGZhy{}description\PYG{+w}{ }\PYG{l+s+s2}{\PYGZdq{}course on DataLad\PYGZhy{}101 on my private laptop\PYGZdq{}}\PYG{+w}{ }\PYGZhy{}c\PYG{+w}{ }text2git\PYG{+w}{ }DataLad\PYGZhy{}101
\end{sphinxVerbatim}

\sphinxAtStartPar
If you want, use the above command instead to provide a description. Its use will not be immediately clear now, but the chapter
{\hyperref[\detokenize{basics/basics-collaboration:chapter-collaboration}]{\sphinxcrossref{\DUrole{std,std-ref}{Collaboration}}}} (\autopageref*{\detokenize{basics/basics-collaboration:chapter-collaboration}}) shows where this description
ends up and how it may be useful.


\end{findoutmore}

\sphinxAtStartPar
Let’s start:

\index{create dataset@\spxentry{create dataset}!with DataLad@\spxentry{with DataLad}}\index{with DataLad@\spxentry{with DataLad}!create dataset@\spxentry{create dataset}}\ignorespaces 
\def\sphinxLiteralBlockLabel{\label{\detokenize{basics/101-101-create:index-2}}}
\begin{sphinxVerbatim}[commandchars=\\\{\}]
\PYG{g+gp}{\PYGZdl{} }datalad\PYG{+w}{ }create\PYG{+w}{ }\PYGZhy{}c\PYG{+w}{ }text2git\PYG{+w}{ }DataLad\PYGZhy{}101
\PYG{g+go}{[INFO] Running procedure cfg\PYGZus{}text2git}
\PYG{g+go}{[INFO] == Command start (output follows) =====}
\PYG{g+go}{[INFO] == Command exit (modification check follows) =====}
\PYG{g+go}{run(ok): /home/me/dl\PYGZhy{}101/DataLad\PYGZhy{}101 (dataset) [VIRTUALENV/bin/python /home/a...]}
\PYG{g+go}{create(ok): /home/me/dl\PYGZhy{}101/DataLad\PYGZhy{}101 (dataset)}
\end{sphinxVerbatim}

\sphinxAtStartPar
This will create a dataset called \sphinxcode{\sphinxupquote{DataLad\sphinxhyphen{}101}} in the directory you are currently
in. For now, disregard \sphinxcode{\sphinxupquote{\sphinxhyphen{}c text2git}}. It applies a configuration template, but there
will be other parts of this book to explain this in detail.

\sphinxAtStartPar
Once created, a DataLad dataset looks like any other directory on your file system.
Currently, it seems empty.

\begin{sphinxVerbatim}[commandchars=\\\{\}]
\PYG{g+gp}{\PYGZdl{} }\PYG{n+nb}{cd}\PYG{+w}{ }DataLad\PYGZhy{}101
\PYG{g+gp}{\PYGZdl{} }ls\PYG{+w}{    }\PYG{c+c1}{\PYGZsh{} ls does not show any output, because the dataset is empty.}
\end{sphinxVerbatim}

\sphinxAtStartPar
However, all files and directories you store within the DataLad dataset
can be tracked (should you want them to be tracked).
\sphinxstyleemphasis{Tracking} in this context means that edits done to a file are automatically
associated with information about the change, the author of the edit,
and the time of this change. This is already informative important on its own
\textendash{} the {\hyperref[\detokenize{glossary:term-provenance}]{\sphinxtermref{\DUrole{xref,std,std-term}{provenance}}}} captured with this can, for example, be used to learn
about a file’s lineage, and can establish trust in it.
But what is especially helpful is that previous states of files or directories
can be restored. Remember the last time you accidentally deleted content
in a file, but only realized \sphinxstyleemphasis{after} you saved it? With DataLad, no
mistakes are forever. We will see many examples of this later in the book,
and such information is stored in what we will refer
to as the \sphinxstyleemphasis{history} of a dataset.

\index{log@\spxentry{log}!Git command@\spxentry{Git command}}\index{Git command@\spxentry{Git command}!log@\spxentry{log}}\index{exit pager@\spxentry{exit pager}!in a terminal@\spxentry{in a terminal}}\index{in a terminal@\spxentry{in a terminal}!exit pager@\spxentry{exit pager}}\index{show history@\spxentry{show history}!with Git@\spxentry{with Git}}\index{with Git@\spxentry{with Git}!show history@\spxentry{show history}}\ignorespaces 
\sphinxAtStartPar
This history is almost as small as it can be at the current state, but let’s take
a look at it. For looking at the history, the code examples will use \sphinxcode{\sphinxupquote{git log}},
a built\sphinxhyphen{}in {\hyperref[\detokenize{glossary:term-Git}]{\sphinxtermref{\DUrole{xref,std,std-term}{Git}}}} command%
\begin{footnote}\sphinxAtStartFootnote
A tool we can recommend as an alternative to \sphinxcode{\sphinxupquote{git log}} is {\hyperref[\detokenize{glossary:term-tig}]{\sphinxtermref{\DUrole{xref,std,std-term}{tig}}}}.
Once installed, exchange any \sphinxcode{\sphinxupquote{git log}} command you see here with the single word \sphinxcode{\sphinxupquote{tig}}.
%
\end{footnote} that works right in your terminal. Your log
\sphinxstyleemphasis{might} be opened in a terminal {\hyperref[\detokenize{glossary:term-pager}]{\sphinxtermref{\DUrole{xref,std,std-term}{pager}}}}
that lets you scroll up and down with your arrow keys, but not enter any more commands.
If this happens, you can get out of \sphinxcode{\sphinxupquote{git log}} by pressing \sphinxcode{\sphinxupquote{q}}. For Windows,
also see \textit{Windows-wit}~{\windowswiticoninline}\textit{\ref{ww-adjustedmode-log}} {\hyperref[\detokenize{basics/101-101-create:ww-adjustedmode-log}]{\sphinxcrossref{\DUrole{std,std-ref}{on showing history}}}} (\autopageref*{\detokenize{basics/101-101-create:ww-adjustedmode-log}}).

\fvset{hllines={, 3, 4, 6, 9, 10, 12,}}%
\begin{sphinxVerbatim}[commandchars=\\\{\}]
\PYG{g+gp}{\PYGZdl{} }git\PYG{+w}{ }log
\PYG{g+go}{commit e0ff3a73✂SHA1}
\PYG{g+go}{Author: Elena Piscopia \PYGZlt{}elena@example.net\PYGZgt{}}
\PYG{g+go}{Date:   Tue Jun 18 16:13:00 2019 +0000}

\PYG{g+go}{    Instruct annex to add text files to Git}

\PYG{g+go}{commit 4ce681d6✂SHA1}
\PYG{g+go}{Author: Elena Piscopia \PYGZlt{}elena@example.net\PYGZgt{}}
\PYG{g+go}{Date:   Tue Jun 18 16:13:00 2019 +0000}

\PYG{g+go}{    [DATALAD] new dataset}
\end{sphinxVerbatim}
\sphinxresetverbatimhllines

\sphinxAtStartPar
We can see two {\hyperref[\detokenize{glossary:term-commit}]{\sphinxtermref{\DUrole{xref,std,std-term}{commit}}}}s in the history of the repository.
Each of them is identified by a unique 40 character sequence, called a
{\hyperref[\detokenize{glossary:term-shasum}]{\sphinxtermref{\DUrole{xref,std,std-term}{shasum}}}}.

\index{log@\spxentry{log}!Git command@\spxentry{Git command}}\index{Git command@\spxentry{Git command}!log@\spxentry{log}}\index{corresponding branch@\spxentry{corresponding branch}!in adjusted mode@\spxentry{in adjusted mode}}\index{in adjusted mode@\spxentry{in adjusted mode}!corresponding branch@\spxentry{corresponding branch}}\index{show history@\spxentry{show history}!on Windows@\spxentry{on Windows}}\index{on Windows@\spxentry{on Windows}!show history@\spxentry{show history}}\ignorespaces \begin{windowswit}[label={ww-adjustedmode-log}, before title={\thetcbcounter\ }, float, floatplacement=tb, check odd page=true]{Your Git log may be more extensive - use ‘git log main’ instead!}
\label{\detokenize{basics/101-101-create:ww-adjustedmode-log}}

\sphinxAtStartPar
The output of \sphinxcode{\sphinxupquote{git log}} shown in the book and the output you will see in your own datasets when executing the same commands may not always match \textendash{} many times you might see commits about a “git\sphinxhyphen{}annex adjusted branch” in your history.
This is expected, and if you want to read up more about this, please progress on to chapter {\hyperref[\detokenize{basics/basics-annex:chapter-gitannex}]{\sphinxcrossref{\DUrole{std,std-ref}{Under the hood: git\sphinxhyphen{}annex}}}} (\autopageref*{\detokenize{basics/basics-annex:chapter-gitannex}}) and afterwards take a look at \dlhbhref{B1B}{this part of git\sphinxhyphen{}annex documentation}.

\sphinxAtStartPar
In order to get a similar experience in your dataset, please add the name of your default {\hyperref[\detokenize{glossary:term-branch}]{\sphinxtermref{\DUrole{xref,std,std-term}{branch}}}} (it will likely have the name \sphinxcode{\sphinxupquote{main}} or \sphinxcode{\sphinxupquote{master}}) to every \sphinxcode{\sphinxupquote{git log}} command.
This should display the same output that the book displays.
The reason behind this is that datasets are using a special {\hyperref[\detokenize{glossary:term-branch}]{\sphinxtermref{\DUrole{xref,std,std-term}{branch}}}} to be functional on Windows.
This branch’s history differs from the history that would be in the default branch.
With this workaround, you will be able to display the dataset history from the same branch that the book and all other operating system display.
Thus, whenever the book code snippet contains a line that starts with \sphinxcode{\sphinxupquote{git log}}, copy it and append the term \sphinxcode{\sphinxupquote{main}} or \sphinxcode{\sphinxupquote{master}}, whichever is appropriate.


\end{windowswit}

\sphinxAtStartPar
Highlighted in this output is information about the author and about
the time, as well as a {\hyperref[\detokenize{glossary:term-commit-message}]{\sphinxtermref{\DUrole{xref,std,std-term}{commit message}}}} that summarizes the
performed action concisely. In this case, both commit messages were written by
DataLad itself. The most recent change is on the top. The first commit
written to the history therefore states that a new dataset was created,
and the second commit is related to the \sphinxcode{\sphinxupquote{\sphinxhyphen{}c text2git}} option (which
uses a configuration template to instruct DataLad to store text files
in Git, but more on this later).
While these commits were produced and described by DataLad,
in most other cases, you will have to create the commit and
an informative commit message yourself.

\sphinxAtStartPar
Congratulations, you just created your first DataLad dataset!
Let us now put some content inside.

\index{create dataset@\spxentry{create dataset}!DataLad concept@\spxentry{DataLad concept}}\index{DataLad concept@\spxentry{DataLad concept}!create dataset@\spxentry{create dataset}}\ignorespaces \begin{gitusernote}[label={index-5}, before title={\thetcbcounter\ }, check odd page=true]{Create internals}
\label{\detokenize{basics/101-101-create:index-5}}

\sphinxAtStartPar
\sphinxcode{\sphinxupquote{datalad create}} uses \sphinxcode{\sphinxupquote{git init}} and \sphinxcode{\sphinxupquote{git annex init}}. Therefore,
the DataLad dataset is a Git repository.
Large file content in the
dataset is tracked with git\sphinxhyphen{}annex. An \sphinxcode{\sphinxupquote{ls \sphinxhyphen{}a}}
reveals that Git has secretly done its work:

\fvset{hllines={, 4, 5, 6,}}%
\begin{sphinxVerbatim}[commandchars=\\\{\}]
\PYG{g+gp}{\PYGZdl{} }ls\PYG{+w}{ }\PYGZhy{}a\PYG{+w}{ }\PYG{c+c1}{\PYGZsh{} show also hidden files}
\PYG{g+go}{.}
\PYG{g+go}{..}
\PYG{g+go}{.datalad}
\PYG{g+go}{.git}
\PYG{g+go}{.gitattributes}
\end{sphinxVerbatim}
\sphinxresetverbatimhllines

\sphinxAtStartPar
\sphinxstylestrong{For non\sphinxhyphen{}Git\sphinxhyphen{}Users:} these hidden \sphinxstyleemphasis{dot\sphinxhyphen{}directories} and \sphinxstyleemphasis{dot\sphinxhyphen{}files} are necessary for all Git operations. Please do not tamper with them, and, importantly, \sphinxstylestrong{do not delete them!}


\end{gitusernote}

\sphinxstepscope


\section{Populate a dataset}
\label{\detokenize{basics/101-102-populate:populate-a-dataset}}\label{\detokenize{basics/101-102-populate:populate}}\label{\detokenize{basics/101-102-populate::doc}}
\sphinxAtStartPar
The first lecture in DataLad\sphinxhyphen{}101 referenced some useful literature.
Even if we end up not reading those books at all, let’s download
them nevertheless and put them into our dataset. You never know, right?
Let’s first create a directory to save books for additional reading in.

\begin{sphinxVerbatim}[commandchars=\\\{\}]
\PYG{g+gp}{\PYGZdl{} }mkdir\PYG{+w}{ }books
\end{sphinxVerbatim}

\index{tree@\spxentry{tree}!terminal command@\spxentry{terminal command}}\index{terminal command@\spxentry{terminal command}!tree@\spxentry{tree}}\ignorespaces 
\sphinxAtStartPar
Let’s take a look at the current directory structure with the tree command%
\begin{footnote}\sphinxAtStartFootnote
\sphinxcode{\sphinxupquote{tree}} is a Unix command to list file system content. If it is not yet installed,
you can get it with your native package manager (e.g.,  \sphinxcode{\sphinxupquote{apt}}, \sphinxcode{\sphinxupquote{brew}}, or conda).
For example, if you use OSX, \sphinxcode{\sphinxupquote{brew install tree}}  will get you this tool.
Windows has its own \sphinxcode{\sphinxupquote{tree}} command.
Note that this \sphinxcode{\sphinxupquote{tree}} works slightly different than its Unix equivalent \sphinxhyphen{} by default, it will only display directories, not files, and the command options it accepts are either \sphinxcode{\sphinxupquote{/f}} (display file names) or \sphinxcode{\sphinxupquote{/a}} (change display of subdirectories to text instead of graphic characters).
%
\end{footnote}:

\begin{sphinxVerbatim}[commandchars=\\\{\}]
\PYG{g+gp}{\PYGZdl{} }tree
\PYG{g+go}{.}
\PYG{g+go}{└── books}

\PYG{g+go}{1 directory, 0 files}
\end{sphinxVerbatim}

\sphinxAtStartPar
Arguably, not the most exciting thing to see. So let’s put some PDFs inside.
Below is a short list of optional readings. We decide to download them (they
are all free, in total about 15 MB), and save them in \sphinxcode{\sphinxupquote{DataLad\sphinxhyphen{}101/books}}.
\begin{itemize}
\item {} 
\sphinxAtStartPar
Additional reading about the command line: \dlhbhref{S6A}{The Linux Command Line}

\item {} 
\sphinxAtStartPar
An intro to Python: \dlhbhref{G2Q}{A byte of Python}

\end{itemize}

\sphinxAtStartPar
You can either visit the links and save them in \sphinxcode{\sphinxupquote{books/}},
or run the following commands%
\begin{footnote}\sphinxAtStartFootnote
\sphinxcode{\sphinxupquote{wget}} is a Unix command for non\sphinxhyphen{}interactively downloading files from the
web. If it is not yet installed, you can get it with your native package manager (e.g.,
\sphinxcode{\sphinxupquote{apt}} or \sphinxcode{\sphinxupquote{brew}}). For example, if you use OSX, \sphinxcode{\sphinxupquote{brew install wget}}
will get you this tool.
%
\end{footnote} to download the books right from the terminal.
Note that we line break the command with \sphinxcode{\sphinxupquote{\textbackslash{}}} line continuation characters. In your own work you can write
commands like this into a single line. If you copy them into your terminal as they
are presented here, make sure to check the \textit{Windows-wit}~{\windowswiticoninline}\textit{\ref{ww-no-multiline-commands}} {\hyperref[\detokenize{basics/101-102-populate:ww-no-multiline-commands}]{\sphinxcrossref{\DUrole{std,std-ref}{on peculiarities of its terminals}}}} (\autopageref*{\detokenize{basics/101-102-populate:ww-no-multiline-commands}}).

\index{line continuation@\spxentry{line continuation}!on Windows in a terminal@\spxentry{on Windows in a terminal}}\index{on Windows in a terminal@\spxentry{on Windows in a terminal}!line continuation@\spxentry{line continuation}}\ignorespaces \begin{windowswit}[label={ww-no-multiline-commands}, before title={\thetcbcounter\ }, float, floatplacement=tbp, check odd page=true]{Terminals other than Git Bash can’t handle multi-line commands}
\label{\detokenize{basics/101-102-populate:ww-no-multiline-commands}}

\sphinxAtStartPar
In Unix shells, \sphinxcode{\sphinxupquote{\textbackslash{}}} can be used to split a command into several lines, for example to aid readability.
Standard Windows terminals (including the Anaconda prompt) do not support this.
They instead use the \sphinxcode{\sphinxupquote{\textasciicircum{}}} character:

\begin{sphinxVerbatim}[commandchars=\\\{\}]
\PYG{g+gp}{\PYGZdl{} }wget\PYG{+w}{ }\PYGZhy{}q\PYG{+w}{ }https://sourceforge.net/projects/linuxcommand/files/TLCL/19.01/TLCL\PYGZhy{}19.01.pdf/download\PYG{+w}{ }\PYGZca{}
\PYG{g+go}{\PYGZhy{}O TLCL.pdf}
\end{sphinxVerbatim}

\sphinxAtStartPar
If you are not using the Git Bash, you either need to copy multi\sphinxhyphen{}line commands into a single line, or use \sphinxcode{\sphinxupquote{\textasciicircum{}}} (make sure that there is \sphinxstylestrong{no space} afterwards) instead of \sphinxcode{\sphinxupquote{\textbackslash{}}}.


\end{windowswit}

\index{download file@\spxentry{download file}!with wget@\spxentry{with wget}}\index{with wget@\spxentry{with wget}!download file@\spxentry{download file}}\ignorespaces 
\def\sphinxLiteralBlockLabel{\label{\detokenize{basics/101-102-populate:index-2}}}
\begin{sphinxVerbatim}[commandchars=\\\{\}]
\PYG{g+gp}{\PYGZdl{} }\PYG{n+nb}{cd}\PYG{+w}{ }books
\PYG{g+gp}{\PYGZdl{} }wget\PYG{+w}{ }\PYGZhy{}q\PYG{+w}{ }https://sourceforge.net/projects/linuxcommand/files/TLCL/19.01/TLCL\PYGZhy{}19.01.pdf/download\PYG{+w}{ }\PYG{l+s+se}{\PYGZbs{}}
\PYG{+w}{  }\PYGZhy{}O\PYG{+w}{ }TLCL.pdf
\PYG{g+gp}{\PYGZdl{} }wget\PYG{+w}{ }\PYGZhy{}q\PYG{+w}{ }https://homepages.uc.edu/\PYGZti{}becktl/byte\PYGZus{}of\PYGZus{}python.pdf\PYG{+w}{ }\PYG{l+s+se}{\PYGZbs{}}
\PYG{+w}{  }\PYGZhy{}O\PYG{+w}{ }byte\PYGZhy{}of\PYGZhy{}python.pdf
\PYG{g+gp}{\PYGZdl{} }\PYG{c+c1}{\PYGZsh{} get back into the root of the dataset}
\PYG{g+gp}{\PYGZdl{} }\PYG{n+nb}{cd}\PYG{+w}{ }../
\end{sphinxVerbatim}

\sphinxAtStartPar
Some machines will not have \sphinxcode{\sphinxupquote{ wget}} available by default, but any command that can
download a file can work as an alternative. See the \textit{Windows-wit}~{\windowswiticoninline}\textit{\ref{ww-curl-instead-wget}} {\hyperref[\detokenize{basics/101-102-populate:ww-curl-instead-wget}]{\sphinxcrossref{\DUrole{std,std-ref}{for the popular alternative
curl}}}} (\autopageref*{\detokenize{basics/101-102-populate:ww-curl-instead-wget}}).

\index{curl instead of wget@\spxentry{curl instead of wget}!on Windows@\spxentry{on Windows}}\index{on Windows@\spxentry{on Windows}!curl instead of wget@\spxentry{curl instead of wget}}\index{download file@\spxentry{download file}!with curl@\spxentry{with curl}}\index{with curl@\spxentry{with curl}!download file@\spxentry{download file}}\ignorespaces \begin{windowswit}[label={ww-curl-instead-wget}, before title={\thetcbcounter\ }, float, floatplacement=tbp, check odd page=true]{You can use curl instead of wget}
\label{\detokenize{basics/101-102-populate:ww-curl-instead-wget}}

\sphinxAtStartPar
Many versions of Windows do not ship with the tool \sphinxcode{\sphinxupquote{wget}}.
You can install it, but it may be easier to use the pre\sphinxhyphen{}installed \sphinxcode{\sphinxupquote{curl}} command:

\begin{sphinxVerbatim}[commandchars=\\\{\}]
\PYG{g+gp}{\PYGZdl{} }\PYG{n+nb}{cd}\PYG{+w}{ }books
\PYG{g+gp}{\PYGZdl{} }curl\PYG{+w}{ }\PYGZhy{}L\PYG{+w}{ }https://sourceforge.net/projects/linuxcommand/files/TLCL/19.01/TLCL\PYGZhy{}19.01.pdf/download\PYG{+w}{ }\PYG{l+s+se}{\PYGZbs{}}
\PYG{+w}{  }\PYGZhy{}o\PYG{+w}{ }TLCL.pdf
\PYG{g+gp}{\PYGZdl{} }curl\PYG{+w}{ }\PYGZhy{}L\PYG{+w}{ }https://github.com/swaroopch/byte\PYGZhy{}of\PYGZhy{}python/releases/download/vadb91fc6fce27c58e3f931f5861806d3ccd1054c/byte\PYGZhy{}of\PYGZhy{}python.pdf\PYG{+w}{ }\PYG{l+s+se}{\PYGZbs{}}
\PYG{+w}{  }\PYGZhy{}o\PYG{+w}{ }byte\PYGZhy{}of\PYGZhy{}python.pdf
\PYG{g+gp}{\PYGZdl{} }\PYG{n+nb}{cd}\PYG{+w}{ }../
\end{sphinxVerbatim}


\end{windowswit}

\sphinxAtStartPar
Let’s see what happened. First of all, in the root of \sphinxcode{\sphinxupquote{DataLad\sphinxhyphen{}101}}, show the directory
structure with tree:

\begin{sphinxVerbatim}[commandchars=\\\{\}]
\PYG{g+gp}{\PYGZdl{} }tree
\PYG{g+go}{.}
\PYG{g+go}{└── books}
\PYG{g+go}{    ├── byte\PYGZhy{}of\PYGZhy{}python.pdf}
\PYG{g+go}{    └── TLCL.pdf}

\PYG{g+go}{1 directory, 2 files}
\end{sphinxVerbatim}

\index{status@\spxentry{status}!DataLad command@\spxentry{DataLad command}}\index{DataLad command@\spxentry{DataLad command}!status@\spxentry{status}}\index{check dataset for modification@\spxentry{check dataset for modification}!with DataLad@\spxentry{with DataLad}}\index{with DataLad@\spxentry{with DataLad}!check dataset for modification@\spxentry{check dataset for modification}}\ignorespaces 
\sphinxAtStartPar
Now what does DataLad do with this new content? One command you will use very
often is \sphinxcode{\sphinxupquote{datalad status}}.
It reports on the state of dataset content, and
regular status reports should become a habit in the wake of \sphinxcode{\sphinxupquote{DataLad\sphinxhyphen{}101}}.

\begin{sphinxVerbatim}[commandchars=\\\{\}]
\PYG{g+gp}{\PYGZdl{} }datalad\PYG{+w}{ }status
\PYG{g+go}{untracked: books (directory)}
\end{sphinxVerbatim}

\index{save@\spxentry{save}!DataLad command@\spxentry{DataLad command}}\index{DataLad command@\spxentry{DataLad command}!save@\spxentry{save}}\index{save dataset modification@\spxentry{save dataset modification}!with DataLad@\spxentry{with DataLad}}\index{with DataLad@\spxentry{with DataLad}!save dataset modification@\spxentry{save dataset modification}}\ignorespaces 
\sphinxAtStartPar
Interesting; the \sphinxcode{\sphinxupquote{books/}} directory is “untracked”. Remember how content
\sphinxstyleemphasis{can} be tracked \sphinxstyleemphasis{if a user wants to}?
Untracked means that DataLad does not know about this directory or its content,
because we have not instructed DataLad to actually track it. This means that DataLad
does not store the downloaded books in its history yet. Let’s change this by
\sphinxstyleemphasis{saving} the files to the dataset’s history with the \sphinxcode{\sphinxupquote{datalad save}} command.

\sphinxAtStartPar
This time, it is your turn to specify a helpful {\hyperref[\detokenize{glossary:term-commit-message}]{\sphinxtermref{\DUrole{xref,std,std-term}{commit message}}}}
with the \sphinxcode{\sphinxupquote{\sphinxhyphen{}m}} option (although the DataLad command is \sphinxcode{\sphinxupquote{datalad save}}, we talk
about commit messages because \sphinxcode{\sphinxupquote{datalad save}} ultimately uses the command
\sphinxcode{\sphinxupquote{git commit}} to do its work):

\begin{sphinxVerbatim}[commandchars=\\\{\}]
\PYG{g+gp}{\PYGZdl{} }datalad\PYG{+w}{ }save\PYG{+w}{ }\PYGZhy{}m\PYG{+w}{ }\PYG{l+s+s2}{\PYGZdq{}add books on Python and Unix to read later\PYGZdq{}}
\PYG{g+go}{add(ok): books/TLCL.pdf (file)}
\PYG{g+go}{add(ok): books/byte\PYGZhy{}of\PYGZhy{}python.pdf (file)}
\PYG{g+go}{save(ok): . (dataset)}
\end{sphinxVerbatim}

\sphinxAtStartPar
If you ever forget to specify a message, or made a typo, not all is lost. A
\textit{Find-out-more}~{\findoutmoreiconinline}\textit{\ref{fom-amend-save}} {\hyperref[\detokenize{basics/101-102-populate:fom-amend-save}]{\sphinxcrossref{\DUrole{std,std-ref}{explains how to amend a saved state}}}} (\autopageref*{\detokenize{basics/101-102-populate:fom-amend-save}}).

\index{amend commit message@\spxentry{amend commit message}!with Git@\spxentry{with Git}}\index{with Git@\spxentry{with Git}!amend commit message@\spxentry{amend commit message}}\ignorespaces \begin{findoutmore}[label={fom-amend-save}, before title={\thetcbcounter\ }, float, floatplacement=tbp, check odd page=true]{“Oh no! I forgot the -m option for ‘datalad save’!”}
\label{\detokenize{basics/101-102-populate:fom-amend-save}}

\sphinxAtStartPar
If you forget to specify a commit message with the \sphinxcode{\sphinxupquote{\sphinxhyphen{}m}} option, DataLad will write
\sphinxcode{\sphinxupquote{{[}DATALAD{]} Recorded changes}} as a commit message into your history.
This is not particularly informative.
You can change the \sphinxstyleemphasis{last} commit message with the Git command
\sphinxcode{\sphinxupquote{git commit \sphinxhyphen{}\sphinxhyphen{}amend}}. This will open up your default editor
and you can edit
the commit message. Careful \textendash{} the default editor might be {\hyperref[\detokenize{glossary:term-vim}]{\sphinxtermref{\DUrole{xref,std,std-term}{vim}}}}!
The section {\hyperref[\detokenize{basics/101-137-history:history}]{\sphinxcrossref{\DUrole{std,std-ref}{Git things done}}}} (\autopageref*{\detokenize{basics/101-137-history:history}}) will show you many more ways in which you can
interact with a dataset’s history.


\end{findoutmore}

\sphinxAtStartPar
As already noted, any files you \sphinxcode{\sphinxupquote{save}} in this dataset, and all modifications
to these files that you \sphinxcode{\sphinxupquote{save}}, are tracked in this history.
Importantly, this file tracking works
regardless of the size of the files \textendash{} a DataLad dataset could be
your private music or movie collection with single files being many GB in size.
This is one aspect that distinguishes DataLad from many other
version control tools, among them Git.
Large content is tracked in an \sphinxstyleemphasis{annex} that is automatically
created and handled by DataLad. Whether text files or larger files change,
all of these changes can be written to your DataLad dataset’s history.

\index{log@\spxentry{log}!Git command@\spxentry{Git command}}\index{Git command@\spxentry{Git command}!log@\spxentry{log}}\index{show last commit@\spxentry{show last commit}!with Git@\spxentry{with Git}}\index{with Git@\spxentry{with Git}!show last commit@\spxentry{show last commit}}\ignorespaces 
\sphinxAtStartPar
Let’s see how the saved content shows up in the history of the dataset with \sphinxcode{\sphinxupquote{git log}}.
The option \sphinxcode{\sphinxupquote{\sphinxhyphen{}n 1}} specifies that we want to take a look at the most recent commit.
In order to get a bit more details, we add the \sphinxcode{\sphinxupquote{\sphinxhyphen{}p}} flag. If you end up in a
{\hyperref[\detokenize{glossary:term-pager}]{\sphinxtermref{\DUrole{xref,std,std-term}{pager}}}}, navigate with up and down arrow keys and leave the log by typing \sphinxcode{\sphinxupquote{q}}:

\fvset{hllines={, 3, 4, 6, 8, 12, 16, 20,}}%
\begin{sphinxVerbatim}[commandchars=\\\{\}]
\PYG{g+gp}{\PYGZdl{} }git\PYG{+w}{ }log\PYG{+w}{ }\PYGZhy{}p\PYG{+w}{ }\PYGZhy{}n\PYG{+w}{ }\PYG{l+m}{1}
\PYG{g+go}{commit b40316a6✂SHA1}
\PYG{g+go}{Author: Elena Piscopia \PYGZlt{}elena@example.net\PYGZgt{}}
\PYG{g+go}{Date:   Tue Jun 18 16:13:00 2019 +0000}

\PYG{g+go}{    add books on Python and Unix to read later}

\PYG{g+go}{diff \PYGZhy{}\PYGZhy{}git a/books/TLCL.pdf b/books/TLCL.pdf}
\PYG{g+go}{new file mode 120000}
\PYG{g+go}{index 0000000..4c84b61}
\PYG{g+go}{\PYGZhy{}\PYGZhy{}\PYGZhy{} /dev/null}
\PYG{g+go}{+++ b/books/TLCL.pdf}
\PYG{g+go}{@@ \PYGZhy{}0,0 +1 @@}
\PYG{g+go}{+../.git/annex/objects/jf/3M/✂/MD5E\PYGZhy{}s2120211\PYGZhy{}\PYGZhy{}06d1efcb✂MD5.pdf}
\PYG{g+go}{\PYGZbs{} No newline at end of file}
\PYG{g+go}{diff \PYGZhy{}\PYGZhy{}git a/books/byte\PYGZhy{}of\PYGZhy{}python.pdf b/books/byte\PYGZhy{}of\PYGZhy{}python.pdf}
\PYG{g+go}{new file mode 120000}
\PYG{g+go}{index 0000000..7a6e51e}
\PYG{g+go}{\PYGZhy{}\PYGZhy{}\PYGZhy{} /dev/null}
\PYG{g+go}{+++ b/books/byte\PYGZhy{}of\PYGZhy{}python.pdf}
\end{sphinxVerbatim}
\sphinxresetverbatimhllines

\sphinxAtStartPar
Now this might look a bit cryptic (and honestly, tig%
\begin{footnote}\sphinxAtStartFootnote
See {\hyperref[\detokenize{glossary:term-tig}]{\sphinxtermref{\DUrole{xref,std,std-term}{tig}}}}. Once installed, exchange any git log command you
see here with the single word \sphinxcode{\sphinxupquote{tig}}.
%
\end{footnote} makes it look prettier).
But this tells us the date and time in which a particular author added two PDFs to
the directory \sphinxcode{\sphinxupquote{books/}}, and thanks to that commit message we have a nice
human\sphinxhyphen{}readable summary of that action. A \textit{Find-out-more}~{\findoutmoreiconinline}\textit{\ref{fom-commit-message-guidance}} {\hyperref[\detokenize{basics/101-102-populate:fom-commit-message-guidance}]{\sphinxcrossref{\DUrole{std,std-ref}{explains what makes
a good message}}}} (\autopageref*{\detokenize{basics/101-102-populate:fom-commit-message-guidance}}).

\index{recommendation@\spxentry{recommendation}!commit message@\spxentry{commit message}}\index{commit message@\spxentry{commit message}!recommendation@\spxentry{recommendation}}\ignorespaces \begin{findoutmore}[label={fom-commit-message-guidance}, before title={\thetcbcounter\ }, float, floatplacement=tb, check odd page=true]{DOs and DON’Ts for commit messages}
\label{\detokenize{basics/101-102-populate:fom-commit-message-guidance}}

\sphinxAtStartPar
\sphinxstylestrong{DOs}
\begin{itemize}
\item {} 
\sphinxAtStartPar
Write a \sphinxstyleemphasis{title line} with 72 characters or less

\item {} 
\sphinxAtStartPar
Use imperative voice, e.g., “Add notes from lecture 2”

\item {} 
\sphinxAtStartPar
If a title line is not enough to express your changes and reasoning behind it, add a body to your commit message: hit enter twice (before closing the quotation marks), and continue writing a brief summary of the changes after a blank line. This summary should explain “what” has been done and “why”, but not “how”. Close the quotation marks, and hit enter to save the change with your message.

\end{itemize}

\sphinxAtStartPar
\sphinxstylestrong{DON’Ts}
\begin{itemize}
\item {} 
\sphinxAtStartPar
Avoid passive voice

\item {} 
\sphinxAtStartPar
Extensive formatting (hashes, asterisks, quotes, …) will most likely make your shell complain

\item {} 
\sphinxAtStartPar
Do not say nasty things about other people

\end{itemize}


\end{findoutmore}

\index{no staging@\spxentry{no staging}!with DataLad@\spxentry{with DataLad}}\index{with DataLad@\spxentry{with DataLad}!no staging@\spxentry{no staging}}\ignorespaces \begin{gitusernote}[label={index-9}, before title={\thetcbcounter\ }, float, floatplacement=tb, check odd page=true]{There is no staging area in DataLad}
\label{\detokenize{basics/101-102-populate:index-9}}

\sphinxAtStartPar
Just as in Git, new files are not tracked from their creation on, but only when
explicitly added to Git (in Git terms, with an initial \sphinxcode{\sphinxupquote{git add}}). But different
from the common Git workflow, DataLad skips the staging area. A \sphinxcode{\sphinxupquote{datalad save}}
combines a \sphinxcode{\sphinxupquote{git add}} and a \sphinxcode{\sphinxupquote{git commit}}, and therefore, the commit message
is specified with \sphinxcode{\sphinxupquote{datalad save}}.


\end{gitusernote}

\sphinxAtStartPar
Cool, so now you have added some files to your dataset history. But what is a bit
inconvenient is that both books were saved \sphinxstyleemphasis{together}. You begin to wonder: “A Python
book and a Unix book do not have that much in common. I probably should not save them
in the same commit. And … what happens if I have files I do not want to track?
\sphinxcode{\sphinxupquote{datalad save \sphinxhyphen{}m "some commit message"}} would save all of what is currently
untracked or modified in the dataset into the history!”

\sphinxAtStartPar
Regarding your first remark, you are absolutely right!
It is good practice to save only those changes
together that belong together. We do not want to squish completely unrelated changes
into the same spot of our history, because it would get very nasty should we want to
revert \sphinxstyleemphasis{some} of the changes without affecting others in this commit.

\sphinxAtStartPar
Luckily, we can point \sphinxcode{\sphinxupquote{datalad save}} to exactly the changes we want it to record.
Let’s try this by adding yet another book, a good reference work about git,
\dlhbhref{G1A}{Pro Git}:

\begin{sphinxVerbatim}[commandchars=\\\{\}]
\PYG{g+gp}{\PYGZdl{} }\PYG{n+nb}{cd}\PYG{+w}{ }books
\PYG{g+gp}{\PYGZdl{} }wget\PYG{+w}{ }\PYGZhy{}q\PYG{+w}{ }https://github.com/progit/progit2/releases/download/2.1.154/progit.pdf
\PYG{g+gp}{\PYGZdl{} }\PYG{n+nb}{cd}\PYG{+w}{ }../
\end{sphinxVerbatim}

\sphinxAtStartPar
\sphinxcode{\sphinxupquote{datalad status}} shows that there is a new untracked file:

\begin{sphinxVerbatim}[commandchars=\\\{\}]
\PYG{g+gp}{\PYGZdl{} }datalad\PYG{+w}{ }status
\PYG{g+go}{untracked: books/progit.pdf (file)}
\end{sphinxVerbatim}

\sphinxAtStartPar
Let’s give \sphinxcode{\sphinxupquote{datalad save}} precisely this file by specifying its path after the commit message:

\begin{sphinxVerbatim}[commandchars=\\\{\}]
\PYG{g+gp}{\PYGZdl{} }datalad\PYG{+w}{ }save\PYG{+w}{ }\PYGZhy{}m\PYG{+w}{ }\PYG{l+s+s2}{\PYGZdq{}add reference book about git\PYGZdq{}}\PYG{+w}{ }books/progit.pdf
\PYG{g+go}{add(ok): books/progit.pdf (file)}
\PYG{g+go}{save(ok): . (dataset)}
\end{sphinxVerbatim}

\sphinxAtStartPar
Regarding your second remark, you are right that a \sphinxcode{\sphinxupquote{datalad save}} without a
path specification would write all of the currently untracked files or modifications
to the history. But check the \textit{Find-out-more}~{\findoutmoreiconinline}\textit{\ref{fom-save-updated-only}} {\hyperref[\detokenize{basics/101-102-populate:fom-save-updated-only}]{\sphinxcrossref{\DUrole{std,std-ref}{on how to tell it otherwise}}}} (\autopageref*{\detokenize{basics/101-102-populate:fom-save-updated-only}}).

\index{save already tracked files only@\spxentry{save already tracked files only}!with DataLad@\spxentry{with DataLad}}\index{with DataLad@\spxentry{with DataLad}!save already tracked files only@\spxentry{save already tracked files only}}\ignorespaces \begin{findoutmore}[label={fom-save-updated-only}, before title={\thetcbcounter\ }, float, floatplacement=tb, check odd page=true]{How to save already tracked dataset components only?}
\label{\detokenize{basics/101-102-populate:fom-save-updated-only}}

\sphinxAtStartPar
A \sphinxcode{\sphinxupquote{datalad save \sphinxhyphen{}m "concise message" \sphinxhyphen{}\sphinxhyphen{}updated}} (or the shorter
form of \sphinxcode{\sphinxupquote{\sphinxhyphen{}\sphinxhyphen{}updated}}, \sphinxcode{\sphinxupquote{\sphinxhyphen{}u}}) will only write \sphinxstyleemphasis{modifications} to the
history, not untracked files. Later, we will also see \sphinxcode{\sphinxupquote{.gitignore}} files
that let you hide content from version control.  However, it is good
practice to safely store away modifications or new content.  This improves
your dataset and workflow, and will be a requirement for executing certain
commands.


\end{findoutmore}

\sphinxAtStartPar
A \sphinxcode{\sphinxupquote{datalad status}} should now be empty, and our dataset’s history should look like this:

\index{show history (compact)@\spxentry{show history}\spxextra{compact}!with Git@\spxentry{with Git}}\index{with Git@\spxentry{with Git}!show history (compact)@\spxentry{show history}\spxextra{compact}}\ignorespaces 
\def\sphinxLiteralBlockLabel{\label{\detokenize{basics/101-102-populate:index-11}}}
\begin{sphinxVerbatim}[commandchars=\\\{\}]
\PYG{g+gp}{\PYGZdl{} }\PYG{c+c1}{\PYGZsh{} lets make the output a bit more concise with the \PYGZhy{}\PYGZhy{}oneline option}
\PYG{g+gp}{\PYGZdl{} }git\PYG{+w}{ }log\PYG{+w}{ }\PYGZhy{}\PYGZhy{}oneline
\PYG{g+go}{a875e49 add reference book about git}
\PYG{g+go}{b40316a add books on Python and Unix to read later}
\PYG{g+go}{e0ff3a7 Instruct annex to add text files to Git}
\PYG{g+go}{4ce681d [DATALAD] new dataset}
\end{sphinxVerbatim}

\sphinxAtStartPar
“Wonderful! I’m getting a hang on this quickly”, you think. “Version controlling
files is not as hard as I thought!”

\sphinxAtStartPar
But downloading and adding content to your dataset “manually” has two
disadvantages: For one, it requires you to download the content and save it.
Compared to a workflow with no DataLad dataset, this is one additional command
you have to perform (\dlhbhref{X1A}{and that additional time adds up, after a while}). But a more
serious disadvantage is that you have no electronic record of the source of the
contents you added. The amount of {\hyperref[\detokenize{glossary:term-provenance}]{\sphinxtermref{\DUrole{xref,std,std-term}{provenance}}}}, the time, date, and author
of file, is already quite nice, but we don’t know anything about where you downloaded
these files from. If you would want to find out, you would have to \sphinxstyleemphasis{remember}
where you got the content from \textendash{} and brains are not made for such tasks.

\sphinxAtStartPar
Luckily, DataLad has a command that will solve both of these problems:
The \sphinxcode{\sphinxupquote{datalad download\sphinxhyphen{}url}} command.
We will dive deeper into the provenance\sphinxhyphen{}related benefits of using it in later chapters, but for now,
we’ll start with best\sphinxhyphen{}practice\sphinxhyphen{}building. \sphinxcode{\sphinxupquote{datalad download\sphinxhyphen{}url}} can retrieve content
from a URL (following any URL\sphinxhyphen{}scheme from https, http, or ftp or s3) and save it
into the dataset together with a human\sphinxhyphen{}readable commit message and a hidden,
machine\sphinxhyphen{}readable record of the origin of the content. This saves you time,
and captures {\hyperref[\detokenize{glossary:term-provenance}]{\sphinxtermref{\DUrole{xref,std,std-term}{provenance}}}} information about the data you add to your dataset.
To experience this, let’s add a final book,
\dlhbhref{T1A}{a beginner’s guide to bash},
to the dataset. We provide the command with a URL, a pointer to the dataset the
file should be saved in (\sphinxcode{\sphinxupquote{.}} denotes “current directory”), and a commit message.

\begin{sphinxVerbatim}[commandchars=\\\{\}]
\PYG{g+gp}{\PYGZdl{} }datalad\PYG{+w}{ }download\PYGZhy{}url\PYG{+w}{ }\PYG{l+s+se}{\PYGZbs{}}
\PYG{+w}{  }https://www.tldp.org/LDP/Bash\PYGZhy{}Beginners\PYGZhy{}Guide/Bash\PYGZhy{}Beginners\PYGZhy{}Guide.pdf\PYG{+w}{ }\PYG{l+s+se}{\PYGZbs{}}
\PYG{+w}{  }\PYGZhy{}\PYGZhy{}dataset\PYG{+w}{ }.\PYG{+w}{ }\PYG{l+s+se}{\PYGZbs{}}
\PYG{+w}{  }\PYGZhy{}m\PYG{+w}{ }\PYG{l+s+s2}{\PYGZdq{}add beginners guide on bash\PYGZdq{}}\PYG{+w}{ }\PYG{l+s+se}{\PYGZbs{}}
\PYG{+w}{  }\PYGZhy{}O\PYG{+w}{ }books/bash\PYGZus{}guide.pdf
\PYG{g+go}{download\PYGZus{}url(ok): /home/me/dl\PYGZhy{}101/DataLad\PYGZhy{}101/books/bash\PYGZus{}guide.pdf (file)}
\PYG{g+go}{add(ok): books/bash\PYGZus{}guide.pdf (file)}
\PYG{g+go}{save(ok): . (dataset)}
\end{sphinxVerbatim}

\sphinxAtStartPar
Afterwards, a fourth book is inside your \sphinxcode{\sphinxupquote{books/}} directory:

\begin{sphinxVerbatim}[commandchars=\\\{\}]
\PYG{g+gp}{\PYGZdl{} }ls\PYG{+w}{ }books
\PYG{g+go}{bash\PYGZus{}guide.pdf}
\PYG{g+go}{byte\PYGZhy{}of\PYGZhy{}python.pdf}
\PYG{g+go}{progit.pdf}
\PYG{g+go}{TLCL.pdf}
\end{sphinxVerbatim}

\sphinxAtStartPar
However, the \sphinxcode{\sphinxupquote{datalad status}} command does not return any output \textendash{}
the dataset state is “clean”:

\begin{sphinxVerbatim}[commandchars=\\\{\}]
\PYG{g+gp}{\PYGZdl{} }datalad\PYG{+w}{ }status
\PYG{g+go}{nothing to save, working tree clean}
\end{sphinxVerbatim}

\sphinxAtStartPar
This is because \sphinxcode{\sphinxupquote{datalad download\sphinxhyphen{}url}} took care of saving for you:

\begin{sphinxVerbatim}[commandchars=\\\{\}]
\PYG{g+gp}{\PYGZdl{} }git\PYG{+w}{ }log\PYG{+w}{ }\PYGZhy{}p\PYG{+w}{ }\PYGZhy{}n\PYG{+w}{ }\PYG{l+m}{1}
\PYG{g+go}{commit 59ac8d32✂SHA1}
\PYG{g+go}{Author: Elena Piscopia \PYGZlt{}elena@example.net\PYGZgt{}}
\PYG{g+go}{Date:   Tue Jun 18 16:13:00 2019 +0000}

\PYG{g+go}{    add beginners guide on bash}

\PYG{g+go}{diff \PYGZhy{}\PYGZhy{}git a/books/bash\PYGZus{}guide.pdf b/books/bash\PYGZus{}guide.pdf}
\PYG{g+go}{new file mode 120000}
\PYG{g+go}{index 0000000..00ca6bd}
\PYG{g+go}{\PYGZhy{}\PYGZhy{}\PYGZhy{} /dev/null}
\PYG{g+go}{+++ b/books/bash\PYGZus{}guide.pdf}
\PYG{g+go}{@@ \PYGZhy{}0,0 +1 @@}
\PYG{g+go}{+../.git/annex/objects/WF/Gq/✂/MD5E\PYGZhy{}s1198170\PYGZhy{}\PYGZhy{}0ab2c121✂MD5.pdf}
\PYG{g+go}{\PYGZbs{} No newline at end of file}
\end{sphinxVerbatim}

\sphinxAtStartPar
At this point in time, the biggest advantage may seem to be the time save. However,
soon you will experience how useful it is to have DataLad keep track for you where
file content came from.

\sphinxAtStartPar
To conclude this section, let’s take a final look at the history of your dataset at
this point:

\begin{sphinxVerbatim}[commandchars=\\\{\}]
\PYG{g+gp}{\PYGZdl{} }git\PYG{+w}{ }log\PYG{+w}{ }\PYGZhy{}\PYGZhy{}oneline
\PYG{g+go}{59ac8d3 add beginners guide on bash}
\PYG{g+go}{a875e49 add reference book about git}
\PYG{g+go}{b40316a add books on Python and Unix to read later}
\PYG{g+go}{e0ff3a7 Instruct annex to add text files to Git}
\PYG{g+go}{4ce681d [DATALAD] new dataset}
\end{sphinxVerbatim}

\sphinxAtStartPar
Well done! Your \sphinxcode{\sphinxupquote{DataLad\sphinxhyphen{}101}} dataset and its history are slowly growing.

\sphinxstepscope


\section{Modify content}
\label{\detokenize{basics/101-103-modify:modify-content}}\label{\detokenize{basics/101-103-modify:modify}}\label{\detokenize{basics/101-103-modify::doc}}
\sphinxAtStartPar
So far, we’ve only added new content to the dataset. And we have not done
much to that content up to this point, to be honest. Let’s see what happens if
we add content, and then modify it.

\sphinxAtStartPar
For this, in the root of \sphinxcode{\sphinxupquote{DataLad\sphinxhyphen{}101}}, create a plain text file
called \sphinxcode{\sphinxupquote{notes.txt}}. It will contain all of the notes that you take
throughout the course.

\sphinxAtStartPar
Let’s write a short summary of how to create a DataLad dataset from scratch:
\begin{quote}

\sphinxAtStartPar
“One can create a new dataset with ‘datalad create
{[}\textendash{}description{]} PATH’. The dataset is created empty”.
\end{quote}

\sphinxAtStartPar
This is meant to be a note you would take in an educational course.
You can take this note and write it to a file with an editor of your choice.
The next code snippet, however, contains this note within the start and end part of a
\dlhbhref{W1G}{heredoc}.
You can also copy the full code snippet, starting
from \sphinxcode{\sphinxupquote{cat \textless{}\textless{} EOT \textgreater{} notes.txt}}, including the \sphinxcode{\sphinxupquote{EOT}} in the last line, in your
terminal to write this note from the terminal (without any editor) into \sphinxcode{\sphinxupquote{notes.txt}}.
Check the \textit{Find-out-more}~{\findoutmoreiconinline}\textit{\ref{fom-heredoc}} {\hyperref[\detokenize{basics/101-103-modify:fom-heredoc}]{\sphinxcrossref{\DUrole{std,std-ref}{on some heredocs background}}}} (\autopageref*{\detokenize{basics/101-103-modify:fom-heredoc}}).

\index{here\sphinxhyphen{}document@\spxentry{here\sphinxhyphen{}document}}\index{heredoc@\spxentry{heredoc}}\ignorespaces \begin{findoutmore}[label={fom-heredoc}, before title={\thetcbcounter\ }, float, floatplacement=tb, check odd page=true]{How does a heredoc (here-document) work?}
\label{\detokenize{basics/101-103-modify:fom-heredoc}}

\sphinxAtStartPar
The code snippet makes sure to write lines of text into a
file (that so far does not exist) called \sphinxcode{\sphinxupquote{notes.txt}}.

\sphinxAtStartPar
To do this, the content of the “document” is wrapped in between
\sphinxstyleemphasis{delimiting identifiers}. Here, these identifiers are \sphinxstyleemphasis{EOT} (short
for “end of text”), but naming is arbitrary as long as the two identifiers
are identical. The first “EOT” identifies the start of the text stream, and
the second “EOT” terminates the text stream.

\sphinxAtStartPar
The characters \sphinxcode{\sphinxupquote{\textless{}\textless{}}} redirect the text stream into
\dlhbhref{W1T}{“standard input” (stdin)},
the standard location that provides the \sphinxstyleemphasis{input} for a command.
Thus, the text stream becomes the input for the
\dlhbhref{W1A}{cat command}, which takes
the input and writes it to
\dlhbhref{W1U}{“standard output” (stdout)}.

\sphinxAtStartPar
Lastly, the \sphinxcode{\sphinxupquote{\textgreater{}}} character takes \sphinxcode{\sphinxupquote{stdout}} can creates a new file
\sphinxcode{\sphinxupquote{notes.txt}} with \sphinxcode{\sphinxupquote{stdout}} as its contents.

\sphinxAtStartPar
It might seem like a slightly convoluted way to create a text file with
a note in it. But it allows to write notes from the terminal, enabling
this book to create commands you can execute with nothing other than your terminal.
You are free to copy\sphinxhyphen{}paste the snippets with the heredocs,
or find a workflow that suites you better. The only thing important is that
you create and modify a \sphinxcode{\sphinxupquote{.txt}} file while working with this book.


\end{findoutmore}

\sphinxAtStartPar
Running this command will create \sphinxcode{\sphinxupquote{notes.txt}} in the
root of your \sphinxcode{\sphinxupquote{DataLad\sphinxhyphen{}101}} dataset, but see the \textit{Windows-wit}~{\windowswiticoninline}\textit{\ref{ww-heredoc}} {\hyperref[\detokenize{basics/101-103-modify:ww-heredoc}]{\sphinxcrossref{\DUrole{std,std-ref}{on terminal compatibility}}}} (\autopageref*{\detokenize{basics/101-103-modify:ww-heredoc}}):

\index{heredoc@\spxentry{heredoc}}\index{heredoc@\spxentry{heredoc}!on Windows in a terminal@\spxentry{on Windows in a terminal}}\index{on Windows in a terminal@\spxentry{on Windows in a terminal}!heredoc@\spxentry{heredoc}}\ignorespaces \begin{windowswit}[label={ww-heredoc}, before title={\thetcbcounter\ }, float, floatplacement=tb, check odd page=true]{Heredocs don’t work under non-Git-Bash Windows terminals}
\label{\detokenize{basics/101-103-modify:ww-heredoc}}

\sphinxAtStartPar
Heredocs rely on Unix\sphinxhyphen{}type redirection and multi\sphinxhyphen{}line commands \textendash{} which is not supported on most native Windows terminals or the Anaconda prompt on Windows.
If you are using an Anaconda prompt or a Windows terminal other than Git Bash, instead of executing heredocs, please open up an editor and paste and save the text into it.

\sphinxAtStartPar
The relevant text in the snippet below would be:

\begin{sphinxVerbatim}[commandchars=\\\{\}]
One can create a new dataset with \PYGZsq{}datalad create [\PYGZhy{}\PYGZhy{}description] PATH\PYGZsq{}.
The dataset is created empty
\end{sphinxVerbatim}

\sphinxAtStartPar
If you are using Git Bash, however, here docs will work just fine.


\end{windowswit}

\index{create heredoc@\spxentry{create heredoc}!in a terminal@\spxentry{in a terminal}}\index{in a terminal@\spxentry{in a terminal}!create heredoc@\spxentry{create heredoc}}\ignorespaces 
\def\sphinxLiteralBlockLabel{\label{\detokenize{basics/101-103-modify:index-2}}}
\begin{sphinxVerbatim}[commandchars=\\\{\}]
\PYG{g+gp}{\PYGZdl{} }cat\PYG{+w}{ }\PYGZlt{}\PYGZlt{}\PYG{+w}{ }EOT\PYG{+w}{ }\PYGZgt{}\PYG{+w}{ }notes.txt
\PYG{g+go}{One can create a new dataset with \PYGZsq{}datalad create [\PYGZhy{}\PYGZhy{}description] PATH\PYGZsq{}.}
\PYG{g+go}{The dataset is created empty}

\PYG{g+go}{EOT}
\end{sphinxVerbatim}

\index{check dataset for modification@\spxentry{check dataset for modification}!with DataLad@\spxentry{with DataLad}}\index{with DataLad@\spxentry{with DataLad}!check dataset for modification@\spxentry{check dataset for modification}}\ignorespaces 
\sphinxAtStartPar
Run \sphinxcode{\sphinxupquote{datalad status}} to confirm that there is a new, untracked file:

\begin{sphinxVerbatim}[commandchars=\\\{\}]
\PYG{g+gp}{\PYGZdl{} }datalad\PYG{+w}{ }status
\PYG{g+go}{untracked: notes.txt (file)}
\end{sphinxVerbatim}

\index{save dataset modification@\spxentry{save dataset modification}!with DataLad@\spxentry{with DataLad}}\index{with DataLad@\spxentry{with DataLad}!save dataset modification@\spxentry{save dataset modification}}\ignorespaces 
\sphinxAtStartPar
Save the current state of this file in your dataset’s history. Because it is the only modification
in the dataset, there is no need to specify a path.

\begin{sphinxVerbatim}[commandchars=\\\{\}]
\PYG{g+gp}{\PYGZdl{} }datalad\PYG{+w}{ }save\PYG{+w}{ }\PYGZhy{}m\PYG{+w}{ }\PYG{l+s+s2}{\PYGZdq{}Add notes on datalad create\PYGZdq{}}
\PYG{g+go}{add(ok): notes.txt (file)}
\PYG{g+go}{save(ok): . (dataset)}
\end{sphinxVerbatim}

\sphinxAtStartPar
But now, let’s see how \sphinxstyleemphasis{changing} tracked content works.
Modify this file by adding another note. After all, you already know how to use
\sphinxcode{\sphinxupquote{datalad save}}, so write a short summary on that as well.

\sphinxAtStartPar
Again, the example uses Unix commands (\sphinxcode{\sphinxupquote{cat}} and redirection, this time however
with \sphinxcode{\sphinxupquote{\textgreater{}\textgreater{}}} to \sphinxstyleemphasis{append} new content to the existing file)
to accomplish this, but you can take any editor of your choice.

\begin{sphinxVerbatim}[commandchars=\\\{\}]
\PYG{g+gp}{\PYGZdl{} }cat\PYG{+w}{ }\PYGZlt{}\PYGZlt{}\PYG{+w}{ }EOT\PYG{+w}{ }\PYGZgt{}\PYGZgt{}\PYG{+w}{ }notes.txt
\PYG{g+go}{The command \PYGZdq{}datalad save [\PYGZhy{}m] PATH\PYGZdq{} saves the file (modifications) to}
\PYG{g+go}{history.}
\PYG{g+go}{Note to self: Always use informative, concise commit messages.}

\PYG{g+go}{EOT}
\end{sphinxVerbatim}

\sphinxAtStartPar
Let’s check the dataset’s current state:

\begin{sphinxVerbatim}[commandchars=\\\{\}]
\PYG{g+gp}{\PYGZdl{} }datalad\PYG{+w}{ }status
\PYG{g+go}{ modified: notes.txt (file)}
\end{sphinxVerbatim}

\sphinxAtStartPar
and save the file in DataLad:

\begin{sphinxVerbatim}[commandchars=\\\{\}]
\PYG{g+gp}{\PYGZdl{} }datalad\PYG{+w}{ }save\PYG{+w}{ }\PYGZhy{}m\PYG{+w}{ }\PYG{l+s+s2}{\PYGZdq{}add note on datalad save\PYGZdq{}}
\PYG{g+go}{add(ok): notes.txt (file)}
\PYG{g+go}{save(ok): . (dataset)}
\end{sphinxVerbatim}

\sphinxAtStartPar
Let’s take another look into our history to see the development of this file.
We are using \sphinxcode{\sphinxupquote{git log \sphinxhyphen{}p \sphinxhyphen{}n 2}} to see last two commits and explore
the difference to the previous state of a file within each commit.

\fvset{hllines={, 6, 25,}}%
\begin{sphinxVerbatim}[commandchars=\\\{\}]
\PYG{g+gp}{\PYGZdl{} }git\PYG{+w}{ }log\PYG{+w}{ }\PYGZhy{}p\PYG{+w}{ }\PYGZhy{}n\PYG{+w}{ }\PYG{l+m}{2}
\PYG{g+go}{commit e310b465✂SHA1}
\PYG{g+go}{Author: Elena Piscopia \PYGZlt{}elena@example.net\PYGZgt{}}
\PYG{g+go}{Date:   Tue Jun 18 16:13:00 2019 +0000}

\PYG{g+go}{    add note on datalad save}

\PYG{g+go}{diff \PYGZhy{}\PYGZhy{}git a/notes.txt b/notes.txt}
\PYG{g+go}{index 3a7a1fe..0142412 100644}
\PYG{g+go}{\PYGZhy{}\PYGZhy{}\PYGZhy{} a/notes.txt}
\PYG{g+go}{+++ b/notes.txt}
\PYG{g+go}{@@ \PYGZhy{}1,3 +1,7 @@}
\PYG{g+go}{ One can create a new dataset with \PYGZsq{}datalad create [\PYGZhy{}\PYGZhy{}description] PATH\PYGZsq{}.}
\PYG{g+go}{ The dataset is created empty}

\PYG{g+go}{+The command \PYGZdq{}datalad save [\PYGZhy{}m] PATH\PYGZdq{} saves the file (modifications) to}
\PYG{g+go}{+history.}
\PYG{g+go}{+Note to self: Always use informative, concise commit messages.}
\PYG{g+go}{+}

\PYG{g+go}{commit 874d766f✂SHA1}
\PYG{g+go}{Author: Elena Piscopia \PYGZlt{}elena@example.net\PYGZgt{}}
\PYG{g+go}{Date:   Tue Jun 18 16:13:00 2019 +0000}

\PYG{g+go}{    Add notes on datalad create}

\PYG{g+go}{diff \PYGZhy{}\PYGZhy{}git a/notes.txt b/notes.txt}
\PYG{g+go}{new file mode 100644}
\end{sphinxVerbatim}
\sphinxresetverbatimhllines

\sphinxAtStartPar
We can see that the history can not only show us the commit message attached to
a commit, but also the precise change that occurred in the text file in the commit.
Additions are marked with a \sphinxcode{\sphinxupquote{+}}, and deletions would be shown with a leading \sphinxcode{\sphinxupquote{\sphinxhyphen{}}}.
From the dataset’s history, we can therefore also find out \sphinxstyleemphasis{how} the text file
evolved over time. That’s quite neat, isn’t it?
Check the \textit{Find-out-more}~{\findoutmoreiconinline}\textit{\ref{fom-log-options}} {\hyperref[\detokenize{basics/101-103-modify:fom-log-options}]{\sphinxcrossref{\DUrole{std,std-ref}{on \textquotesingle{}git log\textquotesingle{} for even more possibilities}}}} (\autopageref*{\detokenize{basics/101-103-modify:fom-log-options}}).

\index{log@\spxentry{log}!Git command@\spxentry{Git command}}\index{Git command@\spxentry{Git command}!log@\spxentry{log}}\index{get help@\spxentry{get help}!with Git@\spxentry{with Git}}\index{with Git@\spxentry{with Git}!get help@\spxentry{get help}}\index{filter history@\spxentry{filter history}!with Git@\spxentry{with Git}}\index{with Git@\spxentry{with Git}!filter history@\spxentry{filter history}}\ignorespaces \begin{findoutmore}[label={fom-log-options}, before title={\thetcbcounter\ }, float, floatplacement=tb, check odd page=true]{‘git log’ has many more useful options}
\label{\detokenize{basics/101-103-modify:fom-log-options}}

\sphinxAtStartPar
\sphinxcode{\sphinxupquote{git log}}, as many other \sphinxcode{\sphinxupquote{Git}} commands, has a good number of options
which you can discover if you run \sphinxcode{\sphinxupquote{git log \sphinxhyphen{}\sphinxhyphen{}help}}.  Those options could
help to find specific changes (e.g., which added or removed a specific word
with \sphinxcode{\sphinxupquote{\sphinxhyphen{}S}}), or change how \sphinxcode{\sphinxupquote{git log}} output will look (e.g.,
\sphinxcode{\sphinxupquote{\sphinxhyphen{}\sphinxhyphen{}word\sphinxhyphen{}diff}} to highlight individual word changes).


\end{findoutmore}

\sphinxstepscope

\index{clone@\spxentry{clone}!DataLad command@\spxentry{DataLad command}}\index{DataLad command@\spxentry{DataLad command}!clone@\spxentry{clone}}\index{clone dataset@\spxentry{clone dataset}!with DataLad@\spxentry{with DataLad}}\index{with DataLad@\spxentry{with DataLad}!clone dataset@\spxentry{clone dataset}}\ignorespaces 

\section{Install datasets}
\label{\detokenize{basics/101-105-install:install-datasets}}\label{\detokenize{basics/101-105-install:installds}}\label{\detokenize{basics/101-105-install:index-0}}\label{\detokenize{basics/101-105-install::doc}}
\sphinxAtStartPar
So far, we have created a \sphinxcode{\sphinxupquote{DataLad\sphinxhyphen{}101}} course dataset. We saved some additional readings
into the dataset, and have carefully made and saved notes on the DataLad
commands we discovered. Up to this point, we therefore know the typical, \sphinxstyleemphasis{local}
workflow to create and populate a dataset from scratch.

\sphinxAtStartPar
But we’ve been told that with DataLad we could very easily get vast amounts of data to our
computer. Rumor has it that this would be only a single command in the terminal!
Therefore, everyone in today’s lecture excitedly awaits today’s topic: Installing datasets.

\sphinxAtStartPar
“With DataLad, users can install \sphinxstyleemphasis{clones} of existing DataLad datasets from paths, URLs, or
open\sphinxhyphen{}data collections” our lecturer begins.
“This makes accessing data fast and easy. A dataset that others could install can be
created by anyone, without a need for additional software. Your own datasets can be
installed by others, should you want that, for example. Therefore, not only accessing
data becomes fast and easy, but also \sphinxstyleemphasis{sharing}.”
“That’s so cool!”, you think. “Exam preparation will be a piece of cake if all of us
can share our mid\sphinxhyphen{}term and final projects easily!”
“But today, let’s only focus on how to install a dataset”, she continues.
“Damn it! Can we not have longer lectures?”, you think and set alarms to all of the
upcoming lecture dates in your calendar.
There is so much exciting stuff to come, you cannot miss a single one.

\sphinxAtStartPar
“Psst!” a student from the row behind reaches over. “There are
a bunch of audio recordings of a really cool podcast, and they have been shared in the form
of a DataLad dataset! Shall we try whether we can install that?”

\sphinxAtStartPar
“Perfect! What a great way to learn how to install a dataset. Doing it
now instead of looking at slides for hours is my preferred type of learning anyway”,
you think as you fire up your terminal and navigate into your \sphinxcode{\sphinxupquote{DataLad\sphinxhyphen{}101}} dataset.

\sphinxAtStartPar
In this demonstration, we are using one of the many openly available datasets that
DataLad provides in a public registry that anyone can access. One of these datasets is a
collection of audio recordings of a great podcast, the longnow seminar series%
\begin{footnote}\sphinxAtStartFootnote
The longnow podcasts are lectures and conversations on long\sphinxhyphen{}term thinking produced by
the LongNow foundation and we can wholeheartedly recommend them for their worldly
wisdoms and compelling, thoughtful ideas. Subscribe to the podcasts at \sphinxurl{https://longnow.org/seminars/podcast}.
Support the foundation by becoming a member: \sphinxurl{https://longnow.org/join}.
%
\end{footnote}.
It consists of audio recordings about long\sphinxhyphen{}term thinking, and while the DataLad\sphinxhyphen{}101
course is not a long\sphinxhyphen{}term thinking seminar, those recordings are nevertheless a
good addition to the large stash of yet\sphinxhyphen{}to\sphinxhyphen{}read text books we piled up.
Let’s get this dataset into our existing \sphinxcode{\sphinxupquote{DataLad\sphinxhyphen{}101}} dataset.

\sphinxAtStartPar
To keep the \sphinxcode{\sphinxupquote{DataLad\sphinxhyphen{}101}} dataset neat and organized, we first create a new directory,
called recordings.

\begin{sphinxVerbatim}[commandchars=\\\{\}]
\PYG{g+gp}{\PYGZdl{} }\PYG{c+c1}{\PYGZsh{} we are in the root of DataLad\PYGZhy{}101}
\PYG{g+gp}{\PYGZdl{} }mkdir\PYG{+w}{ }recordings
\end{sphinxVerbatim}

\sphinxAtStartPar
The command that can be used to obtain a dataset is \sphinxcode{\sphinxupquote{datalad clone}},
but we often refer to the process of cloning a Dataset as \sphinxstyleemphasis{installing}.
Let’s install the longnow podcasts in this new directory.

\sphinxAtStartPar
The \sphinxcode{\sphinxupquote{datalad clone}} command takes a location of an existing dataset to clone. This \sphinxstyleemphasis{source}
can be a URL or a path to a local directory, or an SSH server%
\begin{footnote}\sphinxAtStartFootnote
Additionally, a source can also be a pointer to an open\sphinxhyphen{}data collection,
for example {\hyperref[\detokenize{glossary:term-the-DataLad-superdataset}]{\sphinxtermref{\DUrole{xref,std,std-term}{the DataLad superdataset ///}}}} \textendash{} more on what this is and how to
use it later, though.
%
\end{footnote}. The dataset
to be installed lives on {\hyperref[\detokenize{glossary:term-GitHub}]{\sphinxtermref{\DUrole{xref,std,std-term}{GitHub}}}}, at
\dlhbhref{G2D}{https://github.com/datalad\sphinxhyphen{}datasets/longnow\sphinxhyphen{}podcasts.git},
and we can give its GitHub URL as the  first positional argument.
Optionally, the command also takes as second positional argument a path to the \sphinxstyleemphasis{destination},
\textendash{} a path to where we want to install the dataset to. In this case it is \sphinxcode{\sphinxupquote{recordings/longnow}}.
Because we are installing a dataset (the podcasts) into an existing dataset (the \sphinxcode{\sphinxupquote{DataLad\sphinxhyphen{}101}}
dataset), we also supply a \sphinxcode{\sphinxupquote{\sphinxhyphen{}d/\sphinxhyphen{}\sphinxhyphen{}dataset}} flag to the command.
This specifies the dataset to perform the operation on, and allows us to install
the podcasts as a \sphinxstyleemphasis{subdataset} of \sphinxcode{\sphinxupquote{DataLad\sphinxhyphen{}101}}.
The \textit{Find-out-more}~{\findoutmoreiconinline}\textit{\ref{fom-no-clone-dataset}} {\hyperref[\detokenize{basics/101-105-install:fom-no-clone-dataset}]{\sphinxcrossref{\DUrole{std,std-ref}{on not using this option}}}} (\autopageref*{\detokenize{basics/101-105-install:fom-no-clone-dataset}}) provides some more background.
Because we are in the root
of the \sphinxcode{\sphinxupquote{DataLad\sphinxhyphen{}101}} dataset, the pointer to the dataset is a \sphinxcode{\sphinxupquote{.}} (which is Unix’
way of saying “current directory”). In the \textit{Find-out-more}~{\findoutmoreiconinline}\textit{\ref{fom-clone-nonroot}} {\hyperref[\detokenize{basics/101-105-install:fom-clone-nonroot}]{\sphinxcrossref{\DUrole{std,std-ref}{on dataset option values}}}} (\autopageref*{\detokenize{basics/101-105-install:fom-clone-nonroot}})
you can learn about cloning at other locations than the dataset root.

\sphinxAtStartPar
As before with long commands, we line break the code with a \sphinxcode{\sphinxupquote{\textbackslash{}}}. You can
copy it as it is presented here into your terminal, but in your own work you
can write commands like this into a single line.

\begin{sphinxVerbatim}[commandchars=\\\{\}]
\PYG{g+gp}{\PYGZdl{} }datalad\PYG{+w}{ }clone\PYG{+w}{ }\PYGZhy{}\PYGZhy{}dataset\PYG{+w}{ }.\PYG{+w}{ }\PYG{l+s+se}{\PYGZbs{}}
\PYG{+w}{ }https://github.com/datalad\PYGZhy{}datasets/longnow\PYGZhy{}podcasts.git\PYG{+w}{ }recordings/longnow
\PYG{g+go}{[INFO] Remote origin not usable by git\PYGZhy{}annex; setting annex\PYGZhy{}ignore}
\PYG{g+go}{install(ok): recordings/longnow (dataset)}
\PYG{g+go}{add(ok): recordings/longnow (dataset)}
\PYG{g+go}{add(ok): .gitmodules (file)}
\PYG{g+go}{save(ok): . (dataset)}
\PYG{g+go}{add(ok): .gitmodules (file)}
\PYG{g+go}{save(ok): . (dataset)}
\end{sphinxVerbatim}

\sphinxAtStartPar
This command copied the repository found at the URL \sphinxurl{https://github.com/datalad-datasets/longnow-podcasts}
into the existing \sphinxcode{\sphinxupquote{DataLad\sphinxhyphen{}101}} dataset, into the directory \sphinxcode{\sphinxupquote{recordings/longnow}}.
The optional destination is helpful: If we had not specified the path
\sphinxcode{\sphinxupquote{recordings/longnow}} as a destination for the dataset clone, the command would
have installed the dataset into the root of the \sphinxcode{\sphinxupquote{DataLad\sphinxhyphen{}101}} dataset, and instead
of \sphinxcode{\sphinxupquote{longnow}} it would have used the name of the remote repository “\sphinxcode{\sphinxupquote{longnow\sphinxhyphen{}podcasts}}”.
But the coolest feature of \sphinxcode{\sphinxupquote{datalad clone}} is yet invisible: This command
also recorded where this dataset came from, thus capturing its \sphinxstyleemphasis{origin} as
{\hyperref[\detokenize{glossary:term-provenance}]{\sphinxtermref{\DUrole{xref,std,std-term}{provenance}}}}. Even though this is not obvious at this point in time, later
chapters in this book will demonstrate how useful this information can be.

\index{clone@\spxentry{clone}!DataLad concept@\spxentry{DataLad concept}}\index{DataLad concept@\spxentry{DataLad concept}!clone@\spxentry{clone}}\ignorespaces \begin{gitusernote}[label={index-1}, before title={\thetcbcounter\ }, float, floatplacement=p, check odd page=true]{Clone internals}
\label{\detokenize{basics/101-105-install:index-1}}

\sphinxAtStartPar
The \sphinxcode{\sphinxupquote{datalad clone}} command uses \sphinxcode{\sphinxupquote{git clone}}.
A dataset that is installed from an existing source, e.g., a path or URL,
is the DataLad equivalent of a \sphinxstyleemphasis{clone} in Git.


\end{gitusernote}

\index{clone into another dataset@\spxentry{clone into another dataset}!with DataLad@\spxentry{with DataLad}}\index{with DataLad@\spxentry{with DataLad}!clone into another dataset@\spxentry{clone into another dataset}}\ignorespaces \begin{findoutmore}[label={fom-clone-nonroot}, before title={\thetcbcounter\ }, float, floatplacement=p, check odd page=true]{Do I have to install from the root of datasets?}
\label{\detokenize{basics/101-105-install:fom-clone-nonroot}}

\sphinxAtStartPar
No. Instead of from the \sphinxstyleemphasis{root} of the \sphinxcode{\sphinxupquote{DataLad\sphinxhyphen{}101}} dataset, you could have also
installed the dataset from within the \sphinxcode{\sphinxupquote{recordings}}, or \sphinxcode{\sphinxupquote{books}} directory.
In the case of installing datasets into existing datasets you however need
to adjust the paths that are given with the \sphinxcode{\sphinxupquote{\sphinxhyphen{}d/\sphinxhyphen{}\sphinxhyphen{}dataset}} option:
\sphinxcode{\sphinxupquote{\sphinxhyphen{}d}} needs to specify the path to the root of the dataset. This is
important to keep in mind whenever you do not execute the \sphinxcode{\sphinxupquote{datalad clone}} command
from the root of this dataset. Luckily, there is a shortcut: \sphinxcode{\sphinxupquote{\sphinxhyphen{}d\textasciicircum{}}} will always
point to root of the top\sphinxhyphen{}most dataset. For example, if you navigate into \sphinxcode{\sphinxupquote{recordings}},
the command would be:

\begin{sphinxVerbatim}[commandchars=\\\{\}]
\PYG{g+gp}{\PYGZdl{} }datalad\PYG{+w}{ }clone\PYG{+w}{ }\PYGZhy{}d\PYGZca{}\PYG{+w}{ }https://github.com/datalad\PYGZhy{}datasets/longnow\PYGZhy{}podcasts.git\PYG{+w}{ }longnow
\end{sphinxVerbatim}


\end{findoutmore}
\begin{findoutmore}[label={fom-no-clone-dataset}, before title={\thetcbcounter\ }, float, floatplacement=p, check odd page=true]{What if I do not install into an existing dataset?}
\label{\detokenize{basics/101-105-install:fom-no-clone-dataset}}

\sphinxAtStartPar
If you do not install into an existing dataset, you only need to omit the \sphinxcode{\sphinxupquote{\sphinxhyphen{}d/\sphinxhyphen{}\sphinxhyphen{}dataset}}
option. You can try:

\begin{sphinxVerbatim}[commandchars=\\\{\}]
\PYG{g+gp}{\PYGZdl{} }datalad\PYG{+w}{ }clone\PYG{+w}{ }https://github.com/datalad\PYGZhy{}datasets/longnow\PYGZhy{}podcasts.git
\end{sphinxVerbatim}

\sphinxAtStartPar
anywhere outside of your \sphinxcode{\sphinxupquote{DataLad\sphinxhyphen{}101}} dataset to install the podcast dataset into a new directory
called \sphinxcode{\sphinxupquote{longnow\sphinxhyphen{}podcasts}}. You could even do this inside of an existing dataset.
However, whenever you install datasets into of other datasets, the \sphinxcode{\sphinxupquote{\sphinxhyphen{}d/\sphinxhyphen{}\sphinxhyphen{}dataset}}
option is necessary to not only install the dataset, but also \sphinxstyleemphasis{register} it
automatically into the higher level \sphinxstyleemphasis{superdataset}. The upcoming section will
elaborate on this.


\end{findoutmore}

\sphinxAtStartPar
This is how we can visualize the repository structure, but see the \textit{Windows-wit}~{\windowswiticoninline}\textit{\ref{ww-tree}} {\hyperref[\detokenize{basics/101-105-install:ww-tree}]{\sphinxcrossref{\DUrole{std,std-ref}{on tree}}}} (\autopageref*{\detokenize{basics/101-105-install:ww-tree}})
compatibility notes:

\index{tree@\spxentry{tree}!terminal command@\spxentry{terminal command}}\index{terminal command@\spxentry{terminal command}!tree@\spxentry{tree}}\index{display directory tree@\spxentry{display directory tree}!on Windows@\spxentry{on Windows}}\index{on Windows@\spxentry{on Windows}!display directory tree@\spxentry{display directory tree}}\ignorespaces \begin{windowswit}[label={ww-tree}, before title={\thetcbcounter\ }, float, floatplacement=p, check odd page=true]{use tree}
\label{\detokenize{basics/101-105-install:ww-tree}}

\sphinxAtStartPar
The Windows version of tree requires different parametrization, so please run \sphinxcode{\sphinxupquote{tree}} instead of \sphinxcode{\sphinxupquote{tree \sphinxhyphen{}d}}.


\end{windowswit}

\begin{sphinxVerbatim}[commandchars=\\\{\}]
\PYG{g+gp}{\PYGZdl{} }tree\PYG{+w}{ }\PYGZhy{}d\PYG{+w}{   }\PYG{c+c1}{\PYGZsh{} we limit the output to directories}
\PYG{g+go}{.}
\PYG{g+go}{├── books}
\PYG{g+go}{└── recordings}
\PYG{g+go}{    └── longnow}
\PYG{g+go}{        ├── Long\PYGZus{}Now\PYGZus{}\PYGZus{}Conversations\PYGZus{}at\PYGZus{}The\PYGZus{}Interval}
\PYG{g+go}{        └── Long\PYGZus{}Now\PYGZus{}\PYGZus{}Seminars\PYGZus{}About\PYGZus{}Long\PYGZus{}term\PYGZus{}Thinking}

\PYG{g+go}{5 directories}
\end{sphinxVerbatim}

\sphinxAtStartPar
We can see that \sphinxcode{\sphinxupquote{recordings}} has one subdirectory, our newly installed \sphinxcode{\sphinxupquote{longnow}}
dataset with two subdirectories.
If we navigate into one of them and list its content, we’ll see many \sphinxcode{\sphinxupquote{.mp3}} files (here is an excerpt).

\begin{sphinxVerbatim}[commandchars=\\\{\}]
\PYG{g+gp}{\PYGZdl{} }\PYG{n+nb}{cd}\PYG{+w}{ }recordings/longnow/Long\PYGZus{}Now\PYGZus{}\PYGZus{}Seminars\PYGZus{}About\PYGZus{}Long\PYGZus{}term\PYGZus{}Thinking
\PYG{g+gp}{\PYGZdl{} }ls
\PYG{g+go}{2003\PYGZus{}11\PYGZus{}15\PYGZus{}\PYGZus{}Brian\PYGZus{}Eno\PYGZus{}\PYGZus{}The\PYGZus{}Long\PYGZus{}Now.mp3}
\PYG{g+go}{2003\PYGZus{}12\PYGZus{}13\PYGZus{}\PYGZus{}Peter\PYGZus{}Schwartz\PYGZus{}\PYGZus{}The\PYGZus{}Art\PYGZus{}Of\PYGZus{}The\PYGZus{}Really\PYGZus{}Long\PYGZus{}View.mp3}
\PYG{g+go}{2004\PYGZus{}01\PYGZus{}10\PYGZus{}\PYGZus{}George\PYGZus{}Dyson\PYGZus{}\PYGZus{}There\PYGZus{}s\PYGZus{}Plenty\PYGZus{}of\PYGZus{}Room\PYGZus{}at\PYGZus{}the\PYGZus{}Top\PYGZus{}\PYGZus{}Long\PYGZus{}term\PYGZus{}Thinking\PYGZus{}About\PYGZus{}Large\PYGZus{}scale\PYGZus{}Computing.mp3}
\PYG{g+go}{2004\PYGZus{}02\PYGZus{}14\PYGZus{}\PYGZus{}James\PYGZus{}Dewar\PYGZus{}\PYGZus{}Long\PYGZus{}term\PYGZus{}Policy\PYGZus{}Analysis.mp3}
\PYG{g+go}{2004\PYGZus{}03\PYGZus{}13\PYGZus{}\PYGZus{}Rusty\PYGZus{}Schweickart\PYGZus{}\PYGZus{}The\PYGZus{}Asteroid\PYGZus{}Threat\PYGZus{}Over\PYGZus{}the\PYGZus{}Next\PYGZus{}100\PYGZus{}000\PYGZus{}Years.mp3}
\PYG{g+go}{2004\PYGZus{}04\PYGZus{}10\PYGZus{}\PYGZus{}Daniel\PYGZus{}Janzen\PYGZus{}\PYGZus{}Third\PYGZus{}World\PYGZus{}Conservation\PYGZus{}\PYGZus{}It\PYGZus{}s\PYGZus{}ALL\PYGZus{}Gardening.mp3}
\PYG{g+go}{2004\PYGZus{}05\PYGZus{}15\PYGZus{}\PYGZus{}David\PYGZus{}Rumsey\PYGZus{}\PYGZus{}Mapping\PYGZus{}Time.mp3}
\PYG{g+go}{2004\PYGZus{}06\PYGZus{}12\PYGZus{}\PYGZus{}Bruce\PYGZus{}Sterling\PYGZus{}\PYGZus{}The\PYGZus{}Singularity\PYGZus{}\PYGZus{}Your\PYGZus{}Future\PYGZus{}as\PYGZus{}a\PYGZus{}Black\PYGZus{}Hole.mp3}
\PYG{g+go}{2004\PYGZus{}07\PYGZus{}10\PYGZus{}\PYGZus{}Jill\PYGZus{}Tarter\PYGZus{}\PYGZus{}The\PYGZus{}Search\PYGZus{}for\PYGZus{}Extra\PYGZus{}terrestrial\PYGZus{}Intelligence\PYGZus{}\PYGZus{}Necessarily\PYGZus{}a\PYGZus{}Long\PYGZus{}term\PYGZus{}Strategy.mp3}
\PYG{g+go}{2004\PYGZus{}08\PYGZus{}14\PYGZus{}\PYGZus{}Phillip\PYGZus{}Longman\PYGZus{}\PYGZus{}The\PYGZus{}Depopulation\PYGZus{}Problem.mp3}
\PYG{g+go}{2004\PYGZus{}09\PYGZus{}11\PYGZus{}\PYGZus{}Danny\PYGZus{}Hillis\PYGZus{}\PYGZus{}Progress\PYGZus{}on\PYGZus{}the\PYGZus{}10\PYGZus{}000\PYGZus{}year\PYGZus{}Clock.mp3}
\PYG{g+go}{2004\PYGZus{}10\PYGZus{}16\PYGZus{}\PYGZus{}Paul\PYGZus{}Hawken\PYGZus{}\PYGZus{}The\PYGZus{}Long\PYGZus{}Green.mp3}
\PYG{g+go}{2004\PYGZus{}11\PYGZus{}13\PYGZus{}\PYGZus{}Michael\PYGZus{}West\PYGZus{}\PYGZus{}The\PYGZus{}Prospects\PYGZus{}of\PYGZus{}Human\PYGZus{}Life\PYGZus{}Extension.mp3}
\end{sphinxVerbatim}


\subsection{Dataset content identity and availability information}
\label{\detokenize{basics/101-105-install:dataset-content-identity-and-availability-information}}
\sphinxAtStartPar
Surprised, you turn to your fellow student and wonder about
how fast the dataset was installed. Should
a download of that many \sphinxcode{\sphinxupquote{.mp3}} files not take much more time?

\sphinxAtStartPar
Here you can see another import feature of DataLad datasets
and the \sphinxcode{\sphinxupquote{datalad clone}} command:
Upon installation of a DataLad dataset, DataLad retrieves only small files
(for example, text files or markdown files) and (small) metadata
about the dataset. It does not, however, download any large files
(yet). The metadata exposes the dataset’s file hierarchy
for exploration (note how you are able to list the dataset contents with \sphinxcode{\sphinxupquote{ls}}),
and downloading only this metadata speeds up the installation of a DataLad dataset
of many TB in size to a few seconds. Just now, after installing, the dataset is
small in size:

\index{show file size@\spxentry{show file size}!in a terminal@\spxentry{in a terminal}}\index{in a terminal@\spxentry{in a terminal}!show file size@\spxentry{show file size}}\ignorespaces 
\def\sphinxLiteralBlockLabel{\label{\detokenize{basics/101-105-install:index-4}}}
\begin{sphinxVerbatim}[commandchars=\\\{\}]
\PYG{g+gp}{\PYGZdl{} }\PYG{n+nb}{cd}\PYG{+w}{ }../\PYG{+w}{      }\PYG{c+c1}{\PYGZsh{} in longnow/}
\PYG{g+gp}{\PYGZdl{} }du\PYG{+w}{ }\PYGZhy{}sh\PYG{+w}{      }\PYG{c+c1}{\PYGZsh{} Unix command to show size of contents}
\PYG{g+go}{3.7M	.}
\end{sphinxVerbatim}

\sphinxAtStartPar
This is tiny indeed!

\sphinxAtStartPar
If you executed the previous \sphinxcode{\sphinxupquote{ls}} command in your own terminal, you might have seen
the \sphinxcode{\sphinxupquote{.mp3}} files highlighted in a different color than usually.
On your computer, try to open one of the \sphinxcode{\sphinxupquote{.mp3}} files.
You will notice that you cannot open any of the audio files.
This is not your fault: \sphinxstyleemphasis{None of these files exist on your computer yet}.

\sphinxAtStartPar
Wait, what?

\sphinxAtStartPar
This sounds strange, but it has many advantages. Apart from a fast installation,
it allows you to retrieve precisely the content you need, instead of all the contents
of a dataset. Thus, even if you install a dataset that is many TB in size,
it takes up only few MB of space after the install, and you can retrieve only those
components of the dataset that you need.

\sphinxAtStartPar
Let’s see how large the dataset would be in total if all of the files were present.
For this, we supply an additional option to \sphinxcode{\sphinxupquote{datalad status}}. Make sure to be
(somewhere) inside of the \sphinxcode{\sphinxupquote{longnow}} dataset to execute the following command:

\begin{sphinxVerbatim}[commandchars=\\\{\}]
\PYG{g+gp}{\PYGZdl{} }datalad\PYG{+w}{ }status\PYG{+w}{ }\PYGZhy{}\PYGZhy{}annex
\PYG{g+go}{236 annex\PYGZsq{}d files (15.4 GB recorded total size)}
\PYG{g+go}{nothing to save, working tree clean}
\end{sphinxVerbatim}

\sphinxAtStartPar
Woah! More than 200 files, totaling more than 15 GB?
You begin to appreciate that DataLad did not
download all of this data right away! That would have taken hours given the crappy
internet connection in the lecture hall, and you are not even sure whether your
hard drive has much space left…

\sphinxAtStartPar
But you nevertheless are curious on how to actually listen to one of these \sphinxcode{\sphinxupquote{.mp3}}s now.
So how does one actually “get” the files?

\index{get@\spxentry{get}!DataLad command@\spxentry{DataLad command}}\index{DataLad command@\spxentry{DataLad command}!get@\spxentry{get}}\ignorespaces 
\sphinxAtStartPar
The command to retrieve file content is \sphinxcode{\sphinxupquote{datalad get}}.
You can specify one or more specific files, or \sphinxcode{\sphinxupquote{get}} all of the dataset by
specifying \sphinxcode{\sphinxupquote{datalad get .}} at the root directory of the dataset (with \sphinxcode{\sphinxupquote{.}} denoting “current directory”).

\index{get file content@\spxentry{get file content}!with DataLad@\spxentry{with DataLad}}\index{with DataLad@\spxentry{with DataLad}!get file content@\spxentry{get file content}}\ignorespaces 
\sphinxAtStartPar
First, we get one of the recordings in the dataset \textendash{} take any one of your choice
(here, it’s the first).

\begin{sphinxVerbatim}[commandchars=\\\{\}]
\PYG{g+gp}{\PYGZdl{} }datalad\PYG{+w}{ }get\PYG{+w}{ }Long\PYGZus{}Now\PYGZus{}\PYGZus{}Seminars\PYGZus{}About\PYGZus{}Long\PYGZus{}term\PYGZus{}Thinking/2003\PYGZus{}11\PYGZus{}15\PYGZus{}\PYGZus{}Brian\PYGZus{}Eno\PYGZus{}\PYGZus{}The\PYGZus{}Long\PYGZus{}Now.mp3
\PYG{g+go}{get(ok): Long\PYGZus{}Now\PYGZus{}\PYGZus{}Seminars\PYGZus{}About\PYGZus{}Long\PYGZus{}term\PYGZus{}Thinking/2003\PYGZus{}11\PYGZus{}15\PYGZus{}\PYGZus{}Brian\PYGZus{}Eno\PYGZus{}\PYGZus{}The\PYGZus{}Long\PYGZus{}Now.mp3 (file) [from web...]}
\end{sphinxVerbatim}

\sphinxAtStartPar
Try to open it \textendash{} it will now work.

\sphinxAtStartPar
If you would want to get the rest of the missing data, instead of specifying all files individually,
we can use \sphinxcode{\sphinxupquote{.}} to refer to \sphinxstyleemphasis{all} of the dataset like this:

\begin{sphinxVerbatim}[commandchars=\\\{\}]
\PYG{g+gp}{\PYGZdl{} }datalad\PYG{+w}{ }get\PYG{+w}{ }.
\end{sphinxVerbatim}

\sphinxAtStartPar
However, with a total size of more than 15GB, this might take a while, so do not do that now.
If you did execute the command above, interrupt it by pressing \sphinxcode{\sphinxupquote{CTRL}} + \sphinxcode{\sphinxupquote{C}} \textendash{} Do not worry,
this will not break anything.

\index{show dataset size@\spxentry{show dataset size}!with DataLad@\spxentry{with DataLad}}\index{with DataLad@\spxentry{with DataLad}!show dataset size@\spxentry{show dataset size}}\ignorespaces 
\sphinxAtStartPar
Isn’t that easy?
Let’s see how much content is now present locally. For this, \sphinxcode{\sphinxupquote{datalad status \sphinxhyphen{}\sphinxhyphen{}annex all}}
has a nice summary:

\begin{sphinxVerbatim}[commandchars=\\\{\}]
\PYG{g+gp}{\PYGZdl{} }datalad\PYG{+w}{ }status\PYG{+w}{ }\PYGZhy{}\PYGZhy{}annex\PYG{+w}{ }all
\PYG{g+go}{236 annex\PYGZsq{}d files (35.7 MB/15.4 GB present/total size)}
\PYG{g+go}{nothing to save, working tree clean}
\end{sphinxVerbatim}

\sphinxAtStartPar
This shows you how much of the total content is present locally. With one file,
it is only a fraction of the total size.

\sphinxAtStartPar
Let’s \sphinxcode{\sphinxupquote{get}} a few more recordings, just because it was so mesmerizing to watch
DataLad’s fancy progress bars.

\begin{sphinxVerbatim}[commandchars=\\\{\}]
\PYG{g+gp}{\PYGZdl{} }datalad\PYG{+w}{ }get\PYG{+w}{ }Long\PYGZus{}Now\PYGZus{}\PYGZus{}Seminars\PYGZus{}About\PYGZus{}Long\PYGZus{}term\PYGZus{}Thinking/2003\PYGZus{}11\PYGZus{}15\PYGZus{}\PYGZus{}Brian\PYGZus{}Eno\PYGZus{}\PYGZus{}The\PYGZus{}Long\PYGZus{}Now.mp3\PYG{+w}{ }\PYG{l+s+se}{\PYGZbs{}}
Long\PYGZus{}Now\PYGZus{}\PYGZus{}Seminars\PYGZus{}About\PYGZus{}Long\PYGZus{}term\PYGZus{}Thinking/2003\PYGZus{}12\PYGZus{}13\PYGZus{}\PYGZus{}Peter\PYGZus{}Schwartz\PYGZus{}\PYGZus{}The\PYGZus{}Art\PYGZus{}Of\PYGZus{}The\PYGZus{}Really\PYGZus{}Long\PYGZus{}View.mp3\PYG{+w}{ }\PYG{l+s+se}{\PYGZbs{}}
Long\PYGZus{}Now\PYGZus{}\PYGZus{}Seminars\PYGZus{}About\PYGZus{}Long\PYGZus{}term\PYGZus{}Thinking/2004\PYGZus{}01\PYGZus{}10\PYGZus{}\PYGZus{}George\PYGZus{}Dyson\PYGZus{}\PYGZus{}There\PYGZus{}s\PYGZus{}Plenty\PYGZus{}of\PYGZus{}Room\PYGZus{}at\PYGZus{}the\PYGZus{}Top\PYGZus{}\PYGZus{}Long\PYGZus{}term\PYGZus{}Thinking\PYGZus{}About\PYGZus{}Large\PYGZus{}scale\PYGZus{}Computing.mp3
\PYG{g+go}{get(ok): Long\PYGZus{}Now\PYGZus{}\PYGZus{}Seminars\PYGZus{}About\PYGZus{}Long\PYGZus{}term\PYGZus{}Thinking/2004\PYGZus{}01\PYGZus{}10\PYGZus{}\PYGZus{}George\PYGZus{}Dyson\PYGZus{}\PYGZus{}There\PYGZus{}s\PYGZus{}Plenty\PYGZus{}of\PYGZus{}Room\PYGZus{}at\PYGZus{}the\PYGZus{}Top\PYGZus{}\PYGZus{}Long\PYGZus{}term\PYGZus{}Thinking\PYGZus{}About\PYGZus{}Large\PYGZus{}scale\PYGZus{}Computing.mp3 (file) [from web...]}
\PYG{g+go}{get(ok): Long\PYGZus{}Now\PYGZus{}\PYGZus{}Seminars\PYGZus{}About\PYGZus{}Long\PYGZus{}term\PYGZus{}Thinking/2003\PYGZus{}12\PYGZus{}13\PYGZus{}\PYGZus{}Peter\PYGZus{}Schwartz\PYGZus{}\PYGZus{}The\PYGZus{}Art\PYGZus{}Of\PYGZus{}The\PYGZus{}Really\PYGZus{}Long\PYGZus{}View.mp3 (file) [from web...]}
\PYG{g+go}{action summary:}
\PYG{g+go}{  get (notneeded: 1, ok: 2)}
\end{sphinxVerbatim}

\sphinxAtStartPar
Note that any data that is already retrieved (the first file) is not downloaded again.
DataLad summarizes the outcome of the execution of \sphinxcode{\sphinxupquote{get}} in the end and informs
that the download of one file was \sphinxcode{\sphinxupquote{notneeded}} and the retrieval of the other files was \sphinxcode{\sphinxupquote{ok}}.

\index{get@\spxentry{get}!DataLad concept@\spxentry{DataLad concept}}\index{DataLad concept@\spxentry{DataLad concept}!get@\spxentry{get}}\ignorespaces \begin{gitusernote}[label={index-8}, before title={\thetcbcounter\ }, check odd page=true]{Get internals}
\label{\detokenize{basics/101-105-install:index-8}}

\sphinxAtStartPar
\sphinxcode{\sphinxupquote{datalad get}} uses \sphinxcode{\sphinxupquote{git annex get}} underneath the hood.


\end{gitusernote}

\index{drop file content@\spxentry{drop file content}!with DataLad@\spxentry{with DataLad}}\index{with DataLad@\spxentry{with DataLad}!drop file content@\spxentry{drop file content}}\ignorespaces 

\subsection{Keep whatever you like}
\label{\detokenize{basics/101-105-install:keep-whatever-you-like}}\label{\detokenize{basics/101-105-install:index-9}}
\sphinxAtStartPar
“Oh shit, oh shit, oh shit…” you hear from right behind you. Your fellow student
apparently downloaded the \sphinxstyleemphasis{full} dataset accidentally. “Is there a way to get rid
of file contents in dataset, too?”, they ask. “Yes”, the lecturer responds,
“you can remove file contents by using \sphinxcode{\sphinxupquote{datalad drop}}. This is
really helpful to save disk space for data you can easily reobtain, for example”.

\index{drop@\spxentry{drop}!DataLad command@\spxentry{DataLad command}}\index{DataLad command@\spxentry{DataLad command}!drop@\spxentry{drop}}\ignorespaces 
\sphinxAtStartPar
The \sphinxcode{\sphinxupquote{datalad drop}} command will remove
file contents completely from your dataset.
You should only use this command to remove contents that you can \sphinxcode{\sphinxupquote{datalad get}}
again, or generate again (for example, with next chapter’s \sphinxcode{\sphinxupquote{datalad datalad run}}
command), or that you really do not need anymore.

\sphinxAtStartPar
Let’s remove the content of one of the files that we have downloaded, and check
what this does to the total size of the dataset. Here is the current amount of
retrieved data in this dataset:

\begin{sphinxVerbatim}[commandchars=\\\{\}]
\PYG{g+gp}{\PYGZdl{} }datalad\PYG{+w}{ }status\PYG{+w}{ }\PYGZhy{}\PYGZhy{}annex\PYG{+w}{ }all
\PYG{g+go}{236 annex\PYGZsq{}d files (135.1 MB/15.4 GB present/total size)}
\PYG{g+go}{nothing to save, working tree clean}
\end{sphinxVerbatim}

\sphinxAtStartPar
We drop a single recording’s content that we previously downloaded with
\sphinxcode{\sphinxupquote{datalad get}} …

\begin{sphinxVerbatim}[commandchars=\\\{\}]
\PYG{g+gp}{\PYGZdl{} }datalad\PYG{+w}{ }drop\PYG{+w}{ }Long\PYGZus{}Now\PYGZus{}\PYGZus{}Seminars\PYGZus{}About\PYGZus{}Long\PYGZus{}term\PYGZus{}Thinking/2004\PYGZus{}01\PYGZus{}10\PYGZus{}\PYGZus{}George\PYGZus{}Dyson\PYGZus{}\PYGZus{}There\PYGZus{}s\PYGZus{}Plenty\PYGZus{}of\PYGZus{}Room\PYGZus{}at\PYGZus{}the\PYGZus{}Top\PYGZus{}\PYGZus{}Long\PYGZus{}term\PYGZus{}Thinking\PYGZus{}About\PYGZus{}Large\PYGZus{}scale\PYGZus{}Computing.mp3
\PYG{g+go}{drop(ok): Long\PYGZus{}Now\PYGZus{}\PYGZus{}Seminars\PYGZus{}About\PYGZus{}Long\PYGZus{}term\PYGZus{}Thinking/2004\PYGZus{}01\PYGZus{}10\PYGZus{}\PYGZus{}George\PYGZus{}Dyson\PYGZus{}\PYGZus{}There\PYGZus{}s\PYGZus{}Plenty\PYGZus{}of\PYGZus{}Room\PYGZus{}at\PYGZus{}the\PYGZus{}Top\PYGZus{}\PYGZus{}Long\PYGZus{}term\PYGZus{}Thinking\PYGZus{}About\PYGZus{}Large\PYGZus{}scale\PYGZus{}Computing.mp3 (file)}
\end{sphinxVerbatim}

\sphinxAtStartPar
… and check the size of the dataset again:

\begin{sphinxVerbatim}[commandchars=\\\{\}]
\PYG{g+gp}{\PYGZdl{} }datalad\PYG{+w}{ }status\PYG{+w}{ }\PYGZhy{}\PYGZhy{}annex\PYG{+w}{ }all
\PYG{g+go}{236 annex\PYGZsq{}d files (93.5 MB/15.4 GB present/total size)}
\PYG{g+go}{nothing to save, working tree clean}
\end{sphinxVerbatim}

\sphinxAtStartPar
Dropping the file content of one \sphinxcode{\sphinxupquote{mp3}} file saved roughly 40MB of disk space.
Whenever you need the recording again, it is easy to re\sphinxhyphen{}retrieve it:

\begin{sphinxVerbatim}[commandchars=\\\{\}]
\PYG{g+gp}{\PYGZdl{} }datalad\PYG{+w}{ }get\PYG{+w}{ }Long\PYGZus{}Now\PYGZus{}\PYGZus{}Seminars\PYGZus{}About\PYGZus{}Long\PYGZus{}term\PYGZus{}Thinking/2004\PYGZus{}01\PYGZus{}10\PYGZus{}\PYGZus{}George\PYGZus{}Dyson\PYGZus{}\PYGZus{}There\PYGZus{}s\PYGZus{}Plenty\PYGZus{}of\PYGZus{}Room\PYGZus{}at\PYGZus{}the\PYGZus{}Top\PYGZus{}\PYGZus{}Long\PYGZus{}term\PYGZus{}Thinking\PYGZus{}About\PYGZus{}Large\PYGZus{}scale\PYGZus{}Computing.mp3
\PYG{g+go}{get(ok): Long\PYGZus{}Now\PYGZus{}\PYGZus{}Seminars\PYGZus{}About\PYGZus{}Long\PYGZus{}term\PYGZus{}Thinking/2004\PYGZus{}01\PYGZus{}10\PYGZus{}\PYGZus{}George\PYGZus{}Dyson\PYGZus{}\PYGZus{}There\PYGZus{}s\PYGZus{}Plenty\PYGZus{}of\PYGZus{}Room\PYGZus{}at\PYGZus{}the\PYGZus{}Top\PYGZus{}\PYGZus{}Long\PYGZus{}term\PYGZus{}Thinking\PYGZus{}About\PYGZus{}Large\PYGZus{}scale\PYGZus{}Computing.mp3 (file) [from web...]}
\end{sphinxVerbatim}

\sphinxAtStartPar
Reobtained!

\sphinxAtStartPar
This was only a quick digression into \sphinxcode{\sphinxupquote{datalad drop}}. The main principles
of this command will become clear after chapter
{\hyperref[\detokenize{basics/basics-annex:chapter-gitannex}]{\sphinxcrossref{\DUrole{std,std-ref}{Under the hood: git\sphinxhyphen{}annex}}}} (\autopageref*{\detokenize{basics/basics-annex:chapter-gitannex}}), and its precise use is shown in the paragraph on
{\hyperref[\detokenize{basics/101-136-filesystem:remove}]{\sphinxcrossref{\DUrole{std,std-ref}{removing file contents}}}} (\autopageref*{\detokenize{basics/101-136-filesystem:remove}}).
At this point, however, you already know that datasets allow you do
\sphinxcode{\sphinxupquote{datalad drop}} file contents flexibly. If you want to, you could have more
podcasts (or other data) on your computer than you have disk space available
by using DataLad datasets \textendash{} and that really is a cool feature to have.


\subsection{Dataset archeology}
\label{\detokenize{basics/101-105-install:dataset-archeology}}
\sphinxAtStartPar
You have now experienced how easy it is to (re)obtain shared data with DataLad.
But beyond sharing only the \sphinxstyleemphasis{data} in the dataset, when sharing or installing
a DataLad dataset, all copies also include the dataset’s \sphinxstyleemphasis{history}.

\index{log@\spxentry{log}!Git command@\spxentry{Git command}}\index{Git command@\spxentry{Git command}!log@\spxentry{log}}\index{show history (reverse)@\spxentry{show history}\spxextra{reverse}!with Git@\spxentry{with Git}}\index{with Git@\spxentry{with Git}!show history (reverse)@\spxentry{show history}\spxextra{reverse}}\ignorespaces 
\sphinxAtStartPar
For example, we can find out who created the dataset in the first place
(the output shows an excerpt of \sphinxcode{\sphinxupquote{git log \sphinxhyphen{}\sphinxhyphen{}reverse}}, which displays the
history from first to most recent commit):

\fvset{hllines={, 3,}}%
\begin{sphinxVerbatim}[commandchars=\\\{\}]
\PYG{g+gp}{\PYGZdl{} }git\PYG{+w}{ }log\PYG{+w}{ }\PYGZhy{}\PYGZhy{}reverse
\PYG{g+go}{commit 8df130bb✂SHA1}
\PYG{g+go}{Author: Michael Hanke \PYGZlt{}michael.hanke@gmail.com\PYGZgt{}}
\PYG{g+go}{Date:   Mon Jul 16 16:08:23 2018 +0200}

\PYG{g+go}{    [DATALAD] Set default backend for all files to be MD5E}

\PYG{g+go}{commit 3d0dc8f5✂SHA1}
\PYG{g+go}{Author: Michael Hanke \PYGZlt{}michael.hanke@gmail.com\PYGZgt{}}
\PYG{g+go}{Date:   Mon Jul 16 16:08:24 2018 +0200}

\PYG{g+go}{    [DATALAD] new dataset}
\end{sphinxVerbatim}
\sphinxresetverbatimhllines

\sphinxAtStartPar
But that’s not all. The seminar series is ongoing, and more recordings can get added
to the original repository shared on GitHub.
Because an installed dataset knows the dataset it was installed from,
your local dataset clone can be updated from its origin, and thus get the new recordings,
should there be some. Later in this book, we will see examples of this.

\index{update heredoc@\spxentry{update heredoc}!in a terminal@\spxentry{in a terminal}}\index{in a terminal@\spxentry{in a terminal}!update heredoc@\spxentry{update heredoc}}\index{save dataset modification@\spxentry{save dataset modification}!with DataLad@\spxentry{with DataLad}}\index{with DataLad@\spxentry{with DataLad}!save dataset modification@\spxentry{save dataset modification}}\ignorespaces 
\sphinxAtStartPar
Now you can not only create datasets and work with them locally, you can also consume
existing datasets by installing them. Because that’s cool, and because you will use this
command frequently, make a note of it into your \sphinxcode{\sphinxupquote{notes.txt}}, and \sphinxcode{\sphinxupquote{datalad save}} the
modification.

\begin{sphinxVerbatim}[commandchars=\\\{\}]
\PYG{g+gp}{\PYGZdl{} }\PYG{c+c1}{\PYGZsh{} in the root of DataLad\PYGZhy{}101:}
\PYG{g+gp}{\PYGZdl{} }\PYG{n+nb}{cd}\PYG{+w}{ }../../
\PYG{g+gp}{\PYGZdl{} }cat\PYG{+w}{ }\PYGZlt{}\PYGZlt{}\PYG{+w}{ }EOT\PYG{+w}{ }\PYGZgt{}\PYGZgt{}\PYG{+w}{ }notes.txt
\PYG{g+go}{The command \PYGZsq{}datalad clone URL/PATH [PATH]\PYGZsq{} installs a dataset from}
\PYG{g+go}{e.g., a URL or a path. If you install a dataset into an existing}
\PYG{g+go}{dataset (as a subdataset), remember to specify the root of the}
\PYG{g+go}{superdataset with the \PYGZsq{}\PYGZhy{}d\PYGZsq{} option.}

\PYG{g+go}{EOT}
\PYG{g+gp}{\PYGZdl{} }datalad\PYG{+w}{ }save\PYG{+w}{ }\PYGZhy{}m\PYG{+w}{ }\PYG{l+s+s2}{\PYGZdq{}Add note on datalad clone\PYGZdq{}}
\PYG{g+go}{add(ok): notes.txt (file)}
\PYG{g+go}{save(ok): . (dataset)}
\end{sphinxVerbatim}

\index{placeholder files@\spxentry{placeholder files}!on Mac@\spxentry{on Mac}}\index{on Mac@\spxentry{on Mac}!placeholder files@\spxentry{placeholder files}}\ignorespaces \begin{importantnote}[label={index-13}, before title={\thetcbcounter\ }, check odd page=true]{Empty files can be confusing}
\label{\detokenize{basics/101-105-install:index-13}}

\sphinxAtStartPar
Listing files directly after the installation of a dataset will
work if done in a terminal with \sphinxcode{\sphinxupquote{ls}}.
However, certain file managers (such as OSX’s Finder%
\begin{footnote}\sphinxAtStartFootnote
You can also upgrade your file manager to display file types in a
DataLad datasets (e.g., with the
\dlhbhref{G2B}{git\sphinxhyphen{}annex\sphinxhyphen{}turtle extension}
for Finder)
%
\end{footnote}) may fail to
display files that are not yet present locally (i.e., before a
\sphinxcode{\sphinxupquote{datalad get}} was run). Therefore, be  mindful when exploring
a dataset hierarchy with a file manager \textendash{} it might not show you
the available but not yet retrieved files.
Consider browsing datasets with the {\hyperref[\detokenize{glossary:term-DataLad-Gooey}]{\sphinxtermref{\DUrole{xref,std,std-term}{DataLad Gooey}}}} to be on the safe side.
More about why this is will be explained in section {\hyperref[\detokenize{basics/101-115-symlinks:symlink}]{\sphinxcrossref{\DUrole{std,std-ref}{Data integrity}}}} (\autopageref*{\detokenize{basics/101-115-symlinks:symlink}}).


\end{importantnote}

\sphinxstepscope

\index{dataset nesting@\spxentry{dataset nesting}!with DataLad@\spxentry{with DataLad}}\index{with DataLad@\spxentry{with DataLad}!dataset nesting@\spxentry{dataset nesting}}\ignorespaces 

\section{Dataset nesting}
\label{\detokenize{basics/101-106-nesting:dataset-nesting}}\label{\detokenize{basics/101-106-nesting:nesting}}\label{\detokenize{basics/101-106-nesting:index-0}}\label{\detokenize{basics/101-106-nesting::doc}}
\sphinxAtStartPar
Without noticing, the previous section demonstrated another core principle
and feature of DataLad datasets: \sphinxstyleemphasis{Nesting}.

\sphinxAtStartPar
Within DataLad datasets one can \sphinxstyleemphasis{nest} other DataLad
datasets arbitrarily deep. We for example just installed one dataset, the
\sphinxcode{\sphinxupquote{longnow}} podcasts, \sphinxstyleemphasis{into} another dataset, the \sphinxcode{\sphinxupquote{DataLad\sphinxhyphen{}101}} dataset.
This was done by supplying the \sphinxcode{\sphinxupquote{\sphinxhyphen{}\sphinxhyphen{}dataset}}/\sphinxcode{\sphinxupquote{\sphinxhyphen{}d}} flag in the command call.

\sphinxAtStartPar
At first glance, nesting does not seem particularly spectacular \textendash{}
after all, any directory on a file system can have other directories inside of it.
The possibility for nested Datasets, however, is one of many advantages
DataLad datasets have:

\sphinxAtStartPar
One aspect of nested datasets is that any DataLad dataset
(\sphinxstyleemphasis{subdataset} or \sphinxstyleemphasis{superdataset}) keeps their stand\sphinxhyphen{}alone
history. The top\sphinxhyphen{}level DataLad dataset (the \sphinxstyleemphasis{superdataset}) only stores
\sphinxstyleemphasis{which version} of the subdataset is currently used through an identifier.

\sphinxAtStartPar
Let’s dive into that.
Remember how we had to navigate into \sphinxcode{\sphinxupquote{recordings/longnow}} to see the history,
and how this history was completely independent of the \sphinxcode{\sphinxupquote{DataLad\sphinxhyphen{}101}}
superdataset history? This was the subdataset’s own history.

\sphinxAtStartPar
Apart from stand\sphinxhyphen{}alone histories of super\sphinxhyphen{} or subdatasets, this highlights another
very important advantage that nesting provides: Note that the \sphinxcode{\sphinxupquote{longnow}} dataset
is a completely independent, standalone dataset that was once created and
published. Nesting allows for a modular reuse of any other DataLad dataset,
and this reuse is possible and simple precisely because all of the information
is kept within a (sub)dataset.

\sphinxAtStartPar
But now let’s also check out how the \sphinxstyleemphasis{superdataset’s} (\sphinxcode{\sphinxupquote{DataLad\sphinxhyphen{}101}}) history
looks like after the addition of a subdataset. To do this, make sure you are
\sphinxstyleemphasis{outside} of the subdataset \sphinxcode{\sphinxupquote{longnow}}. Note that the first commit is our recent
addition to \sphinxcode{\sphinxupquote{notes.txt}}, so we’ll look at the second most recent commit in
this excerpt.

\index{show commit patches@\spxentry{show commit patches}!with Git@\spxentry{with Git}}\index{with Git@\spxentry{with Git}!show commit patches@\spxentry{show commit patches}}\ignorespaces 
\def\sphinxLiteralBlockLabel{\label{\detokenize{basics/101-106-nesting:index-1}}}
\fvset{hllines={, 25,}}%
\begin{sphinxVerbatim}[commandchars=\\\{\}]
\PYG{g+gp}{\PYGZdl{} }git\PYG{+w}{ }log\PYG{+w}{ }\PYGZhy{}p\PYG{+w}{ }\PYGZhy{}n\PYG{+w}{ }\PYG{l+m}{3}
\PYG{g+go}{commit 3c016f73✂SHA1}
\PYG{g+go}{Author: Elena Piscopia \PYGZlt{}elena@example.net\PYGZgt{}}
\PYG{g+go}{Date:   Tue Jun 18 16:13:00 2019 +0000}

\PYG{g+go}{    [DATALAD] Added subdataset}

\PYG{g+go}{diff \PYGZhy{}\PYGZhy{}git a/.gitmodules b/.gitmodules}
\PYG{g+go}{new file mode 100644}
\PYG{g+go}{index 0000000..9bc9ee9}
\PYG{g+go}{\PYGZhy{}\PYGZhy{}\PYGZhy{} /dev/null}
\PYG{g+go}{+++ b/.gitmodules}
\PYG{g+go}{@@ \PYGZhy{}0,0 +1,5 @@}
\PYG{g+go}{+[submodule \PYGZdq{}recordings/longnow\PYGZdq{}]}
\PYG{g+go}{+	path = recordings/longnow}
\PYG{g+go}{+	url = https://github.com/datalad\PYGZhy{}datasets/longnow\PYGZhy{}podcasts.git}
\PYG{g+go}{+	datalad\PYGZhy{}id = b3ca2718\PYGZhy{}8901\PYGZhy{}11e8\PYGZhy{}99aa\PYGZhy{}a0369f7c647e}
\PYG{g+go}{+	datalad\PYGZhy{}url = https://github.com/datalad\PYGZhy{}datasets/longnow\PYGZhy{}podcasts.git}
\PYG{g+go}{diff \PYGZhy{}\PYGZhy{}git a/recordings/longnow b/recordings/longnow}
\PYG{g+go}{new file mode 160000}
\PYG{g+go}{index 0000000..dcc34fb}
\PYG{g+go}{\PYGZhy{}\PYGZhy{}\PYGZhy{} /dev/null}
\PYG{g+go}{+++ b/recordings/longnow}
\PYG{g+go}{@@ \PYGZhy{}0,0 +1 @@}
\PYG{g+go}{+Subproject commit dcc34fbe✂SHA1}

\PYG{g+go}{commit e310b465✂SHA1}
\PYG{g+go}{Author: Elena Piscopia \PYGZlt{}elena@example.net\PYGZgt{}}
\PYG{g+go}{Date:   Tue Jun 18 16:13:00 2019 +0000}

\PYG{g+go}{    add note on datalad save}

\PYG{g+go}{diff \PYGZhy{}\PYGZhy{}git a/notes.txt b/notes.txt}
\PYG{g+go}{index 3a7a1fe..0142412 100644}
\PYG{g+go}{\PYGZhy{}\PYGZhy{}\PYGZhy{} a/notes.txt}
\PYG{g+go}{+++ b/notes.txt}
\PYG{g+go}{@@ \PYGZhy{}1,3 +1,7 @@}
\PYG{g+go}{ One can create a new dataset with \PYGZsq{}datalad create [\PYGZhy{}\PYGZhy{}description] PATH\PYGZsq{}.}
\PYG{g+go}{ The dataset is created empty}

\PYG{g+go}{+The command \PYGZdq{}datalad save [\PYGZhy{}m] PATH\PYGZdq{} saves the file (modifications) to}
\PYG{g+go}{+history.}
\end{sphinxVerbatim}
\sphinxresetverbatimhllines

\sphinxAtStartPar
We have highlighted the important part of this rather long commit summary.
Note that you cannot see any \sphinxcode{\sphinxupquote{.mp3}}s being added to the dataset,
as was previously the case when we \sphinxcode{\sphinxupquote{datalad save}}d PDFs that we
downloaded into \sphinxcode{\sphinxupquote{books/}}. Instead,
DataLad stores what it calls a \sphinxstyleemphasis{subproject commit} of the subdataset.
The cryptic character sequence in this line is the {\hyperref[\detokenize{glossary:term-shasum}]{\sphinxtermref{\DUrole{xref,std,std-term}{shasum}}}} we have briefly
mentioned before, and it is the identifier that
DataLad internally used to identify the files and the changes to the files in the subdataset. Exactly this
{\hyperref[\detokenize{glossary:term-shasum}]{\sphinxtermref{\DUrole{xref,std,std-term}{shasum}}}} is what identifies the state of the subdataset.

\sphinxAtStartPar
Navigate back into \sphinxcode{\sphinxupquote{longnow}} and try to find the highlighted shasum in the
subdataset’s history:

\fvset{hllines={, 3,}}%
\begin{sphinxVerbatim}[commandchars=\\\{\}]
\PYG{g+gp}{\PYGZdl{} }\PYG{n+nb}{cd}\PYG{+w}{ }recordings/longnow
\PYG{g+gp}{\PYGZdl{} }git\PYG{+w}{ }log\PYG{+w}{ }\PYGZhy{}\PYGZhy{}oneline
\PYG{g+go}{dcc34fb Update aggregated metadata}
\PYG{g+go}{36a30a1 [DATALAD RUNCMD] Update from feed}
\PYG{g+go}{bafdc04 Uniformize JSON\PYGZhy{}LD context with DataLad\PYGZsq{}s internal extractors}
\PYG{g+go}{004e484 [DATALAD RUNCMD] .datalad/maint/make\PYGZus{}readme.py}
\PYG{g+go}{7ee3ded Sort episodes newest\PYGZhy{}first}
\PYG{g+go}{e829615 Link to the handbook as a source of wisdom}
\PYG{g+go}{4b37790 Fix README generator to parse correct directory}
\end{sphinxVerbatim}
\sphinxresetverbatimhllines

\sphinxAtStartPar
We can see that it is the most recent commit shasum of the subdataset
(albeit we can see only the first seven characters here \textendash{} a \sphinxcode{\sphinxupquote{git log}}
would show you the full shasum). Thus, your dataset does not only know the origin
of its subdataset, but also which version of the subdataset to use,
i.e., it has the identifier of the stage/version in the subdataset’s evolution to be used.
This is what is meant by “the top\sphinxhyphen{}level DataLad dataset (the \sphinxstyleemphasis{superdataset}) only stores
\sphinxstyleemphasis{which version} of the subdataset is currently used through an identifier”.

\sphinxAtStartPar
Importantly, once we learn how to make use of the history of a dataset,
we can set subdatasets to previous states, or \sphinxstyleemphasis{update} them.

\sphinxAtStartPar
By the way, there is actually no need to navigate into different directories all the time.
Check the \textit{Find-out-more}~{\findoutmoreiconinline}\textit{\ref{fom-tmp-cwd}} {\hyperref[\detokenize{basics/101-106-nesting:fom-tmp-cwd}]{\sphinxcrossref{\DUrole{std,std-ref}{on switching the working directory with Git}}}} (\autopageref*{\detokenize{basics/101-106-nesting:fom-tmp-cwd}}) for
an alternative.

\index{temporary working directory change@\spxentry{temporary working directory change}!with Git@\spxentry{with Git}}\index{with Git@\spxentry{with Git}!temporary working directory change@\spxentry{temporary working directory change}}\ignorespaces \begin{findoutmore}[label={fom-tmp-cwd}, before title={\thetcbcounter\ }, float, floatplacement=tb, check odd page=true]{Do I have to navigate into the subdataset to see it’s history?}
\label{\detokenize{basics/101-106-nesting:fom-tmp-cwd}}

\sphinxAtStartPar
Previously, we used \sphinxcode{\sphinxupquote{ cd}} to navigate into the subdataset, and
subsequently opened the Git log. This is necessary, because a \sphinxcode{\sphinxupquote{git log}}
in the superdataset would only return the superdatasets history.
While moving around with \sphinxcode{\sphinxupquote{cd}} is straightforward, you also found it
slightly annoying from time to time to use the \sphinxcode{\sphinxupquote{cd}} command so often and also
to remember in which directory you currently are in. There is one
trick, though: \sphinxcode{\sphinxupquote{git \sphinxhyphen{}C}} and \sphinxcode{\sphinxupquote{datalad \sphinxhyphen{}C}} (note that it is a capital C) let you perform any
Git or DataLad command in a provided path. Providing this option together with a path to
a Git or DataLad command let’s you run the command as if it was started in this path
instead of the current working directory.
Thus, from the root of \sphinxcode{\sphinxupquote{DataLad\sphinxhyphen{}101}}, this command would have given you the
subdataset’s history as well:

\begin{sphinxVerbatim}[commandchars=\\\{\}]
\PYG{g+gp}{\PYGZdl{} }git\PYG{+w}{ }\PYGZhy{}C\PYG{+w}{ }recordings/longnow\PYG{+w}{ }log\PYG{+w}{ }\PYGZhy{}\PYGZhy{}oneline
\end{sphinxVerbatim}


\end{findoutmore}

\sphinxAtStartPar
In the upcoming sections, we’ll experience the perks of dataset nesting
frequently, and everything that might seem vague at this point will become
clearer. To conclude this demonstration,
\hyperref[\detokenize{basics/101-106-nesting:fignesting}]{Fig.\@ \ref{\detokenize{basics/101-106-nesting:fignesting}}} illustrates the current state of our dataset, \sphinxcode{\sphinxupquote{DataLad\sphinxhyphen{}101}}, with its nested subdataset.
Thus, without being consciously aware of it, by taking advantage of dataset
nesting, we took a dataset \sphinxcode{\sphinxupquote{longnow}} and installed it as a
subdataset within the superdataset  \sphinxcode{\sphinxupquote{DataLad\sphinxhyphen{}101}}.

\begin{figure}[tbp]
\centering
\capstart

\noindent\sphinxincludegraphics[width=0.700\linewidth]{{virtual_dstree_dl101}.pdf}
\caption{Virtual directory tree of a nested DataLad dataset}\label{\detokenize{basics/101-106-nesting:id1}}\label{\detokenize{basics/101-106-nesting:fignesting}}\end{figure}

\sphinxAtStartPar
If you have executed the above code snippets, make sure to go back into the
root of the dataset again:

\begin{sphinxVerbatim}[commandchars=\\\{\}]
\PYG{g+gp}{\PYGZdl{} }\PYG{n+nb}{cd}\PYG{+w}{ }../../
\end{sphinxVerbatim}

\sphinxstepscope


\section{Summary}
\label{\detokenize{basics/101-107-summary:summary}}\label{\detokenize{basics/101-107-summary::doc}}
\sphinxAtStartPar
In the last few sections, we have discovered the basics of starting a DataLad dataset from scratch,
and making simple modifications \sphinxstyleemphasis{locally}.
\begin{itemize}
\item {} 
\sphinxAtStartPar
An empty dataset can be created with the \sphinxcode{\sphinxupquote{datalad create}} command. It’s useful to add a description
to the dataset and use the \sphinxcode{\sphinxupquote{\sphinxhyphen{}c text2git}} configuration, but we will see later why.
This is the command structure:

\begin{sphinxVerbatim}[commandchars=\\\{\}]
\PYG{g+gp}{\PYGZdl{} }datalad\PYG{+w}{ }create\PYG{+w}{ }\PYGZhy{}\PYGZhy{}description\PYG{+w}{ }\PYG{l+s+s2}{\PYGZdq{}here is a description\PYGZdq{}}\PYG{+w}{ }\PYGZhy{}c\PYG{+w}{ }text2git\PYG{+w}{ }PATH
\end{sphinxVerbatim}

\item {} 
\sphinxAtStartPar
Thanks to {\hyperref[\detokenize{glossary:term-Git}]{\sphinxtermref{\DUrole{xref,std,std-term}{Git}}}} and {\hyperref[\detokenize{glossary:term-git-annex}]{\sphinxtermref{\DUrole{xref,std,std-term}{git\sphinxhyphen{}annex}}}}, the dataset has a history to track files and their
modifications. Built\sphinxhyphen{}in Git tools (\sphinxcode{\sphinxupquote{git log}}) or external tools (such as \sphinxcode{\sphinxupquote{tig}}) allow to explore
the history.

\item {} 
\sphinxAtStartPar
The \sphinxcode{\sphinxupquote{datalad save}} command records the current state of the dataset to the history. Make it a habit
to specify a concise commit message to summarize the change. If several unrelated modifications
exist in your dataset, specify the path to the precise file (change) that should be saved to history.
Remember, if you run a \sphinxcode{\sphinxupquote{datalad save}} without
specifying a path, all untracked files and all file changes will be committed to the history together!
This is the command structure:

\begin{sphinxVerbatim}[commandchars=\\\{\}]
\PYG{g+gp}{\PYGZdl{} }datalad\PYG{+w}{ }save\PYG{+w}{ }\PYGZhy{}m\PYG{+w}{ }\PYG{l+s+s2}{\PYGZdq{}here is a commit message\PYGZdq{}}\PYG{+w}{ }\PYG{o}{[}PATH\PYG{o}{]}
\end{sphinxVerbatim}

\item {} 
\sphinxAtStartPar
The typical local workflow is simple: \sphinxstyleemphasis{Modify} the dataset by adding or
modifying files, \sphinxstyleemphasis{save} the changes as meaningful units to the history,
\sphinxstyleemphasis{repeat} (see \hyperref[\detokenize{basics/101-107-summary:fig-simple-workflow}]{Fig.\@ \ref{\detokenize{basics/101-107-summary:fig-simple-workflow}}}).

\end{itemize}

\begin{figure}[tbp]
\centering
\capstart

\noindent\sphinxincludegraphics{{local_wf_simple}.png}
\caption{A simple, local version control workflow with DataLad.}\label{\detokenize{basics/101-107-summary:id1}}\label{\detokenize{basics/101-107-summary:fig-simple-workflow}}\end{figure}
\begin{itemize}
\item {} 
\sphinxAtStartPar
\sphinxcode{\sphinxupquote{datalad status}} reports the current state of the dataset. It’s a very helpful command you should
run frequently to check for untracked or modified content.

\item {} 
\sphinxAtStartPar
\sphinxcode{\sphinxupquote{datalad download\sphinxhyphen{}url}} can retrieve files from websources and save them
automatically to your dataset. This does not only save you the time of one
\sphinxcode{\sphinxupquote{datalad save}}, but it also records the source of the file as hidden
{\hyperref[\detokenize{glossary:term-provenance}]{\sphinxtermref{\DUrole{xref,std,std-term}{provenance}}}} information.

\end{itemize}

\sphinxAtStartPar
Furthermore, we have discovered the basics of installing a published DataLad dataset,
and experienced the concept of modular nesting datasets.

\index{clone@\spxentry{clone}!DataLad command@\spxentry{DataLad command}}\index{DataLad command@\spxentry{DataLad command}!clone@\spxentry{clone}}\ignorespaces \begin{itemize}
\item {} 
\sphinxAtStartPar
A published dataset can be installed with the \sphinxcode{\sphinxupquote{datalad clone}} command:

\begin{sphinxVerbatim}[commandchars=\\\{\}]
\PYG{g+gp}{\PYGZdl{} }datalad\PYG{+w}{ }clone\PYG{+w}{ }\PYG{o}{[}\PYGZhy{}\PYGZhy{}dataset\PYG{+w}{ }PATH\PYG{o}{]}\PYG{+w}{ }SOURCE\PYGZhy{}PATH/URL\PYG{+w}{ }\PYG{o}{[}DESTINATION\PYG{+w}{ }PATH\PYG{o}{]}
\end{sphinxVerbatim}

\sphinxAtStartPar
It can be installed “on its own”, or within an existing dataset.

\item {} 
\sphinxAtStartPar
The command takes a location of an existing dataset as a positional argument,
and optionally a path to where you want the dataset to be installed. If you do not specify a path,
the dataset will be installed into the current directory, with the original name of the
dataset.

\item {} 
\sphinxAtStartPar
If a dataset is installed inside of another dataset as a subdataset, the
\sphinxcode{\sphinxupquote{\sphinxhyphen{}\sphinxhyphen{}dataset}}/\sphinxcode{\sphinxupquote{\sphinxhyphen{}d}} option needs to specify the root of the containing dataset, the superdataset.

\item {} 
\sphinxAtStartPar
The source can be a URL, for example of a GitHub repository as in section {\hyperref[\detokenize{basics/101-105-install:installds}]{\sphinxcrossref{\DUrole{std,std-ref}{Install datasets}}}} (\autopageref*{\detokenize{basics/101-105-install:installds}}), but also
paths, or open data collections.

\item {} 
\sphinxAtStartPar
After \sphinxcode{\sphinxupquote{datalad clone}}, only small files and metadata about file availability are present locally.
To retrieve actual file content of larger files, \sphinxcode{\sphinxupquote{datalad get PATH}} downloads large file
content on demand.

\item {} 
\sphinxAtStartPar
\sphinxcode{\sphinxupquote{datalad status \sphinxhyphen{}\sphinxhyphen{}annex}} or \sphinxcode{\sphinxupquote{datalad status \sphinxhyphen{}\sphinxhyphen{}annex all}} are helpful to determine
total repository size and the amount of data that is present locally.

\item {} 
\sphinxAtStartPar
Remember: Super\sphinxhyphen{} and subdatasets have standalone histories. A superdataset stores the currently used
version of a contained subdataset through an identifier.

\end{itemize}


\subsection{Now what can I do with that?}
\label{\detokenize{basics/101-107-summary:now-what-can-i-do-with-that}}
\sphinxAtStartPar
Simple, local workflows allow you to version control changing small files,
for example, your CV, your code, or a book that you are working on, but
you can also add very large files to your datasets history.
Currently, this can be considered “best\sphinxhyphen{}practice building”: Frequent \sphinxcode{\sphinxupquote{datalad status}}
commands, \sphinxcode{\sphinxupquote{datalad save}} commands to save dataset modifications,
and concise {\hyperref[\detokenize{glossary:term-commit-message}]{\sphinxtermref{\DUrole{xref,std,std-term}{commit message}}}}s are the main take always from this. You can already explore
the history of a dataset and you know about many types of provenance information
captured by DataLad, but for now, its been only informative, and has not been used
for anything more fancy. Later on, we will look into utilizing the history
in order to undo mistakes, how the origin of files or datasets becomes helpful
when sharing datasets or removing file contents, and how to make changes to large
content (as opposed to small content we have been modifying so far).

\sphinxAtStartPar
Additionally, you learned the basics on extending the \sphinxcode{\sphinxupquote{DataLad\sphinxhyphen{}101}} dataset and consuming
existing datasets:
You have procedurally experienced how to install a dataset, and simultaneously you have
learned a lot about the principles and features of DataLad datasets.
Cloning datasets and getting their content allows you to consume published datasets.
By nesting datasets within each other, you can reuse datasets in a modular fashion. While this may
appear abstract, upcoming sections will demonstrate many examples of why this can be handy.

\sphinxstepscope


