\chapter{Collaboration}
\label{\detokenize{basics/basics-collaboration:collaboration}}\label{\detokenize{basics/basics-collaboration:chapter-collaboration}}\label{\detokenize{basics/basics-collaboration::doc}}
\noindent{\hspace*{\fill}\sphinxincludegraphics[width=0.500\linewidth]{{collaboration_sketch}.pdf}\hspace*{\fill}}

\sphinxstepscope

\index{data sharing@\spxentry{data sharing}!with DataLad@\spxentry{with DataLad}}\index{with DataLad@\spxentry{with DataLad}!data sharing@\spxentry{data sharing}}\ignorespaces 

\section{Looking without touching}
\label{\detokenize{basics/101-116-sharelocal:looking-without-touching}}\label{\detokenize{basics/101-116-sharelocal:sharelocal1}}\label{\detokenize{basics/101-116-sharelocal:index-0}}\label{\detokenize{basics/101-116-sharelocal::doc}}
\sphinxAtStartPar
Only now, several weeks into the DataLad\sphinxhyphen{}101 course does your room
mate realize that he has enrolled in the course as well, but has not
yet attended at all. “Oh man, can you help me catch up?” he asks
you one day. “Sharing just your notes would be really cool for a
start already!”

\sphinxAtStartPar
“Sure thing”, you say, and decide that it’s probably best if he gets
all of the \sphinxcode{\sphinxupquote{DataLad\sphinxhyphen{}101}} course dataset. Sharing datasets was
something you wanted to look into soon, anyway.

\sphinxAtStartPar
This is one exciting aspect of DataLad datasets that has yet been missing
from this course: How does one share a dataset?
In this section, we will cover the simplest way of sharing a dataset:
on a local or shared file system, via an \sphinxstyleemphasis{installation} with a path as
a source.
\begin{importantnote}[before title={\thetcbcounter\ }, check odd page=true]{More on public data sharing}

\sphinxAtStartPar
Interested in sharing datasets \sphinxstyleemphasis{publicly}? Read this chapter to get a feel
for all relevant basic concepts of sharing datasets. Afterwards, head over
to chapter {\hyperref[\detokenize{basics/basics-thirdparty:chapter-thirdparty}]{\sphinxcrossref{\DUrole{std,std-ref}{Distributing datasets}}}} (\autopageref*{\detokenize{basics/basics-thirdparty:chapter-thirdparty}}) to find out how to share a dataset
on third\sphinxhyphen{}party infrastructure.


\end{importantnote}

\sphinxAtStartPar
In this scenario multiple people can access the very same files at the
same time, often on the same machine (e.g., a shared workstation, or
a server that people can “{\hyperref[\detokenize{glossary:term-SSH}]{\sphinxtermref{\DUrole{xref,std,std-term}{SSH}}}}” into). You might think: “What do I need
DataLad for, if everyone can already access everything?” However,
universal, unrestricted access can easily lead to chaos. DataLad can
help facilitate collaboration without requiring ultimate trust and
reliability of all participants. Essentially, with a shared dataset,
collaborators can see and use your dataset without any danger
of undesired, or uncontrolled modification.

\sphinxAtStartPar
To demonstrate how to share a DataLad dataset on a common file system,
we will pretend that your personal computer
can be accessed by other users. Let’s say that
your room mate has access, and you are making sure that there is
a \sphinxcode{\sphinxupquote{DataLad\sphinxhyphen{}101}} dataset in a different place on the file system
for him to access and work with.

\sphinxAtStartPar
This is indeed a common real\sphinxhyphen{}world use case: Two users on a shared
file system sharing a dataset with each other.
But as we cannot easily simulate a second user in this book,
for now, you will have to share your dataset with yourself.
This endeavor serves several purposes: For one, you will experience a very easy
way of sharing a dataset. Secondly, it will show you
how a dataset can be obtained from a path, instead of a URL as shown in section
{\hyperref[\detokenize{basics/101-105-install:installds}]{\sphinxcrossref{\DUrole{std,std-ref}{Install datasets}}}} (\autopageref*{\detokenize{basics/101-105-install:installds}}). Thirdly, \sphinxcode{\sphinxupquote{DataLad\sphinxhyphen{}101}} is a dataset that can
showcase many different properties of a dataset already, but it will
be an additional learning experience to see how the different parts
of the dataset \textendash{} text files, larger files, subdatasets,
{\hyperref[\detokenize{glossary:term-run-record}]{\sphinxtermref{\DUrole{xref,std,std-term}{run record}}}}s \textendash{} will appear upon installation when shared.
And lastly, you will likely “share a dataset with yourself” whenever you
will be using a particular dataset of your own creation as input for
one or more projects.

\sphinxAtStartPar
“Awesome!” exclaims your room mate as you take out your laptop to
share the dataset. “You are really saving my ass
here. I’ll make up for it when we prepare for the final”, he promises.

\sphinxAtStartPar
To install \sphinxcode{\sphinxupquote{DataLad\sphinxhyphen{}101}} into a different part
of your file system, navigate out of \sphinxcode{\sphinxupquote{DataLad\sphinxhyphen{}101}}, and \textendash{} for
simplicity \textendash{} create a new directory, \sphinxcode{\sphinxupquote{mock\_user}}, right next to it:

\begin{sphinxVerbatim}[commandchars=\\\{\}]
\PYG{g+gp}{\PYGZdl{} }\PYG{n+nb}{cd}\PYG{+w}{ }../
\PYG{g+gp}{\PYGZdl{} }mkdir\PYG{+w}{ }mock\PYGZus{}user
\end{sphinxVerbatim}

\sphinxAtStartPar
For simplicity, pretend that this is a second user’s \textendash{} your room mate’s \textendash{}
home directory. Furthermore, let’s for now disregard anything about
{\hyperref[\detokenize{glossary:term-permissions}]{\sphinxtermref{\DUrole{xref,std,std-term}{permissions}}}}. In a real\sphinxhyphen{}world example you likely would not be able to read and write
to a different user’s directories, but we will talk about permissions later.

\index{clone@\spxentry{clone}!DataLad command@\spxentry{DataLad command}}\index{DataLad command@\spxentry{DataLad command}!clone@\spxentry{clone}}\index{clone dataset (set location description)@\spxentry{clone dataset}\spxextra{set location description}!with DataLad@\spxentry{with DataLad}}\index{with DataLad@\spxentry{with DataLad}!clone dataset (set location description)@\spxentry{clone dataset}\spxextra{set location description}}\ignorespaces 
\sphinxAtStartPar
After creation, navigate into \sphinxcode{\sphinxupquote{mock\_user}} and install the dataset \sphinxcode{\sphinxupquote{DataLad\sphinxhyphen{}101}}.
To do this, use \sphinxcode{\sphinxupquote{datalad clone}}, and provide a path to your original
dataset:

\begin{sphinxVerbatim}[commandchars=\\\{\}]
\PYG{g+gp}{\PYGZdl{} }\PYG{n+nb}{cd}\PYG{+w}{ }mock\PYGZus{}user
\PYG{g+gp}{\PYGZdl{} }datalad\PYG{+w}{ }clone\PYG{+w}{ }\PYGZhy{}\PYGZhy{}description\PYG{+w}{ }\PYG{l+s+s2}{\PYGZdq{}DataLad\PYGZhy{}101 in mock\PYGZus{}user\PYGZdq{}}\PYG{+w}{ }../DataLad\PYGZhy{}101
\PYG{g+go}{install(ok): /home/me/dl\PYGZhy{}101/mock\PYGZus{}user/DataLad\PYGZhy{}101 (dataset)}
\end{sphinxVerbatim}

\sphinxAtStartPar
This will install your dataset \sphinxcode{\sphinxupquote{DataLad\sphinxhyphen{}101}} into your room mate’s home
directory. Note that we have given this new
dataset a description about its location. Note further that we
have not provided the optional destination path to \sphinxcode{\sphinxupquote{datalad clone}},
and hence it installed the dataset under its original name in the current directory.

\sphinxAtStartPar
Together with your room mate, you go ahead and see what this dataset looks
like. Before running the command, try to predict what you will see.

\begin{sphinxVerbatim}[commandchars=\\\{\}]
\PYG{g+gp}{\PYGZdl{} }\PYG{n+nb}{cd}\PYG{+w}{ }DataLad\PYGZhy{}101
\PYG{g+gp}{\PYGZdl{} }tree
\PYG{g+go}{.}
\PYG{g+go}{├── books}
\PYG{g+go}{│   ├── bash\PYGZus{}guide.pdf \PYGZhy{}\PYGZgt{} ../.git/annex/objects/WF/Gq/✂/MD5E\PYGZhy{}s1198170\PYGZhy{}\PYGZhy{}0ab2c121✂MD5.pdf}
\PYG{g+go}{│   ├── byte\PYGZhy{}of\PYGZhy{}python.pdf \PYGZhy{}\PYGZgt{} ../.git/annex/objects/xF/42/✂/MD5E\PYGZhy{}s4161086\PYGZhy{}\PYGZhy{}c832fc13✂MD5.pdf}
\PYG{g+go}{│   ├── progit.pdf \PYGZhy{}\PYGZgt{} ../.git/annex/objects/G6/Gj/✂/MD5E\PYGZhy{}s12465653\PYGZhy{}\PYGZhy{}05cd7ed5✂MD5.pdf}
\PYG{g+go}{│   └── TLCL.pdf \PYGZhy{}\PYGZgt{} ../.git/annex/objects/jf/3M/✂/MD5E\PYGZhy{}s2120211\PYGZhy{}\PYGZhy{}06d1efcb✂MD5.pdf}
\PYG{g+go}{├── code}
\PYG{g+go}{│   └── list\PYGZus{}titles.sh}
\PYG{g+go}{├── notes.txt}
\PYG{g+go}{└── recordings}
\PYG{g+go}{    ├── interval\PYGZus{}logo\PYGZus{}small.jpg \PYGZhy{}\PYGZgt{} ../.git/annex/objects/pw/Mf/✂/MD5E\PYGZhy{}s70348\PYGZhy{}\PYGZhy{}4b2ec0db✂MD5.jpg}
\PYG{g+go}{    ├── longnow}
\PYG{g+go}{    ├── podcasts.tsv}
\PYG{g+go}{    └── salt\PYGZus{}logo\PYGZus{}small.jpg \PYGZhy{}\PYGZgt{} ../.git/annex/objects/fZ/wg/✂/MD5E\PYGZhy{}s76402\PYGZhy{}\PYGZhy{}87da732f✂MD5.jpg}

\PYG{g+go}{4 directories, 9 files}
\end{sphinxVerbatim}

\sphinxAtStartPar
There are a number of interesting things, and your room mate is the
first to notice them:

\sphinxAtStartPar
“Hey, can you explain some things to me?”, he asks. “This directory
here, “\sphinxcode{\sphinxupquote{longnow}}”, why is it empty?”
True, the subdataset has a directory name but apart from this,
the \sphinxcode{\sphinxupquote{longnow}} directory appears empty.

\sphinxAtStartPar
“Also, why do the PDFs in \sphinxcode{\sphinxupquote{books/}} and the \sphinxcode{\sphinxupquote{.jpg}} files
appear so weird? They have
this cryptic path right next to them, and look, if I try to open
one of them, it fails! Did something go wrong when we installed
the dataset?” he worries.
Indeed, the PDFs and pictures appear just as they did in the original dataset
on first sight: They are symlinks pointing to some location in the
object tree. To reassure your room mate that everything is fine you
quickly explain to him the concept of a symlink and the {\hyperref[\detokenize{glossary:term-object-tree}]{\sphinxtermref{\DUrole{xref,std,std-term}{object\sphinxhyphen{}tree}}}}
of {\hyperref[\detokenize{glossary:term-git-annex}]{\sphinxtermref{\DUrole{xref,std,std-term}{git\sphinxhyphen{}annex}}}}.

\index{clone@\spxentry{clone}!DataLad command@\spxentry{DataLad command}}\index{DataLad command@\spxentry{DataLad command}!clone@\spxentry{clone}}\ignorespaces 
\sphinxAtStartPar
“But why does the PDF not open when I try to open it?” he repeats.
True, these files cannot be opened. This mimics our experience when
installing the \sphinxcode{\sphinxupquote{longnow}} subdataset: Right after installation,
the \sphinxcode{\sphinxupquote{.mp3}} files also could not be opened, because their file
content was not yet retrieved. You begin to explain to your room mate
how DataLad retrieves only minimal metadata about which files actually
exist in a dataset upon a \sphinxcode{\sphinxupquote{datalad clone}}. “It’s really handy”,
you tell him. “This way you can decide which book you want to read,
and then retrieve what you need. Everything that is \sphinxstyleemphasis{annexed} is retrieved
on demand. Note though that the text files
contents are present, and the files can be opened \textendash{} this is because
these files are stored in {\hyperref[\detokenize{glossary:term-Git}]{\sphinxtermref{\DUrole{xref,std,std-term}{Git}}}}. So you already have my notes,
and you can decide for yourself whether you want to \sphinxcode{\sphinxupquote{get}} the books.”

\sphinxAtStartPar
To demonstrate this, you decide to examine the PDFs further.
“Try to get one of the books”, you instruct your room mate:

\begin{sphinxVerbatim}[commandchars=\\\{\}]
\PYG{g+gp}{\PYGZdl{} }datalad\PYG{+w}{ }get\PYG{+w}{ }books/progit.pdf
\PYG{g+go}{get(ok): books/progit.pdf (file) [from origin...]}
\end{sphinxVerbatim}

\sphinxAtStartPar
“Opening this file will work, because the content was retrieved from
the original dataset.”, you explain, proud that this worked just as you
thought it would.

\sphinxAtStartPar
Let’s now turn to the fact that the subdataset \sphinxcode{\sphinxupquote{longnow}} contains neither
file content nor file metadata information to explore the contents of the
dataset: there are no subdirectories or any files under \sphinxcode{\sphinxupquote{recordings/longnow/}}.
This is behavior that you have not observed until now.
To fix this and obtain file availability metadata,
you have to run a somewhat unexpected command:

\begin{sphinxVerbatim}[commandchars=\\\{\}]
\PYG{g+gp}{\PYGZdl{} }datalad\PYG{+w}{ }get\PYG{+w}{ }\PYGZhy{}n\PYG{+w}{ }recordings/longnow
\PYG{g+go}{[INFO] Remote origin not usable by git\PYGZhy{}annex; setting annex\PYGZhy{}ignore}
\PYG{g+go}{install(ok): /home/me/dl\PYGZhy{}101/mock\PYGZus{}user/DataLad\PYGZhy{}101/recordings/longnow (dataset) [Installed subdataset in order to get /home/me/dl\PYGZhy{}101/mock\PYGZus{}user/DataLad\PYGZhy{}101/recordings/longnow]}
\end{sphinxVerbatim}

\sphinxAtStartPar
Before we look further into \sphinxcode{\sphinxupquote{datalad get}} and the
\sphinxcode{\sphinxupquote{\sphinxhyphen{}n/\sphinxhyphen{}\sphinxhyphen{}no\sphinxhyphen{}data}} option, let’s first see what has changed after
running the above command (excerpt):

\begin{sphinxVerbatim}[commandchars=\\\{\}]
\PYG{g+gp}{\PYGZdl{} }tree
\PYG{g+go}{.}
\PYG{g+go}{├── books}
\PYG{g+go}{│   ├── bash\PYGZus{}guide.pdf \PYGZhy{}\PYGZgt{} ../.git/annex/objects/WF/Gq/✂/MD5E\PYGZhy{}s1198170\PYGZhy{}\PYGZhy{}0ab2c121✂MD5.pdf}
\PYG{g+go}{│   ├── byte\PYGZhy{}of\PYGZhy{}python.pdf \PYGZhy{}\PYGZgt{} ../.git/annex/objects/xF/42/✂/MD5E\PYGZhy{}s4161086\PYGZhy{}\PYGZhy{}c832fc13✂MD5.pdf}
\PYG{g+go}{│   ├── progit.pdf \PYGZhy{}\PYGZgt{} ../.git/annex/objects/G6/Gj/✂/MD5E\PYGZhy{}s12465653\PYGZhy{}\PYGZhy{}05cd7ed5✂MD5.pdf}
\PYG{g+go}{│   └── TLCL.pdf \PYGZhy{}\PYGZgt{} ../.git/annex/objects/jf/3M/✂/MD5E\PYGZhy{}s2120211\PYGZhy{}\PYGZhy{}06d1efcb✂MD5.pdf}
\PYG{g+go}{├── code}
\PYG{g+go}{│   └── list\PYGZus{}titles.sh}
\PYG{g+go}{├── notes.txt}
\PYG{g+go}{└── recordings}
\PYG{g+go}{    ├── interval\PYGZus{}logo\PYGZus{}small.jpg \PYGZhy{}\PYGZgt{} ../.git/annex/objects/pw/Mf/✂/MD5E\PYGZhy{}s70348\PYGZhy{}\PYGZhy{}4b2ec0db✂MD5.jpg}
\PYG{g+go}{    ├── longnow}
\PYG{g+go}{    │   ├── Long\PYGZus{}Now\PYGZus{}\PYGZus{}Conversations\PYGZus{}at\PYGZus{}The\PYGZus{}Interval}
\PYG{g+go}{    │   │   ├── 2017\PYGZus{}06\PYGZus{}09\PYGZus{}\PYGZus{}How\PYGZus{}Digital\PYGZus{}Memory\PYGZus{}Is\PYGZus{}Shaping\PYGZus{}Our\PYGZus{}Future\PYGZus{}\PYGZus{}Abby\PYGZus{}Smith\PYGZus{}Rumsey.mp3 \PYGZhy{}\PYGZgt{} ../.git/annex/objects/8j/kQ/✂/MD5E\PYGZhy{}s66305442\PYGZhy{}\PYGZhy{}c723d53d✂MD5.mp3}
\PYG{g+go}{    │   │   ├── 2017\PYGZus{}06\PYGZus{}09\PYGZus{}\PYGZus{}Pace\PYGZus{}Layers\PYGZus{}Thinking\PYGZus{}\PYGZus{}Stewart\PYGZus{}Brand\PYGZus{}\PYGZus{}Paul\PYGZus{}Saffo.mp3 \PYGZhy{}\PYGZgt{} ../.git/annex/objects/Qk/9M/✂/MD5E\PYGZhy{}s112801659\PYGZhy{}\PYGZhy{}00a42a1a✂MD5.mp3}
\PYG{g+go}{    │   │   ├── 2017\PYGZus{}06\PYGZus{}09\PYGZus{}\PYGZus{}Proof\PYGZus{}\PYGZus{}The\PYGZus{}Science\PYGZus{}of\PYGZus{}Booze\PYGZus{}\PYGZus{}Adam\PYGZus{}Rogers.mp3 \PYGZhy{}\PYGZgt{} ../.git/annex/objects/FP/96/✂/MD5E\PYGZhy{}s60091960\PYGZhy{}\PYGZhy{}6e48eceb✂MD5.mp3}
\PYG{g+go}{    │   │   ├── 2017\PYGZus{}06\PYGZus{}09\PYGZus{}\PYGZus{}Seveneves\PYGZus{}at\PYGZus{}The\PYGZus{}Interval\PYGZus{}\PYGZus{}Neal\PYGZus{}Stephenson.mp3 \PYGZhy{}\PYGZgt{} ../.git/annex/objects/Wf/5Q/✂/MD5E\PYGZhy{}s66431897\PYGZhy{}\PYGZhy{}aff90c83✂MD5.mp3}
\PYG{g+go}{    │   │   ├── 2017\PYGZus{}06\PYGZus{}09\PYGZus{}\PYGZus{}Talking\PYGZus{}with\PYGZus{}Robots\PYGZus{}about\PYGZus{}Architecture\PYGZus{}\PYGZus{}Jeffrey\PYGZus{}McGrew.mp3 \PYGZhy{}\PYGZgt{} ../.git/annex/objects/Fj/9V/✂/MD5E\PYGZhy{}s61491081\PYGZhy{}\PYGZhy{}c4e88ea0✂MD5.mp3}
\PYG{g+go}{    │   │   ├── 2017\PYGZus{}06\PYGZus{}09\PYGZus{}\PYGZus{}The\PYGZus{}Red\PYGZus{}Planet\PYGZus{}for\PYGZus{}Real\PYGZus{}\PYGZus{}Andy\PYGZus{}Weir.mp3 \PYGZhy{}\PYGZgt{} ../.git/annex/objects/xq/Q3/✂/MD5E\PYGZhy{}s136924472\PYGZhy{}\PYGZhy{}0d107210✂MD5.mp3}
\end{sphinxVerbatim}

\sphinxAtStartPar
Interesting! The file metadata information is now present, and we can
explore the file hierarchy. The file content, however, is not present yet.

\sphinxAtStartPar
What has happened here?

\sphinxAtStartPar
When DataLad installs a dataset, it will by default only obtain the
superdataset, and not any subdatasets. The superdataset contains the
information that a subdataset exists though \textendash{} the subdataset is \sphinxstyleemphasis{registered}
in the superdataset.  This is why the subdataset name exists as a directory.
A subsequent \sphinxcode{\sphinxupquote{datalad get \sphinxhyphen{}n path/to/longnow}} will install the registered
subdataset again, just as we did in the example above.

\sphinxAtStartPar
But what about the \sphinxcode{\sphinxupquote{\sphinxhyphen{}n}} option for \sphinxcode{\sphinxupquote{datalad get}}?
Previously, we used \sphinxcode{\sphinxupquote{datalad get}} to get file content. However,
\sphinxcode{\sphinxupquote{datalad get}} operates on more than just the level of \sphinxstyleemphasis{files} or \sphinxstyleemphasis{directories}.
Instead, it can also operate on the level of \sphinxstyleemphasis{datasets}. Regardless of whether
it is a single file (such as \sphinxcode{\sphinxupquote{books/TLCL.pdf}}) or a registered subdataset
(such as \sphinxcode{\sphinxupquote{recordings/longnow}}), \sphinxcode{\sphinxupquote{datalad get}} will operate on it to 1) install
it \textendash{} if it is a not yet installed subdataset \textendash{} and 2) retrieve the contents of any files.
That makes it very easy to get your file content, regardless of
how your dataset may be structured \textendash{} it is always the same command, and DataLad
blurs the boundaries between superdatasets and subdatasets.

\sphinxAtStartPar
In the above example, we called \sphinxcode{\sphinxupquote{datalad get}} with the option \sphinxcode{\sphinxupquote{\sphinxhyphen{}n/\sphinxhyphen{}\sphinxhyphen{}no\sphinxhyphen{}data}}.
This option prevents that \sphinxcode{\sphinxupquote{datalad get}} obtains the data of individual files or
directories, thus limiting its scope to the level of datasets as only a
\sphinxcode{\sphinxupquote{datalad clone}} is performed. Without this option, the command would
have retrieved all of the subdatasets contents right away. But with \sphinxcode{\sphinxupquote{\sphinxhyphen{}n/\sphinxhyphen{}\sphinxhyphen{}no\sphinxhyphen{}data}},
it only installed the subdataset to retrieve the meta data about file availability.

\index{get all dataset content@\spxentry{get all dataset content}!with DataLad@\spxentry{with DataLad}}\index{with DataLad@\spxentry{with DataLad}!get all dataset content@\spxentry{get all dataset content}}\ignorespaces 
\sphinxAtStartPar
To explicitly install all potential subdatasets \sphinxstyleemphasis{recursively}, that is,
all of the subdatasets inside it as well, one can give the
\sphinxcode{\sphinxupquote{\sphinxhyphen{}r}}/\sphinxcode{\sphinxupquote{\sphinxhyphen{}\sphinxhyphen{}recursive}} option to \sphinxcode{\sphinxupquote{datalad get}}:

\begin{sphinxVerbatim}[commandchars=\\\{\}]
\PYG{g+gp}{\PYGZdl{} }datalad\PYG{+w}{ }get\PYG{+w}{ }\PYGZhy{}n\PYG{+w}{ }\PYGZhy{}r\PYG{+w}{ }\PYGZlt{}subds\PYGZgt{}
\end{sphinxVerbatim}

\sphinxAtStartPar
This would install the \sphinxcode{\sphinxupquote{subds}} subdataset and all potential further
subdatasets inside of it, and the meta data about file hierarchies would
have been available right away for every subdataset inside of \sphinxcode{\sphinxupquote{subds}}. If you
had several subdatasets and would not provide a path to a single dataset,
but, say, the current directory (\sphinxcode{\sphinxupquote{.}} as in \sphinxcode{\sphinxupquote{datalad get \sphinxhyphen{}n \sphinxhyphen{}r .}}), it
would clone all registered subdatasets recursively.

\sphinxAtStartPar
So why is a recursive get not the default behavior?
In {\hyperref[\detokenize{basics/101-106-nesting:nesting}]{\sphinxcrossref{\DUrole{std,std-ref}{Dataset nesting}}}} (\autopageref*{\detokenize{basics/101-106-nesting:nesting}}) we learned that datasets can be nested \sphinxstyleemphasis{arbitrarily} deep.
Upon getting the meta data of one dataset you might not want to also install
a few dozen levels of nested subdatasets right away.

\sphinxAtStartPar
However, there is a middle way%
\begin{footnote}\sphinxAtStartFootnote
Another alternative to a recursion limit to \sphinxcode{\sphinxupquote{datalad get \sphinxhyphen{}n \sphinxhyphen{}r}} is
a dataset configuration that specifies subdatasets that should \sphinxstyleemphasis{not} be
cloned recursively, unless explicitly given to the command with a path. With
this configuration, a superdataset’s maintainer can safeguard users and prevent
potentially large amounts of subdatasets to be cloned.
You can learn more about this configuration in the section {\hyperref[\detokenize{basics/101-123-config2:config2}]{\sphinxcrossref{\DUrole{std,std-ref}{Shipping and overriding configuration}}}} (\autopageref*{\detokenize{basics/101-123-config2:config2}}).
%
\end{footnote}: The \sphinxcode{\sphinxupquote{\sphinxhyphen{}\sphinxhyphen{}recursion\sphinxhyphen{}limit}} option let’s
you specify how many levels of subdatasets should be installed together
with the first subdataset:

\begin{sphinxVerbatim}[commandchars=\\\{\}]
\PYG{g+gp}{\PYGZdl{} }datalad\PYG{+w}{ }get\PYG{+w}{ }\PYGZhy{}n\PYG{+w}{ }\PYGZhy{}r\PYG{+w}{ }\PYGZhy{}\PYGZhy{}recursion\PYGZhy{}limit\PYG{+w}{ }\PYG{l+m}{1}\PYG{+w}{ }\PYGZlt{}subds\PYGZgt{}
\end{sphinxVerbatim}

\sphinxAtStartPar
To summarize what you learned in this section, write a note on how to
install a dataset using a path as a source on a common file system.

\sphinxAtStartPar
Write this note in “your own” (the original) \sphinxcode{\sphinxupquote{DataLad\sphinxhyphen{}101}} dataset, though!

\begin{sphinxVerbatim}[commandchars=\\\{\}]
\PYG{g+gp}{\PYGZdl{} }\PYG{c+c1}{\PYGZsh{} navigate back into the original dataset}
\PYG{g+gp}{\PYGZdl{} }\PYG{n+nb}{cd}\PYG{+w}{ }../../DataLad\PYGZhy{}101
\PYG{g+gp}{\PYGZdl{} }\PYG{c+c1}{\PYGZsh{} write the note}
\PYG{g+gp}{\PYGZdl{} }cat\PYG{+w}{ }\PYGZlt{}\PYGZlt{}\PYG{+w}{ }EOT\PYG{+w}{ }\PYGZgt{}\PYGZgt{}\PYG{+w}{ }notes.txt
\PYG{g+go}{A source to install a dataset from can also be a path, for example as}
\PYG{g+go}{in \PYGZdq{}datalad clone ../DataLad\PYGZhy{}101\PYGZdq{}.}

\PYG{g+go}{Just as in creating datasets, you can add a description on the}
\PYG{g+go}{location of the new dataset clone with the \PYGZhy{}D/\PYGZhy{}\PYGZhy{}description option.}

\PYG{g+go}{Note that subdatasets will not be installed by default, but are only}
\PYG{g+go}{registered in the superdataset \PYGZhy{}\PYGZhy{} you will have to do a}
\PYG{g+go}{\PYGZdq{}datalad get \PYGZhy{}n PATH/TO/SUBDATASET\PYGZdq{} to install the subdataset for file}
\PYG{g+go}{availability meta data. The \PYGZhy{}n/\PYGZhy{}\PYGZhy{}no\PYGZhy{}data options prevents that file}
\PYG{g+go}{contents are also downloaded.}

\PYG{g+go}{Note that a recursive \PYGZdq{}datalad get\PYGZdq{} would install all further}
\PYG{g+go}{registered subdatasets underneath a subdataset, so a safer way to}
\PYG{g+go}{proceed is to set a decent \PYGZhy{}\PYGZhy{}recursion\PYGZhy{}limit:}
\PYG{g+go}{\PYGZdq{}datalad get \PYGZhy{}n \PYGZhy{}r \PYGZhy{}\PYGZhy{}recursion\PYGZhy{}limit 2 \PYGZlt{}subds\PYGZgt{}\PYGZdq{}}

\PYG{g+go}{EOT}
\end{sphinxVerbatim}

\sphinxAtStartPar
Save this note.

\begin{sphinxVerbatim}[commandchars=\\\{\}]
\PYG{g+gp}{\PYGZdl{} }datalad\PYG{+w}{ }save\PYG{+w}{ }\PYGZhy{}m\PYG{+w}{ }\PYG{l+s+s2}{\PYGZdq{}add note about cloning from paths and recursive datalad get\PYGZdq{}}
\PYG{g+go}{add(ok): notes.txt (file)}
\PYG{g+go}{save(ok): . (dataset)}
\end{sphinxVerbatim}

\index{clone@\spxentry{clone}!DataLad concept@\spxentry{DataLad concept}}\index{DataLad concept@\spxentry{DataLad concept}!clone@\spxentry{clone}}\ignorespaces \begin{gitusernote}[label={index-4}, before title={\thetcbcounter\ }, check odd page=true]{Get a clone}
\label{\detokenize{basics/101-116-sharelocal:index-4}}

\sphinxAtStartPar
A dataset that is installed from an existing source, e.g., a path or URL,
is the DataLad equivalent of a \sphinxstyleemphasis{clone} in Git.


\end{gitusernote}

\sphinxstepscope

\index{data sharing@\spxentry{data sharing}!with DataLad@\spxentry{with DataLad}}\index{with DataLad@\spxentry{with DataLad}!data sharing@\spxentry{data sharing}}\index{data availability tracking@\spxentry{data availability tracking}!with git\sphinxhyphen{}annex@\spxentry{with git\sphinxhyphen{}annex}}\index{with git\sphinxhyphen{}annex@\spxentry{with git\sphinxhyphen{}annex}!data availability tracking@\spxentry{data availability tracking}}\ignorespaces 

\section{Where’s Waldo?}
\label{\detokenize{basics/101-117-sharelocal2:where-s-waldo}}\label{\detokenize{basics/101-117-sharelocal2:sharelocal2}}\label{\detokenize{basics/101-117-sharelocal2:index-0}}\label{\detokenize{basics/101-117-sharelocal2::doc}}
\sphinxAtStartPar
So far, you and your room mate have created a copy of the \sphinxcode{\sphinxupquote{DataLad\sphinxhyphen{}101}}
dataset on the same file system but a different place by installing
it from a path.

\sphinxAtStartPar
You have observed that the \sphinxcode{\sphinxupquote{\sphinxhyphen{}r}}/\sphinxcode{\sphinxupquote{\sphinxhyphen{}\sphinxhyphen{}recursive}}
option needs to be given to \sphinxcode{\sphinxupquote{datalad get {[}\sphinxhyphen{}n/\sphinxhyphen{}\sphinxhyphen{}no\sphinxhyphen{}data{]}}}
in order to install further potential subdatasets in one go. Only then
is the subdatasets file content availability metadata present to explore
the file hierarchy available within the subdataset.
Alternatively, a \sphinxcode{\sphinxupquote{datalad get \sphinxhyphen{}n \textless{}subds\textgreater{}}} takes care of installing
exactly the specified registered subdataset.

\sphinxAtStartPar
And you have mesmerized your room mate by showing him how {\hyperref[\detokenize{glossary:term-git-annex}]{\sphinxtermref{\DUrole{xref,std,std-term}{git\sphinxhyphen{}annex}}}}
retrieved large file contents from the original dataset.
Your room mate is excited by this magical command.
You however begin to wonder: how does DataLad know where to look for that original content?

\sphinxAtStartPar
This information comes from git\sphinxhyphen{}annex.
Before getting another PDF, let’s query git\sphinxhyphen{}annex where its content is stored:

\index{whereis@\spxentry{whereis}!git\sphinxhyphen{}annex command@\spxentry{git\sphinxhyphen{}annex command}}\index{git\sphinxhyphen{}annex command@\spxentry{git\sphinxhyphen{}annex command}!whereis@\spxentry{whereis}}\index{show file content availability@\spxentry{show file content availability}!with git\sphinxhyphen{}annex@\spxentry{with git\sphinxhyphen{}annex}}\index{with git\sphinxhyphen{}annex@\spxentry{with git\sphinxhyphen{}annex}!show file content availability@\spxentry{show file content availability}}\ignorespaces 
\def\sphinxLiteralBlockLabel{\label{\detokenize{basics/101-117-sharelocal2:index-1}}}
\begin{sphinxVerbatim}[commandchars=\\\{\}]
\PYG{g+gp}{\PYGZdl{} }\PYG{c+c1}{\PYGZsh{} navigate back into the clone of DataLad\PYGZhy{}101}
\PYG{g+gp}{\PYGZdl{} }\PYG{n+nb}{cd}\PYG{+w}{ }../mock\PYGZus{}user/DataLad\PYGZhy{}101
\PYG{g+gp}{\PYGZdl{} }git\PYG{+w}{ }annex\PYG{+w}{ }whereis\PYG{+w}{ }books/TLCL.pdf
\PYG{g+go}{whereis books/TLCL.pdf (1 copy)}
\PYG{g+go}{  0c450dc0\PYGZhy{}c48c\PYGZhy{}4057\PYGZhy{}a231\PYGZhy{}e2654b689600 \PYGZhy{}\PYGZhy{} me@appveyor\PYGZhy{}vm:\PYGZti{}/dl\PYGZhy{}101/DataLad\PYGZhy{}101 [origin]}
\PYG{g+go}{ok}
\end{sphinxVerbatim}

\sphinxAtStartPar
Oh, another cryptic character sequence \sphinxhyphen{} this time however not a symlink, but an {\hyperref[\detokenize{glossary:term-annex-UUID}]{\sphinxtermref{\DUrole{xref,std,std-term}{annex UUID}}}}.
“That’s hard to read \textendash{} what is it?” your room mate asks.
You can recognize a path to the dataset on your computer, prefixed with the user and hostname of your computer.
“This”, you exclaim, excited about your own realization, “is my dataset’s location I’m sharing it from!” You can check the \textit{Find-out-more}~{\findoutmoreiconinline}\textit{\ref{fom-clone-description}} {\hyperref[\detokenize{basics/101-117-sharelocal2:fom-clone-description}]{\sphinxcrossref{\DUrole{std,std-ref}{on customizing this description}}}} (\autopageref*{\detokenize{basics/101-117-sharelocal2:fom-clone-description}}) and learn how to tweak this accoridng to your taste.

\index{set description for dataset location@\spxentry{set description for dataset location}!with DataLad@\spxentry{with DataLad}}\index{with DataLad@\spxentry{with DataLad}!set description for dataset location@\spxentry{set description for dataset location}}\ignorespaces \begin{findoutmore}[label={fom-clone-description}, before title={\thetcbcounter\ }, float, floatplacement=tb, check odd page=true]{What is this location, and what if I provided a description?}
\label{\detokenize{basics/101-117-sharelocal2:fom-clone-description}}

\sphinxAtStartPar
Back in the very first section of the Basics, {\hyperref[\detokenize{basics/101-101-create:createds}]{\sphinxcrossref{\DUrole{std,std-ref}{Create a dataset}}}} (\autopageref*{\detokenize{basics/101-101-create:createds}}), a {\hyperref[\detokenize{basics/101-101-create:createdescription}]{\sphinxcrossref{\DUrole{std,std-ref}{Find\sphinxhyphen{}out\sphinxhyphen{}more mentioned the ‘\textendash{}description’ option}}}} (\autopageref*{\detokenize{basics/101-101-create:createdescription}})   of \sphinxcode{\sphinxupquote{datalad create}}.
With this option, you can provide a description about the dataset \sphinxstyleemphasis{location}.

\sphinxAtStartPar
The \sphinxcode{\sphinxupquote{git annex whereis}} command, finally, is where such a description
can become handy: If you had created the dataset with

\begin{sphinxVerbatim}[commandchars=\\\{\}]
\PYG{g+gp}{\PYGZdl{} }datalad\PYG{+w}{ }create\PYG{+w}{ }\PYGZhy{}\PYGZhy{}description\PYG{+w}{ }\PYG{l+s+s2}{\PYGZdq{}course on DataLad\PYGZhy{}101 on my private laptop\PYGZdq{}}\PYG{+w}{ }\PYGZhy{}c\PYG{+w}{ }text2git\PYG{+w}{ }DataLad\PYGZhy{}101
\end{sphinxVerbatim}

\sphinxAtStartPar
the command would show \sphinxcode{\sphinxupquote{course on DataLad\sphinxhyphen{}101 on my private laptop}} after
the {\hyperref[\detokenize{glossary:term-UUID}]{\sphinxtermref{\DUrole{xref,std,std-term}{UUID}}}} \textendash{} and thus a more human\sphinxhyphen{}readable description of \sphinxstyleemphasis{where}
file content is stored.
This becomes especially useful when the number of repository copies
increases. If you have only one other dataset it may be easy to
remember what and where it is. But once you have one back\sphinxhyphen{}up
of your dataset on a USB stick, one dataset shared with
Dropbox, and a third one on your institutions
{\hyperref[\detokenize{glossary:term-GitLab}]{\sphinxtermref{\DUrole{xref,std,std-term}{GitLab}}}} instance you will be grateful for the descriptions
you provided these locations with.

\sphinxAtStartPar
The current report of the location of the dataset is in the format
\sphinxcode{\sphinxupquote{user@host:path}}.

\sphinxAtStartPar
If the physical location of a dataset is not relevant, ambiguous, or volatile,
or if it has an {\hyperref[\detokenize{glossary:term-annex}]{\sphinxtermref{\DUrole{xref,std,std-term}{annex}}}} that could move within the foreseeable lifetime of a
dataset, a custom description with the relevant information on the dataset is
superior. If this is not the case, decide for yourself whether you want to use
the \sphinxcode{\sphinxupquote{\sphinxhyphen{}\sphinxhyphen{}description}} option for future datasets or not depending on what you
find more readable \textendash{} a self\sphinxhyphen{}made location description, or an automatic
\sphinxcode{\sphinxupquote{user@host:path}} information.


\end{findoutmore}

\sphinxAtStartPar
The message further informs you that there is only “\sphinxcode{\sphinxupquote{(1 copy)}}” of this file content.
This makes sense: There is only your own, original \sphinxcode{\sphinxupquote{DataLad\sphinxhyphen{}101}} dataset in which this book is saved.

\sphinxAtStartPar
To retrieve file content of an annexed file such as one of these PDFs, git\sphinxhyphen{}annex will try to obtain it from the locations it knows to contain this content.
It uses the UUID to identify these locations.
Every copy of a dataset will get a UUID as a unique identifier.
Note however that just because git\sphinxhyphen{}annex knows a certain location where content was once it does not guarantee that retrieval will work.
If one location is a USB stick that is in your bag pack instead of your USB port, a second location is a hard drive that you deleted all of its previous contents (including dataset content) from,
and another location is a web server, but you are not connected to the internet, git\sphinxhyphen{}annex will not succeed in retrieving contents from these locations.
As long as there is at least one location that contains the file and is accessible, though, git\sphinxhyphen{}annex will get the content.
Therefore, for the books in your dataset, retrieving contents works because you and your room mate share the same file system.
If you’d share the dataset with anyone without access to your file system, \sphinxcode{\sphinxupquote{datalad get}} would not work, because it cannot access your files.

\sphinxAtStartPar
But there is one book that does not suffer from this restriction:
The \sphinxcode{\sphinxupquote{bash\_guide.pdf}}.
This book was not manually downloaded and saved to the dataset with \sphinxcode{\sphinxupquote{wget}} (thus keeping DataLad in the dark about where it came from), but it was obtained with the \sphinxcode{\sphinxupquote{datalad download\sphinxhyphen{}url}} command.
This registered the books original source in the dataset, and here is why that is useful:

\begin{sphinxVerbatim}[commandchars=\\\{\}]
\PYG{g+gp}{\PYGZdl{} }git\PYG{+w}{ }annex\PYG{+w}{ }whereis\PYG{+w}{ }books/bash\PYGZus{}guide.pdf
\PYG{g+go}{whereis books/bash\PYGZus{}guide.pdf (2 copies)}
\PYG{g+go}{  00000000\PYGZhy{}0000\PYGZhy{}0000\PYGZhy{}0000\PYGZhy{}000000000001 \PYGZhy{}\PYGZhy{} web}
\PYG{g+go}{  0c450dc0\PYGZhy{}c48c\PYGZhy{}4057\PYGZhy{}a231\PYGZhy{}e2654b689600 \PYGZhy{}\PYGZhy{} me@appveyor\PYGZhy{}vm:\PYGZti{}/dl\PYGZhy{}101/DataLad\PYGZhy{}101 [origin]}

\PYG{g+go}{  web: https://www.tldp.org/LDP/Bash\PYGZhy{}Beginners\PYGZhy{}Guide/Bash\PYGZhy{}Beginners\PYGZhy{}Guide.pdf}
\PYG{g+go}{ok}
\end{sphinxVerbatim}

\sphinxAtStartPar
Unlike the \sphinxcode{\sphinxupquote{TLCL.pdf}} book, this book has two sources, and one of them is \sphinxcode{\sphinxupquote{web}}.
The second to last line specifies the precise URL you downloaded the file from.
Thus, for this book, your room mate is always able to obtain it (as long as the URL remains valid), even if you would delete your \sphinxcode{\sphinxupquote{DataLad\sphinxhyphen{}101}} dataset.

\sphinxAtStartPar
We can also see a report of the source that git\sphinxhyphen{}annex uses to retrieve the content from if we look at the very end of the \sphinxcode{\sphinxupquote{get}} summary.

\begin{sphinxVerbatim}[commandchars=\\\{\}]
\PYG{g+gp}{\PYGZdl{} }datalad\PYG{+w}{ }get\PYG{+w}{ }books/TLCL.pdf
\PYG{g+gp}{\PYGZdl{} }datalad\PYG{+w}{ }get\PYG{+w}{ }books/bash\PYGZus{}guide.pdf
\PYG{g+go}{get(ok): books/TLCL.pdf (file) [from origin...]}
\PYG{g+go}{get(ok): books/bash\PYGZus{}guide.pdf (file) [from origin...]}
\end{sphinxVerbatim}

\sphinxAtStartPar
Both of these files were retrieved “\sphinxcode{\sphinxupquote{from origin...}}”.
\sphinxcode{\sphinxupquote{Origin}} is Git terminology for “from where the dataset was copied from” \textendash{} \sphinxcode{\sphinxupquote{origin}} therefore is the original \sphinxcode{\sphinxupquote{DataLad\sphinxhyphen{}101}} dataset from which file content can be retrieved from very fast.

\sphinxAtStartPar
If your room mate did not have access to the same file system or you deleted your \sphinxcode{\sphinxupquote{DataLad\sphinxhyphen{}101}} dataset, this output would look differently.
The \sphinxcode{\sphinxupquote{datalad get}} command would fail on the \sphinxcode{\sphinxupquote{TLCL.pdf}} book without a known second source, and \sphinxcode{\sphinxupquote{bash\_guide.pdf}} would be retrieved “\sphinxcode{\sphinxupquote{from web...}}” \sphinxhyphen{} the registered second source, its original download URL.
Let’s see a retrieval from \sphinxcode{\sphinxupquote{web}} in action for another file.
The \sphinxcode{\sphinxupquote{.mp3}} files in the \sphinxcode{\sphinxupquote{longnow}} seminar series have registered web URLs%
\begin{footnote}\sphinxAtStartFootnote
Maybe you wonder what the location \sphinxcode{\sphinxupquote{mih@medusa}} is. It is a copy of the
data on an account belonging to user \sphinxcode{\sphinxupquote{mih}} on the host name \sphinxcode{\sphinxupquote{medusa}}.
Because we do not have the host names’ address, nor log\sphinxhyphen{}in credentials for
this user, we cannot retrieve content from this location. However, somebody
else (for example, the user \sphinxcode{\sphinxupquote{mih}}) could.
%
\end{footnote}.

\begin{sphinxVerbatim}[commandchars=\\\{\}]
\PYG{g+gp}{\PYGZdl{} }\PYG{c+c1}{\PYGZsh{} navigate into the subdirectory}
\PYG{g+gp}{\PYGZdl{} }\PYG{n+nb}{cd}\PYG{+w}{ }recordings/longnow
\PYG{g+gp}{\PYGZdl{} }git\PYG{+w}{ }annex\PYG{+w}{ }whereis\PYG{+w}{ }Long\PYGZus{}Now\PYGZus{}\PYGZus{}Seminars\PYGZus{}About\PYGZus{}Long\PYGZus{}term\PYGZus{}Thinking/2003\PYGZus{}11\PYGZus{}15\PYGZus{}\PYGZus{}Brian\PYGZus{}Eno\PYGZus{}\PYGZus{}The\PYGZus{}Long\PYGZus{}Now.mp3
\PYG{g+gp}{\PYGZdl{} }datalad\PYG{+w}{ }get\PYG{+w}{ }Long\PYGZus{}Now\PYGZus{}\PYGZus{}Seminars\PYGZus{}About\PYGZus{}Long\PYGZus{}term\PYGZus{}Thinking/2003\PYGZus{}11\PYGZus{}15\PYGZus{}\PYGZus{}Brian\PYGZus{}Eno\PYGZus{}\PYGZus{}The\PYGZus{}Long\PYGZus{}Now.mp3
\PYG{g+go}{whereis Long\PYGZus{}Now\PYGZus{}\PYGZus{}Seminars\PYGZus{}About\PYGZus{}Long\PYGZus{}term\PYGZus{}Thinking/2003\PYGZus{}11\PYGZus{}15\PYGZus{}\PYGZus{}Brian\PYGZus{}Eno\PYGZus{}\PYGZus{}The\PYGZus{}Long\PYGZus{}Now.mp3 (2 copies)}
\PYG{g+go}{	00000000\PYGZhy{}0000\PYGZhy{}0000\PYGZhy{}0000\PYGZhy{}000000000001 \PYGZhy{}\PYGZhy{} web}
\PYG{g+go}{	✂UUID✂ \PYGZhy{}\PYGZhy{} mih@medusa:/tmp/seminars\PYGZhy{}on\PYGZhy{}longterm\PYGZhy{}thinking}

\PYG{g+go}{  web: http://podcast.longnow.org/salt/redirect/salt\PYGZhy{}020031114\PYGZhy{}eno\PYGZhy{}podcast.mp3}
\PYG{g+go}{ok}
\PYG{g+go}{get(ok): Long\PYGZus{}Now\PYGZus{}\PYGZus{}Seminars\PYGZus{}About\PYGZus{}Long\PYGZus{}term\PYGZus{}Thinking/2003\PYGZus{}11\PYGZus{}15\PYGZus{}\PYGZus{}Brian\PYGZus{}Eno\PYGZus{}\PYGZus{}The\PYGZus{}Long\PYGZus{}Now.mp3 (file) [from web...]}
\end{sphinxVerbatim}

\sphinxAtStartPar
As you can see at the end of the \sphinxcode{\sphinxupquote{get}} result, the files has been retrieved “\sphinxcode{\sphinxupquote{from web...}}”.
Quite useful, this provenance, right?
Let’s add a note on the \sphinxcode{\sphinxupquote{git annex whereis}} command.
Again, do this in the original \sphinxcode{\sphinxupquote{DataLad\sphinxhyphen{}101}} directory, and do not forget to save it.

\begin{sphinxVerbatim}[commandchars=\\\{\}]
\PYG{g+gp}{\PYGZdl{} }\PYG{c+c1}{\PYGZsh{} navigate back:}
\PYG{g+gp}{\PYGZdl{} }\PYG{n+nb}{cd}\PYG{+w}{ }../../../../DataLad\PYGZhy{}101

\PYG{g+gp}{\PYGZdl{} }\PYG{c+c1}{\PYGZsh{} write the note}
\PYG{g+gp}{\PYGZdl{} }cat\PYG{+w}{ }\PYGZlt{}\PYGZlt{}\PYG{+w}{ }EOT\PYG{+w}{ }\PYGZgt{}\PYGZgt{}\PYG{+w}{ }notes.txt
\PYG{g+go}{The command \PYGZdq{}git annex whereis PATH\PYGZdq{} lists the repositories that have}
\PYG{g+go}{the file content of an annexed file. When using \PYGZdq{}datalad get\PYGZdq{} to}
\PYG{g+go}{retrieve file content, those repositories will be queried.}

\PYG{g+go}{EOT}
\end{sphinxVerbatim}

\begin{sphinxVerbatim}[commandchars=\\\{\}]
\PYG{g+gp}{\PYGZdl{} }datalad\PYG{+w}{ }status
\PYG{g+go}{ modified: notes.txt (file)}
\end{sphinxVerbatim}

\begin{sphinxVerbatim}[commandchars=\\\{\}]
\PYG{g+gp}{\PYGZdl{} }datalad\PYG{+w}{ }save\PYG{+w}{ }\PYGZhy{}m\PYG{+w}{ }\PYG{l+s+s2}{\PYGZdq{}add note on git annex whereis\PYGZdq{}}
\PYG{g+go}{add(ok): notes.txt (file)}
\PYG{g+go}{save(ok): . (dataset)}
\end{sphinxVerbatim}

\sphinxstepscope

\index{reproducibility@\spxentry{reproducibility}!with DataLad@\spxentry{with DataLad}}\index{with DataLad@\spxentry{with DataLad}!reproducibility@\spxentry{reproducibility}}\ignorespaces 

\section{Retrace and reenact}
\label{\detokenize{basics/101-118-sharelocal3:retrace-and-reenact}}\label{\detokenize{basics/101-118-sharelocal3:sharelocal3}}\label{\detokenize{basics/101-118-sharelocal3:index-0}}\label{\detokenize{basics/101-118-sharelocal3::doc}}
\sphinxAtStartPar
“Thanks a lot for sharing your dataset with me! This
is super helpful. I’m sure I’ll catch up in no time!”,
your room mate says confidently. “How far did you get
with the DataLad commands yet?” he asks at last.

\sphinxAtStartPar
“Mhh, I think the last big one was \sphinxcode{\sphinxupquote{datalad run}}.
Actually, let me quickly show you what this command
does. There is something that I’ve been wanting to try
anyway.” you say.

\sphinxAtStartPar
The dataset you shared contained a number of \sphinxcode{\sphinxupquote{datalad run}}
commands. For example, you created the simple \sphinxcode{\sphinxupquote{podcasts.tsv}}
file that listed all titles and speaker names of the longnow
podcasts.

\sphinxAtStartPar
Given that you learned to create “proper” \sphinxcode{\sphinxupquote{datalad run}} commands,
complete with \sphinxcode{\sphinxupquote{\sphinxhyphen{}\sphinxhyphen{}input}} and \sphinxcode{\sphinxupquote{\sphinxhyphen{}\sphinxhyphen{}output}} specification,
anyone should be able to \sphinxcode{\sphinxupquote{datalad rerun}} these commits
easily. This is what you want to try now.

\sphinxAtStartPar
You begin to think about which \sphinxcode{\sphinxupquote{datalad run}} commit would be
the most useful one to take a look at. The creation of
\sphinxcode{\sphinxupquote{podcasts.tsv}} was a bit dull \textendash{} at this point in time, you
didn’t yet know about \sphinxcode{\sphinxupquote{\sphinxhyphen{}\sphinxhyphen{}input}} and \sphinxcode{\sphinxupquote{\sphinxhyphen{}\sphinxhyphen{}output}} arguments,
and the resulting output is present anyway because text files
like this \sphinxcode{\sphinxupquote{.tsv}} file are stored in Git.
However, one of the attempts to resize a picture could be
useful. The input, the podcast logos, is not yet retrieved,
nor is the resulting, resized image. “Let’s go for this!”,
you say, and drag your confused room mate to the computer
screen.

\sphinxAtStartPar
First of all, find the commit shasum of the command you
want to run by taking a look into the history of the dataset
(in the shared dataset):

\begin{sphinxVerbatim}[commandchars=\\\{\}]
\PYG{g+gp}{\PYGZdl{} }\PYG{c+c1}{\PYGZsh{} navigate into the shared copy}
\PYG{g+gp}{\PYGZdl{} }\PYG{n+nb}{cd}\PYG{+w}{ }../mock\PYGZus{}user/DataLad\PYGZhy{}101
\end{sphinxVerbatim}

\fvset{hllines={, 4,}}%
\begin{sphinxVerbatim}[commandchars=\\\{\}]
\PYG{g+gp}{\PYGZdl{} }\PYG{c+c1}{\PYGZsh{} lets view the history}
\PYG{g+gp}{\PYGZdl{} }git\PYG{+w}{ }log\PYG{+w}{ }\PYGZhy{}\PYGZhy{}oneline\PYG{+w}{ }\PYGZhy{}n\PYG{+w}{ }\PYG{l+m}{10}
\PYG{g+go}{41fdaab add note on clean datasets}
\PYG{g+go}{fd4a25e [DATALAD RUNCMD] Resize logo for slides}
\PYG{g+go}{08f2efb [DATALAD RUNCMD] Resize logo for slides}
\PYG{g+go}{87b4f80 add additional notes on run options}
\PYG{g+go}{d0c060f [DATALAD RUNCMD] convert \PYGZhy{}resize 450x450 recordings/longn...}
\PYG{g+go}{5227d93 resized picture by hand}
\PYG{g+go}{6686234 [DATALAD RUNCMD] convert \PYGZhy{}resize 400x400 recordings/longn...}
\PYG{g+go}{4b7a0a0 add note on basic datalad run and datalad rerun}
\PYG{g+go}{cdedbc3 add note datalad and git diff}
\PYG{g+go}{08120c3 [DATALAD RUNCMD] create a list of podcast titles}
\end{sphinxVerbatim}
\sphinxresetverbatimhllines

\sphinxAtStartPar
Ah, there it is, the second most recent commit.
Just as already done in section {\hyperref[\detokenize{basics/101-109-rerun:run2}]{\sphinxcrossref{\DUrole{std,std-ref}{DataLad, rerun!}}}} (\autopageref*{\detokenize{basics/101-109-rerun:run2}}),
take this shasum and plug it into a \sphinxcode{\sphinxupquote{datalad rerun}}
command:

\begin{sphinxVerbatim}[commandchars=\\\{\}]
\PYG{g+gp}{\PYGZdl{} }datalad\PYG{+w}{ }rerun\PYG{+w}{ }fd4a25e6✂SHA1
\PYG{g+go}{[INFO] run commit fd4a25e; (Resize logo for s...)}
\PYG{g+go}{get(ok): recordings/longnow/.datalad/feed\PYGZus{}metadata/logo\PYGZus{}salt.jpg (file) [from web...]}
\PYG{g+go}{run.remove(ok): recordings/salt\PYGZus{}logo\PYGZus{}small.jpg (file) [Removed file]}
\PYG{g+go}{[INFO] == Command start (output follows) =====}
\PYG{g+go}{[INFO] == Command exit (modification check follows) =====}
\PYG{g+go}{run(ok): /home/me/dl\PYGZhy{}101/mock\PYGZus{}user/DataLad\PYGZhy{}101 (dataset) [convert \PYGZhy{}resize 400x400 recordings/longn...]}
\PYG{g+go}{add(ok): recordings/salt\PYGZus{}logo\PYGZus{}small.jpg (file)}
\PYG{g+go}{action summary:}
\PYG{g+go}{  add (ok: 1)}
\PYG{g+go}{  get (notneeded: 1, ok: 1)}
\PYG{g+go}{  run (ok: 1)}
\PYG{g+go}{  run.remove (ok: 1)}
\PYG{g+go}{  save (notneeded: 2)}
\end{sphinxVerbatim}

\sphinxAtStartPar
“This was so easy!” you exclaim. DataLad retrieved the missing
file content from the subdataset and it tried to unlock the output
prior to the command execution. Note that because you did not retrieve
the output, \sphinxcode{\sphinxupquote{recordings/salt\_logo\_small.jpg}}, yet, the missing content
could not be “unlocked”, but is reportedly “removed” prior to the successful rerun.

\sphinxAtStartPar
Your room mate now not only knows how exactly the resized file
came into existence, but he can also reproduce your exact steps to
create it. “This is as reproducible as it can be!” you think in awe.

\sphinxstepscope

\index{update dataset@\spxentry{update dataset}!with DataLad@\spxentry{with DataLad}}\index{with DataLad@\spxentry{with DataLad}!update dataset@\spxentry{update dataset}}\ignorespaces 

\section{Stay up to date}
\label{\detokenize{basics/101-119-sharelocal4:stay-up-to-date}}\label{\detokenize{basics/101-119-sharelocal4:update}}\label{\detokenize{basics/101-119-sharelocal4:sharelocal4}}\label{\detokenize{basics/101-119-sharelocal4:index-0}}\label{\detokenize{basics/101-119-sharelocal4::doc}}
\sphinxAtStartPar
All of what you have seen about sharing dataset was really
cool, and for the most part also surprisingly intuitive.
\sphinxcode{\sphinxupquote{datalad run}} commands or file retrieval worked exactly as
you imagined it to work, and you begin to think that slowly but
steadily you are getting a feel about how DataLad really works.

\sphinxAtStartPar
But to be honest, so far, sharing the dataset with DataLad was
also remarkably unexciting given that you already knew most of
the dataset magic that your room mate currently is still
mesmerized about.
To be honest, you are not yet certain whether
sharing data with DataLad really improves your life up
until this point. After all, you could have just copied
your directory into your \sphinxcode{\sphinxupquote{mock\_user}} directory and
this would have resulted in about the same output, right?

\sphinxAtStartPar
What we will be looking into now is how shared DataLad
datasets can be updated.

\sphinxAtStartPar
Remember that you added some notes on \sphinxcode{\sphinxupquote{datalad clone}},
\sphinxcode{\sphinxupquote{datalad get}}, and \sphinxcode{\sphinxupquote{git annex whereis}} into the
original \sphinxcode{\sphinxupquote{DataLad\sphinxhyphen{}101}}?

\sphinxAtStartPar
This is a change that is not reflected in your “shared”
installation in \sphinxcode{\sphinxupquote{../mock\_user/DataLad\sphinxhyphen{}101}}:

\begin{sphinxVerbatim}[commandchars=\\\{\}]
\PYGZdl{} \PYGZsh{} Inside the installed copy, view the last 15 lines of notes.txt
\PYGZdl{} tail notes.txt
should be specified with an \PYGZhy{}o/\PYGZhy{}\PYGZhy{}output flag. Upon a run or rerun of
the command, the contents of these files will get unlocked so that
they can be modified.

Important! If the dataset is not \PYGZdq{}clean\PYGZdq{} (a datalad status output is
empty), datalad run will not work \PYGZhy{} you will have to save
modifications present in your dataset.
A suboptimal alternative is the \PYGZhy{}\PYGZhy{}explicit flag, used to record only
those changes done to the files listed with \PYGZhy{}\PYGZhy{}output flags.

\end{sphinxVerbatim}

\sphinxAtStartPar
But the original intention of sharing the dataset with
your room mate was to give him access to your notes.
How does he get the notes that you have added in the last
two sections, for example?

\sphinxAtStartPar
This installed copy of \sphinxcode{\sphinxupquote{DataLad\sphinxhyphen{}101}} knows its \sphinxcode{\sphinxupquote{origin}}, i.e.,
the place it was installed from. Using this information,
it can query the original dataset whether any changes
happened since the last time it checked, and if so, retrieve and
integrate them.

\index{update@\spxentry{update}!DataLad command@\spxentry{DataLad command}}\index{DataLad command@\spxentry{DataLad command}!update@\spxentry{update}}\index{update dataset with remote change@\spxentry{update dataset with remote change}!with DataLad@\spxentry{with DataLad}}\index{with DataLad@\spxentry{with DataLad}!update dataset with remote change@\spxentry{update dataset with remote change}}\ignorespaces 
\sphinxAtStartPar
This is done with the \sphinxcode{\sphinxupquote{datalad update \sphinxhyphen{}\sphinxhyphen{}how merge}}
command.

\begin{sphinxVerbatim}[commandchars=\\\{\}]
\PYG{g+gp}{\PYGZdl{} }datalad\PYG{+w}{ }update\PYG{+w}{ }\PYGZhy{}\PYGZhy{}how\PYG{+w}{ }merge
\PYG{g+go}{merge(ok): . (dataset) [Merged origin/main]}
\PYG{g+go}{update.annex\PYGZus{}merge(ok): . (dataset) [Merged annex branch]}
\PYG{g+go}{update(ok): . (dataset)}
\end{sphinxVerbatim}

\sphinxAtStartPar
Importantly, run this command either within the specific
(sub)dataset you are interested in, or provide a path to
the root of the dataset you are interested in with the
\sphinxcode{\sphinxupquote{\sphinxhyphen{}d}}/\sphinxcode{\sphinxupquote{\sphinxhyphen{}\sphinxhyphen{}dataset}} flag. If you would run the command
within the \sphinxcode{\sphinxupquote{longnow}} subdataset, you would query this
subdatasets’ \sphinxcode{\sphinxupquote{origin}} for updates, not the original
\sphinxcode{\sphinxupquote{DataLad\sphinxhyphen{}101}} dataset.

\sphinxAtStartPar
Let’s check the contents in \sphinxcode{\sphinxupquote{notes.txt}} to see whether
the previously missing changes are now present:

\begin{sphinxVerbatim}[commandchars=\\\{\}]
\PYG{g+gp}{\PYGZdl{} }\PYG{c+c1}{\PYGZsh{} view the last 15 lines of notes.txt}
\PYG{g+gp}{\PYGZdl{} }tail\PYG{+w}{ }notes.txt

\PYG{g+go}{Note that a recursive \PYGZdq{}datalad get\PYGZdq{} would install all further}
\PYG{g+go}{registered subdatasets underneath a subdataset, so a safer way to}
\PYG{g+go}{proceed is to set a decent \PYGZhy{}\PYGZhy{}recursion\PYGZhy{}limit:}
\PYG{g+go}{\PYGZdq{}datalad get \PYGZhy{}n \PYGZhy{}r \PYGZhy{}\PYGZhy{}recursion\PYGZhy{}limit 2 \PYGZlt{}subds\PYGZgt{}\PYGZdq{}}

\PYG{g+go}{The command \PYGZdq{}git annex whereis PATH\PYGZdq{} lists the repositories that have}
\PYG{g+go}{the file content of an annexed file. When using \PYGZdq{}datalad get\PYGZdq{} to}
\PYG{g+go}{retrieve file content, those repositories will be queried.}
\end{sphinxVerbatim}

\sphinxAtStartPar
Wohoo, the contents are here!

\sphinxAtStartPar
Therefore, sharing DataLad datasets by installing them
enables you to update the datasets content should the
original datasets’ content change \textendash{} in only a single
command. How cool is that?!

\sphinxAtStartPar
Conclude this section by adding a note about updating a
dataset to your own \sphinxcode{\sphinxupquote{DataLad\sphinxhyphen{}101}} dataset:

\begin{sphinxVerbatim}[commandchars=\\\{\}]
\PYG{g+gp}{\PYGZdl{} }\PYG{c+c1}{\PYGZsh{} navigate back:}
\PYG{g+gp}{\PYGZdl{} }\PYG{n+nb}{cd}\PYG{+w}{ }../../DataLad\PYGZhy{}101

\PYG{g+gp}{\PYGZdl{} }\PYG{c+c1}{\PYGZsh{} write the note}
\PYG{g+gp}{\PYGZdl{} }cat\PYG{+w}{ }\PYGZlt{}\PYGZlt{}\PYG{+w}{ }EOT\PYG{+w}{ }\PYGZgt{}\PYGZgt{}\PYG{+w}{ }notes.txt
\PYG{g+go}{To update a shared dataset, run the command \PYGZdq{}datalad update \PYGZhy{}\PYGZhy{}how merge\PYGZdq{}.}
\PYG{g+go}{This command will query its origin for changes, and integrate the}
\PYG{g+go}{changes into the dataset.}

\PYG{g+go}{EOT}
\end{sphinxVerbatim}

\begin{sphinxVerbatim}[commandchars=\\\{\}]
\PYG{g+gp}{\PYGZdl{} }\PYG{c+c1}{\PYGZsh{} save the changes}
\PYG{g+gp}{\PYGZdl{} }datalad\PYG{+w}{ }save\PYG{+w}{ }\PYGZhy{}m\PYG{+w}{ }\PYG{l+s+s2}{\PYGZdq{}add note about datalad update\PYGZdq{}}
\PYG{g+go}{add(ok): notes.txt (file)}
\PYG{g+go}{save(ok): . (dataset)}
\end{sphinxVerbatim}

\sphinxAtStartPar
PS: You might wonder what a plain \sphinxcode{\sphinxupquote{datalad update}} command with no option does.
If you are a Git\sphinxhyphen{}user and know about branches and merging you can read the
\sphinxcode{\sphinxupquote{Note for Git\sphinxhyphen{}users}}. However, a thorough explanation
and demonstration will be in the next section.

\index{update@\spxentry{update}!DataLad concept@\spxentry{DataLad concept}}\index{DataLad concept@\spxentry{DataLad concept}!update@\spxentry{update}}\ignorespaces \begin{gitusernote}[label={index-2}, before title={\thetcbcounter\ }, check odd page=true]{Update internals}
\label{\detokenize{basics/101-119-sharelocal4:index-2}}

\sphinxAtStartPar
\sphinxcode{\sphinxupquote{datalad update}} is the DataLad equivalent of a \sphinxcode{\sphinxupquote{git fetch}},
\sphinxcode{\sphinxupquote{datalad update \sphinxhyphen{}\sphinxhyphen{}how merge}} is the DataLad equivalent of a
\sphinxcode{\sphinxupquote{git pull}}.
Upon a simple \sphinxcode{\sphinxupquote{datalad update}}, the remote information
is available on a branch separate from the main branch
\textendash{} in most cases this will be \sphinxcode{\sphinxupquote{remotes/origin/main}}.
You can \sphinxcode{\sphinxupquote{git checkout}} this branch or run \sphinxcode{\sphinxupquote{git diff}} to
explore the changes and identify potential merge conflicts.


\end{gitusernote}

\sphinxstepscope

\index{update dataset@\spxentry{update dataset}!with DataLad@\spxentry{with DataLad}}\index{with DataLad@\spxentry{with DataLad}!update dataset@\spxentry{update dataset}}\ignorespaces 

\section{Networking}
\label{\detokenize{basics/101-121-siblings:networking}}\label{\detokenize{basics/101-121-siblings:sibling}}\label{\detokenize{basics/101-121-siblings:index-0}}\label{\detokenize{basics/101-121-siblings::doc}}
\sphinxAtStartPar
To get a hang on the basics of sharing a dataset,
you shared your \sphinxcode{\sphinxupquote{DataLad\sphinxhyphen{}101}} dataset with your
room mate on a common, local file system. Your lucky
room mate now has your notes and can thus try to catch
up to still pass the course.
Moreover, though, he can also integrate all other notes
or changes you make to your dataset, and stay up to date.
This is because a DataLad dataset makes updating shared
data a matter of a single \sphinxcode{\sphinxupquote{datalad update \sphinxhyphen{}\sphinxhyphen{}how merge}} command.

\sphinxAtStartPar
But why does this need to be a one\sphinxhyphen{}way street? “I want to
provide helpful information for you as well!”, says your
room mate. “How could you get any insightful notes that
I make in my dataset, or maybe the results of our upcoming
mid\sphinxhyphen{}term project? Its a bit unfair that I can get your work,
but you cannot get mine.”

\index{register file with URL in dataset@\spxentry{register file with URL in dataset}!with DataLad@\spxentry{with DataLad}}\index{with DataLad@\spxentry{with DataLad}!register file with URL in dataset@\spxentry{register file with URL in dataset}}\ignorespaces 
\sphinxAtStartPar
Consider, for example, that your room mate might have googled about DataLad
a bit. In the depths of the web, he might have found useful additional information, such
a script on \dlhbhref{G15A}{dataset nesting}.
Because he found this very helpful in understanding dataset
nesting concepts, he decided to download it from GitHub, and saved it in the \sphinxcode{\sphinxupquote{code/}} directory.

\sphinxAtStartPar
He does it using the DataLad command \sphinxcode{\sphinxupquote{datalad download\sphinxhyphen{}url}}
that you experienced in section {\hyperref[\detokenize{basics/101-101-create:createds}]{\sphinxcrossref{\DUrole{std,std-ref}{Create a dataset}}}} (\autopageref*{\detokenize{basics/101-101-create:createds}}) already: This command will
download a file just as \sphinxcode{\sphinxupquote{wget}}, but it can also take a commit message
and will save the download right to the history of the dataset that you specify,
while recording its origin as provenance information.

\sphinxAtStartPar
Navigate into your dataset copy in \sphinxcode{\sphinxupquote{mock\_user/DataLad\sphinxhyphen{}101}},
and run the following command

\begin{sphinxVerbatim}[commandchars=\\\{\}]
\PYG{g+gp}{\PYGZdl{} }\PYG{c+c1}{\PYGZsh{} navigate into the installed copy}
\PYG{g+gp}{\PYGZdl{} }\PYG{n+nb}{cd}\PYG{+w}{ }../mock\PYGZus{}user/DataLad\PYGZhy{}101

\PYG{g+gp}{\PYGZdl{} }\PYG{c+c1}{\PYGZsh{} download the shell script and save it in your code/ directory}
\PYG{g+gp}{\PYGZdl{} }datalad\PYG{+w}{ }download\PYGZhy{}url\PYG{+w}{ }\PYG{l+s+se}{\PYGZbs{}}
\PYG{+w}{  }\PYGZhy{}d\PYG{+w}{ }.\PYG{+w}{ }\PYG{l+s+se}{\PYGZbs{}}
\PYG{+w}{  }\PYGZhy{}m\PYG{+w}{ }\PYG{l+s+s2}{\PYGZdq{}Include nesting demo from datalad website\PYGZdq{}}\PYG{+w}{ }\PYG{l+s+se}{\PYGZbs{}}
\PYG{+w}{  }\PYGZhy{}O\PYG{+w}{ }code/nested\PYGZus{}repos.sh\PYG{+w}{ }\PYG{l+s+se}{\PYGZbs{}}
\PYG{+w}{  }https://raw.githubusercontent.com/datalad/datalad.org/7e8e39b1/content/asciicast/seamless\PYGZus{}nested\PYGZus{}repos.sh
\PYG{g+go}{download\PYGZus{}url(ok): /home/me/dl\PYGZhy{}101/mock\PYGZus{}user/DataLad\PYGZhy{}101/code/nested\PYGZus{}repos.sh (file)}
\PYG{g+go}{add(ok): code/nested\PYGZus{}repos.sh (file)}
\PYG{g+go}{save(ok): . (dataset)}
\end{sphinxVerbatim}

\sphinxAtStartPar
Run a quick \sphinxcode{\sphinxupquote{datalad status}}:

\begin{sphinxVerbatim}[commandchars=\\\{\}]
\PYG{g+gp}{\PYGZdl{} }datalad\PYG{+w}{ }status
\PYG{g+go}{nothing to save, working tree clean}
\end{sphinxVerbatim}

\sphinxAtStartPar
Nice, the \sphinxcode{\sphinxupquote{datalad download\sphinxhyphen{}url}} command saved this download
right into the history, and \sphinxcode{\sphinxupquote{datalad status}} does not report
unsaved modifications! We’ll show an excerpt of the last commit
here%
\begin{footnote}\sphinxAtStartFootnote
As this example, simplistically, created a “pretend” room mate by only changing directories, not user accounts, the recorded Git identity of your “room mate” will, of course, be the same as yours.
%
\end{footnote}:

\begin{sphinxVerbatim}[commandchars=\\\{\}]
\PYG{g+gp}{\PYGZdl{} }git\PYG{+w}{ }log\PYG{+w}{ }\PYGZhy{}n\PYG{+w}{ }\PYG{l+m}{1}\PYG{+w}{ }\PYGZhy{}p
\PYG{g+go}{commit 5b6e19a5✂SHA1}
\PYG{g+go}{Author: Elena Piscopia \PYGZlt{}elena@example.net\PYGZgt{}}
\PYG{g+go}{Date:   Tue Jun 18 16:13:00 2019 +0000}

\PYG{g+go}{    Include nesting demo from datalad website}

\PYG{g+go}{diff \PYGZhy{}\PYGZhy{}git a/code/nested\PYGZus{}repos.sh b/code/nested\PYGZus{}repos.sh}
\PYG{g+go}{new file mode 100644}
\PYG{g+go}{index 0000000..f84c817}
\PYG{g+go}{\PYGZhy{}\PYGZhy{}\PYGZhy{} /dev/null}
\PYG{g+go}{+++ b/code/nested\PYGZus{}repos.sh}
\PYG{g+go}{@@ \PYGZhy{}0,0 +1,59 @@}
\end{sphinxVerbatim}

\sphinxAtStartPar
Suddenly, your room mate has a file change that you do not have.
His dataset evolved.

\sphinxAtStartPar
So how do we link back from the copy of the dataset to its
origin, such that your room mate’s changes can be included in
your dataset? How do we let the original dataset “know” about
this copy your room mate has?
Do we need to install the installed dataset of our room mate
as a copy again?

\sphinxAtStartPar
No, luckily, it’s simpler and less convoluted. What we have to
do is to \sphinxstyleemphasis{register} a DataLad {\hyperref[\detokenize{glossary:term-sibling}]{\sphinxtermref{\DUrole{xref,std,std-term}{sibling}}}}: A reference to our room mate’s
dataset in our own, original dataset.

\index{sibling@\spxentry{sibling}!DataLad concept@\spxentry{DataLad concept}}\index{DataLad concept@\spxentry{DataLad concept}!sibling@\spxentry{sibling}}\ignorespaces \begin{gitusernote}[label={index-2}, before title={\thetcbcounter\ }, check odd page=true]{Remote siblings}
\label{\detokenize{basics/101-121-siblings:index-2}}

\sphinxAtStartPar
Git repositories can configure clones of a dataset as \sphinxstyleemphasis{remotes} in
order to fetch, pull, or push from and to them. A \sphinxcode{\sphinxupquote{datalad sibling}}
is the equivalent of a git clone that is configured as a remote.


\end{gitusernote}

\sphinxAtStartPar
Let’s see how this is done.

\index{siblings@\spxentry{siblings}!DataLad command@\spxentry{DataLad command}}\index{DataLad command@\spxentry{DataLad command}!siblings@\spxentry{siblings}}\index{register sibling in dataset@\spxentry{register sibling in dataset}!with DataLad@\spxentry{with DataLad}}\index{with DataLad@\spxentry{with DataLad}!register sibling in dataset@\spxentry{register sibling in dataset}}\ignorespaces 
\sphinxAtStartPar
First of all, navigate back into the original dataset.
In the original dataset, “add” a “sibling” by using
the \sphinxcode{\sphinxupquote{datalad siblings}} command.
The command takes the base command,
\sphinxcode{\sphinxupquote{datalad siblings}}, an action, in this case \sphinxcode{\sphinxupquote{add}}, a path to the
root of the dataset \sphinxcode{\sphinxupquote{\sphinxhyphen{}d .}}, a name for the sibling, \sphinxcode{\sphinxupquote{\sphinxhyphen{}s/\sphinxhyphen{}\sphinxhyphen{}name roommate}},
and a URL or path to the sibling, \sphinxcode{\sphinxupquote{\sphinxhyphen{}\sphinxhyphen{}url ../mock\_user/DataLad\sphinxhyphen{}101}}.
This registers your room mate’s \sphinxcode{\sphinxupquote{DataLad\sphinxhyphen{}101}} as a “sibling” (we will call it
“roommate”) to your own \sphinxcode{\sphinxupquote{DataLad\sphinxhyphen{}101}} dataset.

\begin{sphinxVerbatim}[commandchars=\\\{\}]
\PYG{g+gp}{\PYGZdl{} }\PYG{n+nb}{cd}\PYG{+w}{ }../../DataLad\PYGZhy{}101
\PYG{g+gp}{\PYGZdl{} }\PYG{c+c1}{\PYGZsh{} add a sibling}
\PYG{g+gp}{\PYGZdl{} }datalad\PYG{+w}{ }siblings\PYG{+w}{ }add\PYG{+w}{ }\PYGZhy{}d\PYG{+w}{ }.\PYG{+w}{ }\PYG{l+s+se}{\PYGZbs{}}
\PYG{+w}{  }\PYGZhy{}\PYGZhy{}name\PYG{+w}{ }roommate\PYG{+w}{ }\PYGZhy{}\PYGZhy{}url\PYG{+w}{ }../mock\PYGZus{}user/DataLad\PYGZhy{}101
\PYG{g+go}{.: roommate(+) [../mock\PYGZus{}user/DataLad\PYGZhy{}101 (git)]}
\end{sphinxVerbatim}

\sphinxAtStartPar
There are a few confusing parts about this command: For one, do not be surprised
about the \sphinxcode{\sphinxupquote{\sphinxhyphen{}\sphinxhyphen{}url}} argument \textendash{} it’s called “URL” but it can be a path as well.
Also, do not forget to give a name to your dataset’s sibling. Without the \sphinxcode{\sphinxupquote{\sphinxhyphen{}s}}/
\sphinxcode{\sphinxupquote{\sphinxhyphen{}\sphinxhyphen{}name}} argument the command will fail. The reason behind this is that the default
name of a sibling if no name is given will be the host name of the specified URL,
but as you provide a path and not a URL, there is no host name to take as a default.

\sphinxAtStartPar
As you can see in the command output, the addition of a {\hyperref[\detokenize{glossary:term-sibling}]{\sphinxtermref{\DUrole{xref,std,std-term}{sibling}}}} succeeded:
\sphinxcode{\sphinxupquote{roommate(+){[}../mock\_user/DataLad\sphinxhyphen{}101{]}}} means that your room mate’s dataset
is now known to your own dataset as “roommate”.

\index{list dataset siblings@\spxentry{list dataset siblings}!with DataLad@\spxentry{with DataLad}}\index{with DataLad@\spxentry{with DataLad}!list dataset siblings@\spxentry{list dataset siblings}}\ignorespaces 
\def\sphinxLiteralBlockLabel{\label{\detokenize{basics/101-121-siblings:index-4}}}
\begin{sphinxVerbatim}[commandchars=\\\{\}]
\PYG{g+gp}{\PYGZdl{} }datalad\PYG{+w}{ }siblings
\PYG{g+go}{.: here(+) [git]}
\PYG{g+go}{.: roommate(+) [../mock\PYGZus{}user/DataLad\PYGZhy{}101 (git)]}
\end{sphinxVerbatim}

\sphinxAtStartPar
This command will list all known siblings of the dataset. You can see it
in the resulting list with the name “roommate” you have given to it.
And just as a brief remark: siblings do not have to stay around forever.
Check the \textit{Find-out-more}~{\findoutmoreiconinline}\textit{\ref{fom-remove-sibling}} {\hyperref[\detokenize{basics/101-121-siblings:fom-remove-sibling}]{\sphinxcrossref{\DUrole{std,std-ref}{on removing a sibling}}}} (\autopageref*{\detokenize{basics/101-121-siblings:fom-remove-sibling}}).

\index{remove dataset sibling@\spxentry{remove dataset sibling}!with DataLad@\spxentry{with DataLad}}\index{with DataLad@\spxentry{with DataLad}!remove dataset sibling@\spxentry{remove dataset sibling}}\ignorespaces \begin{findoutmore}[label={fom-remove-sibling}, before title={\thetcbcounter\ }, float, floatplacement=tbp, check odd page=true]{What if I mistyped the name or want to remove the sibling?}
\label{\detokenize{basics/101-121-siblings:fom-remove-sibling}}

\sphinxAtStartPar
You can remove a sibling using \sphinxcode{\sphinxupquote{datalad siblings remove \sphinxhyphen{}s roommate}}


\end{findoutmore}

\sphinxAtStartPar
The fact that the \sphinxcode{\sphinxupquote{DataLad\sphinxhyphen{}101}} dataset now has a sibling means that we
can also \sphinxcode{\sphinxupquote{datalad update}} this repository. Awesome!

\sphinxAtStartPar
Your room mate previously ran a \sphinxcode{\sphinxupquote{datalad update \sphinxhyphen{}\sphinxhyphen{}how merge}} in the section
{\hyperref[\detokenize{basics/101-119-sharelocal4:update}]{\sphinxcrossref{\DUrole{std,std-ref}{Stay up to date}}}} (\autopageref*{\detokenize{basics/101-119-sharelocal4:update}}). This got him
changes \sphinxstyleemphasis{he knew you made} into a dataset that \sphinxstyleemphasis{he so far did not change}.
This meant that nothing unexpected would happen with the
\sphinxcode{\sphinxupquote{datalad update \sphinxhyphen{}\sphinxhyphen{}how merge}}.

\sphinxAtStartPar
But consider the current case: Your room mate made changes to his
dataset, but you do not necessarily know which. You also made
changes to your dataset in the meantime, and added a note on
\sphinxcode{\sphinxupquote{datalad update}}.
How would you know that his changes and
your changes are not in conflict with each other?

\sphinxAtStartPar
This scenario is where a plain \sphinxcode{\sphinxupquote{datalad update}} becomes useful.
If you run a plain \sphinxcode{\sphinxupquote{datalad update}} (which uses the default option \sphinxcode{\sphinxupquote{\sphinxhyphen{}\sphinxhyphen{}how fetch}}), DataLad will query the sibling
for changes, and store those changes in a safe place in your own
dataset, \sphinxstyleemphasis{but it will not yet integrate them into your dataset}.
This gives you a chance to see whether you actually want to have the
changes your room mate made.

\index{update dataset from particular sibling@\spxentry{update dataset from particular sibling}!with DataLad@\spxentry{with DataLad}}\index{with DataLad@\spxentry{with DataLad}!update dataset from particular sibling@\spxentry{update dataset from particular sibling}}\ignorespaces 
\sphinxAtStartPar
Let’s see how it’s done. First, run a plain \sphinxcode{\sphinxupquote{datalad update}} without
the \sphinxcode{\sphinxupquote{\sphinxhyphen{}\sphinxhyphen{}how merge}} option.

\begin{sphinxVerbatim}[commandchars=\\\{\}]
\PYG{g+gp}{\PYGZdl{} }datalad\PYG{+w}{ }update\PYG{+w}{ }\PYGZhy{}s\PYG{+w}{ }roommate
\PYG{g+go}{update(ok): . (dataset)}
\end{sphinxVerbatim}

\sphinxAtStartPar
Note that we supplied the sibling’s name with the \sphinxcode{\sphinxupquote{\sphinxhyphen{}s}}/\sphinxcode{\sphinxupquote{\sphinxhyphen{}\sphinxhyphen{}name}} option.
This is good practice, and allows you to be precise in where you want to get
updates from. It would have worked without the specification (just as a bare
\sphinxcode{\sphinxupquote{datalad update \sphinxhyphen{}\sphinxhyphen{}how merge}} worked for your room mate), because there is only
one other known location, though.

\sphinxAtStartPar
This plain \sphinxcode{\sphinxupquote{datalad update}} “fetched” updates from
the dataset. The changes however, are not yet visible \textendash{} the script that
he added is not yet in your \sphinxcode{\sphinxupquote{code/}} directory:

\begin{sphinxVerbatim}[commandchars=\\\{\}]
\PYG{g+gp}{\PYGZdl{} }ls\PYG{+w}{ }code/
\PYG{g+go}{list\PYGZus{}titles.sh}
\end{sphinxVerbatim}

\sphinxAtStartPar
So where is the file? It is in a different \sphinxstyleemphasis{branch} of your dataset.

\sphinxAtStartPar
If you do not use {\hyperref[\detokenize{glossary:term-Git}]{\sphinxtermref{\DUrole{xref,std,std-term}{Git}}}}, the concept of a {\hyperref[\detokenize{glossary:term-branch}]{\sphinxtermref{\DUrole{xref,std,std-term}{branch}}}} can be a big
source of confusion. There will be sections later in this book that will
elaborate a bit more what branches are, and how to work with them, but
for now envision a branch just like a bunch of drawers on your desk.
The paperwork that you have in front of you right on your desk is your
dataset as you currently see it.
These drawers instead hold documents that you are in principle working on,
just not now \textendash{} maybe different versions of paperwork you currently have in
front of you, or maybe other files than the ones currently in front of you
on your desk.

\sphinxAtStartPar
Imagine that a \sphinxcode{\sphinxupquote{datalad update}} created a small drawer, placed all of
the changed or added files from the sibling inside, and put it on your
desk. You can now take a look into that drawer to see whether you want
to have the changes right in front of you.

\sphinxAtStartPar
The drawer is a branch, and it is usually called \sphinxcode{\sphinxupquote{remotes/origin/main}}.
To look inside of it you can \sphinxcode{\sphinxupquote{git checkout BRANCHNAME}}, or you can
do a \sphinxcode{\sphinxupquote{diff}} between the branch (your drawer) and the dataset as it
is currently in front of you (your desk). We will do the latter, and leave
the former for a different lecture. Make sure to see the
\textit{Windows-wit}~{\windowswiticoninline}\textit{\ref{ww-datalad-diff-calls}} {\hyperref[\detokenize{basics/101-121-siblings:ww-datalad-diff-calls}]{\sphinxcrossref{\DUrole{std,std-ref}{on using \textquotesingle{}datalad diff\textquotesingle{}}}}} (\autopageref*{\detokenize{basics/101-121-siblings:ww-datalad-diff-calls}}) too.

\index{corresponding branch@\spxentry{corresponding branch}!in adjusted mode@\spxentry{in adjusted mode}}\index{in adjusted mode@\spxentry{in adjusted mode}!corresponding branch@\spxentry{corresponding branch}}\index{show dataset modification for particular path@\spxentry{show dataset modification for particular path}!on Windows with DataLad@\spxentry{on Windows with DataLad}}\index{on Windows with DataLad@\spxentry{on Windows with DataLad}!show dataset modification for particular path@\spxentry{show dataset modification for particular path}}\index{diff@\spxentry{diff}!DataLad command@\spxentry{DataLad command}}\index{DataLad command@\spxentry{DataLad command}!diff@\spxentry{diff}}\ignorespaces \begin{windowswit}[label={ww-datalad-diff-calls}, before title={\thetcbcounter\ }, float, floatplacement=tb, check odd page=true]{‘datalad diff’ needs the corresponding branch}
\label{\detokenize{basics/101-121-siblings:ww-datalad-diff-calls}}

\sphinxAtStartPar
Please use the following command instead:

\begin{sphinxVerbatim}[commandchars=\\\{\}]
\PYG{g+gp}{\PYGZdl{} }datalad\PYG{+w}{ }diff\PYG{+w}{ }\PYGZhy{}\PYGZhy{}from\PYG{+w}{ }main\PYG{+w}{ }\PYGZhy{}\PYGZhy{}to\PYG{+w}{ }remotes/roommate/main
\end{sphinxVerbatim}

\sphinxAtStartPar
This syntax specifies the {\hyperref[\detokenize{glossary:term-main}]{\sphinxtermref{\DUrole{xref,std,std-term}{main}}}} {\hyperref[\detokenize{glossary:term-branch}]{\sphinxtermref{\DUrole{xref,std,std-term}{branch}}}} as a starting point for the comparison instead of the current \sphinxcode{\sphinxupquote{adjusted/main(unlocked)}} branch.


\end{windowswit}

\begin{sphinxVerbatim}[commandchars=\\\{\}]
\PYG{g+gp}{\PYGZdl{} }datalad\PYG{+w}{ }diff\PYG{+w}{ }\PYGZhy{}\PYGZhy{}to\PYG{+w}{ }remotes/roommate/main
\PYG{g+go}{    added: code/nested\PYGZus{}repos.sh (file)}
\PYG{g+go}{ modified: notes.txt (file)}
\end{sphinxVerbatim}

\sphinxAtStartPar
This shows us that there is an additional file, and it also shows us
that there is a difference in \sphinxcode{\sphinxupquote{notes.txt}}! Let’s ask
\sphinxcode{\sphinxupquote{git diff}} to show us what the differences in detail (note that it is a shortened excerpt, cut in the middle to reduce its length):
Again, check the \textit{Windows-wit}~{\windowswiticoninline}\textit{\ref{ww-git-diff-calls}} {\hyperref[\detokenize{basics/101-121-siblings:ww-git-diff-calls}]{\sphinxcrossref{\DUrole{std,std-ref}{on using \textquotesingle{}git diff\textquotesingle{}}}}} (\autopageref*{\detokenize{basics/101-121-siblings:ww-git-diff-calls}}) too.

\index{corresponding branch@\spxentry{corresponding branch}!in adjusted mode@\spxentry{in adjusted mode}}\index{in adjusted mode@\spxentry{in adjusted mode}!corresponding branch@\spxentry{corresponding branch}}\index{show dataset modification@\spxentry{show dataset modification}!on Windows with Git@\spxentry{on Windows with Git}}\index{on Windows with Git@\spxentry{on Windows with Git}!show dataset modification@\spxentry{show dataset modification}}\index{diff@\spxentry{diff}!DataLad command@\spxentry{DataLad command}}\index{DataLad command@\spxentry{DataLad command}!diff@\spxentry{diff}}\ignorespaces \begin{windowswit}[label={ww-git-diff-calls}, before title={\thetcbcounter\ }, float, floatplacement=tb, check odd page=true]{‘git diff’ needs the corresponding branch}
\label{\detokenize{basics/101-121-siblings:ww-git-diff-calls}}

\sphinxAtStartPar
Please use the following command instead:

\begin{sphinxVerbatim}[commandchars=\\\{\}]
\PYG{g+gp}{\PYGZdl{} }git\PYG{+w}{ }diff\PYG{+w}{ }main..remotes/roommate/main
\end{sphinxVerbatim}

\sphinxAtStartPar
This is {\hyperref[\detokenize{glossary:term-Git}]{\sphinxtermref{\DUrole{xref,std,std-term}{Git}}}}s syntax for specifying a comparison between two {\hyperref[\detokenize{glossary:term-branch}]{\sphinxtermref{\DUrole{xref,std,std-term}{branch}}}}es.


\end{windowswit}

\begin{sphinxVerbatim}[commandchars=\\\{\}]
\PYG{g+gp}{\PYGZdl{} }git\PYG{+w}{ }diff\PYG{+w}{ }remotes/roommate/main
\PYG{g+go}{diff \PYGZhy{}\PYGZhy{}git a/code/nested\PYGZus{}repos.sh b/code/nested\PYGZus{}repos.sh}
\PYG{g+go}{deleted file mode 100644}
\PYG{g+go}{index f84c817..0000000}
\PYG{g+go}{\PYGZhy{}\PYGZhy{}\PYGZhy{} a/code/nested\PYGZus{}repos.sh}
\PYG{g+go}{+++ /dev/null}
\PYG{g+go}{@@ \PYGZhy{}1,59 +0,0 @@}
\PYG{g+go}{\PYGZhy{}\PYGZsh{}!/bin/bash}
\PYG{g+go}{\PYGZhy{}\PYGZsh{} This script was converted using cast2script from:}
\PYG{g+go}{\PYGZhy{}\PYGZsh{} docs/casts/seamless\PYGZus{}nested\PYGZus{}repos.sh}
\PYG{g+go}{\PYGZhy{}set \PYGZhy{}e \PYGZhy{}u}
\PYG{g+go}{\PYGZhy{}export GIT\PYGZus{}PAGER=cat}
\PYG{g+go}{\PYGZhy{}}
\PYG{g+go}{\PYGZhy{}\PYGZsh{} DataLad provides seamless management of nested Git repositories...}
\PYG{g+go}{\PYGZhy{}}
\PYG{g+go}{\PYGZhy{}\PYGZsh{} Let\PYGZsq{}s create a dataset}
\PYG{g+go}{\PYGZhy{}datalad create demo}
\PYG{g+go}{\PYGZhy{}cd demo}
\PYG{g+go}{diff \PYGZhy{}\PYGZhy{}git a/notes.txt b/notes.txt}
\PYG{g+go}{index 655be7d..ff02f68 100644}
\PYG{g+go}{\PYGZhy{}\PYGZhy{}\PYGZhy{} a/notes.txt}
\PYG{g+go}{+++ b/notes.txt}
\PYG{g+go}{@@ \PYGZhy{}59,3 +59,7 @@ The command \PYGZdq{}git annex whereis PATH\PYGZdq{} lists the repositories that have}
\PYG{g+go}{ the file content of an annexed file. When using \PYGZdq{}datalad get\PYGZdq{} to}
\PYG{g+go}{ retrieve file content, those repositories will be queried.}

\PYG{g+go}{+To update a shared dataset, run the command \PYGZdq{}datalad update \PYGZhy{}\PYGZhy{}how merge\PYGZdq{}.}
\PYG{g+go}{+This command will query its origin for changes, and integrate the}
\PYG{g+go}{+changes into the dataset.}
\PYG{g+go}{+}
\end{sphinxVerbatim}

\sphinxAtStartPar
Let’s digress into what is shown here.
We are comparing the current state of your dataset against
the current state of your room mate’s dataset. Everything marked with
a \sphinxcode{\sphinxupquote{\sphinxhyphen{}}} is a change that your room mate has, but not you: This is the
script that he downloaded!

\sphinxAtStartPar
Everything that is marked with a \sphinxcode{\sphinxupquote{+}} is a change that you have,
but not your room mate: It is the additional note on \sphinxcode{\sphinxupquote{datalad update}}
you made in your own dataset in the previous section.

\sphinxAtStartPar
Cool! So now that you know what the changes are that your room mate
made, you can safely \sphinxcode{\sphinxupquote{datalad update \sphinxhyphen{}\sphinxhyphen{}how merge}} them to integrate
them into your dataset. In technical terms you will
“\sphinxstyleemphasis{merge the branch remotes/roommate/main into main}”.
But the details of this will be stated in a standalone section later.

\sphinxAtStartPar
Note that the fact that your room mate does not have the note
on \sphinxcode{\sphinxupquote{datalad update}} does not influence your note. It will not
get deleted by the merge. You do not set your dataset to the state
of your room mate’s dataset, but you incorporate all changes he made
\textendash{} which is only the addition of the script.

\begin{sphinxVerbatim}[commandchars=\\\{\}]
\PYG{g+gp}{\PYGZdl{} }datalad\PYG{+w}{ }update\PYG{+w}{ }\PYGZhy{}\PYGZhy{}how\PYG{+w}{ }merge\PYG{+w}{ }\PYGZhy{}s\PYG{+w}{ }roommate
\PYG{g+go}{merge(ok): . (dataset) [Merged roommate/main]}
\PYG{g+go}{update.annex\PYGZus{}merge(ok): . (dataset) [Merged annex branch]}
\PYG{g+go}{update(ok): . (dataset)}
\end{sphinxVerbatim}

\sphinxAtStartPar
The exciting question is now whether your room mate’s change is now
also part of your own dataset. Let’s list the contents of the \sphinxcode{\sphinxupquote{code/}}
directory and also peek into the history:

\begin{sphinxVerbatim}[commandchars=\\\{\}]
\PYG{g+gp}{\PYGZdl{} }ls\PYG{+w}{ }code/
\PYG{g+go}{list\PYGZus{}titles.sh}
\PYG{g+go}{nested\PYGZus{}repos.sh}
\end{sphinxVerbatim}

\fvset{hllines={, 2, 4,}}%
\begin{sphinxVerbatim}[commandchars=\\\{\}]
\PYG{g+gp}{\PYGZdl{} }git\PYG{+w}{ }log\PYG{+w}{ }\PYGZhy{}\PYGZhy{}oneline
\PYG{g+go}{6ae8e71 Merge remote\PYGZhy{}tracking branch \PYGZsq{}roommate/main\PYGZsq{}}
\PYG{g+go}{4bb5d39 add note about datalad update}
\PYG{g+go}{5b6e19a Include nesting demo from datalad website}
\PYG{g+go}{adb4b5d add note on git annex whereis}
\PYG{g+go}{1e73592 add note about cloning from paths and recursive datalad get}
\end{sphinxVerbatim}
\sphinxresetverbatimhllines

\sphinxAtStartPar
Wohoo! Here it is: The script now also exists in your own dataset.
You can see the commit that your room mate made when he saved the script,
and you can also see a commit that records how you \sphinxcode{\sphinxupquote{merged}} your
room mate’s dataset changes into your own dataset. The commit message of this
latter commit for now might contain many words yet unknown to you if you
do not use Git, but a later section will get into the details of what
the meaning of “{\hyperref[\detokenize{glossary:term-merge}]{\sphinxtermref{\DUrole{xref,std,std-term}{merge}}}}”, “{\hyperref[\detokenize{glossary:term-branch}]{\sphinxtermref{\DUrole{xref,std,std-term}{branch}}}}”, “refs”
or “{\hyperref[\detokenize{glossary:term-main}]{\sphinxtermref{\DUrole{xref,std,std-term}{main}}}}” is.

\sphinxAtStartPar
For now, you are happy to have the changes your room mate made available.
This is how it should be! You helped him, and he helps you. Awesome!
There actually is a wonderful word for it: \sphinxstyleemphasis{Collaboration}.
Thus, without noticing, you have successfully collaborated for the first
time using DataLad datasets.

\sphinxAtStartPar
Create a note about this, and save it.

\begin{sphinxVerbatim}[commandchars=\\\{\}]
\PYG{g+gp}{\PYGZdl{} }cat\PYG{+w}{ }\PYGZlt{}\PYGZlt{}\PYG{+w}{ }EOT\PYG{+w}{ }\PYGZgt{}\PYGZgt{}\PYG{+w}{ }notes.txt
\PYG{g+go}{To update from a dataset with a shared history, you need to add this}
\PYG{g+go}{dataset as a sibling to your dataset. \PYGZdq{}Adding a sibling\PYGZdq{} means}
\PYG{g+go}{providing DataLad with info about the location of a dataset, and a}
\PYG{g+go}{name for it.}
\PYG{g+go}{Afterwards, a \PYGZdq{}datalad update \PYGZhy{}\PYGZhy{}how merge \PYGZhy{}s name\PYGZdq{} will integrate the}
\PYG{g+go}{changes made to the sibling into the dataset. A safe step in between}
\PYG{g+go}{is to do a \PYGZdq{}datalad update \PYGZhy{}s name\PYGZdq{} and checkout the changes with}
\PYG{g+go}{\PYGZdq{}git/datalad diff\PYGZdq{} to remotes/origin/main}

\PYG{g+go}{EOT}
\PYG{g+gp}{\PYGZdl{} }datalad\PYG{+w}{ }save\PYG{+w}{ }\PYGZhy{}m\PYG{+w}{ }\PYG{l+s+s2}{\PYGZdq{}Add note on adding siblings\PYGZdq{}}
\PYG{g+go}{add(ok): notes.txt (file)}
\PYG{g+go}{save(ok): . (dataset)}
\end{sphinxVerbatim}

\sphinxstepscope


\section{Summary}
\label{\detokenize{basics/101-120-summary:summary}}\label{\detokenize{basics/101-120-summary:summary-sharelocal}}\label{\detokenize{basics/101-120-summary::doc}}
\sphinxAtStartPar
Together with your room mate you have just discovered how
to share, update, and collaborate on a DataLad dataset on a shared file system.
Thus, you have glimpsed into the principles and advantages of
sharing a dataset with a simple example.
\begin{itemize}
\item {} 
\sphinxAtStartPar
To obtain a dataset, one can also use \sphinxcode{\sphinxupquote{datalad clone}} with a path.
Potential subdatasets will not be installed right away. As they are registered in
the superdataset, you can
\begin{itemize}
\item {} 
\sphinxAtStartPar
do \sphinxcode{\sphinxupquote{datalad get \sphinxhyphen{}n/\sphinxhyphen{}\sphinxhyphen{}no\sphinxhyphen{}data}}

\item {} 
\sphinxAtStartPar
or specify the \sphinxcode{\sphinxupquote{\sphinxhyphen{}r}}/\sphinxcode{\sphinxupquote{\sphinxhyphen{}\sphinxhyphen{}recursive}}: \sphinxcode{\sphinxupquote{datalad get \sphinxhyphen{}n \sphinxhyphen{}r \textless{}subds\textgreater{}}}

\end{itemize}

\sphinxAtStartPar
with a decent \sphinxcode{\sphinxupquote{\sphinxhyphen{}R/\sphinxhyphen{}\sphinxhyphen{}recursion\sphinxhyphen{}limit}} choice to install them afterwards.

\item {} 
\sphinxAtStartPar
The configuration of the original dataset determines which types
of files will have their content available right after the installation of
the dataset, and which types of files need to be retrieved via
\sphinxcode{\sphinxupquote{datalad get}}: Any file content stored in {\hyperref[\detokenize{glossary:term-Git}]{\sphinxtermref{\DUrole{xref,std,std-term}{Git}}}} will be available
right away, while all file content that is \sphinxcode{\sphinxupquote{annexed}} only has
small metadata about its availability attached to it. The original
\sphinxcode{\sphinxupquote{DataLad\sphinxhyphen{}101}} dataset used the \sphinxcode{\sphinxupquote{text2git}} configuration template
to store text files such as \sphinxcode{\sphinxupquote{notes.txt}} and \sphinxcode{\sphinxupquote{code/list\_titles.sh}}
in Git \textendash{} these files’ content is therefore available right after
installation.

\item {} 
\sphinxAtStartPar
Annexed content can be retrieved via \sphinxcode{\sphinxupquote{datalad get}} from the
file content sources.

\item {} 
\sphinxAtStartPar
\sphinxcode{\sphinxupquote{git annex whereis PATH}} will list all locations known to contain file
content for a particular file. It is a very
helpful command to find out where file content resides, and how many
locations with copies exist. {\hyperref[\detokenize{glossary:term-git-annex}]{\sphinxtermref{\DUrole{xref,std,std-term}{git\sphinxhyphen{}annex}}}} will try to retrieve file contents from those locations. If you want, you can describe locations with the
\sphinxcode{\sphinxupquote{\sphinxhyphen{}\sphinxhyphen{}description}} provided during a \sphinxcode{\sphinxupquote{datalad create}}.

\item {} 
\sphinxAtStartPar
A shared copy of a dataset includes the datasets history. If well made,
\sphinxcode{\sphinxupquote{datalad run}} commands can then easily be \sphinxcode{\sphinxupquote{rerun}}.

\item {} 
\sphinxAtStartPar
Because an installed dataset knows its origin \textendash{} the place it was
originally installed from \textendash{} it can be kept up\sphinxhyphen{}to\sphinxhyphen{}date with the
\sphinxcode{\sphinxupquote{datalad update}} command. This command will query the origin of the
dataset for updates, and a \sphinxcode{\sphinxupquote{datalad update \sphinxhyphen{}\sphinxhyphen{}how merge}} will integrate
these changes into the dataset copy.

\item {} 
\sphinxAtStartPar
Thus, using DataLad, data can be easily shared and kept up to date
with only two commands: \sphinxcode{\sphinxupquote{datalad clone}} and \sphinxcode{\sphinxupquote{datalad update}}.

\item {} 
\sphinxAtStartPar
By configuring a dataset as a {\hyperref[\detokenize{glossary:term-sibling}]{\sphinxtermref{\DUrole{xref,std,std-term}{sibling}}}}, collaboration becomes easy.

\item {} 
\sphinxAtStartPar
To avoid integrating conflicting modifications of a sibling dataset into your
own dataset, a \sphinxcode{\sphinxupquote{datalad update \sphinxhyphen{}s SIBLINGNAME}} will “\sphinxcode{\sphinxupquote{fetch}}” modifications
and store them on a different {\hyperref[\detokenize{glossary:term-branch}]{\sphinxtermref{\DUrole{xref,std,std-term}{branch}}}} of your dataset. The commands
\sphinxcode{\sphinxupquote{datalad diff}} and \sphinxcode{\sphinxupquote{git diff}} can subsequently help to find
out what changes have been made in the sibling.

\end{itemize}


\subsection{Now what can I do with that?}
\label{\detokenize{basics/101-120-summary:now-what-can-i-do-with-that}}
\sphinxAtStartPar
Most importantly, you have experienced the first way of sharing
and updating a dataset.
The example here may strike you as too simplistic, but in later parts of
the book you will see examples in which datasets are shared on the same
file system in surprisingly useful ways.

\sphinxAtStartPar
Simultaneously, you have observed dataset properties you already knew
(for example, how annexed files need to be retrieved via \sphinxcode{\sphinxupquote{datalad get}}),
but you have also seen novel aspects of a dataset \textendash{} for example, that
subdatasets are not automatically installed by default, how
\sphinxcode{\sphinxupquote{git annex whereis}} can help you find out where file content might be stored,
how useful commands that capture provenance about the origin or creation of files
(such as \sphinxcode{\sphinxupquote{datalad run}} or \sphinxcode{\sphinxupquote{datalad download\sphinxhyphen{}url}}) are,
or how a shared dataset can be updated to reflect changes that were made
to the original dataset.

\sphinxAtStartPar
Also, you have successfully demonstrated a large number of DataLad dataset
principles to your room mate: How content stored in Git is present right
away and how annexed content first needs to be retrieved, how easy a
\sphinxcode{\sphinxupquote{datalad rerun}} is if the original \sphinxcode{\sphinxupquote{datalad run}} command was well
specified, how a datasets history is shared and not only its data.

\sphinxAtStartPar
Lastly, with the configuration of a sibling, you have experienced one
way to collaborate in a dataset, and with \sphinxcode{\sphinxupquote{datalad update \sphinxhyphen{}\sphinxhyphen{}how merge}}
and \sphinxcode{\sphinxupquote{datalad update}}, you also glimpsed into more advances aspects
of Git, namely the concept of a branch.

\sphinxAtStartPar
Therefore, these last few sections have hopefully been a good review
of what you already knew, but also a big knowledge gain, and cause
joyful anticipation of collaboration in a real\sphinxhyphen{}world setting of one
of your own use cases.

\sphinxstepscope


