\chapter{Containerize!}
\label{\detokenize{basics/basics-containers:containerize}}\label{\detokenize{basics/basics-containers:chapter-containersrun}}\label{\detokenize{basics/basics-containers::doc}}
\noindent{\hspace*{\fill}\sphinxincludegraphics[width=0.500\linewidth]{{forward}.pdf}\hspace*{\fill}}

\sphinxstepscope

\index{dataset nesting@\spxentry{dataset nesting}!DataLad concept@\spxentry{DataLad concept}}\index{DataLad concept@\spxentry{DataLad concept}!dataset nesting@\spxentry{dataset nesting}}\ignorespaces 

\section{Turtles all the way down}
\label{\detokenize{basics/101-132-advancednesting:turtles-all-the-way-down}}\label{\detokenize{basics/101-132-advancednesting:nesting2}}\label{\detokenize{basics/101-132-advancednesting:index-0}}\label{\detokenize{basics/101-132-advancednesting::doc}}
\sphinxAtStartPar
You may have noticed how working in the subdataset felt as if you would be
working in an independent dataset \textendash{} there was no information or influence at
all from the top\sphinxhyphen{}level \sphinxcode{\sphinxupquote{DataLad\sphinxhyphen{}101}} superdataset, and you build up a
completely stand\sphinxhyphen{}alone history:

\begin{sphinxVerbatim}[commandchars=\\\{\}]
\PYG{g+gp}{\PYGZdl{} }git\PYG{+w}{ }log\PYG{+w}{ }\PYGZhy{}\PYGZhy{}oneline
\PYG{g+go}{6fc0d0f Provide project description}
\PYG{g+go}{9aadac7 [DATALAD RUNCMD] analyze iris data with classification analysis}
\PYG{g+go}{ca0c747 add script for kNN classification and plotting}
\PYG{g+go}{4f945ed [DATALAD] Added subdataset}
\PYG{g+go}{18f4a98 Apply YODA dataset setup}
\PYG{g+go}{bf231d5 [DATALAD] new dataset}
\end{sphinxVerbatim}

\sphinxAtStartPar
In principle, this is no news to you. From section {\hyperref[\detokenize{basics/101-106-nesting:nesting}]{\sphinxcrossref{\DUrole{std,std-ref}{Dataset nesting}}}} (\autopageref*{\detokenize{basics/101-106-nesting:nesting}}) and the
YODA principles you already know that nesting allows for a modular reuse of
any other DataLad dataset, and that this reuse is possible and simple
precisely because all of the information is kept within a (sub)dataset.

\sphinxAtStartPar
What is new now, however, is that you applied changes to the dataset. While
you already explored the looks and feels of the \sphinxcode{\sphinxupquote{longnow}} subdataset in
previous sections, you now \sphinxstyleemphasis{modified} the contents of the \sphinxcode{\sphinxupquote{midterm\_project}}
subdataset.
How does this influence the superdataset, and how does this look like in the
superdataset’s history? You know from section {\hyperref[\detokenize{basics/101-106-nesting:nesting}]{\sphinxcrossref{\DUrole{std,std-ref}{Dataset nesting}}}} (\autopageref*{\detokenize{basics/101-106-nesting:nesting}}) that the
superdataset only stores the \sphinxstyleemphasis{state} of the subdataset. Upon creation of the
dataset, the very first, initial state of the subdataset was thus recorded in
the superdataset. But now, after you finished your project, your subdataset
evolved. Let’s query the superdataset what it thinks about this.

\begin{sphinxVerbatim}[commandchars=\\\{\}]
\PYG{g+gp}{\PYGZdl{} }\PYG{c+c1}{\PYGZsh{} move into the superdataset}
\PYG{g+gp}{\PYGZdl{} }\PYG{n+nb}{cd}\PYG{+w}{ }../
\PYG{g+gp}{\PYGZdl{} }datalad\PYG{+w}{ }status
\PYG{g+go}{ modified: midterm\PYGZus{}project (dataset)}
\end{sphinxVerbatim}

\sphinxAtStartPar
From the superdataset’s perspective, the subdataset appears as being
“modified”. Note how it is not individual files that show up as “modified”, but
indeed the complete subdataset as a single entity.

\sphinxAtStartPar
What this shows you is that the modifications of the subdataset you performed are not
automatically recorded to the superdataset. This makes sense, after all it
should be up to you to decide whether you want record something or not.
But it is worth repeating: If you modify a subdataset, you will need to save
this \sphinxstyleemphasis{in the superdataset} in order to have a clean superdataset status.

\sphinxAtStartPar
Let’s save the modification of the subdataset into the history of the
superdataset. For this, to avoid confusion, you can specify explicitly to
which dataset you want to save a modification. \sphinxcode{\sphinxupquote{\sphinxhyphen{}d .}} specifies the current
dataset, i.e., \sphinxcode{\sphinxupquote{DataLad\sphinxhyphen{}101}}, as the dataset to save to.
The \textit{Find-out-more}~{\findoutmoreiconinline}\textit{\ref{fom-save-subds}} {\hyperref[\detokenize{basics/101-132-advancednesting:fom-save-subds}]{\sphinxcrossref{\DUrole{std,std-ref}{on saving subdatasets}}}} (\autopageref*{\detokenize{basics/101-132-advancednesting:fom-save-subds}}) provides some more details.

\begin{sphinxVerbatim}[commandchars=\\\{\}]
\PYG{g+gp}{\PYGZdl{} }datalad\PYG{+w}{ }save\PYG{+w}{ }\PYGZhy{}d\PYG{+w}{ }.\PYG{+w}{ }\PYGZhy{}m\PYG{+w}{ }\PYG{l+s+s2}{\PYGZdq{}finished my midterm project\PYGZdq{}}\PYG{+w}{ }midterm\PYGZus{}project
\PYG{g+go}{add(ok): midterm\PYGZus{}project (dataset)}
\PYG{g+go}{add(ok): .gitmodules (file)}
\PYG{g+go}{save(ok): . (dataset)}
\end{sphinxVerbatim}

\index{save modification in nested dataset@\spxentry{save modification in nested dataset}!with DataLad@\spxentry{with DataLad}}\index{with DataLad@\spxentry{with DataLad}!save modification in nested dataset@\spxentry{save modification in nested dataset}}\ignorespaces \begin{findoutmore}[label={fom-save-subds}, before title={\thetcbcounter\ }, float, floatplacement=tb, check odd page=true]{‘datalad save’ on nested datasets}
\label{\detokenize{basics/101-132-advancednesting:fom-save-subds}}

\sphinxAtStartPar
In a superdataset with subdatasets, \sphinxcode{\sphinxupquote{datalad save}} by default
tries to figure out on its own which dataset’s history of all available
datasets a \sphinxcode{\sphinxupquote{datalad save}} should be written to. However, it can reduce
confusion or allow specific operations to be very explicit in the command
call and tell DataLad where to save what kind of modifications to.

\sphinxAtStartPar
If you want to save the current state of the subdataset into the superdataset
(as necessary here), start a \sphinxcode{\sphinxupquote{save}} from the superdataset and have the
\sphinxcode{\sphinxupquote{\sphinxhyphen{}d/\sphinxhyphen{}\sphinxhyphen{}dataset}} option point to its root:

\begin{sphinxVerbatim}[commandchars=\\\{\}]
\PYG{g+gp}{\PYGZdl{} }\PYG{c+c1}{\PYGZsh{} in the root of the superds}
\PYG{g+gp}{\PYGZdl{} }datalad\PYG{+w}{ }save\PYG{+w}{ }\PYGZhy{}d\PYG{+w}{ }.\PYG{+w}{ }\PYGZhy{}m\PYG{+w}{ }\PYG{l+s+s2}{\PYGZdq{}update subdataset\PYGZdq{}}
\end{sphinxVerbatim}

\sphinxAtStartPar
If you are in the superdataset, and you want to save an unsaved modification
in a subdataset to the \sphinxstyleemphasis{subdatasets} history, let \sphinxcode{\sphinxupquote{\sphinxhyphen{}d/\sphinxhyphen{}\sphinxhyphen{}dataset}} point to
the subdataset:

\begin{sphinxVerbatim}[commandchars=\\\{\}]
\PYG{g+gp}{\PYGZdl{} }\PYG{c+c1}{\PYGZsh{} in the superds}
\PYG{g+gp}{\PYGZdl{} }datalad\PYG{+w}{ }save\PYG{+w}{ }\PYGZhy{}d\PYG{+w}{ }path/to/subds\PYG{+w}{ }\PYGZhy{}m\PYG{+w}{ }\PYG{l+s+s2}{\PYGZdq{}modified XY\PYGZdq{}}
\end{sphinxVerbatim}

\sphinxAtStartPar
The recursive option allows you to save any content underneath the specified
directory, and recurse into any potential subdatasets:

\begin{sphinxVerbatim}[commandchars=\\\{\}]
\PYG{g+gp}{\PYGZdl{} }datalad\PYG{+w}{ }save\PYG{+w}{ }.\PYG{+w}{ }\PYGZhy{}\PYGZhy{}recursive
\end{sphinxVerbatim}


\end{findoutmore}

\sphinxAtStartPar
Let’s check which subproject commit is now recorded in the superdataset:

\fvset{hllines={, 14,}}%
\begin{sphinxVerbatim}[commandchars=\\\{\}]
\PYG{g+gp}{\PYGZdl{} }git\PYG{+w}{ }log\PYG{+w}{ }\PYGZhy{}p\PYG{+w}{ }\PYGZhy{}n\PYG{+w}{ }\PYG{l+m}{1}
\PYG{g+go}{commit c5c90178✂SHA1}
\PYG{g+go}{Author: Elena Piscopia \PYGZlt{}elena@example.net\PYGZgt{}}
\PYG{g+go}{Date:   Tue Jun 18 16:13:00 2019 +0000}

\PYG{g+go}{    finished my midterm project}

\PYG{g+go}{diff \PYGZhy{}\PYGZhy{}git a/midterm\PYGZus{}project b/midterm\PYGZus{}project}
\PYG{g+go}{index 18f4a98..6fc0d0f 160000}
\PYG{g+go}{\PYGZhy{}\PYGZhy{}\PYGZhy{} a/midterm\PYGZus{}project}
\PYG{g+go}{+++ b/midterm\PYGZus{}project}
\PYG{g+go}{@@ \PYGZhy{}1 +1 @@}
\PYG{g+go}{\PYGZhy{}Subproject commit 18f4a981✂SHA1}
\PYG{g+go}{+Subproject commit 6fc0d0f5✂SHA1}
\end{sphinxVerbatim}
\sphinxresetverbatimhllines

\sphinxAtStartPar
As you can see in the log entry, the subproject commit changed from the
first commit hash in the subdataset history to the most recent one. With this
change, therefore, your superdataset tracks the most recent version of
the \sphinxcode{\sphinxupquote{midterm\_project}} dataset, and your dataset’s status is clean again.

\sphinxAtStartPar
This time in DataLad\sphinxhyphen{}101 is a convenient moment to dive a bit deeper
into the functions of the \sphinxcode{\sphinxupquote{datalad status}} command. If you are
interested in this, check out the \textit{Find-out-more}~{\findoutmoreiconinline}\textit{\ref{fom-status}} {\hyperref[\detokenize{basics/101-132-advancednesting:fom-status}]{\sphinxcrossref{\DUrole{std,std-ref}{on this topic}}}} (\autopageref*{\detokenize{basics/101-132-advancednesting:fom-status}}).

\index{status@\spxentry{status}!DataLad command@\spxentry{DataLad command}}\index{DataLad command@\spxentry{DataLad command}!status@\spxentry{status}}\index{check dataset for modification@\spxentry{check dataset for modification}!with DataLad@\spxentry{with DataLad}}\index{with DataLad@\spxentry{with DataLad}!check dataset for modification@\spxentry{check dataset for modification}}\ignorespaces \begin{findoutmore}[label={fom-status}, before title={\thetcbcounter\ }, float, check odd page=true]{More on ‘datalad status’}
\label{\detokenize{basics/101-132-advancednesting:fom-status}}

\sphinxAtStartPar
First of all, let’s start with a quick overview of the different content \sphinxstyleemphasis{types}
and content \sphinxstyleemphasis{states} various \sphinxcode{\sphinxupquote{datalad status}} commands in the course
of DataLad\sphinxhyphen{}101 have shown up to this point.
You have seen the following \sphinxstyleemphasis{content types}:
\begin{itemize}
\item {} 
\sphinxAtStartPar
\sphinxcode{\sphinxupquote{file}}, e.g., \sphinxcode{\sphinxupquote{notes.txt}}: any file (or symlink that is a placeholder to an annexed file)

\item {} 
\sphinxAtStartPar
\sphinxcode{\sphinxupquote{directory}}, e.g., \sphinxcode{\sphinxupquote{books}}: any directory that does not qualify for the \sphinxcode{\sphinxupquote{dataset}} type

\item {} 
\sphinxAtStartPar
\sphinxcode{\sphinxupquote{symlink}}, e.g., the \sphinxcode{\sphinxupquote{.jgp}} that was manually unlocked in section {\hyperref[\detokenize{basics/101-110-run2:run3}]{\sphinxcrossref{\DUrole{std,std-ref}{Input and output}}}} (\autopageref*{\detokenize{basics/101-110-run2:run3}}):
any symlink that is not used as a placeholder for an annexed file

\item {} 
\sphinxAtStartPar
\sphinxcode{\sphinxupquote{dataset}}, e.g., the \sphinxcode{\sphinxupquote{midterm\_project}}: any top\sphinxhyphen{}level dataset, or any subdataset
that is properly registered in the superdataset

\end{itemize}

\sphinxAtStartPar
And you have seen the following \sphinxstyleemphasis{content states}: \sphinxcode{\sphinxupquote{modified}} and \sphinxcode{\sphinxupquote{untracked}}.
The section {\hyperref[\detokenize{basics/101-136-filesystem:file-system}]{\sphinxcrossref{\DUrole{std,std-ref}{Dataset ops}}}} (\autopageref*{\detokenize{basics/101-136-filesystem:file-system}}) will show you many instances of \sphinxcode{\sphinxupquote{deleted}} content
state as well.

\sphinxAtStartPar
But beyond understanding the report of \sphinxcode{\sphinxupquote{datalad status}}, there is also
additional functionality:
\sphinxcode{\sphinxupquote{datalad status}} can handle status reports for a whole hierarchy
of datasets, and it can report on a subset of the content across any number of
datasets in this hierarchy by providing selected paths. This is useful as soon
as datasets become more complex and contain subdatasets with changing contents.

\sphinxAtStartPar
When performed without any arguments, \sphinxcode{\sphinxupquote{datalad status}} will report
the state of the current dataset. However, you can specify a path to any
sub\sphinxhyphen{} or superdataset with the \sphinxcode{\sphinxupquote{\sphinxhyphen{}\sphinxhyphen{}dataset}} option.
In order to demonstrate this a bit better, we will make sure that not only the
state of the subdataset \sphinxstyleemphasis{within} the superdataset is modified, but also that the
subdataset contains a modification. For this, let’s add an empty text file into
the \sphinxcode{\sphinxupquote{midterm\_project}} subdataset:

\begin{sphinxVerbatim}[commandchars=\\\{\}]
\PYG{g+gp}{\PYGZdl{} }touch\PYG{+w}{ }midterm\PYGZus{}project/an\PYGZus{}empty\PYGZus{}file
\end{sphinxVerbatim}

\sphinxAtStartPar
If you are in the root of \sphinxcode{\sphinxupquote{DataLad\sphinxhyphen{}101}}, but interested in the status
\sphinxstyleemphasis{within} the subdataset, simply provide a path (relative to your current location)
to the command:

\begin{sphinxVerbatim}[commandchars=\\\{\}]
\PYG{g+gp}{\PYGZdl{} }datalad\PYG{+w}{ }status\PYG{+w}{ }midterm\PYGZus{}project
\PYG{g+go}{untracked: midterm\PYGZus{}project/an\PYGZus{}empty\PYGZus{}file (file)}
\end{sphinxVerbatim}

\sphinxAtStartPar
Alternatively, to achieve the same, specify the superdataset as the \sphinxcode{\sphinxupquote{\sphinxhyphen{}\sphinxhyphen{}dataset}}
and provide a path to the subdataset \sphinxstyleemphasis{with a trailing path separator} like
this:

\begin{sphinxVerbatim}[commandchars=\\\{\}]
\PYG{g+gp}{\PYGZdl{} }datalad\PYG{+w}{ }status\PYG{+w}{ }\PYGZhy{}d\PYG{+w}{ }.\PYG{+w}{ }midterm\PYGZus{}project/
\PYG{g+go}{untracked: midterm\PYGZus{}project/an\PYGZus{}empty\PYGZus{}file (file)}
\end{sphinxVerbatim}

\sphinxAtStartPar
Note that both of these commands return only the \sphinxcode{\sphinxupquote{untracked}} file and not
not the \sphinxcode{\sphinxupquote{modified}} subdataset because we’re explicitly querying only the
subdataset for its status.
If you however, as done outside of this Find\sphinxhyphen{}out\sphinxhyphen{}more, you want to know about
the subdataset record in the superdataset without causing a status query for
the state \sphinxstyleemphasis{within} the subdataset itself, you can also provide an explicit
path to the dataset (without a trailing path separator). This can be used
to specify a specific subdataset in the case of a dataset with many subdatasets:

\begin{sphinxVerbatim}[commandchars=\\\{\}]
\PYG{g+gp}{\PYGZdl{} }datalad\PYG{+w}{ }status\PYG{+w}{ }\PYGZhy{}d\PYG{+w}{ }.\PYG{+w}{ }midterm\PYGZus{}project
\PYG{g+go}{ modified: midterm\PYGZus{}project (dataset)}
\end{sphinxVerbatim}

\sphinxAtStartPar
But if you are interested in both the state within the subdataset, and
the state of the subdataset within the superdataset, you can combine the
two paths:

\begin{sphinxVerbatim}[commandchars=\\\{\}]
\PYG{g+gp}{\PYGZdl{} }datalad\PYG{+w}{ }status\PYG{+w}{ }\PYGZhy{}d\PYG{+w}{ }.\PYG{+w}{ }midterm\PYGZus{}project\PYG{+w}{ }midterm\PYGZus{}project/
\PYG{g+go}{ modified: midterm\PYGZus{}project (dataset)}
\PYG{g+go}{untracked: midterm\PYGZus{}project/an\PYGZus{}empty\PYGZus{}file (file)}
\end{sphinxVerbatim}

\sphinxAtStartPar
Finally, if these subtle differences in the paths are not easy to memorize,
the \sphinxcode{\sphinxupquote{\sphinxhyphen{}r/\sphinxhyphen{}\sphinxhyphen{}recursive}} option will also report you both status aspects:

\begin{sphinxVerbatim}[commandchars=\\\{\}]
\PYG{g+gp}{\PYGZdl{} }datalad\PYG{+w}{ }status\PYG{+w}{ }\PYGZhy{}\PYGZhy{}recursive
\PYG{g+go}{ modified: midterm\PYGZus{}project (dataset)}
\PYG{g+go}{untracked: midterm\PYGZus{}project/an\PYGZus{}empty\PYGZus{}file (file)}
\end{sphinxVerbatim}

\sphinxAtStartPar
Importantly, the regular output from a \sphinxcode{\sphinxupquote{datalad status}} command in the commandline is “condensed” to the most important information by a tailored result renderer.
You can, however, also get \sphinxcode{\sphinxupquote{status}}’ unfiltered full output by switching the \sphinxcode{\sphinxupquote{\sphinxhyphen{}f}}/\sphinxcode{\sphinxupquote{\sphinxhyphen{}\sphinxhyphen{}output\sphinxhyphen{}format}} from \sphinxcode{\sphinxupquote{tailored}} (the default) to \sphinxcode{\sphinxupquote{json}} or, for the same infos as \sphinxcode{\sphinxupquote{json}} but better readability, \sphinxcode{\sphinxupquote{json\_pp}}:

\begin{sphinxVerbatim}[commandchars=\\\{\}]
\PYG{g+gp}{\PYGZdl{} }datalad\PYG{+w}{ }\PYGZhy{}f\PYG{+w}{ }json\PYGZus{}pp\PYG{+w}{ }status\PYG{+w}{ }\PYGZhy{}d\PYG{+w}{ }.\PYG{+w}{ }midterm\PYGZus{}project
\PYG{g+go}{\PYGZob{}}
\PYG{g+go}{  \PYGZdq{}action\PYGZdq{}: \PYGZdq{}status\PYGZdq{},}
\PYG{g+go}{  \PYGZdq{}gitshasum\PYGZdq{}: \PYGZdq{}6fc0d0f5✂SHA1\PYGZdq{},}
\PYG{g+go}{  \PYGZdq{}parentds\PYGZdq{}: \PYGZdq{}/home/me/dl\PYGZhy{}101/DataLad\PYGZhy{}101\PYGZdq{},}
\PYG{g+go}{  \PYGZdq{}path\PYGZdq{}: \PYGZdq{}/home/me/dl\PYGZhy{}101/DataLad\PYGZhy{}101/midterm\PYGZus{}project\PYGZdq{},}
\PYG{g+go}{  \PYGZdq{}prev\PYGZus{}gitshasum\PYGZdq{}: \PYGZdq{}6fc0d0f5✂SHA1\PYGZdq{},}
\PYG{g+go}{  \PYGZdq{}refds\PYGZdq{}: \PYGZdq{}/home/me/dl\PYGZhy{}101/DataLad\PYGZhy{}101\PYGZdq{},}
\PYG{g+go}{  \PYGZdq{}state\PYGZdq{}: \PYGZdq{}modified\PYGZdq{},}
\PYG{g+go}{  \PYGZdq{}status\PYGZdq{}: \PYGZdq{}ok\PYGZdq{},}
\PYG{g+go}{  \PYGZdq{}type\PYGZdq{}: \PYGZdq{}dataset\PYGZdq{}}
\PYG{g+go}{\PYGZcb{}}
\end{sphinxVerbatim}

\sphinxAtStartPar
This still was not all of the available functionality of the
\sphinxcode{\sphinxupquote{datalad status}} command. You could, for example, adjust whether and
how untracked dataset content should be reported with the \sphinxcode{\sphinxupquote{\sphinxhyphen{}\sphinxhyphen{}untracked}}
option, or get additional information from annexed content with the \sphinxcode{\sphinxupquote{\sphinxhyphen{}\sphinxhyphen{}annex}}
option (especially powerful when combined with \sphinxcode{\sphinxupquote{\sphinxhyphen{}f json\_pp}}). To get a complete overview on what you could do, check out the technical
documentation of \sphinxcode{\sphinxupquote{datalad status}} \dlhbhref{D1C}{here}.

\sphinxAtStartPar
Before we leave this Find\sphinxhyphen{}out\sphinxhyphen{}more, lets undo the modification of the subdataset
by removing the untracked file:

\begin{sphinxVerbatim}[commandchars=\\\{\}]
\PYG{g+gp}{\PYGZdl{} }rm\PYG{+w}{ }midterm\PYGZus{}project/an\PYGZus{}empty\PYGZus{}file
\PYG{g+gp}{\PYGZdl{} }datalad\PYG{+w}{ }status\PYG{+w}{ }\PYGZhy{}\PYGZhy{}recursive
\PYG{g+go}{nothing to save, working tree clean}
\end{sphinxVerbatim}


\end{findoutmore}

\sphinxAtStartPar
While there is much more to say and discover about this topic, this section showed the basic handling of datasets that are composed of, arbitrarily deeply, nested datasets.
Each (sub)dataset can be simultaneously: 1) an independently managed data module, usable and reusable in different contexts; and 2) a tightly integrated component that contributes to a look and feel of such a multi\sphinxhyphen{}unit dataset that is not unlike a \dlhbhref{W1M}{mono repo}.
The nesting pattern can be repeated as often as necessary or sensible.
It can be “turtles all the way down”.

\sphinxstepscope

\index{containerized data analysis@\spxentry{containerized data analysis}!with DataLad@\spxentry{with DataLad}}\index{with DataLad@\spxentry{with DataLad}!containerized data analysis@\spxentry{containerized data analysis}}\ignorespaces 

\section{Computational reproducibility with software containers}
\label{\detokenize{basics/101-133-containersrun:computational-reproducibility-with-software-containers}}\label{\detokenize{basics/101-133-containersrun:containersrun}}\label{\detokenize{basics/101-133-containersrun:index-0}}\label{\detokenize{basics/101-133-containersrun::doc}}
\sphinxAtStartPar
Just after submitting your midterm data analysis project, you get together
with your friends. “I’m curious: So what kind of analyses did y’all carry out?”
you ask. The variety of methods and datasets the others used is huge, and
one analysis interests you in particular. Later that day, you decide to
install this particular analysis dataset to learn more about the methods used
in there. However, when you \sphinxcode{\sphinxupquote{datalad rerun}} your friends analysis script,
it throws an error. Hastily, you call her \textendash{} maybe she can quickly fix her
script and resubmit the project with only minor delays. “I don’t know what
you mean”, you hear in return.
“On my machine, everything works fine!”

\sphinxAtStartPar
On its own, DataLad datasets can contain almost anything that is relevant to
ensure reproducibility: Data, code, human\sphinxhyphen{}readable analysis descriptions
(e.g., \sphinxcode{\sphinxupquote{README.md}} files), provenance on the origin of all files
obtained from elsewhere, and machine\sphinxhyphen{}readable records that link generated
outputs to the commands, scripts, and data they were created from.

\sphinxAtStartPar
This however may not be sufficient to ensure that an analysis \sphinxstyleemphasis{reproduces}
(i.e., produces the same or highly similar results), let alone \sphinxstyleemphasis{works} on a
computer different than the one it was initially composed on. This is because
the analysis does not only depend on data and code, but also the
\sphinxstyleemphasis{software environment} that it is conducted in.

\sphinxAtStartPar
A lack of information about the operating system of the computer, the precise
versions of installed software, or their configurations may
make it impossible to replicate your analysis on a different machine, or even
on your own machine once a new software update is installed. Therefore, it is
important to communicate all details about the computational environment for
an analysis as thoroughly as possible. Luckily, DataLad provides an extension
that can link computational environments to datasets, the
\dlhbhref{D1E}{datalad containers}
extension.

\sphinxAtStartPar
This section will give a quick overview on what containers are and
demonstrate how \sphinxcode{\sphinxupquote{datalad\sphinxhyphen{}container}} helps to capture full provenance of an
analysis by linking containers to datasets and analyses.
\begin{importantnote}[before title={\thetcbcounter\ }, check odd page=true]{Install the datalad-container extension}

\sphinxAtStartPar
This section uses the {\hyperref[\detokenize{glossary:term-DataLad-extension}]{\sphinxtermref{\DUrole{xref,std,std-term}{DataLad extension}}}} \sphinxcode{\sphinxupquote{datalad\sphinxhyphen{}container}}.
As other extensions, it is a stand\sphinxhyphen{}alone Python package, and can be installed using {\hyperref[\detokenize{glossary:term-pip}]{\sphinxtermref{\DUrole{xref,std,std-term}{pip}}}}:

\begin{sphinxVerbatim}[commandchars=\\\{\}]
\PYGZdl{}\PYG{+w}{ }pip\PYG{+w}{ }install\PYG{+w}{ }datalad\PYGZhy{}container
\end{sphinxVerbatim}

\sphinxAtStartPar
As with DataLad and other Python packages, you might want to do the installation in a {\hyperref[\detokenize{glossary:term-virtual-environment}]{\sphinxtermref{\DUrole{xref,std,std-term}{virtual environment}}}}.


\end{importantnote}

\index{recipe@\spxentry{recipe}!software container concept@\spxentry{software container concept}}\index{software container concept@\spxentry{software container concept}!recipe@\spxentry{recipe}}\index{image@\spxentry{image}!software container concept@\spxentry{software container concept}}\index{software container concept@\spxentry{software container concept}!image@\spxentry{image}}\index{container@\spxentry{container}!software container concept@\spxentry{software container concept}}\index{software container concept@\spxentry{software container concept}!container@\spxentry{container}}\ignorespaces 

\subsection{Containers}
\label{\detokenize{basics/101-133-containersrun:containers}}\label{\detokenize{basics/101-133-containersrun:index-1}}
\sphinxAtStartPar
To put it simple, computational containers are cut\sphinxhyphen{}down virtual machines that
allow you to package all software libraries and their dependencies (all in the
precise version your analysis requires) into a bundle you can share with
others. On your own and other’s machines, the container constitutes a secluded
software environment that
\begin{itemize}
\item {} 
\sphinxAtStartPar
contains the exact software environment that you specified, ready to run
analyses

\item {} 
\sphinxAtStartPar
does not effect any software outside of the container

\end{itemize}

\sphinxAtStartPar
Unlike virtual machines, software containers do not run a full operating
system on virtualized hardware. Instead, they use basic services of the host operating system
(in a read\sphinxhyphen{}only fashion). This makes them
lightweight and still portable. By sharing software environments with containers,
others (and also yourself) have easy access to the correct software
without the need to modify the software environment of the machine the
container runs on. Thus, containers are ideal to encapsulate the software
environment and share it together with the analysis code and data to ensure
computational reproducibility of your analyses, or to create a suitable
software environment on a computer that you do not have permissions to deploy
software on.

\sphinxAtStartPar
There are a number of different tools to create and use containers, with
\dlhbhref{D9}{Docker} being one of the most well\sphinxhyphen{}known of them.
While being a powerful tool, it is only rarely used on high performance computing
(HPC) infrastructure%
\begin{footnote}\sphinxAtStartFootnote
The main reason why Docker is not deployed on HPC systems is because
it grants users “\dlhbhref{W1V}{superuser privileges}”.
On multi\sphinxhyphen{}user systems such as HPC, users should not have those
privileges, as it would enable them to tamper with other’s or shared
data and resources, posing a severe security threat.
%
\end{footnote}. An alternative is \dlhbhref{S8A}{Singularity}.
Both of these tools share core terminology:
\begin{description}
\sphinxlineitem{{\hyperref[\detokenize{glossary:term-container-recipe}]{\sphinxtermref{\DUrole{xref,std,std-term}{container recipe}}}}}
\sphinxAtStartPar
A text file that lists all required components of the computational environment.
It is made by a human user.

\sphinxlineitem{{\hyperref[\detokenize{glossary:term-container-image}]{\sphinxtermref{\DUrole{xref,std,std-term}{container image}}}}}
\sphinxAtStartPar
This is \sphinxstyleemphasis{built} from the recipe file. It is a static file system inside a file,
populated with the software specified in the recipe, and some initial configuration.

\sphinxlineitem{{\hyperref[\detokenize{glossary:term-container}]{\sphinxtermref{\DUrole{xref,std,std-term}{container}}}}}
\sphinxAtStartPar
A running instance of an image that you can actually use for your computations.
If you want to create and run your own software container, you start by writing
a recipe file and build an image from it. Alternatively, you can can also \sphinxstyleemphasis{pull}
an image built from a publicly shared recipe from the \sphinxstyleemphasis{Hub} of the tool you are using.

\sphinxlineitem{hub}
\sphinxAtStartPar
A storage resource to share and consume images. Examples are
{\hyperref[\detokenize{glossary:term-Singularity-Hub}]{\sphinxtermref{\DUrole{xref,std,std-term}{Singularity\sphinxhyphen{}Hub}}}}, {\hyperref[\detokenize{glossary:term-Docker-Hub}]{\sphinxtermref{\DUrole{xref,std,std-term}{Docker\sphinxhyphen{}Hub}}}}, and \dlhbhref{A2A}{Amazon ECR} which hosts Docker images.

\end{description}

\sphinxAtStartPar
Note that as of now, the \sphinxcode{\sphinxupquote{datalad\sphinxhyphen{}container}} extension supports
Singularity and Docker images.
Singularity furthermore is compatible with Docker \textendash{} you can use
Docker images as a basis for Singularity images, or run Docker images with
Singularity (even without having Docker installed).
See the \textit{Windows-wit}~{\windowswiticoninline}\textit{\ref{ww-docker}} {\hyperref[\detokenize{basics/101-133-containersrun:ww-docker}]{\sphinxcrossref{\DUrole{std,std-ref}{on Docker}}}} (\autopageref*{\detokenize{basics/101-133-containersrun:ww-docker}}) for installation options.
\begin{importantnote}[before title={\thetcbcounter\ }, check odd page=true]{Additional requirement: Singularity}

\sphinxAtStartPar
To use Singularity containers you have to
\dlhbhref{S2B}{install} the software singularity.


\end{importantnote}

\index{installation@\spxentry{installation}!Docker@\spxentry{Docker}}\index{Docker@\spxentry{Docker}!installation@\spxentry{installation}}\index{install Docker@\spxentry{install Docker}!on Windows@\spxentry{on Windows}}\index{on Windows@\spxentry{on Windows}!install Docker@\spxentry{install Docker}}\ignorespaces \begin{windowswit}[label={ww-docker}, before title={\thetcbcounter\ }, float, floatplacement=tb, check odd page=true]{Docker installation}
\label{\detokenize{basics/101-133-containersrun:ww-docker}}

\sphinxAtStartPar
The software singularity is not available for Windows.
Windows users therefore need to install {\hyperref[\detokenize{glossary:term-Docker}]{\sphinxtermref{\DUrole{xref,std,std-term}{Docker}}}}.
The currently recommended way to do so is by installing \dlhbhref{D5A}{Docker Desktop}, and use its “WSL2” backend (a choice one can set during the installation).
In the case of an “outdated WSL kernel version” issue, run \sphinxcode{\sphinxupquote{wsl \sphinxhyphen{}\sphinxhyphen{}update}} in a regular Windows Command Prompt (CMD).
After the installation, run Docker Desktop, and wait several minutes for it to start the Docker engine in the background.
To verify that everything works as it should, run \sphinxcode{\sphinxupquote{docker ps}} in a Windows Command Prompt (CMD).
If it reports an error that asks “Is the docker daemon running?” give it a few more minutes to let Docker Desktop start it.
If it can’t find the docker command, something went wrong during installation.


\end{windowswit}

\index{containers\sphinxhyphen{}add@\spxentry{containers\sphinxhyphen{}add}!DataLad command@\spxentry{DataLad command}}\index{DataLad command@\spxentry{DataLad command}!containers\sphinxhyphen{}add@\spxentry{containers\sphinxhyphen{}add}}\index{containers\sphinxhyphen{}run@\spxentry{containers\sphinxhyphen{}run}!DataLad command@\spxentry{DataLad command}}\index{DataLad command@\spxentry{DataLad command}!containers\sphinxhyphen{}run@\spxentry{containers\sphinxhyphen{}run}}\ignorespaces 

\subsection{Using \sphinxstyleliteralintitle{\sphinxupquote{datalad containers}}}
\label{\detokenize{basics/101-133-containersrun:using-datalad-containers}}\label{\detokenize{basics/101-133-containersrun:index-3}}
\sphinxAtStartPar
One core feature of the \sphinxcode{\sphinxupquote{datalad containers}} extension is that it registers
computational containers with a dataset. This is done with the
\sphinxcode{\sphinxupquote{datalad containers\sphinxhyphen{}add}} command.
Once a container is registered, arbitrary commands can be executed inside of
it, i.e., in the precise software environment the container encapsulates. All it
needs for this it to swap the \sphinxcode{\sphinxupquote{datalad run}} command introduced in
section {\hyperref[\detokenize{basics/101-108-run:run}]{\sphinxcrossref{\DUrole{std,std-ref}{Keeping track}}}} (\autopageref*{\detokenize{basics/101-108-run:run}}) with the \sphinxcode{\sphinxupquote{datalad containers\sphinxhyphen{}run}} command.

\sphinxAtStartPar
Let’s see this in action for the \sphinxcode{\sphinxupquote{midterm\_analysis}} dataset by rerunning
the analysis you did for the midterm project within a Singularity container.
We start by registering a container to the dataset.
For this, we will pull an image from Singularity hub. This image was made
for the online\sphinxhyphen{}handbook, and it contains the relevant Python setup for
the analysis. Its recipe lives in the online\sphinxhyphen{}handbook’s
\dlhbhref{G2K}{resources repository}.
If you are curious how to create a Singularity image, the \textit{Find-out-more}~{\findoutmoreiconinline}\textit{\ref{fom-container-creation}} {\hyperref[\detokenize{basics/101-133-containersrun:fom-container-creation}]{\sphinxcrossref{\DUrole{std,std-ref}{on this topic}}}} (\autopageref*{\detokenize{basics/101-133-containersrun:fom-container-creation}}) has some pointers:

\index{build container image@\spxentry{build container image}!with Singularity@\spxentry{with Singularity}}\index{with Singularity@\spxentry{with Singularity}!build container image@\spxentry{build container image}}\ignorespaces \begin{findoutmore}[label={fom-container-creation}, before title={\thetcbcounter\ }, float, floatplacement=tb, check odd page=true]{How to make a Singularity image}
\label{\detokenize{basics/101-133-containersrun:fom-container-creation}}

\sphinxAtStartPar
Singularity containers are build from image files, often
called “recipes”, that hold a “definition” of the software container and its
contents and components. The
\dlhbhref{S2A}{singularity documentation}
has its own tutorial on how to build such images from scratch.
An alternative to writing the image file by hand is to use
\dlhbhref{G2A}{Neurodocker}. This
command\sphinxhyphen{}line program can help you generate custom Singularity recipes (and
also \sphinxcode{\sphinxupquote{Dockerfiles}}, from which Docker images are built). A wonderful tutorial
on how to use Neurodocker is
\dlhbhref{G14A}{this introduction}
by Michael Notter.

\sphinxAtStartPar
Once a recipe exists, the command

\begin{sphinxVerbatim}[commandchars=\\\{\}]
\PYG{g+gp}{\PYGZdl{} }sudo\PYG{+w}{ }singularity\PYG{+w}{ }build\PYG{+w}{ }\PYGZlt{}NAME\PYGZgt{}\PYG{+w}{ }\PYGZlt{}RECIPE\PYGZgt{}
\end{sphinxVerbatim}

\sphinxAtStartPar
will build a container (called \sphinxcode{\sphinxupquote{\textless{}NAME\textgreater{}}}) from the recipe. Note that this
command requires \sphinxcode{\sphinxupquote{root}} privileges (”\sphinxcode{\sphinxupquote{sudo}}”). You can build the container
on any machine, though, not necessarily the one that is later supposed to
actually run the analysis, e.g., your own laptop versus a compute cluster.


\end{findoutmore}

\index{add container image to dataset@\spxentry{add container image to dataset}!with DataLad@\spxentry{with DataLad}}\index{with DataLad@\spxentry{with DataLad}!add container image to dataset@\spxentry{add container image to dataset}}\ignorespaces 
\sphinxAtStartPar
The \sphinxcode{\sphinxupquote{datalad containers\sphinxhyphen{}add}} command takes an arbitrary
name to give to the container, and a path or URL to a container image:

\begin{sphinxVerbatim}[commandchars=\\\{\}]
\PYG{g+gp}{\PYGZdl{} }\PYG{c+c1}{\PYGZsh{} we are in the midterm\PYGZus{}project subdataset}
\PYG{g+gp}{\PYGZdl{} }datalad\PYG{+w}{ }containers\PYGZhy{}add\PYG{+w}{ }midterm\PYGZhy{}software\PYG{+w}{ }\PYGZhy{}\PYGZhy{}url\PYG{+w}{ }shub://adswa/resources:2
\PYG{g+go}{[INFO] Initializing special remote datalad}
\PYG{g+go}{add(ok): .datalad/config (file)}
\PYG{g+go}{save(ok): . (dataset)}
\PYG{g+go}{add(ok): .datalad/config (file)}
\PYG{g+go}{save(ok): . (dataset)}
\PYG{g+go}{containers\PYGZus{}add(ok): /home/me/dl\PYGZhy{}101/DataLad\PYGZhy{}101/midterm\PYGZus{}project/.datalad/environments/midterm\PYGZhy{}software/image (file)}
\end{sphinxVerbatim}

\index{hub@\spxentry{hub}!Docker@\spxentry{Docker}}\index{Docker@\spxentry{Docker}!hub@\spxentry{hub}}\ignorespaces \begin{findoutmore}[label={fom-container-add}, before title={\thetcbcounter\ }, float, floatplacement=tb, check odd page=true]{How do I add an image from Docker-Hub, Amazon ECR, or a local container?}
\label{\detokenize{basics/101-133-containersrun:fom-container-add}}

\sphinxAtStartPar
Should the image you want to use sit on Dockerhub, specify the \sphinxcode{\sphinxupquote{\sphinxhyphen{}\sphinxhyphen{}url}}
option prefixed with \sphinxcode{\sphinxupquote{docker://}} or \sphinxcode{\sphinxupquote{dhub://}} instead of \sphinxcode{\sphinxupquote{shub://}}:

\begin{sphinxVerbatim}[commandchars=\\\{\}]
\PYG{g+gp}{\PYGZdl{} }datalad\PYG{+w}{ }containers\PYGZhy{}add\PYG{+w}{ }midterm\PYGZhy{}software\PYG{+w}{ }\PYGZhy{}\PYGZhy{}url\PYG{+w}{ }docker://adswa/resources:2
\end{sphinxVerbatim}

\sphinxAtStartPar
If your image lives on Amazon ECR, use a \sphinxcode{\sphinxupquote{dhub://}} prefix followed by the AWS ECR URL as in

\begin{sphinxVerbatim}[commandchars=\\\{\}]
\PYG{g+gp}{\PYGZdl{} }datalad\PYG{+w}{ }containers\PYGZhy{}add\PYG{+w}{ }\PYGZhy{}\PYGZhy{}url\PYG{+w}{ }dhub://12345678.dkr.ecr.us\PYGZhy{}west\PYGZhy{}2.amazonaws.com/maze\PYGZhy{}code/data\PYGZhy{}import:latest\PYG{+w}{ }data\PYGZhy{}import
\end{sphinxVerbatim}

\sphinxAtStartPar
If you want to add a container that exists locally, specify the path to it
like this:

\begin{sphinxVerbatim}[commandchars=\\\{\}]
\PYG{g+gp}{\PYGZdl{} }datalad\PYG{+w}{ }containers\PYGZhy{}add\PYG{+w}{ }midterm\PYGZhy{}software\PYG{+w}{ }\PYGZhy{}\PYGZhy{}url\PYG{+w}{ }path/to/container
\end{sphinxVerbatim}


\end{findoutmore}

\sphinxAtStartPar
This command downloaded the container from Singularity Hub, added it to
the \sphinxcode{\sphinxupquote{midterm\_project}} dataset.
This is not the only possible source for containers.
Check the \textit{Find-out-more}~{\findoutmoreiconinline}\textit{\ref{fom-container-add}} {\hyperref[\detokenize{basics/101-133-containersrun:fom-container-add}]{\sphinxcrossref{\DUrole{std,std-ref}{on more ways to add images}}}} (\autopageref*{\detokenize{basics/101-133-containersrun:fom-container-add}}).
Once added, containers can be listed or removed.
The \textit{Find-out-more}~{\findoutmoreiconinline}\textit{\ref{fom-container-remove}} {\hyperref[\detokenize{basics/101-133-containersrun:fom-container-remove}]{\sphinxcrossref{\DUrole{std,std-ref}{on this topic}}}} (\autopageref*{\detokenize{basics/101-133-containersrun:fom-container-remove}}) shows how.

\sphinxAtStartPar
Adding a container recorded basic information on the
container under its name “midterm\sphinxhyphen{}software” in the dataset’s configuration at
\sphinxcode{\sphinxupquote{.datalad/config}}. You can find out more about them in an upcoming dedicated section on these additional configurations.
Such configurations can, among other things, be important to ensure correct container invocation on specific systems or across systems.
One example is \sphinxstyleemphasis{bind\sphinxhyphen{}mounting} directories into containers, i.e., making a specific directory and its contents available inside a container.
Different containerization software (versions) or configurations of those determine \sphinxstyleemphasis{default bind\sphinxhyphen{}mounts} on a given system.
Thus, depending on the system and the location of the dataset on this system, a shared dataset may be automatically bind\sphinxhyphen{}mounted or not.
To ensure that the dataset is correctly bind\sphinxhyphen{}mounted on all systems, let’s add a call\sphinxhyphen{}format specification with a bind\sphinxhyphen{}mount to the current working directory.

\index{configuration.item@\spxentry{configuration.item}!datalad.containers.\textless{}name\textgreater{}.cmdexec@\spxentry{datalad.containers.\textless{}name\textgreater{}.cmdexec}}\ignorespaces 
\def\sphinxLiteralBlockLabel{\label{\detokenize{basics/101-133-containersrun:index-7}}}
\begin{sphinxVerbatim}[commandchars=\\\{\}]
\PYG{g+gp}{\PYGZdl{} }git\PYG{+w}{ }config\PYG{+w}{ }\PYGZhy{}f\PYG{+w}{ }.datalad/config\PYG{+w}{ }datalad.containers.midterm\PYGZhy{}software.cmdexec\PYG{+w}{ }\PYG{l+s+s1}{\PYGZsq{}singularity exec \PYGZhy{}B \PYGZob{}\PYGZob{}pwd\PYGZcb{}\PYGZcb{} \PYGZob{}img\PYGZcb{} \PYGZob{}cmd\PYGZcb{}\PYGZsq{}}
\PYG{g+gp}{\PYGZdl{} }datalad\PYG{+w}{ }save\PYG{+w}{ }\PYGZhy{}m\PYG{+w}{ }\PYG{l+s+s2}{\PYGZdq{}Modify the container call format to bind\PYGZhy{}mount the working directory\PYGZdq{}}
\PYG{g+go}{add(ok): .datalad/config (file)}
\PYG{g+go}{save(ok): . (dataset)}
\end{sphinxVerbatim}

\index{run command with provenance capture@\spxentry{run command with provenance capture}!with DataLad@\spxentry{with DataLad}}\index{with DataLad@\spxentry{with DataLad}!run command with provenance capture@\spxentry{run command with provenance capture}}\index{run command@\spxentry{run command}!with DataLad containers\sphinxhyphen{}run@\spxentry{with DataLad containers\sphinxhyphen{}run}}\index{with DataLad containers\sphinxhyphen{}run@\spxentry{with DataLad containers\sphinxhyphen{}run}!run command@\spxentry{run command}}\ignorespaces 
\sphinxAtStartPar
Now that we have a complete computational environment linked to the \sphinxcode{\sphinxupquote{midterm\_project}}
dataset, we can execute commands in this environment. Let us, for example, try to repeat
the \sphinxcode{\sphinxupquote{datalad run}} command from the section {\hyperref[\detokenize{basics/101-130-yodaproject:yoda-project}]{\sphinxcrossref{\DUrole{std,std-ref}{YODA\sphinxhyphen{}compliant data analysis projects}}}} (\autopageref*{\detokenize{basics/101-130-yodaproject:yoda-project}}) as a
\sphinxcode{\sphinxupquote{datalad containers\sphinxhyphen{}run}} command.

\sphinxAtStartPar
The previous \sphinxcode{\sphinxupquote{run}} command looked like this:

\begin{sphinxVerbatim}[commandchars=\\\{\}]
\PYG{g+gp}{\PYGZdl{} }datalad\PYG{+w}{ }run\PYG{+w}{ }\PYGZhy{}m\PYG{+w}{ }\PYG{l+s+s2}{\PYGZdq{}analyze iris data with classification analysis\PYGZdq{}}\PYG{+w}{ }\PYG{l+s+se}{\PYGZbs{}}
\PYG{+w}{  }\PYGZhy{}\PYGZhy{}input\PYG{+w}{ }\PYG{l+s+s2}{\PYGZdq{}input/iris.csv\PYGZdq{}}\PYG{+w}{ }\PYG{l+s+se}{\PYGZbs{}}
\PYG{+w}{  }\PYGZhy{}\PYGZhy{}output\PYG{+w}{ }\PYG{l+s+s2}{\PYGZdq{}pairwise\PYGZus{}relationships.png\PYGZdq{}}\PYG{+w}{ }\PYG{l+s+se}{\PYGZbs{}}
\PYG{+w}{  }\PYGZhy{}\PYGZhy{}output\PYG{+w}{ }\PYG{l+s+s2}{\PYGZdq{}prediction\PYGZus{}report.csv\PYGZdq{}}\PYG{+w}{ }\PYG{l+s+se}{\PYGZbs{}}
\PYG{+w}{  }\PYG{l+s+s2}{\PYGZdq{}python3 code/script.py \PYGZob{}inputs\PYGZcb{} \PYGZob{}outputs\PYGZcb{}\PYGZdq{}}
\end{sphinxVerbatim}

\sphinxAtStartPar
How would it look like as a \sphinxcode{\sphinxupquote{containers\sphinxhyphen{}run}} command?

\begin{sphinxVerbatim}[commandchars=\\\{\}]
\PYG{g+gp}{\PYGZdl{} }datalad\PYG{+w}{ }containers\PYGZhy{}run\PYG{+w}{ }\PYGZhy{}m\PYG{+w}{ }\PYG{l+s+s2}{\PYGZdq{}rerun analysis in container\PYGZdq{}}\PYG{+w}{ }\PYG{l+s+se}{\PYGZbs{}}
\PYG{+w}{  }\PYGZhy{}\PYGZhy{}container\PYGZhy{}name\PYG{+w}{ }midterm\PYGZhy{}software\PYG{+w}{ }\PYG{l+s+se}{\PYGZbs{}}
\PYG{+w}{  }\PYGZhy{}\PYGZhy{}input\PYG{+w}{ }\PYG{l+s+s2}{\PYGZdq{}input/iris.csv\PYGZdq{}}\PYG{+w}{ }\PYG{l+s+se}{\PYGZbs{}}
\PYG{+w}{  }\PYGZhy{}\PYGZhy{}output\PYG{+w}{ }\PYG{l+s+s2}{\PYGZdq{}pairwise\PYGZus{}relationships.png\PYGZdq{}}\PYG{+w}{ }\PYG{l+s+se}{\PYGZbs{}}
\PYG{+w}{  }\PYGZhy{}\PYGZhy{}output\PYG{+w}{ }\PYG{l+s+s2}{\PYGZdq{}prediction\PYGZus{}report.csv\PYGZdq{}}\PYG{+w}{ }\PYG{l+s+se}{\PYGZbs{}}
\PYG{+w}{  }\PYG{l+s+s2}{\PYGZdq{}python3 code/script.py \PYGZob{}inputs\PYGZcb{} \PYGZob{}outputs\PYGZcb{}\PYGZdq{}}
\PYG{g+go}{unlock(ok): pairwise\PYGZus{}relationships.png (file)}
\PYG{g+go}{unlock(ok): prediction\PYGZus{}report.csv (file)}
\PYG{g+go}{[INFO] == Command start (output follows) =====}
\PYG{g+go}{[INFO] == Command exit (modification check follows) =====}
\PYG{g+go}{run(ok): /home/me/dl\PYGZhy{}101/DataLad\PYGZhy{}101/midterm\PYGZus{}project (dataset) [singularity exec \PYGZhy{}B /home/me/dl\PYGZhy{}101/Data...]}
\PYG{g+go}{add(ok): pairwise\PYGZus{}relationships.png (file)}
\PYG{g+go}{add(ok): prediction\PYGZus{}report.csv (file)}
\PYG{g+go}{save(ok): . (dataset)}
\PYG{g+go}{action summary:}
\PYG{g+go}{  add (ok: 2)}
\PYG{g+go}{  get (notneeded: 4)}
\PYG{g+go}{  run (ok: 1)}
\PYG{g+go}{  save (notneeded: 1, ok: 1)}
\PYG{g+go}{  unlock (ok: 2)}
\end{sphinxVerbatim}

\sphinxAtStartPar
Almost exactly like a \sphinxcode{\sphinxupquote{datalad run}} command! The only additional parameter
is \sphinxcode{\sphinxupquote{container\sphinxhyphen{}name}}. At this point, though, the \sphinxcode{\sphinxupquote{\sphinxhyphen{}\sphinxhyphen{}container\sphinxhyphen{}name}}
flag is even \sphinxstyleemphasis{optional} because there is only a single container registered to the dataset.
But if your dataset contains more than one container you will \sphinxstyleemphasis{need} to specify
the name of the container you want to use in your command.
The complete command’s structure looks like this:

\begin{sphinxVerbatim}[commandchars=\\\{\}]
\PYG{g+gp}{\PYGZdl{} }datalad\PYG{+w}{ }containers\PYGZhy{}run\PYG{+w}{ }\PYGZhy{}\PYGZhy{}name\PYG{+w}{ }\PYGZlt{}containername\PYGZgt{}\PYG{+w}{ }\PYG{o}{[}\PYGZhy{}m\PYG{+w}{ }...\PYG{o}{]}\PYG{+w}{ }\PYG{o}{[}\PYGZhy{}\PYGZhy{}input\PYG{+w}{ }...\PYG{o}{]}\PYG{+w}{ }\PYG{o}{[}\PYGZhy{}\PYGZhy{}output\PYG{+w}{ }...\PYG{o}{]}\PYG{+w}{ }\PYGZlt{}COMMAND\PYGZgt{}
\end{sphinxVerbatim}

\index{containers\sphinxhyphen{}remove@\spxentry{containers\sphinxhyphen{}remove}!DataLad command@\spxentry{DataLad command}}\index{DataLad command@\spxentry{DataLad command}!containers\sphinxhyphen{}remove@\spxentry{containers\sphinxhyphen{}remove}}\index{containers\sphinxhyphen{}list@\spxentry{containers\sphinxhyphen{}list}!DataLad command@\spxentry{DataLad command}}\index{DataLad command@\spxentry{DataLad command}!containers\sphinxhyphen{}list@\spxentry{containers\sphinxhyphen{}list}}\index{list known containers@\spxentry{list known containers}!with DataLad@\spxentry{with DataLad}}\index{with DataLad@\spxentry{with DataLad}!list known containers@\spxentry{list known containers}}\ignorespaces \begin{findoutmore}[label={fom-container-remove}, before title={\thetcbcounter\ }, float, floatplacement=tb, check odd page=true]{How can I list available containers or remove them?}
\label{\detokenize{basics/101-133-containersrun:fom-container-remove}}

\sphinxAtStartPar
The command \sphinxcode{\sphinxupquote{datalad containers\sphinxhyphen{}list}} will list all containers in
the current dataset:

\begin{sphinxVerbatim}[commandchars=\\\{\}]
\PYG{g+gp}{\PYGZdl{} }datalad\PYG{+w}{ }containers\PYGZhy{}list
\PYG{g+go}{midterm\PYGZhy{}software \PYGZhy{}\PYGZgt{} .datalad/environments/midterm\PYGZhy{}software/image}
\end{sphinxVerbatim}

\sphinxAtStartPar
The command \sphinxcode{\sphinxupquote{datalad containers\sphinxhyphen{}remove}} will remove a container
from the dataset, if there exists a container with name given to the
command. Note that this will remove not only the image from the dataset,
but also the configuration for it in \sphinxcode{\sphinxupquote{.datalad/config}}.


\end{findoutmore}

\sphinxAtStartPar
Here is how the history entry looks like:

\begin{sphinxVerbatim}[commandchars=\\\{\}]
\PYG{g+gp}{\PYGZdl{} }git\PYG{+w}{ }log\PYG{+w}{ }\PYGZhy{}p\PYG{+w}{ }\PYGZhy{}n\PYG{+w}{ }\PYG{l+m}{1}
\PYG{g+go}{commit 4f00ad07✂SHA1}
\PYG{g+go}{Author: Elena Piscopia \PYGZlt{}elena@example.net\PYGZgt{}}
\PYG{g+go}{Date:   Tue Jun 18 16:13:00 2019 +0000}

\PYG{g+go}{    [DATALAD RUNCMD] rerun analysis in container}

\PYG{g+go}{    === Do not change lines below ===}
\PYG{g+go}{    \PYGZob{}}
\PYG{g+go}{     \PYGZdq{}chain\PYGZdq{}: [],}
\PYG{g+go}{     \PYGZdq{}cmd\PYGZdq{}: \PYGZdq{}singularity exec \PYGZhy{}B \PYGZob{}pwd\PYGZcb{} .datalad/environments/midterm\PYGZhy{}software/image python3 code/script.py \PYGZob{}inputs\PYGZcb{} \PYGZob{}outputs\PYGZcb{}\PYGZdq{},}
\PYG{g+go}{     \PYGZdq{}dsid\PYGZdq{}: \PYGZdq{}d95bafc8\PYGZhy{}f2a4\PYGZhy{}d27b\PYGZhy{}dcf4\PYGZhy{}bb99f4bea973\PYGZdq{},}
\PYG{g+go}{     \PYGZdq{}exit\PYGZdq{}: 0,}
\PYG{g+go}{     \PYGZdq{}extra\PYGZus{}inputs\PYGZdq{}: [}
\PYG{g+go}{      \PYGZdq{}.datalad/environments/midterm\PYGZhy{}software/image\PYGZdq{}}
\PYG{g+go}{     ],}
\PYG{g+go}{     \PYGZdq{}inputs\PYGZdq{}: [}
\PYG{g+go}{      \PYGZdq{}input/iris.csv\PYGZdq{}}
\PYG{g+go}{     ],}
\PYG{g+go}{     \PYGZdq{}outputs\PYGZdq{}: [}
\PYG{g+go}{      \PYGZdq{}pairwise\PYGZus{}relationships.png\PYGZdq{},}
\PYG{g+go}{      \PYGZdq{}prediction\PYGZus{}report.csv\PYGZdq{}}
\PYG{g+go}{     ],}
\PYG{g+go}{     \PYGZdq{}pwd\PYGZdq{}: \PYGZdq{}.\PYGZdq{}}
\PYG{g+go}{    \PYGZcb{}}
\PYG{g+go}{    \PYGZca{}\PYGZca{}\PYGZca{} Do not change lines above \PYGZca{}\PYGZca{}\PYGZca{}}

\PYG{g+go}{diff \PYGZhy{}\PYGZhy{}git a/pairwise\PYGZus{}relationships.png b/pairwise\PYGZus{}relationships.png}
\PYG{g+go}{index a24e6b9..963d5a8 120000}
\PYG{g+go}{\PYGZhy{}\PYGZhy{}\PYGZhy{} a/pairwise\PYGZus{}relationships.png}
\PYG{g+go}{+++ b/pairwise\PYGZus{}relationships.png}
\PYG{g+go}{@@ \PYGZhy{}1 +1 @@}
\PYG{g+go}{\PYGZhy{}.git/annex/objects/G3/Mg/✂/MD5E\PYGZhy{}s260649\PYGZhy{}\PYGZhy{}127313ad✂MD5.png}
\PYG{g+go}{\PYGZbs{} No newline at end of file}
\PYG{g+go}{+.git/annex/objects/q1/gp/✂/MD5E\PYGZhy{}s261062\PYGZhy{}\PYGZhy{}025dc493✂MD5.png}
\PYG{g+go}{\PYGZbs{} No newline at end of file}
\end{sphinxVerbatim}

\sphinxAtStartPar
If you would \sphinxcode{\sphinxupquote{datalad rerun}} this commit, it would be re\sphinxhyphen{}executed in the
software container registered to the dataset. If you would share the dataset
with a friend and they would \sphinxcode{\sphinxupquote{datalad rerun}} this commit, the image would first
be obtained from its registered url, and thus your
friend can obtain the correct execution environment automatically.

\sphinxAtStartPar
Note that because this new \sphinxcode{\sphinxupquote{datalad containers\sphinxhyphen{}run}} command modified the
\sphinxcode{\sphinxupquote{midterm\_project}} subdirectory, we need to also save
the most recent state of the subdataset to the superdataset \sphinxcode{\sphinxupquote{DataLad\sphinxhyphen{}101}}.

\begin{sphinxVerbatim}[commandchars=\\\{\}]
\PYG{g+gp}{\PYGZdl{} }\PYG{n+nb}{cd}\PYG{+w}{ }../
\PYG{g+gp}{\PYGZdl{} }datalad\PYG{+w}{ }status
\PYG{g+go}{ modified: midterm\PYGZus{}project (dataset)}
\end{sphinxVerbatim}

\begin{sphinxVerbatim}[commandchars=\\\{\}]
\PYG{g+gp}{\PYGZdl{} }datalad\PYG{+w}{ }save\PYG{+w}{ }\PYGZhy{}d\PYG{+w}{ }.\PYG{+w}{ }\PYGZhy{}m\PYG{+w}{ }\PYG{l+s+s2}{\PYGZdq{}add container and execute analysis within container\PYGZdq{}}\PYG{+w}{ }midterm\PYGZus{}project
\PYG{g+go}{add(ok): midterm\PYGZus{}project (dataset)}
\PYG{g+go}{add(ok): .gitmodules (file)}
\PYG{g+go}{save(ok): . (dataset)}
\end{sphinxVerbatim}

\sphinxAtStartPar
Software containers, the \sphinxcode{\sphinxupquote{datalad\sphinxhyphen{}container}} extension, and DataLad thus work well together
to make your analysis completely reproducible \textendash{} by not only linking code, data,
and outputs, but also the software environment of an analysis. And this does not
only benefit your future self, but also whomever you share your dataset with, as
the information about the container is shared together with the dataset. How cool
is that?

\index{DataLad concept@\spxentry{DataLad concept}!container image registration@\spxentry{container image registration}}\index{container image registration@\spxentry{container image registration}!DataLad concept@\spxentry{DataLad concept}}\ignorespaces 
\sphinxAtStartPar
What changes in .datalad/config when one adds a container?

\begin{sphinxVerbatim}[commandchars=\\\{\}]
\PYG{g+gp}{\PYGZdl{} }cat\PYG{+w}{ }.datalad/config
\PYG{g+go}{[datalad \PYGZdq{}dataset\PYGZdq{}]}
\PYG{g+go}{	id = d95bafc8\PYGZhy{}f2a4\PYGZhy{}d27b\PYGZhy{}dcf4\PYGZhy{}bb99f4bea973}
\PYG{g+go}{[datalad \PYGZdq{}containers.midterm\PYGZhy{}software\PYGZdq{}]}
\PYG{g+go}{	image = .datalad/environments/midterm\PYGZhy{}software/image}
\PYG{g+go}{	cmdexec = singularity exec \PYGZob{}img\PYGZcb{} \PYGZob{}cmd\PYGZcb{}}
\end{sphinxVerbatim}

\sphinxAtStartPar
This recorded the image’s origin on Singularity\sphinxhyphen{}Hub, the location of the
image in the dataset under \sphinxcode{\sphinxupquote{.datalad/environments/\textless{}NAME\textgreater{}/image}}, and it
specifies the way in which the container should be used: The line

\begin{sphinxVerbatim}[commandchars=\\\{\}]
\PYG{n+na}{cmdexec}\PYG{+w}{ }\PYG{o}{=}\PYG{+w}{ }\PYG{l+s}{singularity exec \PYGZob{}img\PYGZcb{} \PYGZob{}cmd\PYGZcb{}}
\end{sphinxVerbatim}

\sphinxAtStartPar
can be read as: “If this container is used, take the \sphinxcode{\sphinxupquote{cmd}} (what you wrap in a
\sphinxcode{\sphinxupquote{datalad containers\sphinxhyphen{}run}} command) and plug it into a
\sphinxcode{\sphinxupquote{ singularity exec}} command. The mode of calling Singularity,
namely \sphinxcode{\sphinxupquote{exec}}, means that the command will be executed inside of the container.

\sphinxAtStartPar
You can configure this call format by modifying it in the config file, or calling \sphinxcode{\sphinxupquote{datalad containers\sphinxhyphen{}add}} with the option \sphinxcode{\sphinxupquote{\sphinxhyphen{}\sphinxhyphen{}call\sphinxhyphen{}fmt \textless{}alternative format\textgreater{}}}.
This can be useful to, for example, automatically bind\sphinxhyphen{}mount the current working directory in the container.
In the alternative call format, the placeholders \sphinxcode{\sphinxupquote{\{img\}}}, \sphinxcode{\sphinxupquote{\{cmd\}}}, and \sphinxcode{\sphinxupquote{\{img\_dspath\}}} (a relative path to the dataset containing the image) are available.
In all other cases with variables that use curly brackets, you need to escape them with another curly bracket.
Here is an example call format that bind\sphinxhyphen{}mounts the current working directory (and thus the dataset) automatically:

\begin{sphinxVerbatim}[commandchars=\\\{\}]
\PYG{g+gp}{\PYGZdl{} }datalad\PYG{+w}{ }containers\PYGZhy{}add\PYG{+w}{ }\PYGZhy{}\PYGZhy{}call\PYGZhy{}fmt\PYG{+w}{ }\PYG{l+s+s1}{\PYGZsq{}singularity exec \PYGZhy{}B \PYGZob{}\PYGZob{}pwd\PYGZcb{}\PYGZcb{} \PYGZhy{}\PYGZhy{}cleanenv \PYGZob{}img\PYGZcb{} \PYGZob{}cmd\PYGZcb{}\PYGZsq{}}
\end{sphinxVerbatim}

\sphinxAtStartPar
Note that the image is saved under \sphinxcode{\sphinxupquote{.datalad/environments}} and the
configuration is done in \sphinxcode{\sphinxupquote{.datalad/config}} \textendash{} as these files are version
controlled and shared with together with a dataset, your software
container and the information where it can be reobtained from are linked
to your dataset.

\sphinxAtStartPar
This is how the \sphinxcode{\sphinxupquote{containers\sphinxhyphen{}add}} command is recorded in your history:

\begin{sphinxVerbatim}[commandchars=\\\{\}]
\PYG{g+gp}{\PYGZdl{} }git\PYG{+w}{ }log\PYG{+w}{ }\PYGZhy{}n\PYG{+w}{ }\PYG{l+m}{1}\PYG{+w}{ }\PYGZhy{}p
\PYG{g+go}{commit 54aad5de✂SHA1}
\PYG{g+go}{Author: Elena Piscopia \PYGZlt{}elena@example.net\PYGZgt{}}
\PYG{g+go}{Date:   Tue Jun 18 16:13:00 2019 +0000}

\PYG{g+go}{    [DATALAD] Configure containerized environment \PYGZsq{}midterm\PYGZhy{}software\PYGZsq{}}

\PYG{g+go}{diff \PYGZhy{}\PYGZhy{}git a/.datalad/config b/.datalad/config}
\PYG{g+go}{index e99ec14..ad3e5d8 100644}
\PYG{g+go}{\PYGZhy{}\PYGZhy{}\PYGZhy{} a/.datalad/config}
\PYG{g+go}{+++ b/.datalad/config}
\PYG{g+go}{@@ \PYGZhy{}1,2 +1,5 @@}
\PYG{g+go}{ [datalad \PYGZdq{}dataset\PYGZdq{}]}
\PYG{g+go}{ 	id = d95bafc8\PYGZhy{}f2a4\PYGZhy{}d27b\PYGZhy{}dcf4\PYGZhy{}bb99f4bea973}
\PYG{g+go}{+[datalad \PYGZdq{}containers.midterm\PYGZhy{}software\PYGZdq{}]}
\PYG{g+go}{+	image = .datalad/environments/midterm\PYGZhy{}software/image}
\PYG{g+go}{+	cmdexec = singularity exec \PYGZob{}img\PYGZcb{} \PYGZob{}cmd\PYGZcb{}}
\PYG{g+go}{diff \PYGZhy{}\PYGZhy{}git a/.datalad/environments/midterm\PYGZhy{}software/image b/.datalad/environments/midterm\PYGZhy{}software/image}
\PYG{g+go}{new file mode 120000}
\PYG{g+go}{index 0000000..75c8b41}
\PYG{g+go}{\PYGZhy{}\PYGZhy{}\PYGZhy{} /dev/null}
\PYG{g+go}{+++ b/.datalad/environments/midterm\PYGZhy{}software/image}
\PYG{g+go}{@@ \PYGZhy{}0,0 +1 @@}
\PYG{g+go}{+../../../.git/annex/objects/F1/K3/✂/MD5E\PYGZhy{}s230694943\PYGZhy{}\PYGZhy{}944b0300✂MD5}
\PYG{g+go}{\PYGZbs{} No newline at end of file}
\end{sphinxVerbatim}

\sphinxstepscope


\section{Summary}
\label{\detokenize{basics/101-134-summary:summary}}\label{\detokenize{basics/101-134-summary:summary-containers}}\label{\detokenize{basics/101-134-summary::doc}}
\sphinxAtStartPar
The last two sections have first of all extended your knowledge on dataset nesting:
\begin{itemize}
\item {} 
\sphinxAtStartPar
When subdatasets are created or installed, they are registered to the superdataset
in their current version state (as identified by their most recent commit’s hash).
For a freshly created subdatasets, the most recent commit is at the same time its
first commit.

\item {} 
\sphinxAtStartPar
Once the subdataset evolves, the superdataset recognizes this as a \sphinxcode{\sphinxupquote{modification}}
of the subdatasets version state. If you want to record this, you need to
\sphinxcode{\sphinxupquote{datalad save}} it in the superdataset:

\begin{sphinxVerbatim}[commandchars=\\\{\}]
\PYG{g+gp}{\PYGZdl{} }datalad\PYG{+w}{ }save\PYG{+w}{ }\PYGZhy{}m\PYG{+w}{ }\PYG{l+s+s2}{\PYGZdq{}a short summary of changes in subds\PYGZdq{}}\PYG{+w}{ }\PYGZlt{}path\PYG{+w}{ }to\PYG{+w}{ }subds\PYGZgt{}
\end{sphinxVerbatim}

\end{itemize}

\sphinxAtStartPar
But more than nesting concepts, they have also extended your knowledge on
reproducible analyses with \sphinxcode{\sphinxupquote{datalad run}} and you have experienced
for yourself why and how software containers can go hand\sphinxhyphen{}in\sphinxhyphen{}hand with DataLad:
\begin{itemize}
\item {} 
\sphinxAtStartPar
A software container encapsulates a complete software environment, independent
from the environment of the computer it runs on. This allows you to create or
use secluded software and also share it together with your analysis to ensure
computational reproducibility. The DataLad extension
\dlhbhref{D1E}{datalad containers}
can make this possible.

\item {} 
\sphinxAtStartPar
The command \sphinxcode{\sphinxupquote{datalad containers\sphinxhyphen{}add}} registers an {\hyperref[\detokenize{glossary:term-container-image}]{\sphinxtermref{\DUrole{xref,std,std-term}{container image}}}} from a path or
URL to your dataset.

\item {} 
\sphinxAtStartPar
If you use \sphinxcode{\sphinxupquote{datalad containers\sphinxhyphen{}run}} instead of \sphinxcode{\sphinxupquote{datalad run}},
you can reproducibly execute a command of your choice \sphinxstyleemphasis{within} the software
environment.

\item {} 
\sphinxAtStartPar
A \sphinxcode{\sphinxupquote{datalad rerun}} of a commit produced with \sphinxcode{\sphinxupquote{datalad containers\sphinxhyphen{}run}}
will re\sphinxhyphen{}execute the command in the same software environment.

\end{itemize}

\index{hub@\spxentry{hub}!Docker@\spxentry{Docker}}\index{Docker@\spxentry{Docker}!hub@\spxentry{hub}}\ignorespaces 

\subsection{Now what can I do with it?}
\label{\detokenize{basics/101-134-summary:now-what-can-i-do-with-it}}\label{\detokenize{basics/101-134-summary:index-0}}
\sphinxAtStartPar
For one, you will not be surprised if you ever see a subdataset being shown as
\sphinxcode{\sphinxupquote{modified}} by \sphinxcode{\sphinxupquote{datalad status}}: You now know that if a subdataset
evolves, it’s most recent state needs to be explicitly saved to the superdatasets
history.

\sphinxAtStartPar
On a different matter, you are now able to capture and share analysis provenance that
includes the relevant software environment. This does not only make your analyses
projects automatically reproducible, but automatically \sphinxstyleemphasis{computationally} reproducible \sphinxhyphen{}
you can make sure that your analyses runs on any computer with Singularity,
regardless of the software environment on this computer. Even if you are unsure how you can wrap up an
environment into a software {\hyperref[\detokenize{glossary:term-container-image}]{\sphinxtermref{\DUrole{xref,std,std-term}{container image}}}} at this point, you could make use of
hundreds of publicly available images on \dlhbhref{S5}{Singularity\sphinxhyphen{}Hub} and
\dlhbhref{D6}{Docker\sphinxhyphen{}Hub}.

\sphinxAtStartPar
With this, you have also gotten a first glimpse into an extension of DataLad: A
Python module you can install with Python package managers such as \sphinxcode{\sphinxupquote{pip}} that
extends DataLad’s functionality.

\sphinxstepscope


