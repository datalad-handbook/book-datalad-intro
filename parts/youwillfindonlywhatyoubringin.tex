\chapter{Find you will only what you bring}
\label{\detokenize{basics/basics-yoda:you-will-find-only-what-you-bring-in}}\label{\detokenize{basics/basics-yoda:chapter-yoda}}\label{\detokenize{basics/basics-yoda::doc}}
\noindent{\hspace*{\fill}\sphinxincludegraphics[height=.25\textheight]{{yoda_bw}.pdf}\hspace*{\fill}}
\begin{quote}

\sphinxAtStartPar
Each choice, the branch of a tree is: What looked like a decision, is after only a pattern of growth.

\begin{flushright}
---Yoda
\end{flushright}
\end{quote}

\sphinxstepscope


\section{A data analysis project with DataLad}
\label{\detokenize{basics/101-126-intro:a-data-analysis-project-with-datalad}}\label{\detokenize{basics/101-126-intro:intromidterm}}\label{\detokenize{basics/101-126-intro::doc}}
\sphinxAtStartPar
Time flies and the semester rapidly approaches the midterms.
In DataLad\sphinxhyphen{}101, students are not given an exam \textendash{} instead, they are
asked to complete and submit a data analysis project with DataLad.
The lecturer hands out the requirements: The project…
\begin{itemize}
\item {} 
\sphinxAtStartPar
needs to be a data analysis project

\item {} 
\sphinxAtStartPar
is to be prepared in the form of a DataLad dataset

\item {} 
\sphinxAtStartPar
should incorporate DataLad whenever possible (data retrieval, publication,
script execution, general version control) and

\item {} 
\sphinxAtStartPar
needs to comply to the YODA principles

\end{itemize}

\sphinxAtStartPar
Luckily, the midterms are only in a couple of weeks, and a lot of the
requirements of the project will be taught in the upcoming sessions.
Therefore, there’s little you can do to prepare for the midterm
than to be extra attentive on the next lectures on the YODA
principles and DataLad’s Python API.

\sphinxstepscope

\index{YODA principles@\spxentry{YODA principles}|spxpagem}\index{data organization@\spxentry{data organization}!with DataLad@\spxentry{with DataLad}}\index{with DataLad@\spxentry{with DataLad}!data organization@\spxentry{data organization}}\index{project management@\spxentry{project management}!with DataLad@\spxentry{with DataLad}}\index{with DataLad@\spxentry{with DataLad}!project management@\spxentry{project management}}\ignorespaces 

\section{YODA: Best practices for data processing}
\label{\detokenize{basics/101-127-yoda:yoda-best-practices-for-data-analyses-in-a-dataset}}\label{\detokenize{basics/101-127-yoda:yoda}}\label{\detokenize{basics/101-127-yoda:id1}}\label{\detokenize{basics/101-127-yoda:index-0}}\label{\detokenize{basics/101-127-yoda::doc}}
\sphinxAtStartPar
The last requirement for the midterm projects reads “needs to comply to the
YODA principles”.
“What are the YODA principles?” you ask, as you have never heard of this
before.
“The topic of today’s lecture: Organizational principles of data
analyses in DataLad datasets. This lecture will show you the basic
principles behind creating, sharing, and publishing reproducible,
understandable, and open data analysis projects with DataLad.”, you
hear in return.


\subsection{The starting point…}
\label{\detokenize{basics/101-127-yoda:the-starting-point}}
\sphinxAtStartPar
Data analyses projects are very common, both in science and industry.
But it can be very difficult to produce a reproducible, let alone
\sphinxstyleemphasis{comprehensible} data analysis project.
Many data analysis projects do not start out with
a stringent organization, or fail to keep the structural organization of a
directory intact as the project develops. Often, this can be due to a lack of
version\sphinxhyphen{}control. In these cases, a project will quickly end up
with many
\dlhbhref{P4A}{almost\sphinxhyphen{}identical scripts suffixed with “\_version\_xyz”},
or a chaotic results structure split between various directories with names
such as \sphinxcode{\sphinxupquote{results/}}, \sphinxcode{\sphinxupquote{results\_August19/}}, \sphinxcode{\sphinxupquote{results\_revision/}} and
\sphinxcode{\sphinxupquote{now\_with\_nicer\_plots/}}. Something like this is a very
common shape a data science project may take after a while:

\enlargethispage{.5\baselineskip}
\begin{sphinxVerbatim}[commandchars=\\\{\}]
\PYG{g+go}{├── code/}
\PYG{g+go}{│   ├── code\PYGZus{}final/}
\PYG{g+go}{│   │   ├── final\PYGZus{}2/}
\PYG{g+go}{│   │   │   ├── main\PYGZus{}script\PYGZus{}fixed.py}
\PYG{g+go}{│   │   │   └── takethisscriptformostthingsnow.py}
\PYG{g+go}{│   │   ├── main\PYGZus{}script.py}
\PYG{g+go}{│   │   ├── utils\PYGZus{}new.py}
\PYG{g+go}{│   │   ├── utils\PYGZus{}2.py}
\PYG{g+go}{│   │   └── main\PYGZus{}analysis\PYGZus{}newparameters.py}
\PYG{g+go}{│   └── main\PYGZus{}script\PYGZus{}DONTUSE.py}
\PYG{g+go}{├── data/}
\PYG{g+go}{│   ├── data\PYGZus{}updated/}
\PYG{g+go}{│   │   └── dataset1/}
\PYG{g+go}{│   │       └── datafile\PYGZus{}a}
\PYG{g+go}{│   ├── dataset1/}
\PYG{g+go}{│   │    └── datafile\PYGZus{}a}
\PYG{g+go}{│   ├── outputs/}
\PYG{g+go}{│   │   ├── figures/}
\PYG{g+go}{│   │   │   ├── figures\PYGZus{}new.py}
\PYG{g+go}{│   │   │   └── figures\PYGZus{}final\PYGZus{}forreal.py}
\PYG{g+go}{│   │   ├── important\PYGZus{}results/}
\PYG{g+go}{│   │   ├── random\PYGZus{}results\PYGZus{}file.tsv}
\PYG{g+go}{│   │   ├── results\PYGZus{}for\PYGZus{}paper/}
\PYG{g+go}{│   │   ├── results\PYGZus{}for\PYGZus{}paper\PYGZus{}revised/}
\PYG{g+go}{│   │   └── results\PYGZus{}new\PYGZus{}data/}
\PYG{g+go}{│   ├── random\PYGZus{}results\PYGZus{}file\PYGZus{}v2.tsv}
\end{sphinxVerbatim}

\sphinxAtStartPar
All data analysis endeavors in directories like this \sphinxstyleemphasis{can} work, for a while,
if there is a person who knows the project well, and works on it all the time.
But it inevitably will get messy once anyone tries to collaborate on a project
like this, or simply goes on a two\sphinxhyphen{}week vacation and forgets whether
the function in \sphinxcode{\sphinxupquote{main\_analysis\_newparameters.py}} or the one in
\sphinxcode{\sphinxupquote{takethisscriptformostthingsnow.py}} was the one that created a particular figure.

\sphinxAtStartPar
But even if a project has an intuitive structure, and \sphinxstyleemphasis{is} version
controlled, in many cases an analysis script will stop working, or maybe worse,
will produce different results, because the software and tools used to
conduct the analysis in the first place got an update. This update may have
come with software changes that made functions stop working, or work differently
than before.
In the same vein, recomputing an analysis project on a different machine than
the one the analysis was developed on can fail if the necessary
software in the required versions is not installed or available on this new machine.
The analysis might depend on software that runs on a Linux machine, but the project
was shared with a Windows user. The environment during analysis development used
Python 2, but the new system has only Python 3 installed. Or one of the dependent
libraries needs to be in version X, but is installed as version Y.

\sphinxAtStartPar
The YODA principles are a clear set of organizational standards for
datasets used for data analysis projects that aim to overcome issues like the
ones outlined above. The name stands for
“YODAs Organigram on Data Analysis”%
\begin{footnote}\sphinxAtStartFootnote
“Why does the acronym contain itself?” you ask confused.
“That’s because it’s a \dlhbhref{W1P}{recursive acronym},
where the first letter stands recursively for the whole acronym.” you get in response.
“This is a reference to the recursiveness within a DataLad dataset \textendash{} all principles
apply recursively to all the subdatasets a dataset has.”
“And what does all of this have to do with Yoda?” you ask mildly amused.
“Oh, well. That’s just because the DataLad team is full of geeks.”
%
\end{footnote}. The principles outlined
in YODA set simple rules for directory names and structures, best\sphinxhyphen{}practices for
version\sphinxhyphen{}controlling dataset elements and analyses, facilitate
usage of tools to improve the reproducibility and accountability
of data analysis projects, and make collaboration easier.
They are summarized in three basic principles, that translate to both
dataset structures and best practices regarding the analysis:
\begin{itemize}
\item {} 
\sphinxAtStartPar
{\hyperref[\detokenize{basics/101-127-yoda:p1}]{\sphinxcrossref{\DUrole{std,std-ref}{P1: One thing, one dataset}}}} (\autopageref*{\detokenize{basics/101-127-yoda:p1}})

\item {} 
\sphinxAtStartPar
{\hyperref[\detokenize{basics/101-127-yoda:p2}]{\sphinxcrossref{\DUrole{std,std-ref}{P2: Record where you got it from, and where it is now}}}} (\autopageref*{\detokenize{basics/101-127-yoda:p2}})

\item {} 
\sphinxAtStartPar
{\hyperref[\detokenize{basics/101-127-yoda:p3}]{\sphinxcrossref{\DUrole{std,std-ref}{P3: Record what you did to it, and with what}}}} (\autopageref*{\detokenize{basics/101-127-yoda:p3}})

\end{itemize}

\sphinxAtStartPar
As you will see, complying to these principles is easy if you
use DataLad. Let’s go through them one by one.


\subsection{P1: One thing, one dataset}
\label{\detokenize{basics/101-127-yoda:p1-one-thing-one-dataset}}\label{\detokenize{basics/101-127-yoda:p1}}
\sphinxAtStartPar
Whenever a particular collection of files could be useful in more
than one context, make them a standalone, modular component.
In the broadest sense, this means to structure your study elements (data, code,
computational environments, results, …) in dedicated directories. For example:
\begin{itemize}
\item {} 
\sphinxAtStartPar
Store \sphinxstylestrong{input data} for an analysis in a dedicated \sphinxcode{\sphinxupquote{inputs/}} directory.
Keep different formats or processing\sphinxhyphen{}stages of your input data as individual,
modular components:  Do not mix raw data, data that is already structured
following community guidelines of the given field, or preprocessed data, but create
one data component for each of them. And if your analysis
relies on two or more data collections, these collections should each be an
individual component, not combined into one.

\item {} 
\sphinxAtStartPar
Store scripts or \sphinxstylestrong{code} used for the analysis of data in a dedicated \sphinxcode{\sphinxupquote{code/}}
directory, outside of the data component of the dataset.

\item {} 
\sphinxAtStartPar
Collect \sphinxstylestrong{results} of an analysis in a dedicated place, outside of the \sphinxcode{\sphinxupquote{inputs/}} directory, and
leave the input data of an analysis untouched by your computations.

\item {} 
\sphinxAtStartPar
Include a place for complete \sphinxstylestrong{execution environments}, such as
\dlhbhref{L1}{singularity images} or
\dlhbhref{D9A}{docker containers}%
\begin{footnote}\sphinxAtStartFootnote
If you want to learn more about Docker and Singularity, or general information
about containerized computational environments for reproducible data science,
check out \dlhbhref{N2A}{this section}
in the wonderful book \dlhbhref{N2}{The Turing Way},
a comprehensive guide to reproducible data science, or read about it in
section {\hyperref[\detokenize{basics/101-133-containersrun:containersrun}]{\sphinxcrossref{\DUrole{std,std-ref}{Computational reproducibility with software containers}}}} (\autopageref*{\detokenize{basics/101-133-containersrun:containersrun}}).
%
\end{footnote}, in
the form of an \sphinxcode{\sphinxupquote{envs/}} directory, if relevant for your analysis.

\item {} 
\sphinxAtStartPar
And if you conduct multiple different analyses, create a dedicated
project for each analysis, instead of conflating them.

\end{itemize}

\sphinxAtStartPar
This, for example, would be a directory structure from the root of a
superdataset of a very comprehensive data analysis project complying to the YODA principles:

\begin{sphinxVerbatim}[commandchars=\\\{\}]
\PYG{g+go}{├── ci/                         \PYGZsh{} continuous integration configuration}
\PYG{g+go}{│   └── .travis.yml}
\PYG{g+go}{├── code/                       \PYGZsh{} your code}
\PYG{g+go}{│   ├── tests/                  \PYGZsh{} unit tests to test your code}
\PYG{g+go}{│   │   └── test\PYGZus{}myscript.py}
\PYG{g+go}{│   └── myscript.py}
\PYG{g+go}{├── docs                        \PYGZsh{} documentation about the project}
\PYG{g+go}{│   ├── build/}
\PYG{g+go}{│   └── source/}
\PYG{g+go}{├── envs                        \PYGZsh{} computational environments}
\PYG{g+go}{│   └── Singularity}
\PYG{g+go}{├── inputs/                     \PYGZsh{} dedicated inputs/, will not be changed by an analysis}
\PYG{g+go}{│   └─── data/}
\PYG{g+go}{│       ├── dataset1/           \PYGZsh{} one stand\PYGZhy{}alone data component}
\PYG{g+go}{│       │   └── datafile\PYGZus{}a}
\PYG{g+go}{│       └── dataset2/}
\PYG{g+go}{│           └── datafile\PYGZus{}a}
\PYG{g+go}{├── important\PYGZus{}results/          \PYGZsh{} outputs away from the input data}
\PYG{g+go}{│   └── figures/}
\PYG{g+go}{├── CHANGELOG.md                \PYGZsh{} notes for fellow humans about your project}
\PYG{g+go}{├── HOWTO.md}
\PYG{g+go}{└── README.md}
\end{sphinxVerbatim}

\sphinxAtStartPar
You can get a few non\sphinxhyphen{}DataLad related advice for structuring your directories in the
\textit{Find-out-more}~{\findoutmoreiconinline}\textit{\ref{fom-yodaproject}} {\hyperref[\detokenize{basics/101-127-yoda:fom-yodaproject}]{\sphinxcrossref{\DUrole{std,std-ref}{on best practices for analysis organization}}}} (\autopageref*{\detokenize{basics/101-127-yoda:fom-yodaproject}}).

\index{recommendation@\spxentry{recommendation}!dataset content organization@\spxentry{dataset content organization}}\index{dataset content organization@\spxentry{dataset content organization}!recommendation@\spxentry{recommendation}}\ignorespaces \begin{findoutmore}[label={fom-yodaproject}, before title={\thetcbcounter\ }, float, floatplacement=tbp, check odd page=true]{More best practices for organizing contents in directories}
\label{\detokenize{basics/101-127-yoda:fom-yodaproject}}

\sphinxAtStartPar
The exemplary YODA directory structure is very comprehensive, and displays many best\sphinxhyphen{}practices for
reproducible data science. For example,
\begin{enumerate}
\sphinxsetlistlabels{\arabic}{enumi}{enumii}{}{.}%
\item {} 
\sphinxAtStartPar
Within \sphinxcode{\sphinxupquote{code/}}, it is best practice to add \sphinxstylestrong{tests} for the code.
These tests can be run to check whether the code still works.

\item {} 
\sphinxAtStartPar
It is even better to further use automated computing such as
\dlhbhref{W1B}{continuous integration (CI) systems},
to test the functionality of your functions and scripts automatically.
If relevant, the setup for continuous integration frameworks (such as
\dlhbhref{A3}{Appveyor}) lives outside of \sphinxcode{\sphinxupquote{code/}},
in a dedicated \sphinxcode{\sphinxupquote{ci/}} directory.

\item {} 
\sphinxAtStartPar
Include \sphinxstylestrong{documents for fellow humans}: Notes in a README.md or a HOWTO.md,
or even proper documentation (for example, using  in a dedicated \sphinxcode{\sphinxupquote{docs/}} directory.
Within these documents, include all relevant metadata for your analysis. If you are
conducting a scientific study, this might be authorship, funding,
change log, etc.

\end{enumerate}

\sphinxAtStartPar
If writing tests for analysis scripts or using continuous integration
is a new idea for you, but you want to learn more, check out
\dlhbhref{N2B}{this chapter on testing}.


\end{findoutmore}

\sphinxAtStartPar
There are many advantages to this modular way of organizing contents.
Having input data as independent components that are not altered (only
consumed) by an analysis does not conflate the data for
an analysis with the results or the code, thus assisting understanding
the project for anyone unfamiliar with it.
But more than just structure, this organization aids modular reuse or
publication of the individual components, for example data. In a
YODA\sphinxhyphen{}compliant dataset, any processing stage of a data component can
be reused in a new project or published and shared. The same is true
for a whole analysis dataset. At one point you might also write a
scientific paper about your analysis in a paper project, and the
whole analysis project can easily become a modular component in a paper
project, to make sharing paper, code, data, and results easy.

\begin{figure}[tbp]
\centering
\capstart

\noindent\sphinxincludegraphics[width=1.000\linewidth]{{dataset_modules_bw}.pdf}
\caption{Data are modular components that can be reused easily.}\label{\detokenize{basics/101-127-yoda:dataset-modules}}\end{figure}

\sphinxAtStartPar
The directory tree above and \hyperref[\detokenize{basics/101-127-yoda:dataset-modules}]{Fig.\@ \ref{\detokenize{basics/101-127-yoda:dataset-modules}}} highlight different aspects
of this principle. The directory tree illustrates the structure of
the individual pieces on the file system from the point of view of
a single top\sphinxhyphen{}level dataset with a particular purpose. For example, it
could be an analysis dataset created by a statistician for a scientific
project, and it could be shared between collaborators or
with others during development of the project. In this
superdataset, code is created that operates on input data to
compute outputs, and the code and outputs are captured,
version\sphinxhyphen{}controlled, and linked to the input data. Each input data in turn
is a (potentially nested) subdataset, but this is not visible
in the directory hierarchy.
\hyperref[\detokenize{basics/101-127-yoda:dataset-modules}]{Fig.\@ \ref{\detokenize{basics/101-127-yoda:dataset-modules}}}, in comparison, emphasizes a process view on a project and
the nested structure of input subdataset:
You can see how the preprocessed data that serves as an input for
the analysis datasets evolves from raw data to
standardized data organization to its preprocessed state. Within
the \sphinxcode{\sphinxupquote{data/}} directory of the file system hierarchy displayed
above one would find data datasets with their previous version as
a subdataset, and this is repeated recursively until one reaches
the raw data as it was originally collected at one point. A finished
analysis project in turn can be used as a component (subdataset) in
a paper project, such that the paper is a fully reproducible research
object that shares code, analysis results, and data, as well as the
history of all of these components.

\sphinxAtStartPar
Principle 1, therefore, encourages to structure data analysis
projects in a clear and modular fashion that makes use of nested
DataLad datasets, yielding comprehensible structures and reusable
components. Having each component version\sphinxhyphen{}controlled \textendash{}
regardless of size \textendash{}  will aid keeping directories clean and
organized, instead of piling up different versions of code, data,
or results.


\subsection{P2: Record where you got it from, and where it is now}
\label{\detokenize{basics/101-127-yoda:p2-record-where-you-got-it-from-and-where-it-is-now}}\label{\detokenize{basics/101-127-yoda:p2}}
\sphinxAtStartPar
It is good to have data, but it is even better if you and anyone you
collaborate or share the project or its components with can find
out where the data came from, or how it
is dependent on or linked to other data. Therefore, this principle
aims to attach this information, the data’s {\hyperref[\detokenize{glossary:term-provenance}]{\sphinxtermref{\DUrole{xref,std,std-term}{provenance}}}}, to the components of
your data analysis project.

\sphinxAtStartPar
Luckily, this is a no\sphinxhyphen{}brainer with DataLad, because the core data structure
of DataLad, the dataset, and many of the DataLad commands already covered
up to now fulfill this principle.

\sphinxAtStartPar
If data components of a project are DataLad datasets, they can
be included in an analysis superdataset as subdatasets. Thanks to
\sphinxcode{\sphinxupquote{datalad clone}}, information on the source of these subdatasets
is stored in the history of the analysis superdataset, and they can even be
updated from those sources if the original data dataset gets extended or changed.
If you are including a file, for example, code from GitHub,
the \sphinxcode{\sphinxupquote{datalad download\sphinxhyphen{}url}} command, introduced in section {\hyperref[\detokenize{basics/101-102-populate:populate}]{\sphinxcrossref{\DUrole{std,std-ref}{Populate a dataset}}}} (\autopageref*{\detokenize{basics/101-102-populate:populate}}),
will record the source of it safely in the dataset’s history. And if you add anything to your dataset,
from simple incremental coding progress in your analysis scripts up to
files that a colleague sent you via email, a plain \sphinxcode{\sphinxupquote{datalad save}}
with a helpful commit message goes a very long way to fulfill this principle
on its own already.

\sphinxAtStartPar
One core aspect of this principle is \sphinxstyleemphasis{linking} between reusable data
resource units (i.e., DataLad subdatasets containing pure data). You will
be happy to hear that this is achieved by simply installing datasets
as subdatasets, as \hyperref[\detokenize{basics/101-127-yoda:fig-subds}]{Fig.\@ \ref{\detokenize{basics/101-127-yoda:fig-subds}}} shows.
This part of this principle will therefore be absolutely obvious to you
because you already know how to install and nest datasets within datasets.
“I might just overcome my impostor syndrome if I experience such advanced
reproducible analysis concepts as being obvious”, you think with a grin.

\begin{figure}[tbp]
\centering
\capstart

\noindent\sphinxincludegraphics[width=0.500\linewidth]{{data_origin_bw}.pdf}
\caption{Schematic illustration of two standalone data datasets installed as subdatasets
into an analysis project.}\label{\detokenize{basics/101-127-yoda:id6}}\label{\detokenize{basics/101-127-yoda:fig-subds}}\end{figure}

\sphinxAtStartPar
But more than linking datasets in a superdataset, linkage also needs to
be established between components of your dataset. Scripts inside of
your \sphinxcode{\sphinxupquote{code/}} directory should point to data not as {\hyperref[\detokenize{glossary:term-absolute-path}]{\sphinxtermref{\DUrole{xref,std,std-term}{absolute path}}}}s
that would only work on your system, but instead as {\hyperref[\detokenize{glossary:term-relative-path}]{\sphinxtermref{\DUrole{xref,std,std-term}{relative path}}}}s
that will work in any shared copy of your dataset. The next section
demonstrates a YODA data analysis project and will show concrete examples of this.

\sphinxAtStartPar
Lastly, this principle also includes \sphinxstyleemphasis{moving}, \sphinxstyleemphasis{sharing}, and \sphinxstyleemphasis{publishing} your
datasets or its components.
It is usually costly to collect data, and economically unfeasible%
\begin{footnote}\sphinxAtStartFootnote
Substitute unfeasible with \sphinxstyleemphasis{wasteful}, \sphinxstyleemphasis{impractical}, or simply \sphinxstyleemphasis{stupid} if preferred.
%
\end{footnote} to keep
it locked in a drawer (or similarly out of reach behind complexities of
data retrieval or difficulties in understanding the data structure).
But conducting several projects on the same dataset yourself, sharing it with
collaborators, or publishing it is easy if the project is a DataLad dataset
that can be installed and retrieved on demand, and is kept clean from
everything that is not part of the data according to principle 1.
Conducting transparent open science is easier if you can link code, data,
and results within a dataset, and share everything together. In conjunction
with principle 1, this means that you can distribute your analysis projects
(or parts of it) in a comprehensible form, exemplified in \hyperref[\detokenize{basics/101-127-yoda:fig-yodads}]{Fig.\@ \ref{\detokenize{basics/101-127-yoda:fig-yodads}}}.

\begin{figure}[tbp]
\centering
\capstart

\noindent\sphinxincludegraphics{{decentralized_publishing_bw}.pdf}
\caption{In a dataset that complies to the YODA principles, modular components
(data, analysis results, papers) can be shared or published individually.}\label{\detokenize{basics/101-127-yoda:id7}}\label{\detokenize{basics/101-127-yoda:fig-yodads}}\end{figure}

\sphinxAtStartPar
Principle 2, therefore, facilitates transparent linkage of datasets and their
components to other components, their original sources, or shared copies.
With the DataLad tools you learned to master up to this point,
you have all the necessary skills to comply to it already.


\subsection{P3: Record what you did to it, and with what}
\label{\detokenize{basics/101-127-yoda:p3-record-what-you-did-to-it-and-with-what}}\label{\detokenize{basics/101-127-yoda:p3}}
\sphinxAtStartPar
This last principle is about capturing \sphinxstyleemphasis{how exactly the content of
every file came to be} that was not obtained from elsewhere. For example,
this relates to results generated from inputs by scripts or commands.
The section {\hyperref[\detokenize{basics/101-108-run:run}]{\sphinxcrossref{\DUrole{std,std-ref}{Keeping track}}}} (\autopageref*{\detokenize{basics/101-108-run:run}}) already outlined the problem of associating
a result with an input and a script. It can be difficult to link a
figure from your data analysis project with an input data file or a
script, even if you created this figure yourself.
The \sphinxcode{\sphinxupquote{datalad run}} command however mitigates these difficulties,
and captures the provenance of any output generated with a
\sphinxcode{\sphinxupquote{datalad run}} call in the history of the dataset. Thus, by using
\sphinxcode{\sphinxupquote{datalad run}} in analysis projects, your dataset knows
which result was generated when, by which author, from which inputs,
and by means of which command.

\sphinxAtStartPar
With another DataLad command one can even go one step further:
The command \sphinxcode{\sphinxupquote{datalad containers\sphinxhyphen{}run}} \sphinxhyphen{} it will be introduced in
section {\hyperref[\detokenize{basics/101-133-containersrun:containersrun}]{\sphinxcrossref{\DUrole{std,std-ref}{Computational reproducibility with software containers}}}} (\autopageref*{\detokenize{basics/101-133-containersrun:containersrun}}) \sphinxhyphen{} performs a command execution within
a configured containerized environment. Thus, not only inputs,
outputs, command, time, and author, but also the \sphinxstyleemphasis{software environment}
are captured as provenance of a dataset component such as a results file,
and, importantly, can be shared together with the dataset in the
form of a software container.

\sphinxAtStartPar
Tip: Make use of \sphinxcode{\sphinxupquote{datalad run}}’s \sphinxcode{\sphinxupquote{\sphinxhyphen{}\sphinxhyphen{}dry\sphinxhyphen{}run}} option to craft your run\sphinxhyphen{}command, as outlined in {\hyperref[\detokenize{basics/101-110-run2:dryrun}]{\sphinxcrossref{\DUrole{std,std-ref}{Dry\sphinxhyphen{}running your run call}}}} (\autopageref*{\detokenize{basics/101-110-run2:dryrun}})!

\sphinxAtStartPar
With this last principle, your dataset collects and stores provenance
of all the contents you created in the wake of your analysis project.
This established trust in your results, and enables others to understand
where files derive from.


\subsection{The YODA procedure}
\label{\detokenize{basics/101-127-yoda:the-yoda-procedure}}\label{\detokenize{basics/101-127-yoda:yodaproc}}
\sphinxAtStartPar
There is one tool that can make starting a yoda\sphinxhyphen{}compliant data analysis
easier: DataLad’s \sphinxcode{\sphinxupquote{yoda}} procedure. Just as the \sphinxcode{\sphinxupquote{text2git}} procedure
from section {\hyperref[\detokenize{basics/101-101-create:createds}]{\sphinxcrossref{\DUrole{std,std-ref}{Create a dataset}}}} (\autopageref*{\detokenize{basics/101-101-create:createds}}), the \sphinxcode{\sphinxupquote{yoda}} procedure can be included in a
\sphinxcode{\sphinxupquote{datalad create}} command and will apply useful configurations
to your dataset:

\begin{sphinxVerbatim}[commandchars=\\\{\}]
\PYG{g+gp}{\PYGZdl{} }datalad\PYG{+w}{ }create\PYG{+w}{ }\PYGZhy{}c\PYG{+w}{ }yoda\PYG{+w}{ }\PYG{l+s+s2}{\PYGZdq{}my\PYGZus{}analysis\PYGZdq{}}
\PYG{g+go}{[INFO   ] Creating a new annex repo at /home/me/repos/testing/my\PYGZus{}analysis}
\PYG{g+go}{create(ok): /home/me/repos/testing/my\PYGZus{}analysis (dataset)}
\PYG{g+go}{[INFO   ] Running procedure cfg\PYGZus{}yoda}
\PYG{g+go}{[INFO   ] == Command start (output follows) =====}
\PYG{g+go}{[INFO   ] == Command exit (modification check follows) =====}
\end{sphinxVerbatim}

\sphinxAtStartPar
Let’s take a look at what configurations and changes come with this procedure:

\begin{sphinxVerbatim}[commandchars=\\\{\}]
\PYG{g+gp}{\PYGZdl{} }tree\PYG{+w}{ }\PYGZhy{}a

\PYG{g+go}{.}
\PYG{g+go}{├── .gitattributes}
\PYG{g+go}{├── CHANGELOG.md}
\PYG{g+go}{├── code}
\PYG{g+go}{│   ├── .gitattributes}
\PYG{g+go}{│   └── README.md}
\PYG{g+go}{└── README.md}
\end{sphinxVerbatim}

\sphinxAtStartPar
Let’s take a closer look into the \sphinxcode{\sphinxupquote{.gitattributes}} files:

\begin{sphinxVerbatim}[commandchars=\\\{\}]
\PYG{g+gp}{\PYGZdl{} }less\PYG{+w}{ }.gitattributes

\PYG{g+go}{**/.git* annex.largefiles=nothing}
\PYG{g+go}{CHANGELOG.md annex.largefiles=nothing}
\PYG{g+go}{README.md annex.largefiles=nothing}

\PYG{g+gp}{\PYGZdl{} }less\PYG{+w}{ }code/.gitattributes

\PYG{g+go}{* annex.largefiles=nothing}
\end{sphinxVerbatim}

\sphinxAtStartPar
Summarizing these two glimpses into the dataset, this configuration has
\begin{enumerate}
\sphinxsetlistlabels{\arabic}{enumi}{enumii}{}{.}%
\item {} 
\sphinxAtStartPar
included a code directory in your dataset

\item {} 
\sphinxAtStartPar
included three files for human consumption (\sphinxcode{\sphinxupquote{README.md}}, \sphinxcode{\sphinxupquote{CHANGELOG.md}})

\item {} 
\sphinxAtStartPar
configured everything in the \sphinxcode{\sphinxupquote{code/}} directory to be tracked by Git, not git\sphinxhyphen{}annex%
\begin{footnote}\sphinxAtStartFootnote
To re\sphinxhyphen{}read how \sphinxcode{\sphinxupquote{.gitattributes}} work, go back to section {\hyperref[\detokenize{basics/101-122-config:config}]{\sphinxcrossref{\DUrole{std,std-ref}{Local configuration}}}} (\autopageref*{\detokenize{basics/101-122-config:config}}), and to remind yourself
about how this worked for the \sphinxcode{\sphinxupquote{text2git}} configuration, go back to section {\hyperref[\detokenize{basics/101-114-txt2git:text2git}]{\sphinxcrossref{\DUrole{std,std-ref}{Data safety}}}} (\autopageref*{\detokenize{basics/101-114-txt2git:text2git}}).
%
\end{footnote}

\item {} 
\sphinxAtStartPar
and configured \sphinxcode{\sphinxupquote{README.md}} and \sphinxcode{\sphinxupquote{CHANGELOG.md}} in the root of the dataset to be
tracked by Git.

\end{enumerate}

\sphinxAtStartPar
Your next data analysis project can thus get a head start with useful configurations
and the start of a comprehensible directory structure by applying the \sphinxcode{\sphinxupquote{yoda}} procedure.


\subsection{Sources}
\label{\detokenize{basics/101-127-yoda:sources}}
\sphinxAtStartPar
This section is based on a comprehensive
\dlhbhref{F1A}{poster} and publicly
available \dlhbhref{G2M}{slides} about the
YODA principles.

\sphinxstepscope

\index{data analysis@\spxentry{data analysis}!with DataLad@\spxentry{with DataLad}}\index{with DataLad@\spxentry{with DataLad}!data analysis@\spxentry{data analysis}}\ignorespaces 

\section{YODA\sphinxhyphen{}compliant data analysis projects}
\label{\detokenize{basics/101-130-yodaproject:yoda-compliant-data-analysis-projects}}\label{\detokenize{basics/101-130-yodaproject:yoda-project}}\label{\detokenize{basics/101-130-yodaproject:index-0}}\label{\detokenize{basics/101-130-yodaproject::doc}}
\sphinxAtStartPar
Now that you know about the YODA principles, it is time to start working on
\sphinxcode{\sphinxupquote{DataLad\sphinxhyphen{}101}}’s midterm project. Because the midterm project guidelines
require a YODA\sphinxhyphen{}compliant data analysis project, you will not only have theoretical
knowledge about the YODA principles, but also gain practical experience.

\sphinxAtStartPar
In principle, you can prepare YODA\sphinxhyphen{}compliant data analyses in any programming
language of your choice. But because you are already familiar with
the \dlhbhref{P8}{Python} programming language, you decide
to script your analysis in Python. Delighted, you find out that there is even
a Python API for DataLad’s functionality that you can read about in {\hyperref[\detokenize{basics/101-130-yodaproject:fom-pythonapi}]{\sphinxcrossref{\DUrole{std,std-ref}{a Find\sphinxhyphen{}out\sphinxhyphen{}more on DataLad in Python}}}} (\autopageref*{\detokenize{basics/101-130-yodaproject:fom-pythonapi}}).

\index{use DataLad API@\spxentry{use DataLad API}!with Matlab@\spxentry{with Matlab}}\index{with Matlab@\spxentry{with Matlab}!use DataLad API@\spxentry{use DataLad API}}\index{use DataLad API@\spxentry{use DataLad API}!with R@\spxentry{with R}}\index{with R@\spxentry{with R}!use DataLad API@\spxentry{use DataLad API}}\ignorespaces \begin{importantnote}[label={index-2}, before title={\thetcbcounter\ }, check odd page=true]{Use DataLad in languages other than Python}
\label{\detokenize{basics/101-130-yodaproject:index-2}}

\sphinxAtStartPar
While there is a dedicated API for Python, DataLad’s functions can of course
also be used with other programming languages, such as Matlab, or R, via standard
system calls.

\sphinxAtStartPar
Even if you do not know or like Python, you can just copy\sphinxhyphen{}paste the code
and follow along \textendash{} the high\sphinxhyphen{}level YODA principles demonstrated in this
section generalize across programming languages.


\end{importantnote}

\sphinxAtStartPar
For your midterm project submission, you decide to create a data analysis on the
\dlhbhref{W1K}{iris flower data set}.
It is a multivariate dataset on 50 samples of each of three species of Iris
flowers (\sphinxstyleemphasis{Setosa}, \sphinxstyleemphasis{Versicolor}, or \sphinxstyleemphasis{Virginica}), with four variables: the length and width of the sepals and petals
of the flowers in centimeters. It is often used in introductory data science
courses for statistical classification techniques in machine learning, and
widely available \textendash{} a perfect dataset for your midterm project!

\index{reproducible paper@\spxentry{reproducible paper}!with DataLad@\spxentry{with DataLad}}\index{with DataLad@\spxentry{with DataLad}!reproducible paper@\spxentry{reproducible paper}}\ignorespaces \begin{importantnote}[label={index-3}, before title={\thetcbcounter\ }, check odd page=true]{Turn data analysis into dynamically generated documents}
\label{\detokenize{basics/101-130-yodaproject:index-3}}

\sphinxAtStartPar
Beyond the contents of this section, we have transformed the example analysis also into a template to write a reproducible paper.
If you are interested in checking that out, please head over to \dlhbhref{G2H}{github.com/datalad\sphinxhyphen{}handbook/repro\sphinxhyphen{}paper\sphinxhyphen{}sketch/}.


\end{importantnote}


\subsection{Raw data as a modular, independent entity}
\label{\detokenize{basics/101-130-yodaproject:raw-data-as-a-modular-independent-entity}}
\sphinxAtStartPar
The first YODA principle stressed the importance of modularity in a data analysis
project: Every component that could be used in more than one context should be
an independent component.

\sphinxAtStartPar
The first aspect this applies to is the input data of your dataset: There can
be thousands of ways to analyze it, and it is therefore immensely helpful to
have a pristine raw iris dataset that does not get modified, but serves as
input for these analysis.
As such, the iris data should become a standalone DataLad dataset.
For the purpose of this analysis, the online\sphinxhyphen{}handbook provides an \sphinxcode{\sphinxupquote{iris\_data}}
dataset at \sphinxurl{https://github.com/datalad-handbook/iris\_data}.
You can either use this provided input dataset, or check the \textit{Find-out-more}~{\findoutmoreiconinline}\textit{\ref{fom-iris}} {\hyperref[\detokenize{basics/101-130-yodaproject:fom-iris}]{\sphinxcrossref{\DUrole{std,std-ref}{on how to create an independent dataset from scratch}}}} (\autopageref*{\detokenize{basics/101-130-yodaproject:fom-iris}}).

\index{create and publish dataset as dependency@\spxentry{create and publish dataset as dependency}!with DataLad@\spxentry{with DataLad}}\index{with DataLad@\spxentry{with DataLad}!create and publish dataset as dependency@\spxentry{create and publish dataset as dependency}}\ignorespaces \begin{findoutmore}[label={fom-iris}, before title={\thetcbcounter\ }, float, floatplacement=tb, check odd page=true]{Creating an independent input dataset}
\label{\detokenize{basics/101-130-yodaproject:fom-iris}}

\sphinxAtStartPar
If you acquire your own data for a data analysis, you will have
to turn it into a DataLad dataset in order to install it as a subdataset.
Any directory with data that exists on
your computer can be turned into a dataset with \sphinxcode{\sphinxupquote{datalad create \sphinxhyphen{}\sphinxhyphen{}force}}
and a subsequent \sphinxcode{\sphinxupquote{datalad save \sphinxhyphen{}m "add data" .}} to first create a dataset inside of
an existing, non\sphinxhyphen{}empty directory, and subsequently save all of its contents into
the history of the newly created dataset.

\sphinxAtStartPar
To create the \sphinxcode{\sphinxupquote{iris\_data}} dataset at \sphinxurl{https://github.com/datalad-handbook/iris\_data}
we first created a DataLad dataset…

\begin{sphinxVerbatim}[commandchars=\\\{\}]
\PYG{g+gp}{\PYGZdl{} }\PYG{c+c1}{\PYGZsh{} make sure to move outside of DataLad\PYGZhy{}101!}
\PYG{g+gp}{\PYGZdl{} }\PYG{n+nb}{cd}\PYG{+w}{ }../
\PYG{g+gp}{\PYGZdl{} }datalad\PYG{+w}{ }create\PYG{+w}{ }iris\PYGZus{}data
\PYG{g+go}{create(ok): /home/me/dl\PYGZhy{}101/iris\PYGZus{}data (dataset)}
\end{sphinxVerbatim}

\sphinxAtStartPar
and subsequently got the data from a publicly available
\dlhbhref{G11A}{GitHub Gist}, a code snippet, or other short standalone information with a
\sphinxcode{\sphinxupquote{datalad download\sphinxhyphen{}url}} command:

\begin{sphinxVerbatim}[commandchars=\\\{\}]
\PYG{g+gp}{\PYGZdl{} }\PYG{n+nb}{cd}\PYG{+w}{ }iris\PYGZus{}data
\PYG{g+gp}{\PYGZdl{} }datalad\PYG{+w}{ }download\PYGZhy{}url\PYG{+w}{ }https://gist.githubusercontent.com/netj/8836201/raw/6f9306ad21398ea43cba4f7d537619d0e07d5ae3/iris.csv
\PYG{g+go}{download\PYGZus{}url(ok): /home/me/dl\PYGZhy{}101/iris\PYGZus{}data/iris.csv (file)}
\PYG{g+go}{add(ok): iris.csv (file)}
\PYG{g+go}{save(ok): . (dataset)}
\end{sphinxVerbatim}

\sphinxAtStartPar
Finally, we \sphinxstyleemphasis{published} the dataset  to {\hyperref[\detokenize{glossary:term-GitHub}]{\sphinxtermref{\DUrole{xref,std,std-term}{GitHub}}}}.

\sphinxAtStartPar
With this setup, the iris dataset (a single comma\sphinxhyphen{}separated (\sphinxcode{\sphinxupquote{.csv}})
file) is downloaded, and, importantly, the dataset recorded \sphinxstyleemphasis{where} it
was obtained from thanks to \sphinxcode{\sphinxupquote{datalad download\sphinxhyphen{}url}}, thus complying
to the second YODA principle.
This way, upon installation of the dataset, DataLad knows where to
obtain the file content from. You can \sphinxcode{\sphinxupquote{datalad clone}} the iris
dataset and find out with a \sphinxcode{\sphinxupquote{git annex whereis iris.csv}} command.


\end{findoutmore}

\sphinxAtStartPar
“Nice, with this input dataset I have sufficient provenance capture for my
input dataset, and I can install it as a modular component”, you think as you
mentally tick off YODA principle number 1 and 2. “But before I can install it,
I need an analysis superdataset first.”

\vspace{2cm}
\noindent{\hspace*{\fill}\sphinxincludegraphics[width=0.500\linewidth]{{reproduced_bw}.pdf}\hspace*{\fill}}

\subsection{Building an analysis dataset}
\label{\detokenize{basics/101-130-yodaproject:building-an-analysis-dataset}}
\sphinxAtStartPar
There is an independent raw dataset as input data, but there is no place
for your analysis to live, yet. Therefore, you start your midterm project
by creating an analysis dataset. As this project is part of \sphinxcode{\sphinxupquote{DataLad\sphinxhyphen{}101}},
you do it as a subdataset of \sphinxcode{\sphinxupquote{DataLad\sphinxhyphen{}101}}.
Remember to specify the \sphinxcode{\sphinxupquote{\sphinxhyphen{}\sphinxhyphen{}dataset}} option of \sphinxcode{\sphinxupquote{datalad create}}
to link it as a subdataset!

\sphinxAtStartPar
You naturally want your dataset to follow the YODA principles, and, as a start,
you use the \sphinxcode{\sphinxupquote{cfg\_yoda}} procedure to help you structure the dataset%
\begin{footnote}\sphinxAtStartFootnote
Note that you could have applied the YODA procedure not only right at
creation of the dataset with \sphinxcode{\sphinxupquote{\sphinxhyphen{}c yoda}}, but also after creation
with the \sphinxcode{\sphinxupquote{datalad run\sphinxhyphen{}procedure}} command:

\sphinxSetupCodeBlockInFootnote
\begin{sphinxVerbatim}[commandchars=\\\{\}]
\PYG{g+gp}{\PYGZdl{} }\PYG{n+nb}{cd}\PYG{+w}{ }midterm\PYGZus{}project
\PYG{g+gp}{\PYGZdl{} }datalad\PYG{+w}{ }run\PYGZhy{}procedure\PYG{+w}{ }cfg\PYGZus{}yoda
\end{sphinxVerbatim}

\sphinxAtStartPar
Both ways of applying the YODA procedure will lead to the same
outcome.
%
\end{footnote}:

\begin{sphinxVerbatim}[commandchars=\\\{\}]
\PYG{g+gp}{\PYGZdl{} }\PYG{c+c1}{\PYGZsh{} inside of DataLad\PYGZhy{}101}
\PYG{g+gp}{\PYGZdl{} }datalad\PYG{+w}{ }create\PYG{+w}{ }\PYGZhy{}c\PYG{+w}{ }yoda\PYG{+w}{ }\PYGZhy{}\PYGZhy{}dataset\PYG{+w}{ }.\PYG{+w}{ }midterm\PYGZus{}project
\PYG{g+go}{[INFO] Running procedure cfg\PYGZus{}yoda}
\PYG{g+go}{[INFO] == Command start (output follows) =====}
\PYG{g+go}{[INFO] == Command exit (modification check follows) =====}
\PYG{g+go}{run(ok): /home/me/dl\PYGZhy{}101/DataLad\PYGZhy{}101/midterm\PYGZus{}project (dataset) [VIRTUALENV/bin/python /home/a...]}
\PYG{g+go}{add(ok): midterm\PYGZus{}project (dataset)}
\PYG{g+go}{add(ok): .gitmodules (file)}
\PYG{g+go}{save(ok): . (dataset)}
\PYG{g+go}{create(ok): midterm\PYGZus{}project (dataset)}
\end{sphinxVerbatim}

\index{subdatasets@\spxentry{subdatasets}!DataLad command@\spxentry{DataLad command}}\index{DataLad command@\spxentry{DataLad command}!subdatasets@\spxentry{subdatasets}}\index{list subdatasets@\spxentry{list subdatasets}!with DataLad@\spxentry{with DataLad}}\index{with DataLad@\spxentry{with DataLad}!list subdatasets@\spxentry{list subdatasets}}\ignorespaces 
\sphinxAtStartPar
The \sphinxcode{\sphinxupquote{datalad subdatasets}} command can report on which subdatasets exist for
\sphinxcode{\sphinxupquote{DataLad\sphinxhyphen{}101}}. This helps you verify that the command succeeded and the
dataset was indeed linked as a subdataset to \sphinxcode{\sphinxupquote{DataLad\sphinxhyphen{}101}}:

\begin{sphinxVerbatim}[commandchars=\\\{\}]
\PYG{g+gp}{\PYGZdl{} }datalad\PYG{+w}{ }subdatasets
\PYG{g+go}{subdataset(ok): midterm\PYGZus{}project (dataset)}
\PYG{g+go}{subdataset(ok): recordings/longnow (dataset)}
\end{sphinxVerbatim}

\sphinxAtStartPar
Not only the \sphinxcode{\sphinxupquote{longnow}} subdataset, but also the newly created
\sphinxcode{\sphinxupquote{midterm\_project}} subdataset are displayed \textendash{} wonderful!

\sphinxAtStartPar
But back to the midterm project now. So far, you have created a pre\sphinxhyphen{}structured
analysis dataset. As a next step, you take care of installing and linking the
raw dataset for your analysis adequately to your \sphinxcode{\sphinxupquote{midterm\_project}} dataset
by installing it as a subdataset. Make sure to install it as a subdataset of
\sphinxcode{\sphinxupquote{midterm\_project}}, and not \sphinxcode{\sphinxupquote{DataLad\sphinxhyphen{}101}}!

\begin{sphinxVerbatim}[commandchars=\\\{\}]
\PYG{g+gp}{\PYGZdl{} }\PYG{n+nb}{cd}\PYG{+w}{ }midterm\PYGZus{}project
\PYG{g+gp}{\PYGZdl{} }\PYG{c+c1}{\PYGZsh{} we are in midterm\PYGZus{}project, thus \PYGZhy{}d . points to the root of it.}
\PYG{g+gp}{\PYGZdl{} }datalad\PYG{+w}{ }clone\PYG{+w}{ }\PYGZhy{}d\PYG{+w}{ }.\PYG{+w}{ }\PYG{l+s+se}{\PYGZbs{}}
\PYG{+w}{  }https://github.com/datalad\PYGZhy{}handbook/iris\PYGZus{}data.git\PYG{+w}{ }\PYG{l+s+se}{\PYGZbs{}}
\PYG{+w}{  }input/
\PYG{g+go}{[INFO] Remote origin not usable by git\PYGZhy{}annex; setting annex\PYGZhy{}ignore}
\PYG{g+go}{install(ok): input (dataset)}
\PYG{g+go}{add(ok): input (dataset)}
\PYG{g+go}{add(ok): .gitmodules (file)}
\PYG{g+go}{save(ok): . (dataset)}
\PYG{g+go}{add(ok): .gitmodules (file)}
\PYG{g+go}{save(ok): . (dataset)}
\end{sphinxVerbatim}

\sphinxAtStartPar
Note that we did not keep its original name, \sphinxcode{\sphinxupquote{iris\_data}}, but rather provided
a path with a new name, \sphinxcode{\sphinxupquote{input}}, because this much more intuitively comprehensible.

\sphinxAtStartPar
After the input dataset is installed, the directory structure of \sphinxcode{\sphinxupquote{DataLad\sphinxhyphen{}101}}
looks like this:

\begin{sphinxVerbatim}[commandchars=\\\{\}]
\PYG{g+gp}{\PYGZdl{} }\PYG{n+nb}{cd}\PYG{+w}{ }../
\PYG{g+gp}{\PYGZdl{} }tree\PYG{+w}{ }\PYGZhy{}d
\PYG{g+gp}{\PYGZdl{} }\PYG{n+nb}{cd}\PYG{+w}{ }midterm\PYGZus{}project
\PYG{g+go}{.}
\PYG{g+go}{├── books}
\PYG{g+go}{├── code}
\PYG{g+go}{├── midterm\PYGZus{}project}
\PYG{g+go}{│   ├── code}
\PYG{g+go}{│   └── input}
\PYG{g+go}{└── recordings}
\PYG{g+go}{    └── longnow}
\PYG{g+go}{        ├── Long\PYGZus{}Now\PYGZus{}\PYGZus{}Conversations\PYGZus{}at\PYGZus{}The\PYGZus{}Interval}
\PYG{g+go}{        └── Long\PYGZus{}Now\PYGZus{}\PYGZus{}Seminars\PYGZus{}About\PYGZus{}Long\PYGZus{}term\PYGZus{}Thinking}

\PYG{g+go}{9 directories}
\end{sphinxVerbatim}

\sphinxAtStartPar
Importantly, all of the subdatasets are linked to the higher\sphinxhyphen{}level datasets,
and despite being inside of \sphinxcode{\sphinxupquote{DataLad\sphinxhyphen{}101}}, your \sphinxcode{\sphinxupquote{midterm\_project}} is an independent
dataset, as is its \sphinxcode{\sphinxupquote{input/}} subdataset. An overview is shown in \hyperref[\detokenize{basics/101-130-yodaproject:fig-linkeddl101}]{Fig.\@ \ref{\detokenize{basics/101-130-yodaproject:fig-linkeddl101}}}.

\begin{figure}[bp]
\centering
\capstart

\noindent\sphinxincludegraphics[width=0.500\linewidth]{{virtual_dstree_dl101_midterm_bw}.pdf}
\caption{Overview of (linked) datasets in DataLad\sphinxhyphen{}101.}\label{\detokenize{basics/101-130-yodaproject:id8}}\label{\detokenize{basics/101-130-yodaproject:fig-linkeddl101}}\end{figure}


\subsection{YODA\sphinxhyphen{}compliant analysis scripts}
\label{\detokenize{basics/101-130-yodaproject:yoda-compliant-analysis-scripts}}
\sphinxAtStartPar
Now that you have an \sphinxcode{\sphinxupquote{input/}} directory with data, and a \sphinxcode{\sphinxupquote{code/}} directory
(created by the YODA procedure) for your scripts, it is time to work on the script
for your analysis. Within \sphinxcode{\sphinxupquote{midterm\_project}}, the \sphinxcode{\sphinxupquote{code/}} directory is where
you want to place your scripts.

\sphinxAtStartPar
But first, you plan your research question. You decide to do a
classification analysis with a k\sphinxhyphen{}nearest neighbors algorithm%
\begin{footnote}\sphinxAtStartFootnote
The choice of analysis method
in this book is rather arbitrary, and understanding the k\sphinxhyphen{}nearest
neighbor algorithm is by no means required for this section.
%
\end{footnote}. The iris
dataset works well for such questions. Based on the features of the flowers
(sepal and petal width and length) you will try to predict what type of
flower (\sphinxstyleemphasis{Setosa}, \sphinxstyleemphasis{Versicolor}, or \sphinxstyleemphasis{Virginica}) a particular flower in the
dataset is. You settle on two objectives for your analysis:
\begin{enumerate}
\sphinxsetlistlabels{\arabic}{enumi}{enumii}{}{.}%
\item {} 
\sphinxAtStartPar
Explore and plot the relationship between variables in the dataset and save
the resulting graphic as a first result.

\item {} 
\sphinxAtStartPar
Perform a k\sphinxhyphen{}nearest neighbor classification on a subset of the dataset to
predict class membership (flower type) of samples in a left\sphinxhyphen{}out test set.
Your final result should be a statistical summary of this prediction.

\end{enumerate}

\sphinxAtStartPar
To compute the analysis you create the following Python script inside of \sphinxcode{\sphinxupquote{code/}}:

\fvset{hllines={, 10, 11, 12, 22, 41,}}%
\begin{sphinxVerbatim}[commandchars=\\\{\}]
\PYG{g+gp}{\PYGZdl{} }cat\PYG{+w}{ }\PYGZlt{}\PYGZlt{}\PYG{+w}{ }EOT\PYG{+w}{ }\PYGZgt{}\PYG{+w}{ }code/script.py
\PYG{g+go}{import argparse}
\PYG{g+go}{import pandas as pd}
\PYG{g+go}{import seaborn as sns}
\PYG{g+go}{from sklearn import model\PYGZus{}selection}
\PYG{g+go}{from sklearn.neighbors import KNeighborsClassifier}
\PYG{g+go}{from sklearn.metrics import classification\PYGZus{}report}

\PYG{g+go}{parser = argparse.ArgumentParser(description=\PYGZdq{}Analyze iris data\PYGZdq{})}
\PYG{g+go}{parser.add\PYGZus{}argument(\PYGZsq{}data\PYGZsq{}, help=\PYGZdq{}Input data (CSV) to process\PYGZdq{})}
\PYG{g+go}{parser.add\PYGZus{}argument(\PYGZsq{}output\PYGZus{}figure\PYGZsq{}, help=\PYGZdq{}Output figure path\PYGZdq{})}
\PYG{g+go}{parser.add\PYGZus{}argument(\PYGZsq{}output\PYGZus{}report\PYGZsq{}, help=\PYGZdq{}Output report path\PYGZdq{})}
\PYG{g+go}{args = parser.parse\PYGZus{}args()}

\PYG{g+gp}{\PYGZsh{} }prepare\PYG{+w}{ }the\PYG{+w}{ }data\PYG{+w}{ }as\PYG{+w}{ }a\PYG{+w}{ }pandas\PYG{+w}{ }dataframe
\PYG{g+go}{df = pd.read\PYGZus{}csv(args.data)}
\PYG{g+go}{attributes = [\PYGZdq{}sepal\PYGZus{}length\PYGZdq{}, \PYGZdq{}sepal\PYGZus{}width\PYGZdq{}, \PYGZdq{}petal\PYGZus{}length\PYGZdq{},\PYGZdq{}petal\PYGZus{}width\PYGZdq{}, \PYGZdq{}class\PYGZdq{}]}
\PYG{g+go}{df.columns = attributes}

\PYG{g+gp}{\PYGZsh{} }create\PYG{+w}{ }a\PYG{+w}{ }pairplot\PYG{+w}{ }to\PYG{+w}{ }plot\PYG{+w}{ }pairwise\PYG{+w}{ }relationships\PYG{+w}{ }\PYG{k}{in}\PYG{+w}{ }the\PYG{+w}{ }dataset
\PYG{g+go}{plot = sns.pairplot(df, hue=\PYGZsq{}class\PYGZsq{}, palette=\PYGZsq{}muted\PYGZsq{})}
\PYG{g+go}{plot.savefig(args.output\PYGZus{}figure)}

\PYG{g+gp}{\PYGZsh{} }perform\PYG{+w}{ }a\PYG{+w}{ }K\PYGZhy{}nearest\PYGZhy{}neighbours\PYG{+w}{ }classification\PYG{+w}{ }with\PYG{+w}{ }scikit\PYGZhy{}learn
\PYG{g+gp}{\PYGZsh{} }Step\PYG{+w}{ }\PYG{l+m}{1}:\PYG{+w}{ }split\PYG{+w}{ }data\PYG{+w}{ }\PYG{k}{in}\PYG{+w}{ }\PYG{n+nb}{test}\PYG{+w}{ }and\PYG{+w}{ }training\PYG{+w}{ }dataset\PYG{+w}{ }\PYG{o}{(}\PYG{l+m}{20}:80\PYG{o}{)}
\PYG{g+go}{array = df.values}
\PYG{g+go}{X = array[:,0:4]}
\PYG{g+go}{Y = array[:,4]}
\PYG{g+go}{test\PYGZus{}size = 0.20}
\PYG{g+go}{seed = 7}
\PYG{g+go}{X\PYGZus{}train, X\PYGZus{}test, Y\PYGZus{}train, Y\PYGZus{}test = model\PYGZus{}selection.train\PYGZus{}test\PYGZus{}split(}
\PYG{g+go}{    X, Y,}
\PYG{g+go}{    test\PYGZus{}size=test\PYGZus{}size,}
\PYG{g+go}{    random\PYGZus{}state=seed)}
\PYG{g+gp}{\PYGZsh{} }Step\PYG{+w}{ }\PYG{l+m}{2}:\PYG{+w}{ }Fit\PYG{+w}{ }the\PYG{+w}{ }model\PYG{+w}{ }and\PYG{+w}{ }make\PYG{+w}{ }predictions\PYG{+w}{ }on\PYG{+w}{ }the\PYG{+w}{ }\PYG{n+nb}{test}\PYG{+w}{ }dataset
\PYG{g+go}{knn = KNeighborsClassifier()}
\PYG{g+go}{knn.fit(X\PYGZus{}train, Y\PYGZus{}train)}
\PYG{g+go}{predictions = knn.predict(X\PYGZus{}test)}
\PYG{g+gp}{\PYGZsh{} }Step\PYG{+w}{ }\PYG{l+m}{3}:\PYG{+w}{ }Save\PYG{+w}{ }the\PYG{+w}{ }classification\PYG{+w}{ }report
\PYG{g+go}{report = classification\PYGZus{}report(Y\PYGZus{}test, predictions, output\PYGZus{}dict=True)}
\PYG{g+go}{df\PYGZus{}report = pd.DataFrame(report).transpose().to\PYGZus{}csv(args.output\PYGZus{}report)}
\PYG{g+go}{EOT}
\end{sphinxVerbatim}
\sphinxresetverbatimhllines

\sphinxAtStartPar
This script will
\begin{itemize}
\item {} 
\sphinxAtStartPar
take three positional arguments: The input data, a path to save a figure under, and path to save the final prediction report under. By including these input and output specifications in a \sphinxcode{\sphinxupquote{datalad run}} command when we run the analysis, we can ensure that input data is retrieved prior to the script execution, and that as much actionable provenance as possible is recorded%
\begin{footnote}\sphinxAtStartFootnote
Alternatively, if you were to use DataLad’s Python API, you could import and expose it as \sphinxcode{\sphinxupquote{dl.\textless{}COMMAND\textgreater{}}} and \sphinxcode{\sphinxupquote{dl.get()}} the relevant files. This however, would not record them as provenance in the dataset’s history.
%
\end{footnote}.

\item {} 
\sphinxAtStartPar
read in the data, perform the analysis, and save the resulting figure and \sphinxcode{\sphinxupquote{.csv}} prediction report into the root of \sphinxcode{\sphinxupquote{midterm\_project/}}. Note how this helps to fulfil YODA principle 1 on modularity:
Results are stored outside of the pristine input subdataset.

\end{itemize}

\sphinxAtStartPar
A short help text explains how the script shall be used:

\begin{sphinxVerbatim}[commandchars=\\\{\}]
\PYG{g+gp}{\PYGZdl{} }python\PYG{+w}{ }code/script.py\PYG{+w}{ }\PYGZhy{}h
\PYG{g+go}{usage: script.py [\PYGZhy{}h] data output\PYGZus{}figure output\PYGZus{}report}

\PYG{g+go}{Analyze iris data}

\PYG{g+go}{positional arguments:}
\PYG{g+go}{   data           Input data (CSV) to process}
\PYG{g+go}{   output\PYGZus{}figure  Output figure path}
\PYG{g+go}{   output\PYGZus{}report  Output report path}

\PYG{g+go}{optional arguments:}
\PYG{g+go}{\PYGZhy{}h, \PYGZhy{}\PYGZhy{}help     show this help message and exit}
\end{sphinxVerbatim}

\sphinxAtStartPar
The script execution would thus be \sphinxcode{\sphinxupquote{python3 code/script.py \textless{}path\sphinxhyphen{}to\sphinxhyphen{}input\textgreater{} \textless{}path\sphinxhyphen{}to\sphinxhyphen{}figure\sphinxhyphen{}output\textgreater{} \textless{}path\sphinxhyphen{}to\sphinxhyphen{}report\sphinxhyphen{}output\textgreater{}}}.
When parametrizing the input and output path parameters, we just need make sure that all paths  are \sphinxstyleemphasis{relative}, such that the \sphinxcode{\sphinxupquote{midterm\_project}} analysis is completely self\sphinxhyphen{}contained within the dataset, contributing to fulfill the second YODA principle.

\sphinxAtStartPar
Let’s run a quick \sphinxcode{\sphinxupquote{datalad status}}…

\begin{sphinxVerbatim}[commandchars=\\\{\}]
\PYG{g+gp}{\PYGZdl{} }datalad\PYG{+w}{ }status
\PYG{g+go}{untracked: code/script.py (file)}
\end{sphinxVerbatim}

\index{tag dataset version@\spxentry{tag dataset version}!with DataLad@\spxentry{with DataLad}}\index{with DataLad@\spxentry{with DataLad}!tag dataset version@\spxentry{tag dataset version}}\ignorespaces 
\sphinxAtStartPar
… and save the script to the subdataset’s history. As the script completes your
analysis setup, we \sphinxstyleemphasis{tag} the state of the dataset to refer to it easily at a later
point with the \sphinxcode{\sphinxupquote{\sphinxhyphen{}\sphinxhyphen{}version\sphinxhyphen{}tag}} option of \sphinxcode{\sphinxupquote{datalad save}}. Check the
\textit{Find-out-more}~{\findoutmoreiconinline}\textit{\ref{fom-what-is-a-tag}} {\hyperref[\detokenize{basics/101-130-yodaproject:fom-what-is-a-tag}]{\sphinxcrossref{\DUrole{std,std-ref}{on tags}}}} (\autopageref*{\detokenize{basics/101-130-yodaproject:fom-what-is-a-tag}}) for details.

% beauty pagebreak
\newpage

\begin{sphinxVerbatim}[commandchars=\\\{\}]
\PYG{g+gp}{\PYGZdl{} }datalad\PYG{+w}{ }save\PYG{+w}{ }\PYGZhy{}m\PYG{+w}{ }\PYG{l+s+s2}{\PYGZdq{}add script for kNN classification and plotting\PYGZdq{}}\PYG{+w}{ }\PYG{l+s+se}{\PYGZbs{}}
\PYG{+w}{  }\PYGZhy{}\PYGZhy{}version\PYGZhy{}tag\PYG{+w}{ }ready4analysis\PYG{+w}{ }\PYG{l+s+se}{\PYGZbs{}}
\PYG{+w}{  }code/script.py
\PYG{g+go}{add(ok): code/script.py (file)}
\PYG{g+go}{save(ok): . (dataset)}
\end{sphinxVerbatim}

\index{tag@\spxentry{tag}!Git concept@\spxentry{Git concept}}\index{Git concept@\spxentry{Git concept}!tag@\spxentry{tag}}\index{show@\spxentry{show}!Git command@\spxentry{Git command}}\index{Git command@\spxentry{Git command}!show@\spxentry{show}}\index{rerun command@\spxentry{rerun command}!with DataLad@\spxentry{with DataLad}}\index{with DataLad@\spxentry{with DataLad}!rerun command@\spxentry{rerun command}}\ignorespaces \begin{findoutmore}[label={fom-what-is-a-tag}, before title={\thetcbcounter\ }, float, floatplacement=tb, check odd page=true]{What is a tag?}
\label{\detokenize{basics/101-130-yodaproject:fom-what-is-a-tag}}

\sphinxAtStartPar
A {\hyperref[\detokenize{glossary:term-tag}]{\sphinxtermref{\DUrole{xref,std,std-term}{tag}}}} is a marker that you can attach to a commit in your dataset history.
Tags can have any name, and can help you and others to identify certain commits
or dataset states in the history of a dataset. Let’s take a look at how the tag
you just created looks like in your history with \sphinxcode{\sphinxupquote{git show}}.
Note how we can use a tag just as easily as a commit {\hyperref[\detokenize{glossary:term-shasum}]{\sphinxtermref{\DUrole{xref,std,std-term}{shasum}}}}:

\begin{sphinxVerbatim}[commandchars=\\\{\}]
\PYG{g+gp}{\PYGZdl{} }git\PYG{+w}{ }show\PYG{+w}{ }ready4analysis
\PYG{g+go}{commit ca0c7477✂SHA1}
\PYG{g+go}{Author: Elena Piscopia \PYGZlt{}elena@example.net\PYGZgt{}}
\PYG{g+go}{Date:   Tue Jun 18 16:13:00 2019 +0000}

\PYG{g+go}{    add script for kNN classification and plotting}

\PYG{g+go}{diff \PYGZhy{}\PYGZhy{}git a/code/script.py b/code/script.py}
\PYG{g+go}{new file mode 100644}
\PYG{g+go}{index 0000000..c7a6ea9}
\PYG{g+go}{\PYGZhy{}\PYGZhy{}\PYGZhy{} /dev/null}
\PYG{g+go}{+++ b/code/script.py}
\PYG{g+go}{@@ \PYGZhy{}0,0 +1,43 @@}
\end{sphinxVerbatim}

\sphinxAtStartPar
This tag thus identifies the version state of the dataset in which this script
was added.
Later we can use this tag to identify the point in time at which
the analysis setup was ready \textendash{} much more intuitive than a 40\sphinxhyphen{}character shasum!
This is handy in the context of a \sphinxcode{\sphinxupquote{datalad rerun}}, for example:

\begin{sphinxVerbatim}[commandchars=\\\{\}]
\PYG{g+gp}{\PYGZdl{} }datalad\PYG{+w}{ }rerun\PYG{+w}{ }\PYGZhy{}\PYGZhy{}since\PYG{+w}{ }ready4analysis
\end{sphinxVerbatim}

\sphinxAtStartPar
would rerun any \sphinxcode{\sphinxupquote{datalad run}} command in the history performed between tagging
and the current dataset state.


\end{findoutmore}

\sphinxAtStartPar
Finally, with your directory structure being modular and intuitive,
the input data installed, the script ready, and the dataset status clean,
you can wrap the execution of the script in a \sphinxcode{\sphinxupquote{datalad run}} command. Note that
simply executing the script would work as well \textendash{} thanks to DataLad’s Python API.
But using \sphinxcode{\sphinxupquote{datalad run}} will capture full provenance, and will make
re\sphinxhyphen{}execution with \sphinxcode{\sphinxupquote{datalad rerun}} easy.
In case you run into trouble, check the \textit{Windows-wit}~{\windowswiticoninline}\textit{\ref{ww-python-not-3}} {\hyperref[\detokenize{basics/101-130-yodaproject:ww-python-not-3}]{\sphinxcrossref{\DUrole{std,std-ref}{on different Python command names}}}} (\autopageref*{\detokenize{basics/101-130-yodaproject:ww-python-not-3}}).
\begin{importantnote}[before title={\thetcbcounter\ }, float, floatplacement=tb, check odd page=true]{Additional software requirements: pandas, seaborn, sklearn}

\sphinxAtStartPar
Note that you need to have the following Python packages installed to run the
analysis%
\begin{footnote}\sphinxAtStartFootnote
It is recommended (but optional) to create a
\dlhbhref{P1D}{virtual environment} and
install the required Python packages inside of it:

\sphinxSetupCodeBlockInFootnote
\begin{sphinxVerbatim}[commandchars=\\\{\}]
\PYG{g+gp}{\PYGZdl{} }\PYG{c+c1}{\PYGZsh{} create and enter a new virtual environment (optional)}
\PYG{g+gp}{\PYGZdl{} }virtualenv\PYG{+w}{ }\PYGZhy{}\PYGZhy{}python\PYG{o}{=}python3\PYG{+w}{ }\PYGZti{}/env/handbook
\PYG{g+gp}{\PYGZdl{} }.\PYG{+w}{ }\PYGZti{}/env/handbook/bin/activate
\end{sphinxVerbatim}

\sphinxSetupCodeBlockInFootnote
\begin{sphinxVerbatim}[commandchars=\\\{\}]
\PYG{g+gp}{\PYGZdl{} }\PYG{c+c1}{\PYGZsh{} install the Python packages from PyPi via pip}
\PYG{g+gp}{\PYGZdl{} }pip\PYG{+w}{ }install\PYG{+w}{ }seaborn\PYG{+w}{ }pandas\PYG{+w}{ }sklearn
\end{sphinxVerbatim}
%
\end{footnote}:
%\begin{itemize}
%\item {} 
%\sphinxAtStartPar
\dlhbhref{P3}{pandas},
%
%\item {} 
%\sphinxAtStartPar
\dlhbhref{P7}{seaborn},
%
%\item {} 
%\sphinxAtStartPar
\dlhbhref{S4}{sklearn}

%\end{itemize}

\sphinxAtStartPar
The packages can be installed via {\hyperref[\detokenize{glossary:term-pip}]{\sphinxtermref{\DUrole{xref,std,std-term}{pip}}}}.
However, if you do not want to install any
Python packages, do not execute the remaining code examples in this section
\textendash{} an upcoming section on \sphinxcode{\sphinxupquote{datalad containers\sphinxhyphen{}run}} will allow you to
perform the analysis without changing your Python software\sphinxhyphen{}setup.


\end{importantnote}

\index{python instead of python3@\spxentry{python instead of python3}!on Windows@\spxentry{on Windows}}\index{on Windows@\spxentry{on Windows}!python instead of python3@\spxentry{python instead of python3}}\ignorespaces \begin{windowswit}[label={ww-python-not-3}, before title={\thetcbcounter\ }, float, floatplacement=tb, check odd page=true]{You may need to use ‘python’, not ‘python3’}
\label{\detokenize{basics/101-130-yodaproject:ww-python-not-3}}

\sphinxAtStartPar
If executing the code below returns an exit code of 9009, there may be no \sphinxcode{\sphinxupquote{python3}} \textendash{} instead, it is called just \sphinxcode{\sphinxupquote{python}}.
Please run the following instead (adjusted for line breaks, you should be able to copy\sphinxhyphen{}paste this as a whole):

\begin{sphinxVerbatim}[commandchars=\\\{\}]
\PYG{g+go}{\PYGZgt{} datalad run \PYGZhy{}m \PYGZdq{}analyze iris data with classification analysis\PYGZdq{} \PYGZca{}}
\PYG{g+go}{ \PYGZhy{}\PYGZhy{}input \PYGZdq{}input/iris.csv\PYGZdq{} \PYGZca{}}
\PYG{g+go}{ \PYGZhy{}\PYGZhy{}output \PYGZdq{}pairwise\PYGZus{}relationships.png\PYGZdq{} \PYGZca{}}
\PYG{g+go}{ \PYGZhy{}\PYGZhy{}output \PYGZdq{}prediction\PYGZus{}report.csv\PYGZdq{} \PYGZca{}}
\PYG{g+go}{ \PYGZdq{}python code/script.py \PYGZob{}inputs\PYGZcb{} \PYGZob{}outputs\PYGZcb{}\PYGZdq{}}
\end{sphinxVerbatim}


\end{windowswit}

% beauty pagebreak
\newpage

\index{run command with provenance capture@\spxentry{run command with provenance capture}!with DataLad@\spxentry{with DataLad}}\index{with DataLad@\spxentry{with DataLad}!run command with provenance capture@\spxentry{run command with provenance capture}}\ignorespaces 
\def\sphinxLiteralBlockLabel{\label{\detokenize{basics/101-130-yodaproject:index-9}}}
\begin{sphinxVerbatim}[commandchars=\\\{\}]
\PYG{g+gp}{\PYGZdl{} }datalad\PYG{+w}{ }run\PYG{+w}{ }\PYGZhy{}m\PYG{+w}{ }\PYG{l+s+s2}{\PYGZdq{}analyze iris data with classification analysis\PYGZdq{}}\PYG{+w}{ }\PYG{l+s+se}{\PYGZbs{}}
\PYG{+w}{  }\PYGZhy{}\PYGZhy{}input\PYG{+w}{ }\PYG{l+s+s2}{\PYGZdq{}input/iris.csv\PYGZdq{}}\PYG{+w}{ }\PYG{l+s+se}{\PYGZbs{}}
\PYG{+w}{  }\PYGZhy{}\PYGZhy{}output\PYG{+w}{ }\PYG{l+s+s2}{\PYGZdq{}pairwise\PYGZus{}relationships.png\PYGZdq{}}\PYG{+w}{ }\PYG{l+s+se}{\PYGZbs{}}
\PYG{+w}{  }\PYGZhy{}\PYGZhy{}output\PYG{+w}{ }\PYG{l+s+s2}{\PYGZdq{}prediction\PYGZus{}report.csv\PYGZdq{}}\PYG{+w}{ }\PYG{l+s+se}{\PYGZbs{}}
\PYG{+w}{  }\PYG{l+s+s2}{\PYGZdq{}python3 code/script.py \PYGZob{}inputs\PYGZcb{} \PYGZob{}outputs\PYGZcb{}\PYGZdq{}}
\PYG{g+go}{get(ok): input/iris.csv (file) [from web...]}
\PYG{g+go}{[INFO] == Command start (output follows) =====}
\PYG{g+go}{[INFO] == Command exit (modification check follows) =====}
\PYG{g+go}{run(ok): /home/me/dl\PYGZhy{}101/DataLad\PYGZhy{}101/midterm\PYGZus{}project (dataset) [python3 code/script.py input/iris.csv pa...]}
\PYG{g+go}{add(ok): pairwise\PYGZus{}relationships.png (file)}
\PYG{g+go}{add(ok): prediction\PYGZus{}report.csv (file)}
\PYG{g+go}{save(ok): . (dataset)}
\end{sphinxVerbatim}

\sphinxAtStartPar
As the successful command summary indicates, your analysis seems to work! Two
files were created and saved to the dataset: \sphinxcode{\sphinxupquote{pairwise\_relationships.png}}
and \sphinxcode{\sphinxupquote{prediction\_report.csv}}. If you want, take a look and interpret
your analysis. But what excites you even more than a successful data science
project on first try is that you achieved complete provenance capture:
\begin{itemize}
\item {} 
\sphinxAtStartPar
Every single file in this dataset is associated with an author and a time
stamp for each modification thanks to \sphinxcode{\sphinxupquote{datalad save}}.

\item {} 
\sphinxAtStartPar
The raw dataset knows where the data came from thanks to \sphinxcode{\sphinxupquote{datalad clone}}
and \sphinxcode{\sphinxupquote{datalad download\sphinxhyphen{}url}}.

\item {} 
\sphinxAtStartPar
The subdataset is linked to the superdataset thanks to
\sphinxcode{\sphinxupquote{datalad clone \sphinxhyphen{}d}}.

\item {} 
\sphinxAtStartPar
The \sphinxcode{\sphinxupquote{datalad run}} command took care of linking the outputs of your
analysis with the script and the input data it was generated from, fulfilling
the third YODA principle.

\end{itemize}

\sphinxAtStartPar
Let’s take a look at the history of the \sphinxcode{\sphinxupquote{midterm\_project}} analysis
dataset:

\begin{sphinxVerbatim}[commandchars=\\\{\}]
\PYG{g+gp}{\PYGZdl{} }git\PYG{+w}{ }log\PYG{+w}{ }\PYGZhy{}\PYGZhy{}oneline
\PYG{g+go}{9aadac7 [DATALAD RUNCMD] analyze iris data with classification analysis}
\PYG{g+go}{ca0c747 add script for kNN classification and plotting}
\PYG{g+go}{4f945ed [DATALAD] Added subdataset}
\PYG{g+go}{18f4a98 Apply YODA dataset setup}
\PYG{g+go}{bf231d5 [DATALAD] new dataset}
\end{sphinxVerbatim}

\sphinxAtStartPar
“Wow, this is so clean and intuitive!” you congratulate yourself. “And I think
this was and will be the fastest I have ever completed a midterm project!”
But what is still missing is a human readable description of your dataset.
The YODA procedure kindly placed a \sphinxcode{\sphinxupquote{README.md}} file into the root of your
dataset that you can use for this%
\begin{footnote}\sphinxAtStartFootnote
All \sphinxcode{\sphinxupquote{README.md}} files the YODA procedure created are
version controlled by Git, not git\sphinxhyphen{}annex, thanks to the
configurations that YODA supplied. This makes it easy to change the
\sphinxcode{\sphinxupquote{README.md}} file. The previous section detailed how the YODA procedure
configured your dataset. If you want to re\sphinxhyphen{}read the full chapter on
configurations and run\sphinxhyphen{}procedures, start with section {\hyperref[\detokenize{basics/101-122-config:config}]{\sphinxcrossref{\DUrole{std,std-ref}{Local configuration}}}} (\autopageref*{\detokenize{basics/101-122-config:config}}).
%
\end{footnote}.

\begin{sphinxVerbatim}[commandchars=\\\{\}]
\PYG{g+gp}{\PYGZdl{} }\PYG{c+c1}{\PYGZsh{} with the \PYGZgt{}| redirection we are replacing existing contents in the file}
\PYG{g+gp}{\PYGZdl{} }cat\PYG{+w}{ }\PYG{l+s}{\PYGZlt{}\PYGZlt{} EOT \PYGZgt{}| RE}ADME.md

\PYG{g+gp}{\PYGZsh{} }Midterm\PYG{+w}{ }YODA\PYG{+w}{ }Data\PYG{+w}{ }Analysis\PYG{+w}{ }Project

\PYG{g+gp}{\PYGZsh{}}\PYG{c+c1}{\PYGZsh{} Dataset structure}

\PYG{g+go}{\PYGZhy{} All inputs (i.e. building blocks from other sources) are located in input/.}
\PYG{g+go}{\PYGZhy{} All custom code is located in code/.}
\PYG{g+go}{\PYGZhy{} All results (i.e., generated files) are located in the root of the dataset:}
\PYG{g+go}{  \PYGZhy{} \PYGZdq{}prediction\PYGZus{}report.csv\PYGZdq{} contains the main classification metrics.}
\PYG{g+go}{  \PYGZhy{} \PYGZdq{}output/pairwise\PYGZus{}relationships.png\PYGZdq{} is a plot of the relations between features.}

\PYG{g+go}{EOT}
\end{sphinxVerbatim}

\begin{sphinxVerbatim}[commandchars=\\\{\}]
\PYG{g+gp}{\PYGZdl{} }datalad\PYG{+w}{ }status
\PYG{g+go}{ modified: README.md (file)}
\end{sphinxVerbatim}

\begin{sphinxVerbatim}[commandchars=\\\{\}]
\PYG{g+gp}{\PYGZdl{} }datalad\PYG{+w}{ }save\PYG{+w}{ }\PYGZhy{}m\PYG{+w}{ }\PYG{l+s+s2}{\PYGZdq{}Provide project description\PYGZdq{}}\PYG{+w}{ }README.md
\PYG{g+go}{add(ok): README.md (file)}
\PYG{g+go}{save(ok): . (dataset)}
\end{sphinxVerbatim}

\sphinxAtStartPar
Note that one feature of the YODA procedure was that it configured certain files
(for example, everything inside of \sphinxcode{\sphinxupquote{code/}}, and the \sphinxcode{\sphinxupquote{README.md}} file in the
root of the dataset) to be saved in Git instead of git\sphinxhyphen{}annex. This was the
reason why the \sphinxcode{\sphinxupquote{README.md}} in the root of the dataset was easily modifiable.
See the \textit{Find-out-more}~{\findoutmoreiconinline}\textit{\ref{fom-save-to-git}} {\hyperref[\detokenize{basics/101-130-yodaproject:fom-save-to-git}]{\sphinxcrossref{\DUrole{std,std-ref}{on an alternative approach}}}} (\autopageref*{\detokenize{basics/101-130-yodaproject:fom-save-to-git}}) that can achieve the same result.

\index{save@\spxentry{save}!DataLad command@\spxentry{DataLad command}}\index{DataLad command@\spxentry{DataLad command}!save@\spxentry{save}}\index{save file content directly in Git (no annex)@\spxentry{save file content directly in Git}\spxextra{no annex}!with DataLad@\spxentry{with DataLad}}\index{with DataLad@\spxentry{with DataLad}!save file content directly in Git (no annex)@\spxentry{save file content directly in Git}\spxextra{no annex}}\ignorespaces \begin{findoutmore}[label={fom-save-to-git}, before title={\thetcbcounter\ }, float, floatplacement=tb, check odd page=true]{Saving contents to Git regardless of configuration}
\label{\detokenize{basics/101-130-yodaproject:fom-save-to-git}}

\sphinxAtStartPar
The \sphinxcode{\sphinxupquote{yoda}} procedure in \sphinxcode{\sphinxupquote{midterm\_project}} applied a different configuration
within \sphinxcode{\sphinxupquote{.gitattributes}} than the \sphinxcode{\sphinxupquote{text2git}} procedure did in \sphinxcode{\sphinxupquote{DataLad\sphinxhyphen{}101}}.
Within \sphinxcode{\sphinxupquote{DataLad\sphinxhyphen{}101}}, any text file is automatically stored in {\hyperref[\detokenize{glossary:term-Git}]{\sphinxtermref{\DUrole{xref,std,std-term}{Git}}}}.
This is not true in \sphinxcode{\sphinxupquote{midterm\_project}}: Only the existing \sphinxcode{\sphinxupquote{README.md}} files and
anything within \sphinxcode{\sphinxupquote{code/}} are stored \textendash{} everything else will be annexed.
That means that if you create any other file, even text files, inside of
\sphinxcode{\sphinxupquote{midterm\_project}} (but not in \sphinxcode{\sphinxupquote{code/}}), it will be managed by {\hyperref[\detokenize{glossary:term-git-annex}]{\sphinxtermref{\DUrole{xref,std,std-term}{git\sphinxhyphen{}annex}}}}
and content\sphinxhyphen{}locked after a \sphinxcode{\sphinxupquote{datalad save}} \textendash{} an inconvenience if it
would be a file that is small enough to be handled by Git.

\sphinxAtStartPar
Luckily, there is a handy shortcut to saving files in Git that does not
require you to edit configurations in \sphinxcode{\sphinxupquote{.gitattributes}}: The \sphinxcode{\sphinxupquote{\sphinxhyphen{}\sphinxhyphen{}to\sphinxhyphen{}git}}
option for \sphinxcode{\sphinxupquote{datalad save}}.

\begin{sphinxVerbatim}[commandchars=\\\{\}]
\PYG{g+gp}{\PYGZdl{} }datalad\PYG{+w}{ }save\PYG{+w}{ }\PYGZhy{}m\PYG{+w}{ }\PYG{l+s+s2}{\PYGZdq{}add sometextfile.txt\PYGZdq{}}\PYG{+w}{ }\PYGZhy{}\PYGZhy{}to\PYGZhy{}git\PYG{+w}{ }sometextfile.txt
\end{sphinxVerbatim}


\end{findoutmore}

\sphinxAtStartPar
After adding this short description to your \sphinxcode{\sphinxupquote{README.md}}, your dataset now also
contains sufficient human\sphinxhyphen{}readable information to ensure that others can understand
everything you did easily.
The only thing left to do is to hand in your assignment. According to the
syllabus, this should be done via {\hyperref[\detokenize{glossary:term-GitHub}]{\sphinxtermref{\DUrole{xref,std,std-term}{GitHub}}}}. Check the \textit{Find-out-more}~{\findoutmoreiconinline}\textit{\ref{fom-github}} {\hyperref[\detokenize{basics/101-130-yodaproject:fom-github}]{\sphinxcrossref{\DUrole{std,std-ref}{on this platform}}}} (\autopageref*{\detokenize{basics/101-130-yodaproject:fom-github}}), if you are not familiar with it yet.

\vspace{1cm}
\noindent{\hspace*{\fill}\sphinxincludegraphics[height=4cm]{{pull_request_bw}.pdf}\hspace*{\fill}}


\index{dataset hosting@\spxentry{dataset hosting}!GitHub@\spxentry{GitHub}}\ignorespaces \begin{findoutmore}[label={fom-github}, before title={\thetcbcounter\ }, float, floatplacement=p, check odd page=true]{What is GitHub?}
\label{\detokenize{basics/101-130-yodaproject:fom-github}}

\sphinxAtStartPar
GitHub is a web based hosting service for Git repositories. Among many
different other useful perks it adds features that allow collaboration on
Git repositories. \dlhbhref{G8}{GitLab} is a similar
service with highly similar features, but its source code is free and open,
whereas GitHub is a subsidiary of Microsoft.

\sphinxAtStartPar
Web\sphinxhyphen{}hosting services like GitHub and {\hyperref[\detokenize{glossary:term-GitLab}]{\sphinxtermref{\DUrole{xref,std,std-term}{GitLab}}}} integrate wonderfully with
DataLad. They are especially useful for making your dataset publicly available,
if you have figured out storage for your large files otherwise (as large content
cannot be hosted for free by GitHub). You can make DataLad publish large file content to one location
and afterwards automatically push an update to GitHub, such that
users can install directly from GitHub/GitLab and seemingly also obtain large file
content from GitHub. GitHub can also resolve subdataset links to other GitHub
repositories, which lets you navigate through nested datasets in the web\sphinxhyphen{}interface.

\vspace{1mm}
\noindent{\hspace*{\fill}\sphinxincludegraphics[width=.92\linewidth]{{screenshot_midtermproject_bw}.png}\hspace*{\fill}}
\vspace{1mm}

\sphinxAtStartPar
The above screenshot shows the linkage between the analysis project you will create
and its subdataset. Clicking on the subdataset (highlighted) will take you to the iris dataset
the online\sphinxhyphen{}handbook provides, shown below.

\vspace{1mm}
\noindent{\hspace*{\fill}\sphinxincludegraphics[width=.92\linewidth]{{screenshot_midtermproject_submodule_bw}.png}\hspace*{\fill}}
\vspace{1mm}

\end{findoutmore}

\index{create\sphinxhyphen{}sibling\sphinxhyphen{}github@\spxentry{create\sphinxhyphen{}sibling\sphinxhyphen{}github}!DataLad command@\spxentry{DataLad command}}\index{DataLad command@\spxentry{DataLad command}!create\sphinxhyphen{}sibling\sphinxhyphen{}github@\spxentry{create\sphinxhyphen{}sibling\sphinxhyphen{}github}}\ignorespaces 

\subsection{Publishing the dataset to GitHub}
\label{\detokenize{basics/101-130-yodaproject:publishing-the-dataset-to-github}}\label{\detokenize{basics/101-130-yodaproject:publishtogithub}}\label{\detokenize{basics/101-130-yodaproject:index-12}}
\sphinxAtStartPar
For this, you need to
\begin{itemize}
\item {} 
\sphinxAtStartPar
create a GitHub account, if you do not yet have one

\item {} 
\sphinxAtStartPar
create a repository for this dataset on GitHub,

\item {} 
\sphinxAtStartPar
configure this GitHub repository to be a {\hyperref[\detokenize{glossary:term-sibling}]{\sphinxtermref{\DUrole{xref,std,std-term}{sibling}}}} of the \sphinxcode{\sphinxupquote{midterm\_project}} dataset,

\item {} 
\sphinxAtStartPar
and \sphinxstyleemphasis{publish} your dataset to GitHub.

\end{itemize}

\index{create\sphinxhyphen{}sibling\sphinxhyphen{}gitlab@\spxentry{create\sphinxhyphen{}sibling\sphinxhyphen{}gitlab}!DataLad command@\spxentry{DataLad command}}\index{DataLad command@\spxentry{DataLad command}!create\sphinxhyphen{}sibling\sphinxhyphen{}gitlab@\spxentry{create\sphinxhyphen{}sibling\sphinxhyphen{}gitlab}}\ignorespaces 
\sphinxAtStartPar
Luckily, DataLad can make this very easy with the
\sphinxcode{\sphinxupquote{datalad create\sphinxhyphen{}sibling\sphinxhyphen{}github}}
command (or, for \dlhbhref{G8}{GitLab}, \sphinxcode{\sphinxupquote{datalad create\sphinxhyphen{}sibling\sphinxhyphen{}gitlab}}).

\sphinxAtStartPar
The two commands have different arguments and options.
Here, we look at \sphinxcode{\sphinxupquote{datalad create\sphinxhyphen{}sibling\sphinxhyphen{}github}}.
The command takes a repository name and GitHub authentication credentials
(either in the command line call with options \sphinxcode{\sphinxupquote{github\sphinxhyphen{}login \textless{}TOKEN\textgreater{}}}, with an \sphinxstyleemphasis{oauth} \dlhbhref{G3B}{token} stored in the Git
configuration, or interactively).

\index{GitHub token@\spxentry{GitHub token}!credential@\spxentry{credential}}\index{credential@\spxentry{credential}!GitHub token@\spxentry{GitHub token}}\ignorespaces \begin{importantnote}[label={index-14}, before title={\thetcbcounter\ }, check odd page=true]{Generate a GitHub token}
\label{\detokenize{basics/101-130-yodaproject:index-14}}

\sphinxAtStartPar
GitHub \dlhbhref{G9A}{deprecated user\sphinxhyphen{}password authentication} and instead supports authentication via personal access token.
To ensure successful authentication, don’t supply your password, but create a personal access token at \dlhbhref{G2P}{github.com/settings/tokens}%
\begin{footnote}\sphinxAtStartFootnote
Instead of using GitHub’s WebUI you could also obtain a token using the command line GitHub interface (\sphinxurl{https://github.com/sociomantic-tsunami/git-hub}) by running \sphinxcode{\sphinxupquote{git hub setup}} (if no 2FA is used).
%
\end{footnote} instead, and either
\begin{itemize}
\item {} 
\sphinxAtStartPar
supply the token with the argument \sphinxcode{\sphinxupquote{\sphinxhyphen{}\sphinxhyphen{}github\sphinxhyphen{}login \textless{}TOKEN\textgreater{}}} from the command line,

\item {} 
\sphinxAtStartPar
or supply the token from the command line when queried for a password

\end{itemize}


\end{importantnote}

\sphinxAtStartPar
Based on the credentials and the
repository name, it will create a new, empty repository on GitHub, and
configure this repository as a sibling of the dataset.
If you are asked to enter a credential, and things feel weird, check
the \textit{Windows-wit}~{\windowswiticoninline}\textit{\ref{ww-ghost-credentials}} {\hyperref[\detokenize{basics/101-130-yodaproject:ww-ghost-credentials}]{\sphinxcrossref{\DUrole{std,std-ref}{on typing passwords}}}} (\autopageref*{\detokenize{basics/101-130-yodaproject:ww-ghost-credentials}}).

\index{credential@\spxentry{credential}!entry@\spxentry{entry}}\index{typed credentials are not displayed@\spxentry{typed credentials are not displayed}!on Windows@\spxentry{on Windows}}\index{on Windows@\spxentry{on Windows}!typed credentials are not displayed@\spxentry{typed credentials are not displayed}}\ignorespaces \begin{windowswit}[label={ww-ghost-credentials}, before title={\thetcbcounter\ }, float, floatplacement=tb, check odd page=true]{Your shell will not display credentials}
\label{\detokenize{basics/101-130-yodaproject:ww-ghost-credentials}}

\sphinxAtStartPar
Don’t be confused if you are prompted for your GitHub credentials, but can’t seem to type \textendash{} the terminal protects your private information by not displaying what you type.
Simply type in what is requested, and press enter.


\end{windowswit}

\begin{sphinxVerbatim}[commandchars=\\\{\}]
\PYG{g+gp}{\PYGZdl{} }datalad\PYG{+w}{ }create\PYGZhy{}sibling\PYGZhy{}github\PYG{+w}{ }\PYGZhy{}d\PYG{+w}{ }.\PYG{+w}{ }midtermproject
\PYG{g+go}{.: github(\PYGZhy{}) [https://github.com/adswa/midtermproject.git (git)]}
\PYG{g+go}{\PYGZsq{}https://github.com/adswa/midtermproject.git\PYGZsq{} configured as sibling \PYGZsq{}github\PYGZsq{} for \PYGZlt{}Dataset path=/home/me/dl\PYGZhy{}101/DataLad\PYGZhy{}101/midterm\PYGZus{}project\PYGZgt{}}
\end{sphinxVerbatim}

\sphinxAtStartPar
Verify that this worked by listing the siblings of the dataset:

\begin{sphinxVerbatim}[commandchars=\\\{\}]
\PYG{g+gp}{\PYGZdl{} }datalad\PYG{+w}{ }siblings
\PYG{g+go}{[WARNING] Failed to determine if github carries annex.}
\PYG{g+go}{.: here(+) [git]}
\PYG{g+go}{.: github(\PYGZhy{}) [https://github.com/adswa/midtermproject.git (git)]}
\end{sphinxVerbatim}

\index{sibling (GitHub)@\spxentry{sibling}\spxextra{GitHub}!DataLad concept@\spxentry{DataLad concept}}\index{DataLad concept@\spxentry{DataLad concept}!sibling (GitHub)@\spxentry{sibling}\spxextra{GitHub}}\ignorespaces \begin{gitusernote}[label={index-16}, before title={\thetcbcounter\ }, check odd page=true]{Create-sibling-github internals}
\label{\detokenize{basics/101-130-yodaproject:index-16}}

\sphinxAtStartPar
Creating a sibling on GitHub will create a new empty repository under the
account that you provide and set up a \sphinxstyleemphasis{remote} to this repository. Upon a
\sphinxcode{\sphinxupquote{datalad push}} to this sibling, your datasets history
will be pushed there.


\end{gitusernote}

\index{push@\spxentry{push}!DataLad concept@\spxentry{DataLad concept}}\index{DataLad concept@\spxentry{DataLad concept}!push@\spxentry{push}}\index{push (dataset)@\spxentry{push}\spxextra{dataset}!with DataLad@\spxentry{with DataLad}}\index{with DataLad@\spxentry{with DataLad}!push (dataset)@\spxentry{push}\spxextra{dataset}}\ignorespaces 
\sphinxAtStartPar
On GitHub, you will see a new, empty repository with the name
\sphinxcode{\sphinxupquote{midtermproject}}. However, the repository does not yet contain
any of your dataset’s history or files. This requires \sphinxstyleemphasis{publishing} the current
state of the dataset to this {\hyperref[\detokenize{glossary:term-sibling}]{\sphinxtermref{\DUrole{xref,std,std-term}{sibling}}}} with the \sphinxcode{\sphinxupquote{datalad push}}
command.
\begin{importantnote}[before title={\thetcbcounter\ }, check odd page=true]{Learn how to push “on the job”}

\sphinxAtStartPar
Publishing is one of the remaining big concepts that this book tries to
convey. However, publishing is a complex concept that encompasses a large
proportion of this book’s content so far as a prerequisite. In order to be
not too overwhelmingly detailed, the upcoming sections will approach
\sphinxcode{\sphinxupquote{datalad push}} from a “learning\sphinxhyphen{}by\sphinxhyphen{}doing” perspective:
First, you will see a \sphinxcode{\sphinxupquote{datalad push}} to GitHub, and “On the looks and feels of a published dataset”
will already give a practical glimpse into the
difference between annexed contents and contents stored in Git when pushed
to GitHub. The chapter {\hyperref[\detokenize{basics/basics-thirdparty:chapter-thirdparty}]{\sphinxcrossref{\DUrole{std,std-ref}{Distributing datasets}}}} (\autopageref*{\detokenize{basics/basics-thirdparty:chapter-thirdparty}}) will extend on this,
but the section {\hyperref[\detokenize{basics/101-141-push:push}]{\sphinxcrossref{\DUrole{std,std-ref}{The datalad push command}}}} (\autopageref*{\detokenize{basics/101-141-push:push}})
will finally combine and link all the previous contents to give a comprehensive
and detailed wrap up of the concept of publishing datasets. In this section,
you will also find a detailed overview on how \sphinxcode{\sphinxupquote{datalad push}} works and which
options are available. If you are impatient or need an overview on publishing,
feel free to skip ahead. If you have time to follow along, reading the next
sections will get you towards a complete picture of publishing a bit more
small\sphinxhyphen{}stepped and gently.
For now, we will start with learning by doing, and
the fundamental basics of \sphinxcode{\sphinxupquote{datalad push}}: The command
will make the last saved state of your dataset available (i.e., publish it)
to the {\hyperref[\detokenize{glossary:term-sibling}]{\sphinxtermref{\DUrole{xref,std,std-term}{sibling}}}} you provide with the \sphinxcode{\sphinxupquote{\sphinxhyphen{}\sphinxhyphen{}to}} option.


\end{importantnote}

\begin{sphinxVerbatim}[commandchars=\\\{\}]
\PYG{g+gp}{\PYGZdl{} }datalad\PYG{+w}{ }push\PYG{+w}{ }\PYGZhy{}\PYGZhy{}to\PYG{+w}{ }github
\PYG{g+go}{copy(ok): pairwise\PYGZus{}relationships.png (file) [to github...]}
\PYG{g+go}{copy(ok): prediction\PYGZus{}report.csv (file) [to github...]}
\PYG{g+go}{publish(ok): . (dataset) [refs/heads/git\PYGZhy{}annex\PYGZhy{}\PYGZgt{}github:refs/heads/git\PYGZhy{}annex ✂FROM✂..✂TO✂]}
\PYG{g+go}{publish(ok): . (dataset) [refs/heads/main\PYGZhy{}\PYGZgt{}github:refs/heads/main [new branch]]}
\end{sphinxVerbatim}

\sphinxAtStartPar
Thus, you have now published your dataset’s history to a public place for others
to see and clone. Now we will explore how this may look and feel for others.
\begin{gitusernote}[before title={\thetcbcounter\ }, check odd page=true]{Push internals}

\sphinxAtStartPar
The \sphinxcode{\sphinxupquote{datalad push}} uses \sphinxcode{\sphinxupquote{git push}}, and \sphinxcode{\sphinxupquote{git annex copy}} under
the hood. Publication targets need to either be configured remote Git repositories,
or git\sphinxhyphen{}annex special remotes (if they support data upload).


\end{gitusernote}

\sphinxAtStartPar
There is one important detail first, though: By default, your tags will not be published.
Thus, the tag \sphinxcode{\sphinxupquote{ready4analysis}} is not pushed to GitHub, and currently this
version identifier is unavailable to anyone else but you.
The reason for this is that tags are viral \textendash{} they can be removed locally, and old
published tags can cause confusion or unwanted changes. In order to publish a tag,
an additional \sphinxcode{\sphinxupquote{git push}}  with the \sphinxcode{\sphinxupquote{\sphinxhyphen{}\sphinxhyphen{}tags}} option is required:

\index{push@\spxentry{push}!DataLad concept@\spxentry{DataLad concept}}\index{DataLad concept@\spxentry{DataLad concept}!push@\spxentry{push}}\index{push (tag)@\spxentry{push}\spxextra{tag}!with Git@\spxentry{with Git}}\index{with Git@\spxentry{with Git}!push (tag)@\spxentry{push}\spxextra{tag}}\ignorespaces 
\def\sphinxLiteralBlockLabel{\label{\detokenize{basics/101-130-yodaproject:index-18}}}
\begin{sphinxVerbatim}[commandchars=\\\{\}]
\PYG{g+gp}{\PYGZdl{} }git\PYG{+w}{ }push\PYG{+w}{ }github\PYG{+w}{ }\PYGZhy{}\PYGZhy{}tags
\end{sphinxVerbatim}

\index{push (tag)@\spxentry{push}\spxextra{tag}!with DataLad@\spxentry{with DataLad}}\index{with DataLad@\spxentry{with DataLad}!push (tag)@\spxentry{push}\spxextra{tag}}\ignorespaces \begin{gitusernote}[label={index-19}, before title={\thetcbcounter\ }, check odd page=true]{Pushing tags}
\label{\detokenize{basics/101-130-yodaproject:index-19}}

\sphinxAtStartPar
Note that this is a \sphinxcode{\sphinxupquote{git push}}, not \sphinxcode{\sphinxupquote{datalad push}}.
Tags could be pushed upon a \sphinxcode{\sphinxupquote{datalad push}}, though, if one
configures (what kind of) tags to be pushed. This would need to be done
on a per\sphinxhyphen{}sibling basis in \sphinxcode{\sphinxupquote{.git/config}} in the \sphinxcode{\sphinxupquote{remote.*.push}}
configuration. If you had a {\hyperref[\detokenize{glossary:term-sibling}]{\sphinxtermref{\DUrole{xref,std,std-term}{sibling}}}} “github”, the following
configuration would push all tags that start with a \sphinxcode{\sphinxupquote{v}} upon a
\sphinxcode{\sphinxupquote{datalad push \sphinxhyphen{}\sphinxhyphen{}to github}}:

\begin{sphinxVerbatim}[commandchars=\\\{\}]
\PYG{g+gp}{\PYGZdl{} }git\PYG{+w}{ }config\PYG{+w}{ }\PYGZhy{}\PYGZhy{}local\PYG{+w}{ }remote.github.push\PYG{+w}{ }\PYG{l+s+s1}{\PYGZsq{}refs/tags/v*\PYGZsq{}}
\end{sphinxVerbatim}

\sphinxAtStartPar
This configuration would result in the following entry in \sphinxcode{\sphinxupquote{.git/config}}:

\begin{sphinxVerbatim}[commandchars=\\\{\}]
\PYG{k}{[remote \PYGZdq{}github\PYGZdq{}]}
\PYG{+w}{      }\PYG{n+na}{url}\PYG{+w}{ }\PYG{o}{=}\PYG{+w}{ }\PYG{l+s}{git@github.com/adswa/midtermproject.git}
\PYG{+w}{      }\PYG{n+na}{fetch}\PYG{+w}{ }\PYG{o}{=}\PYG{+w}{ }\PYG{l+s}{+refs/heads/*:refs/remotes/github/*}
\PYG{+w}{      }\PYG{n+na}{annex\PYGZhy{}ignore}\PYG{+w}{ }\PYG{o}{=}\PYG{+w}{ }\PYG{l+s}{true}
\PYG{+w}{      }\PYG{n+na}{push}\PYG{+w}{ }\PYG{o}{=}\PYG{+w}{ }\PYG{l+s}{refs/tags/v*}
\end{sphinxVerbatim}


\end{gitusernote}

\sphinxAtStartPar
Yay! Consider your midterm project submitted! Others can now install your
dataset and check out your data science project \textendash{} and even better: they can
reproduce your data science project easily from scratch!

\index{work on published YODA dataset@\spxentry{work on published YODA dataset}!with DataLad@\spxentry{with DataLad}}\index{with DataLad@\spxentry{with DataLad}!work on published YODA dataset@\spxentry{work on published YODA dataset}}\index{rerun command@\spxentry{rerun command}!with DataLad@\spxentry{with DataLad}}\index{with DataLad@\spxentry{with DataLad}!rerun command@\spxentry{rerun command}}\ignorespaces 

\subsection{On the looks and feels of a published dataset}
\label{\detokenize{basics/101-130-yodaproject:on-the-looks-and-feels-of-a-published-dataset}}\label{\detokenize{basics/101-130-yodaproject:index-20}}
\sphinxAtStartPar
Now that you have created and published such a YODA\sphinxhyphen{}compliant dataset, you
are understandably excited how this dataset must look and feel for others.
Therefore, you decide to install this dataset into a new location on your
computer, just to get a feel for it.

\sphinxAtStartPar
Replace the \sphinxcode{\sphinxupquote{url}} in the \sphinxcode{\sphinxupquote{datalad clone}} command with the path
to your own \sphinxcode{\sphinxupquote{midtermproject}} GitHub repository, or clone the “public”
\sphinxcode{\sphinxupquote{midterm\_project}} repository that is available via the online\sphinxhyphen{}handbook’s GitHub
organization at \dlhbhref{G2G}{github.com/datalad\sphinxhyphen{}handbook/midterm\_project}:

\begin{sphinxVerbatim}[commandchars=\\\{\}]
\PYG{g+gp}{\PYGZdl{} }\PYG{n+nb}{cd}\PYG{+w}{ }../../
\PYG{g+gp}{\PYGZdl{} }datalad\PYG{+w}{ }clone\PYG{+w}{ }\PYG{l+s+s2}{\PYGZdq{}https://github.com/adswa/midtermproject.git\PYGZdq{}}
\PYG{g+go}{[INFO] Remote origin not usable by git\PYGZhy{}annex; setting annex\PYGZhy{}ignore}
\PYG{g+go}{install(ok): /home/me/dl\PYGZhy{}101/midtermproject (dataset)}
\end{sphinxVerbatim}

\sphinxAtStartPar
Let’s start with the subdataset, and see whether we can retrieve the
input \sphinxcode{\sphinxupquote{iris.csv}} file. This should not be a problem, since its origin
is recorded:

\begin{sphinxVerbatim}[commandchars=\\\{\}]
\PYG{g+gp}{\PYGZdl{} }\PYG{n+nb}{cd}\PYG{+w}{ }midtermproject
\PYG{g+gp}{\PYGZdl{} }datalad\PYG{+w}{ }get\PYG{+w}{ }input/iris.csv
\PYG{g+go}{[INFO] Remote origin not usable by git\PYGZhy{}annex; setting annex\PYGZhy{}ignore}
\PYG{g+go}{install(ok): /home/me/dl\PYGZhy{}101/midtermproject/input (dataset) [Installed subdataset in order to get /home/me/dl\PYGZhy{}101/midtermproject/input/iris.csv]}
\PYG{g+go}{get(ok): input/iris.csv (file) [from web...]}
\end{sphinxVerbatim}

\sphinxAtStartPar
Nice, this worked well. The output files, however, cannot be easily
retrieved:

\begin{sphinxVerbatim}[commandchars=\\\{\}]
\PYG{g+gp}{\PYGZdl{} }datalad\PYG{+w}{ }get\PYG{+w}{ }prediction\PYGZus{}report.csv\PYG{+w}{ }pairwise\PYGZus{}relationships.png
\PYG{g+go}{get(error): pairwise\PYGZus{}relationships.png (file) [not available; (Note that these git remotes have annex\PYGZhy{}ignore set: origin)]}
\PYG{g+go}{get(error): prediction\PYGZus{}report.csv (file) [not available; (Note that these git remotes have annex\PYGZhy{}ignore set: origin)]}
\end{sphinxVerbatim}

\sphinxAtStartPar
Why is that? This is the first detail of publishing datasets we will dive into.
When publishing dataset content to GitHub with \sphinxcode{\sphinxupquote{datalad push}}, it is
the dataset’s \sphinxstyleemphasis{history}, i.e., everything that is stored in Git, that is
published. The file \sphinxstyleemphasis{content} of these particular files, though, is managed
by {\hyperref[\detokenize{glossary:term-git-annex}]{\sphinxtermref{\DUrole{xref,std,std-term}{git\sphinxhyphen{}annex}}}} and not stored in Git, and
thus only information about the file name and location is known to Git.
Because GitHub does not host large data for free, annexed file content always
needs to be deposited somewhere else (e.g., a web server) to make it
accessible via \sphinxcode{\sphinxupquote{datalad get}}. The chapter {\hyperref[\detokenize{basics/basics-thirdparty:chapter-thirdparty}]{\sphinxcrossref{\DUrole{std,std-ref}{Distributing datasets}}}} (\autopageref*{\detokenize{basics/basics-thirdparty:chapter-thirdparty}})
will demonstrate how this can be done. For this dataset, it is not
necessary to make the outputs available, though: Because all provenance
on their creation was captured, we can simply recompute them with the
\sphinxcode{\sphinxupquote{datalad rerun}} command. If the tag was published we can simply
rerun any \sphinxcode{\sphinxupquote{datalad run}} command since this tag:

\begin{sphinxVerbatim}[commandchars=\\\{\}]
\PYG{g+gp}{\PYGZdl{} }datalad\PYG{+w}{ }rerun\PYG{+w}{ }\PYGZhy{}\PYGZhy{}since\PYG{+w}{ }ready4analysis
\end{sphinxVerbatim}

\sphinxAtStartPar
But without the published tag, we can rerun the analysis by specifying its
shasum:

\begin{sphinxVerbatim}[commandchars=\\\{\}]
\PYG{g+gp}{\PYGZdl{} }datalad\PYG{+w}{ }rerun\PYG{+w}{ }d715890b✂SHA1
\PYG{g+go}{[INFO] run commit d715890; (analyze iris data...)}
\PYG{g+go}{run.remove(ok): pairwise\PYGZus{}relationships.png (file) [Removed file]}
\PYG{g+go}{run.remove(ok): prediction\PYGZus{}report.csv (file) [Removed file]}
\PYG{g+go}{[INFO] == Command start (output follows) =====}
\PYG{g+go}{action summary:}
\PYG{g+go}{  get (notneeded: 2)}
\PYG{g+go}{[INFO] == Command exit (modification check follows) =====}
\PYG{g+go}{run(ok): /home/me/dl\PYGZhy{}101/midtermproject (dataset) [python3 code/script.py]}
\PYG{g+go}{add(ok): pairwise\PYGZus{}relationships.png (file)}
\PYG{g+go}{add(ok): prediction\PYGZus{}report.csv (file)}
\PYG{g+go}{save(ok): . (dataset)}
\PYG{g+go}{action summary:}
\PYG{g+go}{  add (ok: 2)}
\PYG{g+go}{  get (notneeded: 3)}
\PYG{g+go}{  run (ok: 1)}
\PYG{g+go}{  run.remove (ok: 2)}
\PYG{g+go}{  save (notneeded: 1, ok: 1)}
\end{sphinxVerbatim}

\sphinxAtStartPar
Hooray, your analysis was reproduced! You happily note that rerunning your
analysis was incredibly easy \textendash{} it would not even be necessary to have any
knowledge about the analysis at all to reproduce it!
With this, you realize again how letting DataLad take care of linking input,
output, and code can make your life and others’ lives so much easier.
Applying the YODA principles to your data analysis was very beneficial indeed.
Proud of your midterm project you cannot wait to use those principles the
next time again.

\index{push@\spxentry{push}!DataLad concept@\spxentry{DataLad concept}}\index{DataLad concept@\spxentry{DataLad concept}!push@\spxentry{push}}\ignorespaces \begin{gitusernote}[label={index-21}, before title={\thetcbcounter\ }, check odd page=true]{Push internals}
\label{\detokenize{basics/101-130-yodaproject:index-21}}

\sphinxAtStartPar
The \sphinxcode{\sphinxupquote{datalad push}} uses \sphinxcode{\sphinxupquote{git push}}, and \sphinxcode{\sphinxupquote{git annex copy}} under
the hood. Publication targets need to either be configured remote Git repositories,
or git\sphinxhyphen{}annex {\hyperref[\detokenize{glossary:term-special-remote}]{\sphinxtermref{\DUrole{xref,std,std-term}{special remote}}}}s (if they support data upload).


\end{gitusernote}

\sphinxstepscope


\section{Summary}
\label{\detokenize{basics/101-128-summary_yoda:summary}}\label{\detokenize{basics/101-128-summary_yoda:summary-yoda}}\label{\detokenize{basics/101-128-summary_yoda::doc}}
\sphinxAtStartPar
The YODA principles are a small set of guidelines that can make a huge
difference towards reproducibility, comprehensibility, and transparency
in a data analysis project. By applying them in your own midterm analysis
project, you have experienced their immediate benefits.

\sphinxAtStartPar
You also noticed that these standards are not complex \textendash{} quite the opposite,
they are very intuitive.
They structure essential components of a data analysis project \textendash{}
data, code, potentially computational environments, and lastly also the results \textendash{}
in a modular and practical way, and use basic principles and commands
of DataLad you are already familiar with.

\sphinxAtStartPar
There are many advantages to this organization of contents.
\begin{itemize}
\item {} 
\sphinxAtStartPar
Having input data as independent dataset(s) that are not influenced (only
consumed) by an analysis allows for a modular reuse of pure data datasets,
and does not conflate the data of an analysis with the results or the code.
You have experienced this with the \sphinxcode{\sphinxupquote{iris\_data}} subdataset.

\item {} 
\sphinxAtStartPar
Keeping code within an independent, version\sphinxhyphen{}controlled directory, but as a part
of the analysis dataset, makes sharing code easy and transparent, and helps
to keep directories neat and organized. Moreover,
with the data as subdatasets, data and code can be automatically shared together.
By complying to this principle, you were able to submit both code and data
in a single superdataset.

\item {} 
\sphinxAtStartPar
Keeping an analysis dataset fully self\sphinxhyphen{}contained with relative instead of
absolute paths in scripts is critical to ensure that an analysis reproduces
easily on a different computer.

\item {} 
\sphinxAtStartPar
DataLad’s Python API makes all of DataLad’s functionality available in
Python, either as standalone functions that are exposed via \sphinxcode{\sphinxupquote{datalad.api}},
or as methods of the \sphinxcode{\sphinxupquote{Dataset}} class.
This provides an alternative to the command line, but it also opens up the
possibility of performing DataLad commands directly inside of scripts.

\item {} 
\sphinxAtStartPar
Including the computational environment into an analysis dataset encapsulates
software and software versions, and thus prevents re\sphinxhyphen{}computation failures
(or sudden differences in the results) once
software is updated, and software conflicts arising on different machines
than the one the analysis was originally conducted on. You have not yet
experienced how to do this first\sphinxhyphen{}hand, but you will in a later section.

\item {} 
\sphinxAtStartPar
Having all of these components as part of a DataLad dataset allows version
controlling all pieces within the analysis regardless of their size, and
generates provenance for everything, especially if you make use of the tools
that DataLad provides. This way, anyone can understand and even reproduce
your analysis without much knowledge about your project.

\item {} 
\sphinxAtStartPar
The \sphinxcode{\sphinxupquote{cfg\_yoda}} procedure is a good starting point to build your next data analysis
project up on.

\end{itemize}


\subsection{Now what can I do with it?}
\label{\detokenize{basics/101-128-summary_yoda:now-what-can-i-do-with-it}}
\sphinxAtStartPar
Using tools that DataLad provides you are able to make the most out of
your data analysis project. The YODA principles are a guide to accompany
you on your path to reproducibility and provenance\sphinxhyphen{}tracking.

\sphinxAtStartPar
What should have become clear in this section is that you are already
equipped with enough DataLad tools and knowledge that complying to these
standards felt completely natural and effortless in your midterm analysis
project.


\index{use DataLad API@\spxentry{use DataLad API}!with Python@\spxentry{with Python}}\index{with Python@\spxentry{with Python}!use DataLad API@\spxentry{use DataLad API}}\ignorespaces \phantomsection\label{\detokenize{basics/101-130-yodaproject:pythonapi}}\begin{findoutmore}[label={fom-pythonapi}, before title={\thetcbcounter\ }, float, check odd page=true]{DataLad’s Python API}
\label{\detokenize{basics/101-130-yodaproject:fom-pythonapi}}
\phantomsection\label{\detokenize{basics/101-130-yodaproject:python}}
\sphinxAtStartPar
Whatever you can do with DataLad from the command line, you can also do it with
DataLad’s Python API.
Thus, DataLad’s functionality can also be used within interactive Python sessions
or Python scripts.
All of DataLad’s user\sphinxhyphen{}oriented commands are exposed via \sphinxcode{\sphinxupquote{datalad.api}}.
Thus, any command can be imported as a stand\sphinxhyphen{}alone command like this:

\begin{sphinxVerbatim}[commandchars=\\\{\}]
\PYG{g+gp}{\PYGZgt{}\PYGZgt{}\PYGZgt{} }\PYG{k+kn}{from} \PYG{n+nn}{datalad}\PYG{n+nn}{.}\PYG{n+nn}{api} \PYG{k+kn}{import} \PYG{o}{\PYGZlt{}}\PYG{n}{COMMAND}\PYG{o}{\PYGZgt{}}
\end{sphinxVerbatim}

\sphinxAtStartPar
Alternatively, to import all commands, one can use

\begin{sphinxVerbatim}[commandchars=\\\{\}]
\PYG{g+gp}{\PYGZgt{}\PYGZgt{}\PYGZgt{} }\PYG{k+kn}{import} \PYG{n+nn}{datalad}\PYG{n+nn}{.}\PYG{n+nn}{api} \PYG{k}{as} \PYG{n+nn}{dl}
\end{sphinxVerbatim}

\sphinxAtStartPar
and subsequently access commands as \sphinxcode{\sphinxupquote{dl.get()}}, \sphinxcode{\sphinxupquote{dl.clone()}}, and so forth.

\sphinxAtStartPar
The \dlhbhref{D1D}{developer documentation}
of DataLad lists an overview of all commands, but naming is congruent to the
command line interface. The only functionality that is not available at the
command line is \sphinxcode{\sphinxupquote{datalad.api.Dataset}}, DataLad’s core Python data type.
Just like any other command, it can be imported like this:

\begin{sphinxVerbatim}[commandchars=\\\{\}]
\PYG{g+gp}{\PYGZgt{}\PYGZgt{}\PYGZgt{} }\PYG{k+kn}{from} \PYG{n+nn}{datalad}\PYG{n+nn}{.}\PYG{n+nn}{api} \PYG{k+kn}{import} \PYG{n}{Dataset}
\end{sphinxVerbatim}

\sphinxAtStartPar
or like this:

\begin{sphinxVerbatim}[commandchars=\\\{\}]
\PYG{g+gp}{\PYGZgt{}\PYGZgt{}\PYGZgt{} }\PYG{k+kn}{import} \PYG{n+nn}{datalad}\PYG{n+nn}{.}\PYG{n+nn}{api} \PYG{k}{as} \PYG{n+nn}{dl}
\PYG{g+gp}{\PYGZgt{}\PYGZgt{}\PYGZgt{} }\PYG{n}{dl}\PYG{o}{.}\PYG{n}{Dataset}\PYG{p}{(}\PYG{p}{)}
\end{sphinxVerbatim}

\sphinxAtStartPar
A \sphinxcode{\sphinxupquote{Dataset}} is a \dlhbhref{P1B}{class}
that represents a DataLad dataset. In addition to the
stand\sphinxhyphen{}alone commands, all of DataLad’s functionality is also available via
\dlhbhref{P1C}{methods}
of this class. Thus, these are two equally valid ways to create a new
dataset with DataLad in Python:

\begin{sphinxVerbatim}[commandchars=\\\{\}]
\PYG{g+gp}{\PYGZgt{}\PYGZgt{}\PYGZgt{} }\PYG{k+kn}{from} \PYG{n+nn}{datalad}\PYG{n+nn}{.}\PYG{n+nn}{api} \PYG{k+kn}{import} \PYG{n}{create}\PYG{p}{,} \PYG{n}{Dataset}
\PYG{g+go}{\PYGZsh{} create as a stand\PYGZhy{}alone command}
\PYG{g+gp}{\PYGZgt{}\PYGZgt{}\PYGZgt{} }\PYG{n}{create}\PYG{p}{(}\PYG{n}{path}\PYG{o}{=}\PYG{l+s+s1}{\PYGZsq{}}\PYG{l+s+s1}{scratch/test}\PYG{l+s+s1}{\PYGZsq{}}\PYG{p}{)}
\PYG{g+go}{[INFO   ] Creating a new annex repo at /.../scratch/test}
\PYG{g+go}{Out[3]: \PYGZlt{}Dataset path=/home/me/scratch/test\PYGZgt{}}
\PYG{g+go}{\PYGZsh{} create as a dataset method}
\PYG{g+gp}{\PYGZgt{}\PYGZgt{}\PYGZgt{} }\PYG{n}{ds} \PYG{o}{=} \PYG{n}{Dataset}\PYG{p}{(}\PYG{n}{path}\PYG{o}{=}\PYG{l+s+s1}{\PYGZsq{}}\PYG{l+s+s1}{scratch/test}\PYG{l+s+s1}{\PYGZsq{}}\PYG{p}{)}
\PYG{g+gp}{\PYGZgt{}\PYGZgt{}\PYGZgt{} }\PYG{n}{ds}\PYG{o}{.}\PYG{n}{create}\PYG{p}{(}\PYG{p}{)}
\PYG{g+go}{[INFO   ] Creating a new annex repo at /.../scratch/test}
\PYG{g+go}{Out[3]: \PYGZlt{}Dataset path=/home/me/scratch/test\PYGZgt{}}
\end{sphinxVerbatim}

\sphinxAtStartPar
As shown above, the only required parameter for a Dataset is the \sphinxcode{\sphinxupquote{path}} to
its location, and this location may or may not exist yet.

\sphinxAtStartPar
Stand\sphinxhyphen{}alone functions have a \sphinxcode{\sphinxupquote{dataset=}} argument, corresponding to the
\sphinxcode{\sphinxupquote{\sphinxhyphen{}d/\sphinxhyphen{}\sphinxhyphen{}dataset}} option in their command\sphinxhyphen{}line equivalent. You can specify
the \sphinxcode{\sphinxupquote{dataset=}} argument with a path (string) to your dataset (such as
\sphinxcode{\sphinxupquote{dataset=\textquotesingle{}.\textquotesingle{}}} for the current directory, or \sphinxcode{\sphinxupquote{dataset=\textquotesingle{}path/to/ds\textquotesingle{}}} to
another location). Alternatively, you can pass a \sphinxcode{\sphinxupquote{Dataset}} instance to it:

\begin{sphinxVerbatim}[commandchars=\\\{\}]
\PYG{g+gp}{\PYGZgt{}\PYGZgt{}\PYGZgt{} }\PYG{k+kn}{from} \PYG{n+nn}{datalad}\PYG{n+nn}{.}\PYG{n+nn}{api} \PYG{k+kn}{import} \PYG{n}{save}\PYG{p}{,} \PYG{n}{Dataset}
\PYG{g+go}{\PYGZsh{} use save with dataset specified as a path}
\PYG{g+gp}{\PYGZgt{}\PYGZgt{}\PYGZgt{} }\PYG{n}{save}\PYG{p}{(}\PYG{n}{dataset}\PYG{o}{=}\PYG{l+s+s1}{\PYGZsq{}}\PYG{l+s+s1}{path/to/dataset/}\PYG{l+s+s1}{\PYGZsq{}}\PYG{p}{)}
\PYG{g+go}{\PYGZsh{} use save with dataset specified as a dataset instance}
\PYG{g+gp}{\PYGZgt{}\PYGZgt{}\PYGZgt{} }\PYG{n}{ds} \PYG{o}{=} \PYG{n}{Dataset}\PYG{p}{(}\PYG{l+s+s1}{\PYGZsq{}}\PYG{l+s+s1}{path/to/dataset}\PYG{l+s+s1}{\PYGZsq{}}\PYG{p}{)}
\PYG{g+gp}{\PYGZgt{}\PYGZgt{}\PYGZgt{} }\PYG{n}{save}\PYG{p}{(}\PYG{n}{dataset}\PYG{o}{=}\PYG{n}{ds}\PYG{p}{,} \PYG{n}{message}\PYG{o}{=}\PYG{l+s+s2}{\PYGZdq{}}\PYG{l+s+s2}{saving all modifications}\PYG{l+s+s2}{\PYGZdq{}}\PYG{p}{)}
\PYG{g+go}{\PYGZsh{} use save as a dataset method (no dataset argument)}
\PYG{g+gp}{\PYGZgt{}\PYGZgt{}\PYGZgt{} }\PYG{n}{ds}\PYG{o}{.}\PYG{n}{save}\PYG{p}{(}\PYG{n}{message}\PYG{o}{=}\PYG{l+s+s2}{\PYGZdq{}}\PYG{l+s+s2}{saving all modifications}\PYG{l+s+s2}{\PYGZdq{}}\PYG{p}{)}
\end{sphinxVerbatim}

\sphinxAtStartPar
\sphinxstylestrong{Use cases for DataLad’s Python API}

\sphinxAtStartPar
Using the command line or the Python API of DataLad are both valid ways to accomplish the same results.
Depending on your workflows, using the Python API can help to automate dataset operations, provides an alternative
to the command line, or could be useful for scripting reproducible data analyses.
One unique advantage of the Python API is the \sphinxcode{\sphinxupquote{Dataset}}:
As the Python API does not suffer from the startup time cost of the command line,
there is the potential for substantial speed\sphinxhyphen{}up when doing many calls to the API,
and using a persistent Dataset object instance.
You will also notice that the output of Python commands can be more verbose as the result records returned by each command do not get filtered by command\sphinxhyphen{}specific result renderers.
Thus, the outcome of \sphinxcode{\sphinxupquote{dl.status(\textquotesingle{}myfile\textquotesingle{})}} matches that of \sphinxcode{\sphinxupquote{datalad status}} only when \sphinxcode{\sphinxupquote{\sphinxhyphen{}f}}/\sphinxcode{\sphinxupquote{\sphinxhyphen{}\sphinxhyphen{}output\sphinxhyphen{}format}} is set to \sphinxcode{\sphinxupquote{json}} or \sphinxcode{\sphinxupquote{json\_pp}}, as illustrated below.

\begin{sphinxVerbatim}[commandchars=\\\{\}]
\PYG{g+gp}{\PYGZgt{}\PYGZgt{}\PYGZgt{} }\PYG{k+kn}{import} \PYG{n+nn}{datalad}\PYG{n+nn}{.}\PYG{n+nn}{api} \PYG{k}{as} \PYG{n+nn}{dl}
\PYG{g+gp}{\PYGZgt{}\PYGZgt{}\PYGZgt{} }\PYG{n}{dl}\PYG{o}{.}\PYG{n}{status}\PYG{p}{(}\PYG{l+s+s1}{\PYGZsq{}}\PYG{l+s+s1}{myfile}\PYG{l+s+s1}{\PYGZsq{}}\PYG{p}{)}
\PYG{g+go}{[\PYGZob{}\PYGZsq{}type\PYGZsq{}: \PYGZsq{}file\PYGZsq{},}
\PYG{g+go}{\PYGZsq{}gitshasum\PYGZsq{}: \PYGZsq{}915983d6576b56792b4647bf0d9fa04d83ce948d\PYGZsq{},}
\PYG{g+go}{\PYGZsq{}bytesize\PYGZsq{}: 85,}
\PYG{g+go}{\PYGZsq{}prev\PYGZus{}gitshasum\PYGZsq{}: \PYGZsq{}915983d6576b56792b4647bf0d9fa04d83ce948d\PYGZsq{},}
\PYG{g+go}{\PYGZsq{}state\PYGZsq{}: \PYGZsq{}clean\PYGZsq{},}
\PYG{g+go}{\PYGZsq{}path\PYGZsq{}: \PYGZsq{}/home/me/my\PYGZhy{}ds/myfile\PYGZsq{},}
\PYG{g+go}{\PYGZsq{}parentds\PYGZsq{}: \PYGZsq{}/home/me/my\PYGZhy{}ds\PYGZsq{},}
\PYG{g+go}{\PYGZsq{}status\PYGZsq{}: \PYGZsq{}ok\PYGZsq{},}
\PYG{g+go}{\PYGZsq{}refds\PYGZsq{}: \PYGZsq{}/home/me/my\PYGZhy{}ds\PYGZsq{},}
\PYG{g+go}{\PYGZsq{}action\PYGZsq{}: \PYGZsq{}status\PYGZsq{}\PYGZcb{}]}
\end{sphinxVerbatim}

\begin{sphinxVerbatim}[commandchars=\\\{\}]
\PYG{g+gp}{\PYGZdl{} }datalad\PYG{+w}{ }\PYGZhy{}f\PYG{+w}{ }json\PYGZus{}pp\PYG{+w}{ }status\PYG{+w}{ }myfile
\PYG{g+go}{ \PYGZob{}\PYGZdq{}action\PYGZdq{}: \PYGZdq{}status\PYGZdq{},}
\PYG{g+go}{  \PYGZdq{}bytesize\PYGZdq{}: 85,}
\PYG{g+go}{  \PYGZdq{}gitshasum\PYGZdq{}: \PYGZdq{}915983d6576b56792b4647bf0d9fa04d83ce948d\PYGZdq{},}
\PYG{g+go}{  \PYGZdq{}parentds\PYGZdq{}: \PYGZdq{}/home/me/my\PYGZhy{}ds\PYGZdq{},}
\PYG{g+go}{  \PYGZdq{}path\PYGZdq{}: \PYGZdq{}/home/me/my\PYGZhy{}ds/myfile\PYGZdq{},}
\PYG{g+go}{  \PYGZdq{}prev\PYGZus{}gitshasum\PYGZdq{}: \PYGZdq{}915983d6576b56792b4647bf0d9fa04d83ce948d\PYGZdq{},}
\PYG{g+go}{  \PYGZdq{}refds\PYGZdq{}: \PYGZdq{}/home/me/my\PYGZhy{}ds/\PYGZdq{},}
\PYG{g+go}{  \PYGZdq{}state\PYGZdq{}: \PYGZdq{}clean\PYGZdq{},}
\PYG{g+go}{  \PYGZdq{}status\PYGZdq{}: \PYGZdq{}ok\PYGZdq{},}
\PYG{g+go}{  \PYGZdq{}type\PYGZdq{}: \PYGZdq{}file\PYGZdq{}\PYGZcb{}}
\end{sphinxVerbatim}


\end{findoutmore}



\sphinxstepscope


